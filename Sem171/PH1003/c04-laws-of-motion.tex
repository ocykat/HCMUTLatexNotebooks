\chapter{The laws of motion}

    \hi{The concept of Force}

        \hii{Definition}
            \begin{itemize}
                \item A force cause an object to \textit{accelerate}.
                \item The \textbf{net force} acting on an object is defined as the vector
                sum of all forces acting on the object. If the net force exerted on the
                object is zero, the acceleration of the object is zero and its velocity
                remains constant.
                \item When the velocity of an object is constant (including when the object
                is at rest), the object is said to be in \textbf{equilibrium}.
            \end{itemize}
            
        \hii{Two types of forces}
            \par There are two types of forces:
            \begin{itemize}
            \item \textbf{Contact forces} which involve physical contact between two objects.
            \item \textbf{Field forces} which do not involve physical contact between two
                objects.
            \end{itemize}

        \hii{Four fundamental forces}
            \begin{itemize}
                \item Gravitational forces
                \item Electromagnetic forces
                \item Nuclear forces
                \item Weak forces
            \end{itemize}

    \hi{Newton's Laws}
        \hii{Newton's First Law and Inertial Frame}
            \par Newton's First Law of motion, sometimes called the \textit{law of inertia},
            defines a special set of reference frames called \textit{inertial frames}. The
            law can be stated as follow:
            \par \textit{If an object does not interact with other objects, it is possible
            to identify a reference frame in which the object has zero acceleration}.
            \par Such inertial frame is called an \textbf{inertial frame of reference}.
            \par Any reference frame that moves with constant velocity relative to an
            inertial frame is itself an inertia frame.
            \par The law can also be stated in a more practical way:
            \par \textit{In the absence of external forces, when viewed from an inertial
            reference frame, an object at rest remains at rest and an object in motion
            continues in motion with a constant velocity (that is, with a constant speed in
            a straight line)}.

        \hii{Newton's Second Law}
            \par \textit{When viewed from an inertial reference frame, the acceleration of an
            object is directly propotional to the net force acting on it and inversely propotional
            to its mass}.
            \begin{equation}
                \sum F = ma
            \end{equation}
            \par in which:
                \begin{itemize}
                    \item $\sum F$: net force $[N]$ or $[kg \mul m/s]$
                    \item $m$: mass $[kg]$
                    \item $a$: acceleration $[m/s^2]$
                \end{itemize}

        \hii{Newton's Third Law}
            \par \textit{If two objects interact, the force $F_{12}$ exerted by object 1 on object
            2 is equal in magnitude and opposite in direction to the force $F_{21}$ exerted by
            object 2 on object 1.}
            \begin{equation}
                F_{12} = -F_{21}
            \end{equation}

        \hii{Problem Solving Hints}
            \par Procedure:
            \begin{itemize}
                \item Draw a diagram to help conceptualize the problem.
                \item Categorize the problem:
                    \begin{itemize}
                        \item If $a = 0$, the particle is in equilibrium.
                            This means: $\sum F = 0$.
                        \item If $a \neq 0$, the particle is undergoing an acceleration.
                            This means: $\sum F = ma$.
                    \end{itemize}
                \item Establish convenient coordinate axes for each object and find the components
                    of the force in these axes. (making a table is a good idea).
                \item Apply Newton's second law in component form.
                \item Solve the component equations for the unknowns. The number of unknowns and
                    the number of independent equations must be the same.
                \item Point out the direction of the unknown vectors (if any). \textit{(Example:
                    The direction of $\vt{a}$ relative to the positive $x$ axis is: $45 \degree$)}.
            \end{itemize}

    \hi{Force of Friction}
        \hii{Normal Force}
            \par The normal force is defined as the net force compressing two parallel surfaces
            together, and its direction is perpendicular to the surfaces.
        \hii{Force of Friction}
            \par When an object is in motion either on a surface or in a viscous medium such as air
            or water, there is resistance to the motion because the object interacts with its
            surroundings. We call such resistance a \textbf{force of friction}.
        \hii{Static Friction}
            \par Static friction is friction between two or more solid objects that are not
            moving relative to each other.
            \par The static friction force must be overcome by an applied force before an object
            can move. The maximum possible friction force between two surfaces before sliding
            begins is the product of the coefficient of static friction and the normal force.
            \begin{equation}
                F_{s-max} = \mu_{s} F_{n}
            \end{equation}
            in which:
            \begin{itemize}
                \item $F_{max}$: maximum possible friction force $[N]$
                \item $\mu_{s}$: coefficient of static friction
                \item $F_{n}$: normal force $[N]$
            \end{itemize}
        \hii{Kinetic Friction}
            \par Kinetic friction occurs when two objects are moving relative to each other and rub
            togeter.
            \par The kinetic friction force between two surfaces after sliding begins is the product
            of the coefficient of kinetic friction and the normal force.
            \begin{equation}
                F_{k} = \mu_{k} F_{n}
            \end{equation}
            in which:
            \begin{itemize}
                \item $F_{max}$: maximum possible friction force $[N]$
                \item $\mu_{s}$: coefficient of static friction
                \item $F_{n}$: normal force $[N]$
            \end{itemize}

\pagebreak
