\chapter{Pointers and Memory}

\hi{Pointers}
  \hii{Memory Address}
    \par When variables are stored in the memory, each variable has its own
      address. The computer uses this address to locate where a variable is
      when it is called.

  \hii{Pointers}
    \par A \tb{pointer} is a variable whose value is an address.

  \hii{Operations related to pointers}

    \hiii{Address-of operator (\&)}
      \par An \tb{address-of} operator \tb{obtains the address} of a variable in
        the memory.
      \par \ti{Note}: An address-of operator should be applied on a \ti{normal variable}.

\begin{lstlisting}[language=C++]
int a = 10;
int* p = &a; // p stores the address of variable a
\end{lstlisting}

    \hiii{Deference operator(*)}
      \par A \tb{deference} operator \tb{obtains the value} located at the memory
        address stored by the pointer.
      \par \ti{Note}:
        \begin{enumerate}[1.]
          \item A deference operator should be applied on a \tb{pointer}.
          \item Do not mistake between two different uses of the \lstinline{*}
            symbol. The first use is to declare a pointer. The second use is to
            deference a pointer.
        \end{enumerate}
\begin{lstlisting}[language=C++]
int a = 10;
int* p = &a; // p stores the address of variable a
int b = *p;  // b = 10
\end{lstlisting}

\hi{Memory}
  \hii{Types of memory}
    \par When doing programming, we should distinguish between two different
      types of memory: \tb{stack memory} and \tb{heap memory}.
    \par In a computer, stack and heap are both parts of the RAM memory:

