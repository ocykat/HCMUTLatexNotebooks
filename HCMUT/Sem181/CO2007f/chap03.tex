\begin{multicols}{3}

\hi{Arithmetics}
  
  \hii{Overflow}
    \par \tb{Overflow} cannot occur by:
      \begin{itemize} 
        \item adding operands of different sign
        \item subtracting operands of the same sign
      \end{itemize}
    \begin{tabular}{|c|c|c|c|}
    \hline
    \tb{Operation} & \tb{A} & \tb{B} & \tb{Result indicating overflow} \\ \hline
    A + B            & +          & +          & -                     \\ \hline
    A + B            & -          & -          & +                     \\ \hline
    A - B            & +          & -          & -                     \\ \hline
    A - B            & -          & +          & +                     \\ \hline
    \end{tabular}

  \hii{Multiplication}
      \par If sign bit is ignore: n bits $\times$ m bits = (n$+$m) bits
      \begin{algorithm}[H]
        %\caption{}
        \begin{algorithmic}[1]
          \Procedure{MULTIPLY}{a, b}
            \Let{prod}{0}
            \For{i = 1 to 32}
              \If{\Call{LSB}{b} = 1}
                \Let{prod}{prod + a}
              \ENDIF
              \Let{a}{a $\SHIFTLEFT$ 1}
              \Let{b}{b $\SHIFTRIGHT$ 1}
            \ENDFOR
          \ENDPROCEDURE
        \end{algorithmic}
      \end{algorithm}
    \par A 64-bit register is required to store the multiplicand, since it has to
      be shifted left 32 times.
     \par Instructions for multiplication:
      \begin{itemize}
        \item \lstinline{mult rs, rt}: multiply two signed numbers and store the
          64-bit product in HI/LO
        \item \lstinline{multu rs, rt}: multiply two unsigned numbers and store the
          64-bit product in HI/LO
        \item \lstinline{mfhi rd}: move the value of \lstinline{HI} to \lstinline{rd}
        \item \lstinline{mflo rd}: move the value of \lstinline{LO} to \lstinline{rd}
        \item \lstinline{rd, rs, rt}: store the least-significant 32 bits of
          the product to rd.
      \end{itemize} 

  \hii{Division}
      \begin{algorithm}[H]
        %\caption{}
        \begin{algorithmic}[1]
          \Procedure{DIVIDE}{a, b}
            \Let{q}{0}
            \Let{r}{a}
            \Let{b}{b $\SHIFTLEFT$ 32}
            \For{i = 1 to 33}
              \Let{r}{r - a}
              \If{r $\geq$ 0}
                \Let{q}{q $\SHIFTLEFT$ 1}
                \Let{\Call{LSB}{q}}{1}
              \Else
                \Let{q}{q $\SHIFTLEFT$ 1}
                \Let{\Call{LSB}{q}}{0}
                \Let{r}{r + a}
              \ENDIF
              \Let{b}{b $\SHIFTRIGHT$ 1}
            \ENDFOR
          \ENDPROCEDURE
        \end{algorithmic}
      \end{algorithm}
    \par The correct signed division algorithm negates the quotient if the
    signs of the operands are opposite and makes the sign of the nonzero
    remainder match the dividend.
    \par Instructions for division:
      \begin{itemize}
        \item \lstinline{div rs, rt}: does division on two signed numbers and
          store the results in HI/LO.
        \item \lstinline{divu rs, rt}: does division on two unsigned numbers and
          store the 64-bit product in HI/LO
        \item \lstinline{mfhi rd}: move the value of \lstinline{HI} to \lstinline{rd}
        \item \lstinline{mflo rd}: move the value of \lstinline{LO} to \lstinline{rd}
      \end{itemize}
    \par The HI register stores the remainder and the LO register stores the
      quotient after a division.

  \hii{IEEE 754 floating-point standard}
    \par Order from MSB to LSB: 1 sign bit, exponent bits, fraction bits.
    \par For \tb{single-precision} numbers:
      \begin{itemize}
        \item exponent: 8 bits
        \item fraction: 23 bits
      \end{itemize}
    \par For \tb{double-precision} numbers:
      \begin{itemize}
        \item exponent: 11 bits
        \item fraction: 20 + 32 = 52 bits
      \end{itemize}
    \par Representation of floating-point numbers:
        \begin{align*}
          (-1)^{S} \times (1 + \mbox{ Fraction }) \times 2^{(\mbox{ Exponent }
          - \mbox{ Bias }}
        \end{align*}
      \begin{itemize}
        \item For single precision: the bias is \tb{127}.
        \item For double precision: the bias is \tb{1023}.
      \end{itemize}

  \hii{Floating-Point Addition}
    \par \tb{Algorithm}:
    \begin{algorithm}[H]
      %\caption{}
      \begin{algorithmic}[1]
        \Function{FPAdd}{addend1, addend2}
          \Let{a}{\Call{Larger}{addend1, addend2}}
          \Let{b}{\Call{Smaller}{addend1, addend2}}
          \While{b.exponent $!=$ a.exponent}
            \Let{b}{b $\SHIFTRIGHT$ 1}
          \EndWhile
          \Let{s}{a + b}
          \While{s \tb{not} normalized}
            \If{s $\geq$ 10}
              \While{s \tb{not} normalized}
                \Let{s}{s $\SHIFTRIGHT$ 1}
                \Let{s.exponent}{s.exponent + 1}
              \EndWhile
            \ElsIf{s $\leq$ 0}
              \While{s \tb{not} normalized}
                \Let{s}{s $\SHIFTLEFT$ 1}
                \Let{s.exponent}{s.exponent - 1}
              \EndWhile
            \EndIf
            \If{overflow \tb{or} underflow} \tb{throw} Exception
            \EndIf
            \State \Call{Round}{s}
          \EndWhile
          \RETURN{s}
        \EndFunction
      \end{algorithmic}
    \end{algorithm}

      \hii{Floating-Point Multiplication}
    \par \tb{Algorithm}:
    \begin{algorithm}[H]
      %\caption{}
      \begin{algorithmic}[1]
        \Function{FPMult}{a, b}
          \Let{prod.exponent}{a.exponent + b.exponent - bias}
          \Let{prod}{a $\times$ b}
          \While{prod \tb{not} normalized}
            \If{prod $\geq$ 10}
              \While{prod \tb{not} normalized}
                \Let{prod}{prod $\SHIFTRIGHT$ 1}
                \Let{prod.exponent}{prod.exponent + 1}
              \EndWhile
            \ElsIf{prod $\leq$ 0}
              \While{prod \tb{not} normalized}
                \Let{prod}{prod $\SHIFTLEFT$ 1}
                \Let{prod.exponent}{prod.exponent - 1}
              \EndWhile
            \EndIf
            \If{overflow \tb{or} underflow} \tb{throw} Exception
            \EndIf
            \State \Call{Round}{prod}
          \EndWhile
          \RETURN{prod}
        \EndFunction
      \end{algorithmic}
    \end{algorithm}
  
\end{multicols}
\noindent\makebox[\linewidth]{\rule{\paperwidth}{0.4pt}}