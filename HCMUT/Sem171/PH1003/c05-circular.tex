\chapter{Circular Motion and Other Applications of Newton's Laws}
    \hi{Uniform Circular Motion}
        \hii{Centripetal acceleration}
            \par When a particle moving with uniform speed $v$ in a circular path of radius $r$
            experiences an acceleration that has a magnitude:
            \begin{equation}
                a_{c} = \dfrac{v^2}{r}
            \end{equation}
            \par The acceleration is called \impt{centripetal acceleration} because it is directed
            toward the centre of the circular path.
        \hii{Centripetal force}
            \par Applying Newton's second law along the radial direction, the net force causing
            the centripetal acceleration can be evaluated:
            \begin{equation}
                \sum F = ma_{c} = \dfrac{mv^2}{r}
            \end{equation}
    
    \hi{Nonuniform Circular Motion}
        \par If a particle moves with varying speed in a circular path, in addition to the
        \impt{radial component} of acceleration, there is a \impt{tangential component} having
        the magninude:
        \begin{equation}
            a_{t} = \dfrac{dv}{dt}
        \end{equation}
        and responsible for the change in the speed of the particle with time.
        \par Because:
        \begin{equation}
            \sum a = a_{r} + a_{t}
        \end{equation}
        applying the Newton's second law, we have:
        \begin{equation}
            \sum F = F_{r} + F_{t}
        \end{equation}

    \hi{Motion in Accelerated Frames}
        \hii{Validity of Newton's laws}
            \par Newton's laws are only valid only when observations are made in an inertial
            frame of reference.

        \hii{Non-inertial Frame of Reference}
            \par A \impt{non-inertial} reference frame is a frame of reference that is undergoing
            acceleration with respect to an inertial frame.

        \hii{Fictitous Force}
            \par A \impt{fictitous force} (also called \impt{inertial force}) is an apparent
            force that acts on all masses whose motion is described using a non-inertial frame
            of reference.

            \par In a coordinate system S' which moves by translation relative to an inertial
            system S, the motion of a mechanical system takes place as if the coordinate system
            were inertial, but on every point of mass m an additional "inertial force" acted:
            \begin{equation}
                F = -ma
            \end{equation}
            where a is the acceleration of the system S'.

        \hii{Equilibrium in an Accelerated Frame}
            \par A \impt{fictitous force} (also called \impt{inertial force}) is an apparent
            force that acts on all masses whose motion is described using a non-inertial frame
            of reference.
            \par When the frame accelerates with the acceleration $a$, if the object in the frame
            is in equilibrium, then according to an inertial observer, there is a force $F$
            exerting on the object with the magnitude:
            \begin{equation}
                \sum F = ma
            \end{equation}

    \hi{Motion in Presence of Resistive Forces}