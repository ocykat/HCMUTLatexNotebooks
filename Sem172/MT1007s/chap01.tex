\chapter{Systems of Linear Equations and Matrices}

\hi{Introduction to Systems of Linear Equations}

  \hii{Linear Equations}

    \par In two dimensions a line in a rectangular $xy$-coordinate system
    can be represented by an equation of the form:

    \begin{align*}
      ax + by = c \quad (a, b \mb{not both} 0)
    \end{align*}

    \par In three dimensions a plane in a rectangular $xyz$-coordinate
    system can be represented by an equation of the form:

    \begin{align*}
      ax + by = c \quad (a, b, c \mb{not all} 0)
    \end{align*}

    \par These equations are called \tb{linear equations}.

    \par A linear equation of $n$ variables can be expressed in the form:

    \begin{align*}
      a_{1}x_{1} + a_{2}x_{2} + \ldots + a_{n}x_{n} = b
    \end{align*}

    \par If $b = 0$, the equation is called a \tb{homogeneous linear
    equation} in the variables $x_{1}, x_{2}, \ldots, x_{n}$.

  \hii{Linear Systems}
    \par A finite set of linear equation is called a \tb{system of linear
    equations} or a \tb{linear system}.

    \par A general linear system of $m$ equations in the $n$ unknowns
    $x_{1}, x_{2}, \ldots, x_{n}$ can be written as:

    \begin{align*}
      \begin{cases}
        a_{11}x_{1} + a_{12}x_{2} + \ldots + a_{1n}x_{n} = b_{1} \\
        a_{21}x_{1} + a_{22}x_{2} + \ldots + a_{2n}x_{n} = b_{2} \\
        \ldots \\
        a_{n1}x_{1} + a_{n2}x_{2} + \ldots + a_{nn}x_{n} = b_{n}
      \end{cases}
    \end{align*}

    \par A \tb{solution} of a linear system in $n$ unknowns
    $x_{1}, x_{2}, \ldots, x_{n}$ is a sequence of $n$ numbers
    $s_{1}, s_{2}, \ldots, s_{n}$ for which the substitution:
    \begin{align*}
      x_{1} = s_{1}, x_{2} = s_{2}, \ldots, x_{n} = s_{n}
    \end{align*}
    makes each equation a true statement.
    \par A solution can be written in the form of an \tb{ordered n-tuple}.
    \begin{align*}
      (s_{1}, s_{2}, \ldots, s_{n})
    \end{align*}

    \par We say that a linear system is \tb{consistent} if it has at least
    one solution, and \tb{inconsistent} if it has no solutions.

  \hii{Augmented Matrices and Elementary Row Operations}

    \par A linear system can be abbreviated with a matrix called
    \tb{augmented matrix}:

    \begin{align*}
      \begin{pmatrix}
        a_{11} & a_{12} & \ldots & a_{1n} & b_{1} \\
        a_{21} & a_{22} & \ldots & a_{2n} & b_{2} \\
        \vdots & \vdots & \vdots & \vdots & \ldots \\
        a_{n1} & a_{n2} & \ldots & a_{nn} & b_{n} \\
      \end{pmatrix}
    \end{align*}

    \par The method of solving a linear system is to perform appropriate
    algebraic operations on the system that:
    \begin{itemize}
      \item do not alter the solution set 
      \item produce a succession of increasingly simpler systems, until
        the solution is obtained if any.
    \end{itemize}
    \par These algebraic operations are typically used:
    \begin{enumerate}
      \item Multiply an equation through by a nonzero constant.
      \item Interchange two equations.
      \item Add a constant times one equation to another.
    \end{enumerate}

  \hii{Elementary row operations}
    \par The mentioned algebraic operations on linear systems are 
    equivalent to these on matrices:

    \begin{enumerate}
      \item Multiply a row through by a nonzero constant.
      \item Interchange two rows.
      \item Add a constant times one row to another.
    \end{enumerate}

    \par These operations are called \tb{elementary row operations} on a
    matrix.


\hi{Gaussian Elimination}

  \hii{Considerations in Solving Linear Systems}
    \par Although there are more than one way to solve linear equations,
    Gauusian Elimination is arguably the most systematic, and the
    most popular when it comes to dealing with a large system.

  \hii{Echelon Forms}
    \par A matrix is of \tb{Echelon form} (or more precisely,
    \tb{reduced row Echelon form}) if it consists of these properties:
    \begin{enumerate}
      \item If a row does not consist entirely of zeros, then the first
      nonzero number in the row is a 1. We call this a \tb{leading 1}.
      \item If there are any rows that consist entirely of zeros, then
      they are grouped together at the bottom of the matrix.
      \item In any two successive rows that do not consist entirely of
      zeros, the leading 1 in the lower row occurs farther to the right
      than the leading 1 in the higher row.
      \item Each column that contains a leading 1 has zeros everywhere
      else in that column.
    \end{enumerate}
    \par A matrix which has the first 3 properties is said to be in
    \tb{row echelon form}.


\hi{Matrices and Matrix Operations}
  \hiiBEGIN{Matrix Notation and Terminology}
    \hiii{Matrix and Entries}
      \begin{itemize}
        \item A \tb{matrix} is a rectangular array of numbers.
        \item A $m \times n$ matrix is a matrix with $m$ \tb{rows}
          and $n$ \tb{columns}.
        \item Each number in a matrix is called an \tb{entry}.
        \item The entry that occurs in row $i$ and column $j$ of a matrix $A$
          is denoted by $a_{ij}$.
        \item A matrix can also be denoted by $[a_{ij}]_{m \times n}$ or
          $[a_{ij}]$.
        \item An entry can also be denoted as $(A)_{ij}$.
      \end{itemize}
    \hiii{Row and Column Matrices}
      \begin{itemize}
        \item A matrix with only one column is called a \tb{column vector} or
          \tb{column matrix}.
        \item A matrix with only one row is called a \tb{row vector} or
          \tb{row matrix}.
      \end{itemize}
    \hiii{Square Matrix}
      \par 
  \hiiEND
