\chapter{Instructions: Language of the Computer}


\hi{Operations of the Computer Hardware}
  \BOOKSECTION{2.2}
  \par \tb{Instruction types}:
  \begin{itemize}
    \item Arithmetic
    \item Data transfer
    \item Logical
    \item Conditional branch
    \item Unconditional jump
  \end{itemize}

\hi{Operands}
  \hii{Registers}
    \par In the MIPS architecture:
    \begin{itemize}
      \item the size of a register is \tb{32 bits}.
      \item a group of 32 bits is called \tb{word}.
      \item there are 32 different registers.
      \item each word must start at addresses that are multiples of 4 (known
        as alignment restriction).
    \end{itemize}
    \par The address of a register is the address of either its leftmost byte
      (MSb) or its rightmost byte (LSb). Architecture that has register address
      to be the MSb is called big-endian. Architecture that has register address
      to be the LSb is called little-endian.
  \hii{Arithmetic operands}
    \par One arithmetic operand is followed by 3 registers.
    \begin{lstlisting}
# $s1 contains value of a
# $s2 contains value of b
# $t0 contains value of c
add $s1, $s2, $t0 # a = b + c
    \end{lstlisting}
  \hii{Memory operands}
    \par The data transfer instruction that copies data from memory to a
      register is called \tb{load}.
    \par The data transfer instruction that copies data from a register back
      to memory is called \tb{store}.
    \begin{lstlisting}
# $s3 keeps the address of array a
lw $t0, 8($s3) # reg $t0 gets the value of a[8]
    \end{lstlisting}
  \hii{Constant/Immediate Operands}
    \par There is a special register: \reg{zero}.
