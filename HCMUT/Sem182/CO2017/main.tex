\documentclass[12pt, a4paper]{report}

% MARGINS
\usepackage[margin=0.75in]{geometry}

% FONT
\renewcommand{\familydefault}{\sfdefault}

% LINE SPACING
\setlength{\parskip}{6pt}

% HEADINGS
\setcounter{secnumdepth}{4}
\newcommand{\hi}{\section}
\newcommand{\hii}{\subsection}
\newcommand{\hiii}{\subsubsection}
\newcommand{\hiiiBEGIN}[1]{\subsubsection \begin{enumerate}}
\newcommand{\hiiiEND}{\end{enumerate}}
\newcommand{\hiv}{\item\textbf}

% INDENTATION
\usepackage{indentfirst}

% TABLE OF CONTENTS
% links
\usepackage{hyperref}
\hypersetup{
  colorlinks,
  citecolor=black,
  filecolor=black,
  linkcolor=black,
  urlcolor=black
}

% APPENDICES
\usepackage[toc,page]{appendix}

% TEXT
% Bold, italic, underlined text
\newcommand{\tb}[1]{\textbf{#1}}
\newcommand{\ti}[1]{\textit{#1}}
\newcommand{\tbi}[1]{\textbf{\textit{#1}}}
\newcommand{\tu}[1]{\underline{#1}}
\newcommand{\tbu}[1]{\textbf{\underline{#1}}}
\newcommand{\smf}[1]{\small #1 \normalsize}
\newenvironment{smfont}{\small}{\normalsize}

% MATH
\usepackage{amsmath}
\usepackage{amssymb}
\usepackage{gensymb}

% Equation box
\usepackage{empheq}
\newenvironment{eqbox}
  {\setkeys{EmphEqEnv}{align}\setkeys{EmphEqOpt}{box=\fbox}\EmphEqMainEnv}
  {\endEmphEqMainEnv}

% Arrows
\newcommand{\ra}{\Rightarrow}
\newcommand{\lra}{\Leftrightarrow}

% Sum, product, limit
\newcommand{\SUM}[1]{\sum\limits #1}
\newcommand{\PROD}[1]{\prod\limits #1}

% Bold in math mode
\usepackage{bm}

% Absolute value
\DeclarePairedDelimiter\abs{\lvert}{\rvert}%
\DeclarePairedDelimiter\norm{\lVert}{\rVert}%
%   Swap the definition of \abs* and \norm*, so that \abs
%   and \norm resizes the size of the brackets, and the
%   starred version does not.
\makeatletter
\let\oldabs\abs
\def\abs{\@ifstar{\oldabs}{\oldabs*}}
\let\oldnorm\norm
\def\norm{\@ifstar{\oldnorm}{\oldnorm*}}
\makeatother

% Floor and ceiling
\usepackage{mathtools}
\DeclarePairedDelimiter\ceil{\lceil}{\rceil}
\DeclarePairedDelimiter\floor{\lfloor}{\rfloor}
\DeclarePairedDelimiter\set\{\}

% Inner Product
\newcommand{\iprod}[1]{\langle #1 \rangle}

% Derivatives
\newcommand{\dif}[2]{\dfrac{d #1}{d #2}}
\newcommand{\ddif}[2]{\dfrac{d^2 #1}{d #2^2}}
\newcommand{\difi}[2]{d #1/d #2}
\newcommand{\diff}[2]{\dfrac{d}{d #2} #1}
\newcommand{\pd}[2]{\dfrac{\partial #1}{\partial #2}}
\newcommand{\pdds}[2]{\dfrac{\partial^{2} #1}{\partial #2^{2}}}
\newcommand{\pdd}[3]{\dfrac{\partial^{2} #1}{\partial #2 \partial #3}}
\newcommand{\pddsi}[2]{{\partial^{2} #1} / {\partial #2^{2}}}
\newcommand{\pddi}[3]{{\partial^{2} #1} / {\partial #2 \partial #3}}

% Integrals
\usepackage{esint}
\newcommand{\INT}{\int \limits}
\newcommand{\OINT}{\oint \limits}
\newcommand{\IINT}{\iint \limits}
\newcommand{\IIINT}{\iiint \limits}

% Ratios
\newcommand{\ratio}[3]{\dfrac{#1_{#2}}{#1_{#3}}}

% Manual equation number
\newcommand{\eqnumber}[1]{\qquad\mbox{\tb{{(#1)}}}}

% FOOTNOTE
%   one foot note stays on one page
\interfootnotelinepenalty=10000

\newcommand{\fnmark}{\footnotemark}
\newcommand{\fnmarksame}{\footnotemark[\value{footnote}]}
\newcommand{\fntext}{\footnotetext}

% === IMAGES ===
\usepackage{graphicx}
\usepackage{caption}
\usepackage{subcaption}
\usepackage{float}
\newcommand{\img}[3][]
{
  \begin{figure}[H]
    \centering
    \includegraphics[#1]{#2}
    \caption*{#3}
  \end{figure}
}
% Usage: \img[width=...]{<path_to_img>}{<caption>}

% PSEUDOCODE
\usepackage{algorithm}
\usepackage{algorithmicx}
\usepackage[noend]{algpseudocode}
\usepackage{caption}

\renewcommand{\thealgorithm}{\arabic{chapter}.\arabic{algorithm}}

\newcommand*\Let[2]{\State #1 $\gets$ #2}
\algrenewcommand\algorithmicrequire{\textbf{Input:}}
\algrenewcommand\algorithmicensure{\textbf{Output:}}
\newcommand{\INPUT}[1]{\Require{#1} \Statex}
\newcommand{\OUTPUT}[1]{\Ensure{#1} \Statex}
\newcommand{\INPUTOUTPUT}[2]{\Require{#1} \Ensure{#2} \Statex}
\newcommand{\LET}[2]{\Let{$#1$}{$#2$}}
\newcommand{\FOR}[2]{\For{$#1 \gets #2$}}
\newcommand{\ENDFOR}{\EndFor}
\newcommand{\TO}{\textrm{ \tb{to} }}
\newcommand{\DOWNTO}{\textrm{ \tb{downto} }}
\newcommand{\AND}{\textrm{ \tb{and} }}
\newcommand{\OR}{\textrm{ \tb{or} }}
\newcommand{\XOR}{\textrm{ \tb{xor} }}
\newcommand{\GETS}{\gets}
\newcommand{\IF}[1]{\If{$#1$}}
\newcommand{\ELSEIF}[1]{\ElsIf{$#1$}}
\newcommand{\ELSE}{\Else}
\newcommand{\ENDIF}{\EndIf}
\newcommand{\WHILE}[1]{\While{$#1$}}
\newcommand{\ENDWHILE}{\EndWhile}
\newcommand{\FUNCTION}[2]{\Function{#1}{$#2$}}
\newcommand{\ENDFUNCTION}{\EndFunction}
\newcommand{\PROCEDURE}[2]{\Procedure{#1}{$#2$}}
\newcommand{\ENDPROCEDURE}{\EndProcedure}
\newcommand{\CALLFUNC}[2]{\State \Call{#1}{$#2$}}
\newcommand{\CALLPROC}[2]{\State \Call{#1}{$#2$}}
\newcommand{\RETURN}[1]{\State \Return{#1}}
\newcommand{\SHIFTLEFT}{\ll}
\newcommand{\SHIFTRIGHT}{\gg}

% FRENCH CONVENTION FOR VERTICAL ARITHMETIC
\usepackage{xlop}

% CODE
\usepackage{listings}
%\makeatletter
%\lst@CCPutMacro\lst@ProcessOther {"2D}{\lst@ttfamily{-{}}{-{}}}
%\@empty\z@\@empty
%\makeatother
\usepackage{inconsolata}
\lstset{
  language=[x86masm]Assembler,
  aboveskip=1mm,
  belowskip=1mm,
  numbers=left,
  frame=tb,
  % language=[mips]Assembler,
  basicstyle=\ttfamily
  \footnotesize,
  breaklines=true,
  mathescape=false
}

\usepackage{varwidth}
\newenvironment{fboxenv}
  {
    \begin{minipage}{\dimexpr\textwidth-2\fboxsep-2\fboxrule\relax}
  }
  {
    \end{minipage}
  }

\usepackage[shortlabels]{enumitem}

% SOURCE
%\newcommand{\code}[1]{\texttt{#1}}
\newcommand{\refsource}[1]{Source: {#1}}

\begin{document}
\tableofcontents

\hi{Logic}

\hii{Natural Deduction Rules}

% AndIntroduction
\newsavebox\AndIntro
\sbox\AndIntro{
  \AxiomC{$\phi$}
  \AxiomC{$\psi$}
  \RightLabel{$\land i$}
  \BinaryInfC{$\phi \land \psi$}
  \DisplayProof
}

% AndElimination
\newsavebox\AndElim
\sbox\AndElim{
  \AxiomC{$\phi \land \psi$}
  \RightLabel{$\land e_1$}
  \UnaryInfC{$\phi$}
  \DisplayProof
  \hskip 1cm
  \AxiomC{$\phi \land \psi$}
  \RightLabel{$\land e_2$}
  \UnaryInfC{$\psi$}
  \DisplayProof
}

% Double-Negation Introduction
\newsavebox\DNegIntro
\sbox\DNegIntro{
  \AxiomC{$\phi$}
  \RightLabel{$\lnotnot i$}
  \UnaryInfC{$\lnotnot \phi$}
  \DisplayProof
}

% Double-Negation Elimination 
\newsavebox\DNegElim
\sbox\DNegElim{
  \AxiomC{$\lnotnot \phi$}
  \RightLabel{$\lnotnot e$}
  \UnaryInfC{$\phi$}
  \DisplayProof
}

% 

\begin{tabular}{|c|c|c|}
  \hline
  Rule              & Introduction      & Elimination      \\ \hline
  Conjunction (AND) & \usebox\AndIntro  & \usebox\AndElim  \\ \hline
  Double Negation   & \usebox\DNegIntro & \usebox\DNegElim \\ \hline
\end{tabular}

\setcounter{chapter}{2}
\begin{multicols*}{3}

\hi{Distributions}
  \hii{Discrete Distributions}

  \hiii{Binomial Distribution}
    \begin{itemize}
      \item \tb{Intepretation}:
        \begin{itemize}
          \item The number of winning game in $n$ games, given that the chance to win any one game is $p$.
          \item The number of red balls obtained after picking \tb{with replacement} from $n$ balls from a box of red and blue balls, given that the chance to pick a red ball at one time is $p$.
        \end{itemize}
      \item \tb{Notation}:
        \[
          X \sim B(n, p)
        \]
      \item \tb{P.M.F}:
        \[
          f(k) = P(X = k) = {n \choose k} p^k q^{n - k}
        \]
      \item \tb{Expectation}:
        \[
          \mu = E(X) = np
        \]
      \item \tb{Variance}:
        \[
          \sigma^2 = V(X) = npq = np(1 - p)
        \]
    \end{itemize}

  \hiii{Hypergeometric Distribution}
    \begin{itemize}
      \item \tb{Intepretation}:
        \begin{itemize}
          \item The number of red balls obtained after picking \tb{without replacement} $n$ balls from $N$ balls from a box of red and blue balls, given that the number of red balls in the box is $m$.
        \end{itemize}
      \item \tb{Notation}:
        \[
          X \sim H(N, m, n)
        \]
      \item \tb{P.M.F}:
        \[
          f(k) = P(X = k) = \dfrac{{m \choose k} {N - m \choose n - k}}{{N \choose n}}
        \]
      \item \tb{Expectation}:
        \[
          \mu = E(X) = np, \text{ where } p = \frac{m}{N}
        \]
      \item \tb{Variance}:
        \[
          \sigma^2 = V(X) = npq \frac{N - n}{N - 1}, \text{ where } p = \frac{m}{N}
        \]
    \end{itemize}

  \hiii{Poisson Distribution}
    \begin{itemize}
      \item \tb{Intepretation}:
        \begin{itemize}
          \item The number of events occuring in a fixed period of time.
        \end{itemize}
      \item \tb{Notation}:
        \[
          X \sim Po(\lambda)
        \]
      \item \tb{P.M.F}:
        \[
          f(k) = P(X = k) = \frac{\lambda^k}{k!} e^{-\lambda}
        \]
      \item \tb{Expectation}:
        \[
          \mu = E(X) = \lambda
        \]
      \item \tb{Variance}:
        \[
          \sigma^2 = V(X) = \lambda
        \]
    \end{itemize}

  \hii{Continuous Distributions}

  \hiii{Uniform Distribution}
    \begin{itemize}
      \item \tb{Notation}:
        \[
          X \sim U(a, b)
        \]
      \item \tb{P.D.F}:
        \[
          f(x) =
            \begin{cases}
              \dfrac{1}{b - a} & x \in [a, b]\\
              0 & \text{otherwise}
            \end{cases}
        \]
      \item \tb{C.D.F}:
        \[
          F(x) =
            \begin{cases}
              0 & x < a \\
              \dfrac{x - a}{b - a} & x \in [a, b)\\
              1 & x \geq b
            \end{cases}
        \]
      \item \tb{Expectation}:
        \[
          \mu = E(X) = \frac{a + b}{2}
        \]
      \item \tb{Variance}:
        \[
          \sigma^2 = V(X) = \frac{(b - a)^2}{12}
        \]
    \end{itemize}

  \hiii{Normal Distribution}
    \begin{itemize}
      \item \tb{Notation}:
        \[
          X \sim N(\mu, \sigma^2)
        \]
      \item \tb{Standardizing}:
        \[
          Y = \frac{X - \mu}{\sigma} \sim N(0, 1)
        \]
      \item \tb{C.D.F}:
        \par For $X \sim N(0, 1)$:
        \[
          \Phi(x) = \int\limits_{-\infty}^{x} f(u) du
        \]
      \item \tb{Expectation}:
        \[
          \mu = E(X) = \mu
        \]
      \item \tb{Variance}:
        \[
          \sigma^2 = V(X) = \sigma^2
        \]
      \item \tb{Properties}
        \begin{itemize}
          \item If $X \sim N(\mu, \sigma^2)$ and $Y = aX + b$:
            \[
              Y \sim N(a\mu + b, a^2 \sigma^2)
            \]
          \item If $X \sim N(\mu, \sigma^2)$:
            \[
              \sum\limits_{i = 1}^{n} X_i \sim N\bigg( \sum\limits_{i = 1}^{n} \mu_i, \sum\limits_{i = 1}^{n} \sigma_i^2 \bigg)
            \]
        \end{itemize}
    \end{itemize}


\end{multicols*}


\begin{multicols*}{3}

\hi{Appendix}

  \hii{Integrals}
    \begin{itemize}
      \item \[\int x^n dx = \frac{1}{n + 1} x^{n + 1} + C \]
      \item \[\int \frac{1}{x} dx = \ln|x| + C \]
      \item \[\int \frac{1}{t^2} dx = - \frac{1}{t} + C\]
      \item \[\int e^x dx = e^x + C\]
      \item \[\int a^x dx = \frac{a^x}{\ln(a)} + C\]
      \item \[\int e^{ax} dx = \frac{e^{ax}}{a} + C\]
      \item \[\int x e^x dx = (x - 1)e^x + C\]
      \item \tb{Integration by parts}:
        \[
          \int udv = uv - v \int du
        \]
        \par Choose $u$ in this order: log - algebraic - trig - exp
    \end{itemize}

\end{multicols*}

\begin{multicols}{3}
\hi{Confidence Intervals}

  \hii{Definitions and Terminology}
    \hiii{Terminology}
      \begin{itemize}
        \item A \tb{population} is a collection of objects about which information is sought.
        \item A \tb{sample} is a part of the population that is observed.
        \item A \tb{parameter} is a numerical characteristic of a population.
        \item A \tb{statistics} is a numerical function of the sampled data, used to estimate an unknown parameter
      \end{itemize}

    \hiii{Mean}
      \begin{itemize}
      \item \tb{Population Mean}: denoted $\mu$, is the average of all values in the \tb{entire population}.
      \item \tb{Sample Mean}: denoted $\hat{\mu} = \bar{X}$, is the average value of a \tb{sample}. For a sample with $n$ values:
        \[
          \hat{\mu} = \bar{X} = \frac{x_1 + \ldots + x_n}{n}
        \]
      \end{itemize}

    \hiii{Median}
      \par List all data values in sorted order from smallest to largest. For a sample with $n$ values:
      \begin{itemize}
        \item If $n \not \mid 2$: the mean is one single value at the middle position.
        \item If $n \mid 2$: the mean is the average of two values at the middle positions.
      \end{itemize}

    \hiii{Variance}
      \par The \tb{sample variance}, denoted $s^2$, is the approximation of the \tb{population variance} $\sigma^2$.
      \[
        s^2 = \frac{\sum (x_i - \bar{X})^2}{n - 1} = \frac{\sum (x_i - \hat{\mu})^2}{n - 1}
      \]
      where $n - 1$ is called the \tb{degree of freedom}.

    \hiii{Confidence Level and Significance Level}
      \begin{itemize}
        \item \tb{Confidence level} $\gamma$ is the measure of the degree of reliability of the interval.
        \item \tb{Significance level} $\alpha$ is the probability we allow ourselves to be wrong when we are estimating a parameter with a confidence interval.
      \end{itemize}
      \[
        \gamma + \alpha = 1
      \]

  \hii{Case 1: Normal Population with Known Standard Deviation}
    \hiii{Confidence Interval of the Population Mean}
    \par If $x_1, \ldots, x_n$ are independent and identically distributed (i.i.d.), following a normal distribution $N(\mu, \sigma^2)$ and $\alpha = 1 - \gamma$, then the \tb{confidence interval of the population mean} is:
    \[
      \mu = \hat{\mu} \pm z_{\alpha / 2} \times \frac{\sigma}{\sqrt{n}}
    \]

    \hii{Sample Size According to Maximum Error}
      \par If $\bar{X}$ is used to estimate $\mu$, then we can be $100(1 - a)\%$ confident that the error will not exceed a specified amount $\epsilon$ when the sample size is:
      \[
        n = \bigg\lceil \frac{\sigma z_{\alpha/2}}{\epsilon} \bigg\rceil^2
      \]

    \hii{One-Sided Convidence Interval of the Population Mean}
      \begin{itemize}
        \item A $100(1 - \alpha)\%$ upper-confidence bound for $\mu$ is
        \[
          \mu \leq \hat{\mu} + z_{\alpha} \times \frac{\sigma}{\sqrt(n)}
        \]
        \item A $100(1 - \alpha)\%$ lower-confidence bound for $\mu$ is
        \[
          \mu \geq \hat{\mu} - z_{\alpha} \times \frac{\sigma}{\sqrt(n)}
        \]
      \end{itemize}

  \hii{Case 2: Normal Population with Unknown Standard Deviation}
    \hiii{Confidence Interval of the Population Mean}
     \par If $x_1, \ldots, x_n$ are independent and identically distributed (i.i.d.) then the \tb{confidence interval of the population mean} is:
     \[
      \mu = \hat{\mu} \pm t_{n - 1, \alpha / 2} \times \frac{s}{\sqrt(n)}
     \]
    \hiii{Confidence Interval of the Population Variance}
      \begin{itemize}
        \item \tb{Two-sided Confidence Interval}
        \par Choose $c_1$ and $c_2$ so that the area in each tail of $\chi_{n - 1}^2$ distribution is $\alpha / 2$. Then the $\gamma$-confidence interval
        for the unknown variance $\sigma^2$ is:
        \[
          \frac{(n - 1)s^2}{c_2} \leq \sigma^2 \leq \frac{(n - 1)s^2}{c_1}
        \]
        \item \tb{One-sided Confidence Interval}
        \par Choose $c_1$ and $c_2$ so that the area in each tail of $\chi_{n - 1}^2$ distribution is $\alpha$. Then the $\gamma$-confidence interval
        for the unknown variance $\sigma^2$ are:
        \[
          \sigma^2 \geq \frac{(n - 1)s^2}{c_2}
        \]
        and
        \[
          \sigma^2 \leq \frac{(n - 1)s^2}{c_1}
        \]
      \end{itemize}

    \hii{Case 3: Large Sample Size}
      \hiii{Confidence Interval of the Population Mean}
      \par If $x_1, \ldots, x_n$ are independent and identically distributed (i.i.d.) and $n$ is large, then:
        \[
          \mu = \hat{\mu} \pm z_{\alpha / 2} \times \frac{s}{\sqrt(n)}
        \]

\end{multicols}

\clearpage

\begin{multicols*}{3}
  
\hi{Hypothesis Testing for One Sample}

\hii{Terminology}
  \begin{itemize}
    \item \tb{null hypothesis}:  the claim that is initially assumed to be true, denoted by $H_0$
    \item \tb{alternative hypothesis}: the assertion that is contradictory to $H_0$, denoted by $H_1$
    \item \tb{test of hypotheses}: a method for using sample data to decide \tb{whether the null hypothesis should be rejected}.
    \item \tb{significant level}: the probability of a type I error, denoted by $\alpha$
  \end{itemize}

\hii{Tips}
  \begin{itemize}
    \item The \tb{equal sign} only appears in the \tb{null hypothesis} $H_0$.
    \item The thing we want to prove appears in the \tb{alternative hypothesis} $H_1$. 
  \end{itemize}

\hii{Case 1: Normal Population with Known Standard Deviation}
  \hiii{Hypothesis Testing}
  \begin{itemize}
    \item \tb{Step 1}: Compute the statistic:
      \[
        z = \frac{\bar{x} - \mu_0}{\sigma / \sqrt{n}}
      \]
    \item \tb{Step 2}: Apply the decision rule:
      \begin{center}
        \begin{tabular}{|c|c|}
          \hline
          \textbf{$H_1$}   & \textbf{Rejection Region} \\ \hline
          $\mu \neq \mu_0$ & $|z| > z_{\alpha/2}$      \\ \hline
          $\mu < \mu_0$    & $z < -z_{\alpha}$         \\ \hline
          $\mu > \mu_0$    & $z > z_{\alpha}$          \\ \hline
        \end{tabular}
      \end{center}
  \end{itemize}

    \hiii{Hypothesis Testing with Propotion}
    \begin{itemize}
      \item \tb{Step 1}: Compute the statistic:
        \[
          z = \frac{\hat{p} - p_0}{\sqrt{\hat{p}\hat{q}/n}}
        \]
      \item \tb{Step 2}: Apply the decision rule:
        \begin{center}
          \begin{tabular}{|c|c|}
            \hline
            \textbf{$H_1$}   & \textbf{Rejection Region} \\ \hline
            $p \neq p_0$ & $|z| > z_{\alpha/2}$      \\ \hline
            $p < p_0$    & $z < -z_{\alpha}$         \\ \hline
            $p > p_0$    & $z > z_{\alpha}$          \\ \hline
          \end{tabular}
        \end{center}
    \end{itemize}

\hii{Case 2: Normal Population with Unknown Standard Deviation}

\begin{itemize}
  \item \tb{Step 1}: Compute the statistic:
    \[
      t = \frac{\bar{x} - \mu_0}{s / \sqrt{n}}
    \]
  \item \tb{Step 2}: Apply the decision rule:
      \begin{center}
        \begin{tabular}{|c|c|}
          \hline
          \textbf{$H_1$}   & \textbf{Rejection Region}   \\ \hline
          $\mu \neq \mu_0$ & $|t| > t_{\alpha/2, n - 1}$ \\ \hline
          $\mu < \mu_0$    & $t < -t_{\alpha, n - 1}$    \\ \hline
          $\mu > \mu_0$    & $t > t_{\alpha, n - 1}$     \\ \hline
        \end{tabular}
      \end{center}
\end{itemize}

\hii{Case 3: Any Distribution - Large Sample Size}

\begin{itemize}
  \item \tb{Step 1}: Compute the statistic:
    \[
      z = \frac{\bar{x} - \mu_0}{s / \sqrt{n}}
    \]
  \item \tb{Step 2}: Apply the decision rule:
      \begin{center}
        \begin{tabular}{|c|c|}
          \hline
          \textbf{$H_1$}   & \textbf{Rejection Region} \\ \hline
          $\mu \neq \mu_0$ & $|z| > z_{\alpha/2}$      \\ \hline
          $\mu < \mu_0$    & $z < -z_{\alpha}$         \\ \hline
          $\mu > \mu_0$    & $z > z_{\alpha}$          \\ \hline
        \end{tabular}
      \end{center}
\end{itemize}

\par \tb{Testing with Population Proportion - Large Sample Size}

\begin{itemize}
  \item \tb{Step 1}: Compute the statistic:
    \[
      z = \frac{\hat{p} - p_0}{\sqrt{p_0 q_0 / n}}
    \]
  \item \tb{Step 2}: Apply the decision rule:
      \begin{center}
        \begin{tabular}{|c|c|}
          \hline
          \textbf{$H_1$}   & \textbf{Rejection Region} \\ \hline
          $p \neq p_0$ & $|z| > z_{\alpha/2}$      \\ \hline
          $p < p_0$    & $z < -z_{\alpha}$         \\ \hline
          $p > p_0$    & $z > z_{\alpha}$          \\ \hline
        \end{tabular}
      \end{center}
\end{itemize}


\end{multicols*}

\chapter{Basic SQL}

\hi{SQL Data Definition and Data Types}
  \hii{Terminology}
    \begin{itemize}
      \item A \tb{table} is equivalent to a \tb{relation}.
      \item A \tb{row} is equivalent to a \tb{tuple}.
      \item A \tb{column} is equivalent to an \tb{attribute}.
    \end{itemize}

  \hii{Schema}
    \hiii{Schema}
      \par A \tb{schema} is a group of related tables.
      \par An \tb{SQL schema}:
        \begin{itemize}
          \item is identified by a \tb{schema name}
          \item includes an authorization identifier to indicate the user or account who owns the schema
          \item includes descriptors for each element in the schema.
        \end{itemize}

      \par \tb{Creating a schema}:
      % \begin{lstlisting}[style=SQL]
% \begin{minted}[linenos,tabsize=2,breaklines]{SQL}
\begin{lstlisting}
CREATE SCHEMA `schema_name' AUTHORIZATION `author_name'
\end{lstlisting}

  \hii{Catalog}
    \par A \tb{catalog} is a named collection of schemas.

  \hii{CREATLE TABLE command in SQL}
    \par The CREATE TABLE command is used to specify a new relation by
      \begin{itemize}
      \item giving it a name
      \item specifying its attributes and initial constraints.
      \end{itemize}
    \par The attributes are specified first, and each attribute is given a name, a data type to specify its domain of values, and possibly attribute constraints, such as NOT NULL. The key, entity integrity, and referential integrity constraints can be specified within the CREATE TABLE statement
after the attributes are declared, or they can be added later using the ALTER TABLE command.
    \par The relations declared through CREATE TABLE statements are called base tables (or base relations); this means that the table and its rows are actually created and stored as a file by the DBMS. Base relations are distinguished from virtual relations, created through the CREATE VIEW statement (see Chapter 7), which may or may not correspond to an actual physical file.
    \par In SQL, columns are ordered while rows are not.

  \hii{Attribute Data Types and Domains in SQL}

\hi{Specifying Constraints in SQL}
  \hii{Specifying Attribute Constraints and Attribute Defaults}
    \par SQL allows NULLs as attribute values, a constraint NOT NULL may be specified if NULL is not permitted for a particular attribute.
    \par Primary keys are always implicitly specified as NOT NULL.
    \par It is also possible to define a default value for an attribute by appending the clause
      \lstinline{DEFAULT <value>} to an attribute definition.

  \hii{Specifying Key and Referential Integrity Constraints}
    \par The PRIMARY KEY clause specifies one or more attributes that make up the primary
key of a relation. If a primary key has a single attribute, the clause can follow the
attribute directly.
    \par The UNIQUE clause specifies alternate (unique) keys, also known as candidate keys.
    \par Referential integrity is specified via the FOREIGN KEY clause.
    \par The default action that SQL takes for an integrity violation is to reject the update operation that will cause a violation, which is known as the RESTRICT option. However, the schema designer can specify an alternative action to be taken by attaching a referential triggered action clause to any foreign key constraint. The options include SET NULL, CASCADE, and SET DEFAULT. An
option must be qualified with either ON DELETE or ON UPDATE.
    \par A constraint may be given a constraint name, following the keyword CONSTRAINT .

  \hii{Specifying Constraints on Tuples Using CHECK}
    \par In addition to key and referential integrity constraints, which are specified by special keywords, other table constraints can be specified through additional CHECK clauses at the end of a CREATE TABLE statement. These can be called row-based constraints because they apply to each row individually and are checked whenever a row is inserted or modified.

\hi{Basic Retrieval Queries in SQL}

  \hii{Important Notes on SQL}
    \par SQL allows a table (relation) to have two or more tuples that are identical in all their attribute values. Hence, in general, an SQL table is not a set of tuples, because a set does not
allow two identical members; rather, it is a multiset (sometimes called a bag) of tuples. Some SQL relations are constrained to be sets because a key constraint has been declared or because the DISTINCT option has been used with the SELECT statement.

    \par The SELECT statement of SQL \ti{is not the same} as the SELECT operation of relational algebra.

  \hii{The SELECT-FROM-WHERE Structure of Basic SQL Queries}
    % \begin{minted}[linenos,tabsize=2,breaklines]{SQL}
    \begin{lstlisting}
SELECT <attribute_list>
FROM <table_list>
WHERE <condition>;
    \end{lstlisting}
      where
    \begin{itemize}
      \item \lstinline{<attribute list>} is a list of attribute names whose values are to be retrieved by the query.
      \item \lstinline{<table list>} is a list of the relation names required to process the query.
      \item \lstinline{<condition>} is a conditional (Boolean) expression that identifies the tuples to be retrieved by the query.
    \end{itemize}

    \par \tb{Logical comparison operators}: \lstinline{=, <, <=, >, >=, <>}

    \begin{itemize}
      \item The \lstinline{SELECT} clause specifies the attributes whose values are to be retrieved, which are called the \tb{projection attributes}.
      \item The \lstinline{WHERE} clause specifies the Boolean condition that must be true for any retrieved tuple, which is known as the \tb{selection condition}.
      \item A selection condition that joins two different tuples is called a \tb{join condition}.
      \item A query that involves only selection and join conditions plus projection attributes is known as a \tb{select-project-join} query.
    \end{itemize}

  \hii{Ambiguous Attribute Names, Aliasing, Renaming, and Tuple Variables}
    \par In SQL, the same name can be used for many attributes as long as they are in different tables. If this is the case, and a multitable query refers to two or more attributes with the same name, we must qualify the attribute name with the relation name to prevent ambiguity. This is done by prefixing the relation name to the attribute name and separating the two by a period.
    \par The ambiguity of attribute names also arises in the case of queries that refer to the
same relation twice (recursive ?). In this case, we are required to declare alternative relation names E and S , called aliases or tuple variables, for the EMPLOYEE relation. An alias can follow the keyword AS.
      % \begin{minted}[linenos,tabsize=2,breaklines]{SQL}
      \begin{lstlisting}
SELECT E.Fname, E.Lname, S.Fname, S.Lname
FROM EMPLOYEE AS E, EMPLOYEE AS S
WHERE E.Super_ssn = S.Ssn;
      \end{lstlisting}
    \par It is also possible to rename the relation \tb{attributes} within the query in SQL by giving them aliases.

  \hii{Unspecified WHERE Clause and Use of the Asterisk}
    \par A missing WHERE clause indicates no condition on tuple selection; hence, all tuples of the relation specified in the FROM clause qualify and are selected for the query result.

    \par If more than one relation is specified in the FROM clause and there is no WHERE clause, then the CROSS PRODUCT —all possible tuple combinations—of these relations is selected.

    \par To retrieve all the attribute values of the selected tuples, we do not have to list the attribute names explicitly in SQL; we just specify an asterisk (*), which stands for all the attributes. The * can also be prefixed by the relation name or alias; for example, EMPLOYEE.* refers to all attributes of the EMPLOYEE table.

  \hii{Tables as Sets in SQL}
    \par If we do want to eliminate duplicate tuples from the result of an SQL query, we use the keyword DISTINCT in the SELECT clause, meaning that only distinct tuples should remain in the result. In general, a query with SELECT DISTINCT eliminates duplicates, whereas a query with SELECT ALL does not. Specifying SELECT with neither ALL nor DISTINCT —as in our previous examples—is equivalent to SELECT ALL. If we do want to eliminate duplicate tuples from the result of an SQL query, we use the keyword DISTINCT in the SELECT clause, meaning that only distinct tuples should remain in the result. In general, a query with SELECT DISTINCT eliminates duplicates, whereas a query with SELECT ALL does not. Specifying SELECT with neither ALL nor DISTINCT is
equivalent to SELECT ALL .
    \par SQL has directly incorporated some of the set operations from mathematical set
theory, which are also part of relational algebra: set union (UNION), set difference (EXCEPT), and set intersection (INTERSECT) operations. The relations resulting from these set operations are sets of tuples; that is, duplicate tuples are eliminated from the result. These set operations apply only to \tb{type compatible relations}, so we must make sure that the two relations on which we apply the operation have the same attributes and that the attributes appear in the same order in both relations.
  \par SQL also has corresponding multiset operations, which are followed by the keyword ALL (UNION ALL, EXCEPT ALL, INTERSECT ALL). Their results are multisets (duplicates are not eliminated). Basically, each tuple—whether it is a duplicate or not— is considered as a different tuple when applying these operations.

  \hii{Substring Pattern Matching and Arithmetic Operators}
    \par \tb{Example}: Retrieve all employees whose address is in Houston, Texas.
      % \begin{minted}[linenos,tabsize=2,breaklines]{SQL}
      \begin{lstlisting}
SELECT Fname, Lname
FROM   EMPLOYEE
WHERE  Address LIKE ‘%Houston,TX%’;
      \end{lstlisting}
    \par \tb{Example}: Find all employees who were born during the 1950s.
      % \begin{minted}[linenos,tabsize=2,breaklines]{SQL}
      \begin{lstlisting}
SELECT Fname, Lname
FROM EMPLOYEE
WHERE Bdate LIKE ‘_ _ 5 _ _ _ _ _ _ _’;
      \end{lstlisting}

  \hii{Ordering of Query Results}
    \par SQL allows the user to order the tuples in the result of a query by the values of one or more of the attributes that appear in the query result.
    \par \tb{Example}: Retrieve a list of employees and the projects they are working on, ordered by department and, within each department, ordered alphabetically by last name, then first name.
      % \begin{minted}[linenos,tabsize=2,breaklines]{SQL}
      \begin{lstlisting}
SELECT D.Dname, E.Lname, E.Fname, P.Pname
FROM DEPARTMENT AS D, EMPLOYEE AS E, WORKS_ON AS W, PROJECT AS P
WHERE  D.Dnumber = E.Dno AND E.Ssn = W.Essn AND W.Pno = P.Pnumber
ORDER BY D.Dname, E.Lname, E.Fname;
      \end{lstlisting}
    \par Ascending and Descending:
       % \begin{minted}[linenos,tabsize=2,breaklines]{SQL}
       \begin{lstlisting}
    ORDER BY D.Dname DESC, E.Lname ASC, E.Fname ASC
      \end{lstlisting}

\hi{INSERT, DELETE, and UPDATE Statements in SQL}
  \hii{The INSERT Command}

    \par In its simplest form, INSERT is used to add a single tuple (row) to a relation (table). We must specify the relation name and a list of values for the tuple. The values should be listed in the same order in which the corresponding attributes were specified in the CREATE TABLE command.

      % \begin{minted}[linenos,tabsize=2,breaklines]{SQL}
      \begin{lstlisting}
INSERT INTO EMPLOYEE
VALUES ( 'Richard', 'K', 'Marini', '653298653', '1962-12-30', '98 Oak Forest, Katy, TX', 'M', 37000, ‘653298653’, 4 );
      \end{lstlisting}

    \par A second form of the INSERT statement allows the user to specify explicit attribute
    names that correspond to the values provided in the INSERT command. This is use-
    ful if a relation has many attributes but only a few of those attributes are assigned
    values in the new tuple. However, the values must include all attributes with NOT
    NULL specification and no default value. Attributes with NULL allowed or DEFAULT
    values are the ones that can be left out.

      % \begin{minted}[linenos,tabsize=2,breaklines]{SQL}
      \begin{lstlisting}
INSERT INTO EMPLOYEE(Fname, Lname, Dno, Ssn)
VALUES (‘Richard’, ‘Marini’, 4, ‘653298653’);
      \end{lstlisting}

    \par A DBMS that fully implements SQL should support and enforce all the integrity
constraints that can be specified in the DDL.

  \hii{The DELETE Command}
    \par The DELETE command removes tuples from a relation. It includes a WHERE clause, similar to that used in an SQL query, to select the tuples to be deleted. Tuples are explicitly deleted from only one table at a time. However, the deletion may propagate to tuples in other relations if referential triggered actions are specified in the referential integrity constraints of the DDL.
    \par A missing WHERE clause specifies that all tuples in the relation are to be deleted; however, the table remains in the database as an empty table. We must use the DROP TABLE command to remove the table definition.
    % \begin{minted}[linenos,tabsize=2,breaklines]{SQL}
    \begin{lstlisting}
DELETE FROM EMPLOYEE
WHERE Lname = ‘Brown’;
    \end{lstlisting}

  \hii{The UPDATE Command}
    \par The UPDATE command is used to modify attribute values of one or more selected tuples. As in the DELETE command, a WHERE clause in the UPDATE command selects the tuples to be modified from a single relation. However, updating a primary key value may propagate to the foreign key values of tuples in other relations if such a referential triggered action is specified in the referential integrity constraints of the DDL.
    \par An additional SET clause in the UPDATE command specifies the attributes to be modified and their new values.
    % \begin{minted}[linenos,tabsize=2,breaklines]{SQL}
    \begin{lstlisting}
UPDATE PROJECT
SET Plocation = ‘Bellaire’, Dnum = 5
WHERE Pnumber = 10;

UPDATE EMPLOYEE
SET Salary = Salary * 1.1
WHERE Dno = 5;
    \end{lstlisting}

\hi{Aggregate Functions}
  \par \tb{Purpose}: summarize information from multiple tuples into a single-tuple summary.

  \par \tb{List of built-in aggregate functions}:
    \begin{itemize}
      \item SUM
      \item AVG
      \item COUNT
      \item MAX
      \item MIN
    \end{itemize}

\par These functions can be used in the SELECT clause or in a HAVING clause

  \hii{Built-in Aggregate Functions}

    \hiii{SUM and AVG}
      \begin{itemize}
        \item set/multiset of numeric values
      \end{itemize}

      % \begin{minted}[linenos,tabsize=2,breaklines]{SQL}
      \begin{lstlisting}
        SELECT SUM(attribute_name)
        FROM   table_name;
      \end{lstlisting}

    \hiii{MAX and MIN}
      \begin{itemize}
        \item Comparable attributes (the domain values have a total ordering among one another)
      \end{itemize}

    \hiii{COUNT}
      \par Returns the \ti{number of tuples or values} that satisfy a condition.
      \par The asterisk refers to \ti{all rows (tuples)}.
      \par Examples:
        \begin{itemize}
          \item Simple count all
          \item Count non-distinct
          \item Count distinct
        \end{itemize}

\hi{Grouping: GROUP BY and HAVING Clauses}
  \par \tb{Objective}: Partition in to non-overlapping subsets (or groups) and apply aggregate functions.

  \hii{GROUP BY}
    \par Each group consist of the tuples that have the same value of some attributes, called \tb{grouping attributes}

    \par The grouping attributes can be defined by the \lstinline{GROUP BY} clause. The GROUP BY clause specifies the grouping attributes, which should also appear in the SELECT clause, so that the value resulting from applying each aggregate function to a group of tuples appears along with the value of the grouping attribute(s).

  \hii{HAVING}
    \par Retrieve the values of aggregate functions only for groups that satisfy certain conditions.

\setcounter{chapter}{7}
\clearpage
\begin{multicols}{3}

\hi{MSI Logic Circuits}

  \hii{Terminology}
    \begin{itemize}
      \item \tb{Decoder}:
        \begin{itemize}
          \item \tb{Input}: $n$-bit
          \item \tb{Output}: only $\pmb{1}$ corresponding output
        \end{itemize}
      \item \tb{7-seg Decoder}:
        \begin{itemize}
          \item \tb{Common-Anode LED Display}: outputs are active-LOW
          \item \tb{Common-Cathode LED Display}: outputs are active-HIGH
        \end{itemize}
      \item \tb{Encoder}
        \begin{itemize}
          \item \tb{Input}: only 1 is active at a time
          \item \tb{Output}: $n$-bit
        \end{itemize}
      \item \tb{Priority Encoder}: the output is corresponding to the
        \tb{largest} active input.
      \item \tb{Multiplexer}: (MUX, also Data Selector)
        \begin{itemize}
          \item \tb{Input}: $n$ SELECT inputs and $\leq 2^{N}$ DATA inputs
          \item \tb{Output}: only 1 output for all
        \end{itemize}
      \item \tb{Demultiplexer}: (DEMUX, also Data Distributor)
        \begin{itemize}
          \item \tb{Input}: $n$ SELECT inputs and 1 DATA input
          \item \tb{Output}: $\leq 2^{N}$ outputs, only 1 receives the DATA input
        \end{itemize}
      \item \tb{Magnitude Comparator}: circuit to compare two $n$-bit numbers
        If the two $n$-bit input numbers are equal, the comparison will also
        based on the \tb{cascading input}.
        \par \ti{Connecting MCs: LSB first}
      \item \tb{Code Converter}: circuit to convert one type to another type
        of binary code
      \item \tb{Data Busing}: multiple device have their outputs connected to
        a common set of bus lines
        \par \ti{Only outputs of one device can be transmitted on the bus lines
        at a time. To achieve that, we use \tb{tristate registers}}.
  \end{itemize}

\end{multicols}

\begin{appendices}
  \chapter*{UNIX Process Memory}

\par Memory of a process can be divided into segments:
  \begin{itemize}
    \item \tb{code segment}: contains the binary code of the program which is running as a process
    \item \tb{data segment}: contains the initialized global variables and data structures.
    \item \tb{BSS (block started by symbol}:  contains the uninitialized global data structures
    \item \tb{stack segment}: contains the local variables, return addresses, etc. for a particular process
  \end{itemize}

\par A process can execute in two modes:
  \begin{itemize}
    \item \tb{Kernel mode}: When a process makes a system call, the kernel takes control and does the requested service on behalf of the process. The process is said to be ``in kernel space” during this time.
    \item \tb{User mode}: Normal mode of a process. The process is said to be ``in userland" during this time.
  \end{itemize}

\par Userland’s view of the segments:
\begin{itemize}
  \item \tb{code segment}: consists of the code of the program, including all functions. If \lstinline{p = foo()} where \lstinline{foo()} is an arbitrary function, then \lstinline{p} must point to some location within the code segment.
  \item \tb{data segment}: consists of the initialized global variables of a program. If \lstinline{N} is a global variable, then the address of \lstinline{N} must be in the data segment.
  \item \tb{BSS}: consists of the uninitialized global variables of a program. If \lstinline{N} is a global variable, then the address of \lstinline{N} must be in the data segment.
  \item \tb{stack}: consists of local variables.
  \item \tb{heap}: consists of memory dynamically allocated during process run time.
\end{itemize}



\end{appendices}

\end{document}
