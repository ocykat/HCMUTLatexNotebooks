\chapter{Properties of Electric Charges}

\hi{Properties of Electric Charges}
    \begin{itemize}
        \item There are two kinds of electric charges:
            \begin{itemize}
                \item Positive
                \item Negative
            \end{itemize}
            The name \impt{positive} and \impt{negative} were given by Benjamin Franklin.
        \item Charges of the same sign repel one another and charges of the opposite signs attract
            one another.
        \item In an isolated system, electric charge is always conserved.
        \item Electric charge always occurs as some integral multiple of a fundamental amount $e$.
            In modern terems, the electric charge is is said to be \impt{quantized}.
    \end{itemize}

\hi{Charging an Object}
    \par{There are three ways to charge an object}
    \begin{itemize}
        \item by \impt{friction}
        \item by \impt{contact}
        \item by \impt{induction}
    \end{itemize}

\hi{Coulomb's Law}
    \par According to Coulomb's experiments, the \impt{electric force} between two stationary
    charged particles has the following properties:
    \begin{itemize}
        \item is inversely proportional to the square of the separation $r$ between the particles
            and directed along the line joining them.
        \item is proportional to the product of the charges $q_{1}$ and $q_{2}$ on the two
            particles.
        \item is attractive if the charges are of opposite sign and repulsive if the charges have
            the same sign.
        \item is a conservative force.
    \end{itemize}
    \par A \impt{point charge} is a particle of zero size that carries an electric charge.
    \par Coulomb's law:
    \begin{equation}
        F_{e} = k_{e} \frac{|q_{1}||q_{2}|}{r^{2}}
    \end{equation}
    where
    \begin{itemize}
        \item $k_{e} = 9 \cdot 10^{9} Nm^{2}/C^{2}$: the Coulomb constant
    \end{itemize}
    \par The Coulomb constant is also written in the form:
    \begin{equation}
        k_{e} = \frac{1}{4 \pi \epsilon_{0}}
    \end{equation}
    where $\epsilon_{0}$ is known as the \impt{permittivity of free space}.

\hi{The Electric Field}
    \hii{Definition}
        \par \impt{The electric field vector $E$} at a point in space is defined as the electric
        force $F_{e}$ acting on a positive test charge $q_{0}$ placed at that point divided by
        the test charge.
        \begin{equation}
            E = \frac{F_{e}}{q_{0}}
        \end{equation}
        \par An electric field exists at a point if a test charge at that point experiences an
        electric force.
    \hii{Direction of an electric field}
        \par The electric field at position $P$ is:
        \begin{itemize}
            \item \impt{directed away} from $q$ if $q > 0$.
            \item \impt{directed toward} $q$ if $q < 0$.
        \end{itemize}
    \hii{Electric field of a group of charges}
        \par At any point $P$, the total electric field due to a group of source charges equals the
        vector sum of the electric fields of all the charges.

\hi{Electric Field of a Continuous Charge Distribution}
    \hii{Method of Evaluation}
        \par To evaluate the electric field of a continuous charge distribution, we divide the
        distribution into many parts, each with charge $\Dt q$.
        \begin{align*}
            E = k \frac{\Dt q}{r^{2}} \hat{r}
        \end{align*}
        \par Suppose each part is infinitely small:
        \begin{align*}
            E = k \SUM{_{\Dt q_{i} \to 0} \frac{\Dt q}{r^{2}} \hat{r}} \\
            = k \INT{\frac{dq}{r^{2}} \hat{r}}
        \end{align*}
    \hiiBEGIN{Uniform Distribution of Charge on an object}
        \hiii{Linear Density of Charge}
            \begin{equation}
                \lambda = \frac{Q}{L} = \frac{dQ}{dL}
            \end{equation}
        \hiii{Surface Density of Charge}
            \begin{equation}
                \sigma = \frac{Q}{S} = \frac{dQ}{dS}
            \end{equation}
        \hiii{Volume Density of Charge}
            \begin{equation}
                \rho = \frac{Q}{V} = \frac{dQ}{dV}
            \end{equation}
    \hiiEND

\hi{Electric Field Lines}
    \par \impt{Electric Field Lines}, first introduced by Faraday, are related to the electric
    field in a region of space in the following manner:
    \begin{itemize}
        \item The electric field vector $E$ is tangent to the electric field line at each point.
            The line has a direction, indicated by an arrowhead, that is the same as that of the
            electric field vector.
        \item The number of lines per unit area through a surface perpendicular to the lines is
            proportional to the magnitude of the electric field in that region. Thus, the field
            lines are close together where the electric field is strong and far apart where the
            field is weak.
    \end{itemize}
    \par The rules for drawing electric field lines:
    \begin{itemize}
        \item The lines must begin on a positive charge and terminate on a negative charge.
            In the case of an excess of one type of charge, some lines will begin or end infinitely
            far away.
        \item The number of lines drawn leaving a positive charge or approaching a negative charge
            is proportional to the magnitude of the charge.
        \item No two field lines can cross.
    \end{itemize}

\hi{Motion of Charged Particles in a Uniform Electric Field}
