\chapter{What is Bitcoin?}

\par In the simplest terms, \ti{Bitcoin} is just another currency. The term Bitcoin refers to the entire currency system, whereas bitcoins are the basic units of the currency. (1)

\par A \ti{digital currency} is one that can be easily stored and used on a computer. By this definition, even dollars can be considered a digital currency, since they can be easily sent to others or used to shop online, but their supply is controlled by a centralized bank organization. In contrast, gold coins are \ti{decentralized}, meaning that no central authority controls the supply of gold in the world. In fact, anyone can dig for gold, create new coins, and distribute them. However,unlike digital currencies, it’s not easy to use gold coins to pay for goods (at least not with exact change!), and it’s impossible to transfer gold coins over the Internet. Because Bitcoin combines these two properties, it is somewhat like digital gold. Never before has there been a currency with both these two properties, and its impact on our increasingly digital, globalized world may turnout to be significant.

\par Sometimes called a stateless currency, Bitcoin is not associated with any nation. However, you should not consider Bitcoin to be in the same category as private currencies, hundreds of which have existed in various forms in the past. 2 Private currencies, whether issued by a person, a company, or a nonstate organization, arecentrally controlled and run the risk of collapse due to bankruptcy or other economic failure. Bitcoin is not a company, nor does a single person or organization issue or control bitcoins; therefore, it has no central point of failure. For this reason, nobody can inflate the currency supply and create hyperinflation crises, such as those that occurred in post–World War I Germanyand more recently in Zimbabwe. 3

\par Many people are asking about the motive behind the creation of Bitcoin, so let’s explore the currency’s purpose.

\img[width=4cm]{img/chap01-01.jpg}

\hi{Why Bitcoin Now?}

\par Until recently, people could not send digital cash back and forth to each other in a reliable way without a central mediator. A trusted central mediator such as PayPal cantrack payments and money transfers in a privately held account ledger, but it wasn’t clear how a group of strangers who \ti{do not} trust each other could accomplish the same transactions dependably. 4 Sometimes referred to as the Byzantine Generals’ Problem, this fundamental conundrum also emerges in computer science, specifically in how toachieve consensus on a distributed network.

\par In 2008, the problem was elegantly solved by Bitcoin’s inventor, known pseudonymously as Satoshi Nakamoto. Satoshi’s significant breakthrough made it possible for a digital currency to exist without relying on a central authority. Satoshi described the solution to the ByzantineGenerals’ Problem and the invention of Bitcoin in a white paper titled “Bitcoin: A Peer-to-Peer Electronic Cash System.” But the creation of the software that demonstrated the concept in practice was released a year later.

\par Although the first version of the software was written by Satoshi, it quickly became a community project as thesoftware was improved and maintained by hundreds of volunteers. Currently, the software is open source, and anyone can read and contribute to it. In January of 2009, the first bitcoins were distributed using the early Bitcoin software, and since then transactions have been running smoothly. Slowly but surely, an increasing number of people have started using Bitcoin, andwhat began as an experiment is now a multibillion dollar economy that processes hundreds of thousands of transactions per day (and is growing quickly).

\hi{Why Bitcoin Now?}

\par Bitcoin is an inherently international currency; anyone can send bitcoins toanyone else in the world, in any amount, almost instantly. In addition, it is becoming increasingly possible to travel the world and spend bitcoins without having to change them into the local currency. Because no middleman is involved, transaction fees are negligible. Unlike with credit cards, which require giving online merchants your personal information, youcan use bitcoins to shop online while maintaining your privacy. There is no risk of losing your savings due to runaway inflation because bitcoins were designed to have a fixed supply. Bitcoins are also fundamentally impossible to counterfeit.

\par As a merchant, you can start accepting bitcoins as payment immediately without filling out tediouspaperwork (compared to setting up the credit card transaction process). You can also own bitcoins without anyone else knowing, and no third party or government can seize your money. (The privacy this feature entails may protect the security and freedom of political dissidents living under repressive regimes, for example.)

\par Thanks to all of its benefits, Bitcoin continues to increase in popularity; however, anyone familiar with Bitcoin will agree the technology behind it is difficult to explain and understand. At first blush, it’s hard to grasp how bitcoins are stored, how they are used, or even where they come from.

\hi{The Complexity and Confusion of Bitcoin}

\par Rarely do we get to see the creation of a new currency, let alone one that is so different from previous currencies. This creates major challenges in comprehension and comfort for most people.

\par Bitcoin can be compared to the advent of paper currency years ago when everyone was using gold andsilver coins. Then, it must have seemed strange and confusing to attribute value to little pieces of paper instead of precious metals. Today, paper currency feels fairly safe, and trading paper for a purely digital asset like bitcoins seems odd. Furthermore, the economic and social consequences of switching to a decentralized digital currency are still unclear. Even Satoshi and theearly volunteers who helped develop the concept could not have imagined precisely how Bitcoin would be used and valued by society, much as the creators of the Internet in the 1980s could not have predicted how transformative it would become.

\par Confusion also stems from the fact that Bitcoin is a truly complex technology. It relies not only on Satoshi’sbreakthrough to achieving consensus on a distributed network but also on modern cryptographic techniques, such as digital signatures, public/private key pairs, and secure hashing. (These cryptographic concepts are covered in detail in Chapter 7.) The issuing of new currency occurs through a cryptographic lottery called mining that anyone can participate in. Miningsimultaneously processes transactions made by Bitcoin users. To resist abuse from those who might want to destroy the network, Bitcoin’s design uses game theory to align the incentives of those who maintain the network and those who want to act in their own selfish interest. (Bitcoin mining and game theory is explained in detail in Chapter 8.)

\par Put simply, you cannot learn and completely understand Bitcoin in a single afternoon. However, we hope this book will help you understand the basics of Bitcoin as quickly as possible.

\hi{What's in This Book?}
\par To make sense of the Bitcointechnology and phenomenon, you must view it from multiple perspectives. This book is organized around those perspectives.
  \begin{itemize}
    \item First, we’ll look at Bitcoin from the perspective of a basic user. In Chapters 2–4 we describe how Bitcoin works and how you can acquire, spend, and safely store bitcoins—so you can actually start using Bitcoin.
    \item Next, in Chapters 5 and 6, we take a philosophical perspective on Bitcoin. Chapter 5 is an adventure story told from the perspective of Crowley the cryptographer. Crowley gets stranded on an island and needs to figure out how to efficiently exchange goods with inhabitants of other distant islands. Crowley knows about Bitcoin from a chance encounter with Satoshi but has significant doubts about the currency. In the story, he works through his doubts (which may be similar to yours) by giving Bitcoin a chance. Chapter 6 continues in this philosophical vein by looking at the potentially broader impact of Bitcoin and the potentially uneasy relationship of Bitcoin andits users with nation states whose currencies compete with Bitcoin.
    \item Then, we’ll look at Bitcoin from the perspective of an advanced user. Chapters 7– 9 describe the cryptographic methods behind Bitcoin, the details of bitcoin mining, and the nuances of various third- party wallet software solutions.
    \item Finally, in Chapter 10, we’ll look at what the distant future might look like in a world where Bitcoin has gone mainstream.
    \item For programmers and developers who are new to Bitcoin, the appendices show you how to write your own programs to send and receive bitcoins.
  \end{itemize}

\par As you read this book,keep in mind just how new Bitcoin is as a technology. For fields like particle physics, Egyptian history, or constitutional law, we can turn to authority figures that have devoted the better part of their lives to studying those subjects; by comparison nobody is really an expert on Bitcoin. Just as there were no electricians before the discovery of electricity or programminggurus before computers were invented, arguably no Bitcoin experts exist today. We are all Bitcoin beginners, and no one can predict with any clarity how Bitcoin will evolve, even a year or two into the future.

\par On the upside, this means that if Bitcoin becomes widely used in the future, the potential exists for you to become one of the earlyexperts in Bitcoin, since you are studying this technology at such an early stage. We hope you will be inspired by the ideas behind Bitcoin and will make your own contributions to this wonderful technology in years to come. Now, let’s learn some Bitcoin basics.


