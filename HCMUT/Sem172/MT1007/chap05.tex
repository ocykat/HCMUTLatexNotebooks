\chapter{Eigenvalueus and Eigenvectors}

\hi{Eigenvalues and Eigenvectors}
  \hii{Definitions}
    \par Let $A$ be a \tb{square matrix} over $K$.
    \par A number $\lambda_{0} \in K$ is called an \tb{eigenvalue} of the
      matrix A, if there exists a nonzero vector $X_{0}$ such that
      $AX_{0} = \lambda_{0} X_{0}$.
    \par The vector $X_{0}$ such that $AX_{0} = \lambda_{0} X_{0}$ is called
      an \tb{eigenvector} of $A$ corresponding to $\lambda_{0}$.
  \hii{Characteristic Equation}
    \par \tb{Equation}
      \begin{eqbox}
        det(A - \lambda I) = 0
      \end{eqbox}
    \par Explanation:
      \begin{smfont}
        \par Assume that $\lambda_{0}$ is an eigenvalue of $A$.
          \begin{flalign*}
            & \ra \exists X_{0} \neq 0: AX_{0} = \lambda_{0} X_{0} && \\
            & \ra AX_{0} = \lambda_{0} X_{0} && \\
            & \ra AX_{0} - \lambda_{0} X_{0} = 0 && \\
            & \ra (A - \lambda_{0}I)X_{0} = 0
          \end{flalign*}
        \par Because $X_{0} \neq 0$:
          \begin{flalign*}
            & \ra det(A - \lambda_{0}I) = 0 \qquad (*)
          \end{flalign*}
      \end{smfont}
  \hii{Problem: Find all eigenvalues and eigenvectors of a square matrix}
    \begin{itemize}
      \item \tb{Step 1}: Solve the characteristic equation.
        \begin{align*}
          det(A - \lambda I) = 0
        \end{align*}
        All roots of the equation are eigenvalues.
      \item \tb{Step 2}: Find the corresonding eigenvectors for each eigenvalue.
        \par Let $\lambda_{k}$ be any eigenvalue.
        Solve the system:
        \begin{align*}
          (A - \lambda_{k}I)X = 0
        \end{align*}
        \par All nonzero solutions of the system are eigenvectors of the matrix
          $A$ corresponding to $\lambda_{k}$.
    \end{itemize}
  \hii{Formula: Characteristic equation of 3-by-3 matrix}
    \begin{eqbox}
      - \lambda^{3} + trace(A) \lambda^{2} - (A_{11} + A_{22} + A_{33}) \lambda
      + det(A) = 0
    \end{eqbox}

\hi{Diagonalization}
  \hii{Definitions}
    \hiii{Spectrum}
      \par A set of all eigenvalues of the square matrix $A$ is called the
        \tb{spectrum} of $A$ and is denoted by $\delta(A)$.
    \hiii{Algebraic multiplicity}
      \par Let $\lambda_{0}$ be an eigenvalue of $A$. The \tb{algebraic
        multiplicity} of $\lambda_{0}$ is its multiplicity (degree)
        as the root of the characteristic equation. 
    \hiii{Eigenspace}
      \par All solutions of the system $(A - \lambda_{k} I) X = 0$ form
        a subspace $E_{\lambda_{k}}$. This subspace is called an \tb{eigenspace}
        of $\lambda_{k}$.
    \hiii{Geometric multiplicity}
      \par The dimension of the eigenspace $E_{\lambda_{k}}$ is called
        the \tb{geometric multiplicity} of $\lambda_{k}$.
    \hiii{Diagonalizability}
      \par A square matrix $A$ is \tb{diagonalizable} if
      \begin{align*}
        A = P D P^{-1}
      \end{align*}
      where $D$ is diagonal and $P$ is invertible.
  \hii{Theorems}
    \hiii{Theorem 1}
      \begin{itemize}
        \item $\forall \lambda_{0} \in \delta(A):
          1 \leq geo(\lambda_{0}) \leq alg(\lambda_{0})$
        \item Let $\lambda_{1}, \lambda_{2}, \ldots, \lambda_{m}$ be $m$
          distinct eigenvalues and $X_{1}, X_{2}, \ldots, X_{m}$ be $m$
          eigenvectors of $A$ corresponding to $\lambda_{1},
          \lambda_{2}, \ldots, \lambda_{m}$. Then the set
          $\{X_{1}, X_{2}, \ldots, X_{m}\}$ 
      \end{itemize}
    \hiii{Theorem 2}
      \par Let $A$ be a square matrix of order $n$. A is diagonalizable if
        and only if there exists exactly $n$ linear independent eigenvectors.
      \par \tb{Corollary}: A square matrix $A$ is diagonalizable if and only
        if $\forall \lambda_{k} \in \delta(A)$:

  \hii{Problem: Diagonalizing matrix $A$}
    \begin{itemize}
      \item \tb{Step 1}: Fiding eigenvalues and their algebraic multiplicity
        by solving the characteristic equation.
      \item \tb{Step 2}: Finding basis for every eigenspace by solving the
        system $(A - \lambda I) X = 0$.
        \begin{align*}
          geo(\lambda_{k}) = dim(E_{\lambda_{k}})
        \end{align*}
      \item \tb{Step 3}: Conclusion
        \par If $\forall \lambda_{k} \in \delta(A):
          geo(\lambda_{k}) = alg(\lambda_{k})$ then $A$ is diagonalizable. 
    \end{itemize}

\hi{Orthogonal Diagonalization}
  \hii{Definitions}
