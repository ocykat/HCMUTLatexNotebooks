\chapter{Energy and Energy Transfer}
    \hi{System and Environment}
        \par In physics, a \impt{physical system} is a portion of the physical universe chosen for
        analysis. Everything outside the system is known as the environment. The environment
        is ignored except for its effects on the system.

    \hi{The Scalar Product of Two Vectors}
        \par The \impt{scalar product}, or \impt{dot product} of any two vectors $\vt{a}$ and
        $\vt{b}$ is a scalar quantity equal to the product of the magnitudes of the two vectors
        and the cosine of the angle $\theta$ between them.
        \begin{equation*}
            \vt{a} \mul \vt{b} = |\vt{a}||\vt{b}|\cos \theta
        \end{equation*}

    \hi{Work done by a Constant Force}
        \par The \impt{work} $W$ done on a system by an agent exerting a constant force on the system
        is the dot product of the magnitude $F$ of the force, the magnitude $\Delta r$ of the
        displacement of the point of application of the force, and $\cos \theta$, where $\theta$ is
        the angle between the force and displacement vectors:
        \begin{equation}
            W = \vt{F} \mul \vt{r} = F \Delta r \cos \theta
        \end{equation}
        in which:
        \begin{itemize}
            \item $W$: work done by the agent $[J]$ or $[N \mul m]$
            \item $F$: force exerted by the agent [N]
            \item $r$: magnitude of the displacement [m]
            \item $\theta$: angle between the force and displacement vector [rad]
        \end{itemize}

    \hi{Work Done by a Varying Force}
        \par The total work done by the net force $ \sum F_{x}$ as the particle moves from $r_{i}$
        to $r_{f}$:
        \begin{equation} \label{eq:work01}
            \sum W = \INT{_{x_{i}}^{x_{f}} \sum \vt{F}d\vt{r}}
        \end{equation}

    \hi{Kinetic Energy}
        \par \impt{Kinetic energy} $K$ of a particle of mass $m$ moving with a speed $v$ is defined
        as:
        \begin{equation} \label{eq:kinetic01}
            K = \dfrac{1}{2}mv^{2}
        \end{equation}

    \hi{Kinetic Energy and the Work-Kinetic Energy Theorem}
        \par Based on the equation \eqref{eq:work01}, we have:
        \begin{align*}
            \begin{split}
                \sum W & = \INT{_{x_{i}}^{x_{f}} \sum Fdx} \\
                & = \INT{_{x_{i}}^{x_{f}} \sum madx} \mbox{\qquad (the Second Newton's law)} \\
                & = \INT{_{x_{i}}^{x_{f}} \sum m \dfrac{dv}{dt}dx} \\
                & = \INT{_{v_{i}}^{v_{f}} \sum mvdv} \\
                & = \dfrac{1}{2}mv_{f}^{2} - \dfrac{1}{2}mv_{i}^{2} \\
            \end{split}
        \end{align*}
        \par In combination with the equation \eqref{eq:kinetic01}, we conclude that:
        \begin{equation}
            \sum W = K_{f} - K_{i} = \Delta K \\
        \end{equation}

    \hi{Isolated and Nonisolated System}
        \par If a system does not interact with its environment, it is an \impt{isolated system}.
        \par On the other hand, if the system is acted on by various forces, resulting in a change
        in its kinetic energy, it is a \impt{nonisolated system}. 
        \par The energy associated with an object's temperature is called \impt{internal energy}. 
        If positive work has been done on an object but there is no increase in the object's
        kinetic energy, then the energy has been transformed into internal energy. In this case, the
        work-kinetic energy theorem cannot be applied.

    \hi{Power}
        \hii{Average Power}
            \begin{equation}
                \avg{P} = \dfrac{W}{t}
            \end{equation}
        \hii{Instantaneous Power}
            \begin{equation}
                P = \dfrac{dW}{dt} = F \mul \dfrac{dr}{dt} = F \mul v
            \end{equation}