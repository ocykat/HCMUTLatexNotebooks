\chapter{Rotation of Rigid Bodies}


\hi{Angular Velocity and Acceleration}
  \par Define $z$ as the axis of rotation.
  \hii{Angular Velocity}
    \par Average angular velocity:
      \begin{eqbox}
        \omega_z = \frac{\Delta\theta}{\Delta t}
      \end{eqbox}
    \par Instantaneous angular velocity:
      \begin{eqbox}
        \omega_z = \dif{\Delta\theta}{\Delta t}
      \end{eqbox}
  \hii{Angular Acceleration}
    \par Average angular acceleration:
      \begin{eqbox}
        \alpha_z = \frac{\Delta\omega_z}{\Delta t}
      \end{eqbox}
    \par Instantaneous angular acceleration:
      \begin{eqbox}
        \alpha_z = \dif{\Delta\omega_z}{\Delta t} = \ddif{\Delta\theta}{\Delta t}
      \end{eqbox}


\hi{Rotation with Constant Angular Acceleration}


\hi{Energy in Rotational Motion - Moment of Inertia}
  \par Kinetic energy of a rotating particle:
    \begin{align*}
      K_i = \frac{1}{2} m_i v_i^2 = \frac{1}{2} m_i r_i^2 \omega_i^2
    \end{align*}
  \par For a rigid body consisting of particles:
    \begin{align*}
      K = \SUM_{i} \frac{1}{2}  m_i r_i^2 \omega_i^2
    \end{align*}
  \par For a rigid body, $\forall i, \omega_i = \omega$.
    \begin{align*}
      K = \frac{1}{2} \bigg(\SUM_{i} m_i r_i^2\bigg) \omega^2
    \end{align*}
  \par \tb{Moment of inertia} is defined as:
  \begin{eqbox}
    I = \SUM_{i} m_i r_i^2
  \end{eqbox}
  \par Then, rotational kinetic energy can be calculated by:
  \begin{eqbox}
    K = \frac{1}{2} I\omega^2
  \end{eqbox}


\hi{Parallel-Axis Theorem}
  \begin{eqbox}
    I_p = I_{CM} + md^2
  \end{eqbox}
  where
  \begin{itemize}
    \item $I_{CM}$: moment of inertia of the body about the axis through the
      center of mass
    \item $I_p$: moment of inertia of the body about an axis $p$ parallel to
      the axis through the center of mass
    \item $m$: mass of the body
    \item $d$: distance between two axes
  \end{itemize}


\hi{Moment of inertia calculations}
