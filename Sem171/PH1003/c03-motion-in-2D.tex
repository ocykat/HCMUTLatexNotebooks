\chapter{Motion in Two Dimensions}
    \hi{The Position, Velocity and Acceleration Vectors}
        \hiiBEGIN{Position}
            \hiii{Position Vector}
                \par Position vector is the vector drawn from the origin of some
                coordinate system to the particle located in the $xy$ plane.

            \hiii{Displacement Vector}
                \par Displacement vector is the vector indicating the change in position
                of a particle in some time interval.
                \begin{center}
                    \begin{equation}
                        \vt{\Delta r} = \vt{r_{f}} - \vt{r_{i}}
                    \end{equation}
                \end{center}
                \textit{in which:}
                \begin{itemize}
                    \item $\vt{\Delta r}$: displacement
                    \item $\vt{r_{i}}$: initial position
                    \item $\vt{r_{f}}$: final position
                \end{itemize}
        \hiiEND

        \hiiBEGIN{Velocity}
            \hiii{Average Velocity}
                The average velocity of a particle during the time interval $\Delta t$is 
                defined as the displacement of the particle divided by the time interval.
                \begin{center}
                    \begin{equation}
                        \avg{\vt{\Delta v}} = \frac{\vt{\Delta r}}{t}
                    \end{equation}
                \end{center}

            \hiii{Instantaneous Velocity}
                The instantaneous velocity of a particle is defined as the limit of the
                average velocity $\frac{\Delta r}{\Delta t}$ as $\Delta t$ approaches zero:
                \begin{center}
                    \begin{equation}
                        v = \lim_{x \to 0} \frac{\Delta r}{\Delta t} = \frac{dx}{dt}
                    \end{equation}
                \end{center}
        \hiiEND

        \hiiBEGIN{Acceleration}
            \hiii{Average Acceleration}
                The average acceleration of a particle is defined as the change in the
                instantaneous velocity vector $\vt{\Delta v}$ divided by the time interval
                $\Delta t$ during which that change occurs.
                \begin{center}
                    \begin{equation}
                        \avg{\vt{\Delta a}} = \frac{\vt{\Delta v}}{t}
                    \end{equation}
                \end{center}

            \hiii{Instantaneous Acceleration}
                The instantaneous acceleration of a particle is defined as the limit of the
                average acceleration $\frac{\Delta v}{\Delta t}$ as $\Delta t$ approaches
                zero:
                \begin{center}
                    \begin{equation}
                        a = \lim_{t \to 0} \frac{\Delta v}{\Delta t} = \frac{dv}{dt}
                    \end{equation}
                \end{center}
        \hiiEND
    
    \pagebreak

    \hi{Two-Dimensional Motion with Constant Acceleration}
        \hii{Position vector}
            \par When a particle moves with constant acceleration, the position vector for it
            in the $xy$ plane can be written as
            \begin{equation}
                r = x \hat{i} + y \hat{j}
            \end{equation}
            where $x$, $y$ and $r$ change with time as the particle moves while the unit
            vector $\hat{i}$ and $\hat{j}$ remain constant.

        \hii{Velocity vector}
            \begin{equation}
                v = \frac{dr}{dt} = \frac{dx}{dt} \hat{i} + \frac{dy}{dt} \hat{j}
                = v_{x} \hat{i} + v_{y} \hat{j}
            \end{equation}
            \par Because $a$ is constant:
            \begin{equation}
                \begin{aligned}
                    v_{f} &= (v_{xi} + a_{x}t) \hat{i} + (v_{yi} + a_{y}t) \hat{j} \\
                    &= (v_{xi} \hat{i} + v_{yi} \hat{j}) + (a_{x} \hat{i} + a_{y} \hat{j})t \\
                    &= v_{i} + at
                \end{aligned}
            \end{equation}

        \hii{Equations of kinematics}
            \begin{equation}
                r_{f} = r_{i} + v_{i}t + \frac{1}{2} at^{2}
            \end{equation}

    \pagebreak

    \hi{Projectile Motion}
        \hii{Definition}
            \par A projectile motion is a motion in which two assumptions are made:
            \begin{itemize}
                \item The free-fall acceleration $g$ is constant over the range of motion
                and is directed downward.
                \item The effect of air resistance is negligible.
            \end{itemize}

        \hii{Velocity}
            \par At the beginning, the velocity components are:
            \begin{equation}
                \begin{aligned}
                v_{xi} = v_{i} \cos \theta_{i} \qquad
                v_{yi} = v_{i} \sin \theta_{i}
                \end{aligned}
            \end{equation}

            \par After some time interval $t$, these components become:
            \begin{equation}
                \begin{aligned}
                v_{xi} = v_{i} \cos \theta_{i} \qquad
                v_{yi} = v_{i} \sin \theta_{i} - gt
                \end{aligned}
            \end{equation}

        \hii{Position}
            \begin{equation} 
                \label{eq:01}
                x_{f} = x_{i} + v_{ix}t
            \end{equation}

            \begin{equation}
                \label{eq:02}
                y_{f} = y_{i} + v_{iy}t - \frac{1}{2} gt^{2}
            \end{equation}

            \par If the particle starts from the origin of the referene frame, which
            means $x_{i} = 0$ and $y_{i} = 0$, then:
            \begin{equation} 
                \label{eq:03}
                x = v_{ix}t = v_{i}\cos\theta_{i}t
            \end{equation}

            \begin{equation}
                \label{eq:04}
                y = v_{iy}t - \frac{1}{2}gt^{2} = v_{i}\sin\theta_{i}t - \frac{1}{2}gt^{2}
            \end{equation}


            \par From the equation \eqref{eq:03}:
            \begin{equation}
                t = \frac{x}{v_{i}\cos\theta_{i}}
            \end{equation}

            \par Substitute $t$ in the equation \eqref{eq:04}:
            \begin{equation}
                \begin{aligned}
                \label{eq:05}
                    y &= v_{i}\sin\theta_{i} \mul \frac{x}{v_{i}\cos\theta_{i}}
                    - \frac{1}{2}g (\,\frac{x}{v_{i}\cos\theta_{i}})\,^{2} \\
                    &= x\tan{\theta_{i}} - \frac{gx^{2}}{2v^{2}_{i}\cos^{2}\theta_{i}}
                \end{aligned}
            \end{equation}

            Because the equation \eqref{eq:05} is in the form of a quadratic equation,
            the path is parabolic.

    \pagebreak

    \hi{Uniform Circular Motion}
        \hii{Definition}
            Uniform circular motion can be described as the motion of an object in a circle
            at a constant speed. As an object moves in a circle, it is constantly changing
            its direction. At all instances, the object is moving tangent to the circle.

        \hii{Centripetal Acceleration}
            \begin{equation}
                a_{c} = \frac{v^{2}}{r}
            \end{equation}

        \hii{Period}
            \begin{equation}
                T = \frac{2\pi r}{v}
            \end{equation}

    \pagebreak
    
    \hi{Tangential and Radial Acceleration}
        \hii{Definition}
            \par When a particle moves along a curved path, the total acceleration $\vt{a}$ 
            changes from point to point. This vector can be resolved into two components:
            \begin{itemize}
                \item A radial component $\vt{a_{r}}$
                \item A tangential component $\vt{a_{r}}$
            \end{itemize}

        \hii{Model circle}
            \par At each point on the path, there exists a model circle on which the radial
            component and a tangential component are defined.

        \hii{Radial Acceleration}
            \par The Radial Acceleration lies along the radius of the model circle, arises from
            the change in direction of the velocity vector.
            \begin{equation}
                a_{r} = -a{c} = -\frac{v^{2}}{r}
            \end{equation}
            \par The negative sign indicates that the direction of the centripetal acceleration
            is toward the center of the circle representing the radius of curvature, which
            is opposite the direction of the radial unit vector $\hat{r}$.

        \hii{Tangential Acceleration}
        \par The Tangential Acceleration is perpendicular to the radius, parallel to the
        instantaneous velocity, and causes the change in the speed of the particle.
        \begin{equation}
            a_{t} = \frac{d|v|}{dt}
        \end{equation}

    \pagebreak 

    \hi{Relative Velocity and Relative Acceleration}
        \par Consider a particle which is being observed by two observers:
        \begin{itemize} 
            \item One in reference frame S, fixed relative to Earth
            \item One in reference frame S', moving relative to S with a constant velocity
                $\vt{v_{0}}$
        \end{itemize} 

        \par Suppose at t = 0, the origins of the two reference frames coincide in space. Thus,
        at time t, the origins of the referene frames will be separated by a distance
        $v_{0}t$.
        \par If we denote:
        \begin{itemize}
            \item $\vt{r}$ as the position vector of the particle related to the origin of S.
            \item $\vt{r'}$ as the position vector of the particle related to the origin of S'.
        \end{itemize} 
        \par then we have:
        \begin{equation}
        \label{eq:06}
            r' = r - v_{0}t
        \end{equation}

        \par By differentiating equation \eqref{eq:06} and note that $v_{0}$ is constant, we obtain:
        \begin{equation}
        \label{eq:07}
            \begin{aligned}
                & \frac{dr'}{dt} = \frac{dr}{dt} - v_{0} \\
                & \ra v' = v - v_{0}
            \end{aligned}
        \end{equation}

        \par The equations \eqref{eq:06} and \eqref{eq:07} are known as \textbf{Galilean
        transformation equations}.

        \par Taking the time derivative of equation \eqref{eq:07}, we obtain:
        \begin{equation}
            \frac{dv'}{dt} = \frac{dv}{dt} - \frac{dv_{0}}{dt}
        \end{equation}
        \par Because $v_{0}$ is constant, $\dfrac{dv_{0}}{dt} = 0$. Therefore, the acceleration of
        the particle measured by an observer in one frame of the reference is the same as
        that measured by any other observer \textit{moving with constant velocity related to
        the first frame}.
        \begin{equation}
            a' = a.
        \end{equation}

    \pagebreak
