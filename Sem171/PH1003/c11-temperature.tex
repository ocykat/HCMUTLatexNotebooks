\chapter{Temperature}

\hi{Temperature and the Zeroth Law of Thermodynamics}
    \hii{Thermal Equilibrium}
        \par Thermal equilibrium is a situation in which two objects would not exchange
        energy by heat or electromagnetic radiation if they were placed in thermal
        contact.
    \hii{Zeroth Law of Thermodynamics}
        \par If objects A and B are separately in thermal equilibrium with a third object
        C, then A and B are in thermal equilibrium with each other.
    \hii{Temperature}
        \par Two objects in thermal equilibrium with each other are at the same
        temperature.

\hi{Absolute Temperature Scale}
    \hii{Experiment with Gas Thermometers}
        \par Suppose that temperartures are measured with gas thermometers containing
        dfferent gases at different initial pressures.
        \par If we visualize the result on a pressure - temperature $P-T$ graph, the
        results obtained by one particular thermometer would form a straight line.
        In order words, $P$ and $T$ have a linear relationship.
        \par Experiments show that in every case, the pressure is zero when the
        temperature is about $-273 \degree C$.

\hi{Thermal Expansion of Solid and Liquids}
    \hii{Thermal Expansion}
        \par As temperature increases, its volume increases. This phenomenon is known as
        \impt{thermal expansion}.
        \par Thermal expansion is a consequence of the change in the \textit{average
        separation} between the atoms in an object.
    \hii{Linear Expansion}
        \begin{equation}
            \alpha = \frac{\Dt L / L_{i}}{\Dt T}
        \end{equation}
        or
        \begin{eqbox}
            \Dt L = \alpha L_{i}\Dt T
        \end{eqbox}
        or
        \begin{equation}
            L_{f} - L_{i} = \alpha L_{i} (T_{f} - T_{i})
        \end{equation}
        where:
        \begin{itemize}
            \item $\alpha$: average coefficient of linear expansion
        \end{itemize}
    \hii{Volume Expansion}
        \begin{equation}
            \beta = \frac{\Dt V / V_{i}}{\Dt T}
        \end{equation}
        or
        \begin{eqbox}
            \Dt V = \beta V_{i}\Dt T
        \end{eqbox}
        or
        \begin{equation}
            V_{f} - V_{i} = \beta V_{i} (T_{f} - T_{i})
        \end{equation}
        where:
        \begin{itemize}
            \item $\beta$: average coefficient of volume expansion
        \end{itemize}

\hi{Macroscopic Description of an Ideal Gas}
    \hii{Equation of State for an Ideal Gas}
        \begin{eqbox}
            PV = nRT
        \end{eqbox}
        where
        \begin{itemize}
            \item $P$: pressure $[Pa]$ or $[N/m^{2}]$
            \item $V$: volume $[m^{3}]$
            \item $n$: moles of gas $[mol]$
            \item $R$: ideal gas constant/universal gas constant
                $(R = 8.314 J/mol \cdot K)$
            \item $T$: temperature $[K]$
        \end{itemize}
        \par The ideal gas law can also be expressed in terms of the total number of
        molecules $N$:
        \begin{eqbox}
            PV = Nk_{B}T
        \end{eqbox}
        where $k_{B}$ is the Boltzmann's constant.