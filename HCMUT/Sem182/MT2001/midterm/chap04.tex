\begin{multicols}{2}
\hi{Confidence Intervals}

  \hii{Definitions and Terminology}
    \hiii{Terminology}
      \begin{itemize}
        \item A \tb{population} is a collection of objects about which information is sought.
        \item A \tb{sample} is a part of the population that is observed.
        \item A \tb{parameter} is a numerical characteristic of a population.
        \item A \tb{statistics} is a numerical function of the sampled data, used to estimate an unknown parameter
      \end{itemize}

    \hiii{Mean}
      \begin{itemize}
      \item \tb{Population Mean}: denoted $\mu$, is the average of all values in the \tb{entire population}.
      \item \tb{Sample Mean}: denoted $\hat{\mu} = \bar{X}$, is the average value of a \tb{sample}. For a sample with $n$ values:
        \[
          \hat{\mu} = \bar{X} = \frac{x_1 + \ldots + x_n}{n}
        \]
      \end{itemize}


    \hiii{Variance}
      \par The \tb{sample variance}, denoted $s^2$, is the approximation of the \tb{population variance} $\sigma^2$.
      \[
        s^2 = \frac{\sum (x_i - \bar{X})^2}{n - 1} = \frac{\sum (x_i - \hat{\mu})^2}{n - 1}
      \]
      where $n - 1$ is called the \tb{degree of freedom}.

    \hiii{Median and IQR}
      \par List all data values \ti{in sorted order from smallest to largest}. For a sample with $n$ values:
      \begin{itemize}
        \item If $n \nmid 2$:
          \begin{itemize}
            \item The \tb{median} is one single value at the middle position.
            \item \tu{To calculate the IQR}: Divide the list into two equal halves \tb{not including the median}.
          \end{itemize}
        \item If $n \mid 2$:
          \begin{itemize}
            \item The \tb{median} is the average of two values at the middle positions.
            \item \tu{To calculate the IQR}: Divide the list into two equal halves \tb{including two middle values}.
          \end{itemize}
      \end{itemize}
      \par The median of the first half is called the \tb{first quatile} or the \tb{lower quatile}. 
      \par The median of the second half is called the \tb{third quatile} or the \tb{upper quatile}.
      \par The \tb{interquartile range (IQR)} is defined as:
        \[
          \text{IQR} = \text{upper quatile} - \text{lower quatile}
        \]
      \par The \tb{mode} is the value (or multiple values) that appears most often in the sample.

    \hiii{Confidence Level and Significance Level}
      \begin{itemize}
        \item \tb{Confidence level} $\gamma$ is the measure of the degree of reliability of the interval.
        \item \tb{Significance level} $\alpha$ is the probability we allow ourselves to be wrong when we are estimating a parameter with a confidence interval.
      \end{itemize}
      \[
        \gamma + \alpha = 1
      \]

\end{multicols}
\clearpage
\begin{multicols}{3}

  \hii{Case 1: Normal Population with Known Standard Deviation}
    \hiii{Confidence Interval of the Population Mean}
    \par If $x_1, \ldots, x_n$ are independent and identically distributed (i.i.d.), following a normal distribution $N(\mu, \sigma^2)$ and $\alpha = 1 - \gamma$, then the \tb{confidence interval of the population mean} is:
    \[
      \mu = \hat{\mu} \pm z_{\alpha / 2} \times \frac{\sigma}{\sqrt{n}}
    \]

    \hiii{Sample Size According to Maximum Error}
      \par If $\bar{X}$ is used to estimate $\mu$, then we can be $100(1 - a)\%$ confident that the error will not exceed a specified amount $\epsilon$ when the sample size is:
      \[
        n = \bigg\lceil \frac{\sigma z_{\alpha/2}}{\epsilon} \bigg\rceil^2
      \]

    \hiii{One-Sided Convidence Interval of the Population Mean}
      \begin{itemize}
        \item A $100(1 - \alpha)\%$ upper-confidence bound for $\mu$ is
        \[
          \mu \leq \hat{\mu} + z_{\alpha} \times \frac{\sigma}{\sqrt{n}}
        \]
        \item A $100(1 - \alpha)\%$ lower-confidence bound for $\mu$ is
        \[
          \mu \geq \hat{\mu} - z_{\alpha} \times \frac{\sigma}{\sqrt{n}}
        \]
      \end{itemize}

  \hii{Case 2: Normal Population with Unknown Standard Deviation}
    \hiii{Confidence Interval of the Population Mean}
     \par If $x_1, \ldots, x_n$ are independent and identically distributed (i.i.d.) then the \tb{confidence interval of the population mean} is:
     \[
      \mu = \hat{\mu} \pm t_{n - 1, \alpha / 2} \times \frac{s}{\sqrt{n}}
     \]
    \hiii{Confidence Interval of the Population Variance}
      \begin{itemize}
        \item \tb{Two-sided Confidence Interval}
        \par Choose $c_1$ and $c_2$ so that the area in each tail of $\chi_{n - 1}^2$ distribution is $\alpha / 2$. Then the $\gamma$-confidence interval
        for the unknown variance $\sigma^2$ is:
        \[
          \frac{(n - 1)s^2}{c_2} \leq \sigma^2 \leq \frac{(n - 1)s^2}{c_1}
        \]
        \item \tb{One-sided Confidence Interval}
        \par Choose $c_1$ and $c_2$ so that the area in each tail of $\chi_{n - 1}^2$ distribution is $\alpha$. Then the $\gamma$-confidence interval
        for the unknown variance $\sigma^2$ are:
        \[
          \sigma^2 \geq \frac{(n - 1)s^2}{c_2}
        \]
        and
        \[
          \sigma^2 \leq \frac{(n - 1)s^2}{c_1}
        \]
      \end{itemize}

    \hii{Case 3: Large Sample Size}
      \hiii{Confidence Interval of the Population Mean}
      \par If $x_1, \ldots, x_n$ are independent and identically distributed (i.i.d.) and $n$ is large, then:
        \[
          \mu = \hat{\mu} \pm z_{\alpha / 2} \times \frac{s}{\sqrt{n}}
        \]

    \hiii{Population Proportion}
      \par Let $X \sim B(n, p)$ and assume $np \geq 10$, $nq \geq 10$. Then:
      \begin{itemize}
        \item An approximate $100\gamma\%$ \tb{confidence interval} for $p$ is:
          \[
            p = \hat{p} \pm z_{\alpha / 2} \times \sqrt{\frac{\hat{p} \hat{q}}{n}}
          \]
        \item The approximate $100\gamma\%$ \tb{lower confidence bound} is:
          \[
            p \geq \hat{p} - z_{\alpha} \times \sqrt{\frac{\hat{p} \hat{q}}{n}}
          \]
        \item The approximate $100\gamma\%$ \tb{upper confidence bound} is:
          \[
            p \leq \hat{p} - z_{\alpha} \times \sqrt{\frac{\hat{p} \hat{q}}{n}}
          \]
      \end{itemize}

\end{multicols}

\clearpage
