\chapter{Linear Mapping - Linear Transformation}

\hi{Linear Mapping}
  \hii{Definition}
    \par Let $V$ and $W$ be two vector spaces over $K$. A mapping
      $f: V \to W$ is called a \tb{linear mapping} if the following
      conditions hold:
    \begin{enumerate}
      \item $\forall x, y \in V, f(x + y) = f(x) + f(y)$
      \item $\forall x \in V, \forall \alpha \in K: f(\alpha x) = \alpha f(x)$
        \par \smf{\ti{Corollary}: $\forall x, y \in V,
          \forall \alpha, \beta \in K,
          f(\alpha x + \beta y) = \alpha f(x) + \beta f(y)$}
    \end{enumerate}
    \par A linear mapping usually has a matrix representation:
      \begin{eqbox}
        f(x) = Mx
      \end{eqbox}
  \hii{Linear Mapping and Spanning Set}
    \hiii{Explanation}
      \par Let $f: V \to W$ be a linear mapping;
        $x \in V$ and
        $E = \{e_{1}, e_{2}, \ldots, e_{n}\}$ be a spanning set of $V$.
      \begin{flalign*}
        & x = \alpha_{1} e_{1}
             + \alpha_{2} e_{2} + \ldots
             + \alpha_{n} e_{n} && \\
        & \ra f(x) = f(
               \alpha_{1} e_{1}
             + \alpha_{2} e_{2} + \ldots
             + \alpha_{n} e_{n}
        ) && \\
        & \ra f(x) =
               f(\alpha_{1} e_{1})
             + f(\alpha_{2} e_{2}) + \ldots
             + f(\alpha_{n} e_{n})
        && \\
        & \ra f(x) =
               \alpha_{1}f(e_{1})
             + \alpha_{2}f(e_{2}) + \ldots
             + \alpha_{n}f(e_{n})
        && \\
      \end{flalign*}

    \hiii{Problem}
      \par Given $x \in V$, a spanning set $E$ of $V$ and
        $f(e_{i}), i: 1 \to n$. Calculate $f(x)$.

      \begin{itemize}
        \item \tb{Method 1}: Using vector representation and linear system:
          \begin{itemize}
            \item \tb{Step 1}: Because $E$ is a spanning set of $V$:
              \begin{flalign*}
                & x = \SUM{_{i = 1}^{n} \alpha_{i} e_{i}} && \\
                & f(x) = \SUM{_{i = 1}^{n} \alpha_{i} f(e_{i})} && \\
              \end{flalign*}
            \item \tb{Step 2}: Solve the linear system for $a_{i}, i; 1 \to n$:
              \begin{flalign*}
                & x = \SUM{_{i = 1}^{n} \alpha_{i} e_{i}} && \\
                & \ra
                  \begin{cases}
                    \alpha_{1}e_{11} + \alpha_{2}e_{21}
                       + \ldots + \alpha_{n}e_{n1} = x_{1} \\
                    \alpha_{1}e_{12} + \alpha_{2}e_{22}
                       + \ldots + \alpha_{n}e_{n2} = x_{2} \\
                    \ldots \\
                    \alpha_{1}e_{1n} + \alpha_{2}e_{2n}
                       + \ldots + \alpha_{n}e_{nn} = x_{n} \\
                  \end{cases}
                &&\\
              \end{flalign*}
            \item \tb{Step 3}: Calculate f(x):
              \begin{flalign*}
                f(x) = \SUM{_{i = 1}^{n} \alpha_{i} f(e_{i})}
              \end{flalign*}
          \end{itemize}

        \item \tb{Method 2}: Using matrix representation:
          \begin{itemize}
            \item \tb{Step 1}: $E$ is a spanning set of $V$:
              \begin{flalign*}
                & x = \SUM{_{i = 1}^{n} \alpha_{i} e_{i}} && \\
                & \ra x
                = \begin{pmatrix}
                    e_{1} & e_{2} & \ldots & e_{n}
                  \end{pmatrix}
                  \begin{pmatrix}
                    \alpha_{1} \\
                    \alpha_{2} \\
                    \ldots     \\
                    \alpha_{n}
                  \end{pmatrix}
                = E
                  \begin{pmatrix}
                    \alpha_{1} \\
                    \alpha_{2} \\
                    \ldots     \\
                    \alpha_{n}
                  \end{pmatrix}
                && \\
                & f(x) = \SUM{_{i = 1}^{n} \alpha_{i} f(e_{i})} && \\
                & \ra f(x)
                = \begin{pmatrix}
                    f(e_{1}) & f(e_{2}) & \ldots & f(e_{n})
                  \end{pmatrix}
                  \begin{pmatrix}
                    \alpha_{1} \\
                    \alpha_{2} \\
                    \ldots     \\
                    \alpha_{n}
                  \end{pmatrix}
                = F(E)
                  \begin{pmatrix}
                    \alpha_{1} \\
                    \alpha_{2} \\
                    \ldots     \\
                    \alpha_{n}
                  \end{pmatrix}
                && \\
              \end{flalign*}
            \item \tb{Step 2}: Solve the linear system for $a_{i}, i; 1 \to n$:
              \begin{flalign*}
                & x = E
                  \begin{pmatrix}
                    \alpha_{1} \\
                    \alpha_{2} \\
                    \ldots     \\
                    \alpha_{n}
                  \end{pmatrix}
                \ra
                  \begin{pmatrix}
                    \alpha_{1} \\
                    \alpha_{2} \\
                    \ldots     \\
                    \alpha_{n}
                  \end{pmatrix}
                  = E^{-1} x    \qquad (1)
                &&
              \end{flalign*}
            \item \tb{Step 3}: Calculate $f(x)$:
              \begin{flalign*}
                & f(x) = F(E)
                  \begin{pmatrix}
                    \alpha_{1} \\
                    \alpha_{2} \\
                    \ldots     \\
                    \alpha_{n}
                  \end{pmatrix}
                \ra
                  \begin{pmatrix}
                    \alpha_{1} \\
                    \alpha_{2} \\
                    \ldots     \\
                    \alpha_{n}
                  \end{pmatrix}
                  = F(E)^{-1} f(x)     \qquad (2)
                &&
              \end{flalign*}
          \end{itemize}
          \par From $(1)$ and $(2)$:
              \begin{flalign*}
                & E^{-1} x = F(E)^{-1} f(x) && \\
                & \ra f(x) = F(E) E^{-1} x && \\
              \end{flalign*}
      \end{itemize}

\hi{Matrix Representation of Linear Mapping}
  \par Given:
    \begin{itemize}
      \item the linear mapping $f: V \to W$
      \item basis of $V$: $E = \{e_{1}, e_{2}, \ldots, e_{n}\}$
      \item basis of $W$: $F = \{w_{1}, w_{2}, \ldots, w_{n}\}$
    \end{itemize}

  \par A matrix:
    \begin{eqbox}\label{motm-01}
      A =
      \begin{pmatrix}
        [f(e_{1})]_{F} & [f(e_{2})]_{F} & \ldots & [f(e_{n})]_{F}
      \end{pmatrix}
    \end{eqbox}
  is called the \tb{matrix of the mapping} in two bases $E$ and $F$.

  \par $\forall x \in V$, we have:
    \begin{eqbox} \label{motm-02}
      [f(x)]_{F} = A[x]_{E}
    \end{eqbox}

  \par From equation \eqref{motm-01}:
  \begin{flalign*}
      & A =
        \begin{pmatrix}
          [f(e_{1})]_{F} & [f(e_{2})]_{F} & \ldots & [f(e_{n})]_{F}
        \end{pmatrix}
      && \\
      & = F^{-1}
        \begin{pmatrix}
          f(e_{1}) & f(e_{2}) & \ldots & f(e_{n})
        \end{pmatrix}
  \end{flalign*}
  \par Assume that $f(x) = Mx$.
    \begin{flalign*}
      & A = F^{-1}
        \begin{pmatrix}
          Me_{1} & Me_{2} & \ldots & Me_{n}
        \end{pmatrix}
      && \\
      & = F^{-1} M
        \begin{pmatrix}
          e_{1} & e_{2} & \ldots & e_{n}
        \end{pmatrix}
      && \\
      & = F^{-1}ME
    \end{flalign*}

  \par From equation \eqref{motm-02}:
  \begin{flalign*}
    & [f(x)]_{F} = A[x]_{E} && \\
    & \ra F^{-1} f(x) = A E^{-1} x && \\
    & \ra f(x) = F A E^{-1} x && \\
  \end{flalign*}

  \begin{eqbox}
    A = F^{-1} ME \\
    f(x) = F M E^{-1} x
  \end{eqbox}

  \par If the mapping is $f: V \to V$, we have the similar results:
  \begin{align*}
    A = E^{-1} ME \\
    f(x) = E M E^{-1} x
  \end{align*}


\hi{Kernel and Image of a Linear Mapping}
  \hii{Definitions}
    \par Let $f: V \to W$ be a linear mapping.
    \begin{itemize}
      \item The subset $ker(f) = \{\forall x \in V | f(x) = 0 \}$ is called
        the \tb{kernel} of the linear mapping.
      \item The subset $im(f) = \{\forall y \in W | \exists x \in V: y = f(x) \}$
        is called the \tb{image} of the linear mapping.
    \end{itemize}
  \hii{Theorems}
    \begin{itemize}
      \item $ker(f)$ is a subspace of $V$.
      \item $im(f)$ is a subspace of $W$.
      \item $dim(ker(f)) + dim(im(f)) = dim(V)$
    \end{itemize}
  \hii{Problems}
    \begin{enumerate}
      \item Find one basis and a dimension of $ker(f)$.
        \begin{itemize}
          \item \tb{Step 1}:
            \begin{flalign*}
              & \forall x \in V, find f(x). &&
            \end{flalign*}
          \item \tb{Step 2}:
            \begin{flalign*}
              & \forall x \in ker(f): f(x) = 0 && \\
              & \ra Ax = 0 \qquad \mbox{(homogeneous system)} &&
            \end{flalign*}
            \par Solve the system, find a general solution.
          \item \tb{Step 3}: From the general solution, deduce the
            basis and the dimension of $ker(f)$.
        \end{itemize}
      \item Find one basis and a dimension of $im(f)$.
        \begin{itemize}
          \item \tb{Step 1}: Select a spanning set $E$ of $V$:
            \begin{flalign*}
              & E = \{e_{1}, e_{2}, \ldots, e_{n}\} &&
            \end{flalign*}
          \item \tb{Step 2}: Compute
            \begin{flalign*}
              & f(e_{1}), f(e_{2}), \ldots, f(e_{n}) &&
            \end{flalign*}
          \item \tb{Step 3}:
            \begin{flalign*}
              & \forall y \in im(f): \exists x \in V: y = f(x) && \\
              & \ra y = f(\alpha_{1}e_{1}
                        + \alpha_{2}e_{2} + \ldots +
                          \alpha_{n}e_{n}) && \\
              & = \alpha_{1}f(e_{1}) + \alpha_{2}f(e_{2})
                        + \ldots + \alpha_{n}f(e_{n}) && \\
              & \ra \{f(e_{1}), f(e_{2}), \ldots, f(e_{n}) = B
                \mbox{ is a spanning set for } im(f)
              && \\
              & \ra dim(im(f)) = rank(B) &&
            \end{flalign*}
            \par The basis of $im(f)$ is all nonzero rows of the Echelon form
              of $B$.
        \end{itemize}
    \end{enumerate}
