\chapter{Dynamics of Rotational Motion}
\hi{Torque}
  \begin{itemize}
    \item Vector form:
    \begin{eqbox}
      \tau_z = \vec{F} \times \vec{r}
    \end{eqbox}

    \item Magnitude:
    \begin{eqbox}
      \tau_z = Fr\sin(\theta) = F_{tan}r
    \end{eqbox}
    where
    \begin{itemize}
      \item $tau$: torque
      \item $\vec{F}$: force
      \item $\vec{r}$: lever arm
    \end{itemize}

  \end{itemize}


\hi{Torque and Angular Acceleration for a Rigid Body}
  \par According to Newton's Second Law, for a rotating particle under a force $F$:
    \begin{align*}
      F_{i.tan} = m_i a_{i.tan}
    \end{align*}
    where
    \begin{itemize}
      \item $F_{i.tan}$: tangential component of the force $F$
      \item $m$: mass of the particle
      \item $a_{i.tan}$: tangential component of acceleration
    \end{itemize}
  \par Relationship between tangential acceleration and rotational acceleration:
  \begin{align*}
    a_{i.tan} = r_i \alpha_z
  \end{align*}
  \par Therefore:
  \begin{flalign*}
    & \qquad F_{i.tan} = m_i r_i \alpha_z && \\
    & \ra F_{i.tan} r_i = m_i r_i^2 \alpha_z && \\
    & \ra \SUM (F_{i.tan} r_i) = \SUM (m_i r_i^2) \alpha_z && \\
    & \ra \SUM (\tau_{z.i.tan}) = \SUM (I_i) \alpha_z && \\
    & \ra \tau_z =  I \alpha_z &&
  \end{flalign*}
  \begin{eqbox}
    \tau_z = I \alpha_z
  \end{eqbox}
  where
  \begin{itemize}
    \item $\tau_z$: torque
    \item $I$: moment of inertia
    \item $\alpha_z$: angular acceleration
  \end{itemize}


\hi{Angular Momentum}
  \hii{Definition}
    \BOOKSECTION{10.5}
    \begin{align*}
      \vec{L} = \vec{p} \times \vec{r} = m\vec{v} \times \vec{r}
    \end{align*}
    where:
    \begin{itemize}
      \item $\vec{L}$: angular momentum
      \item $\vec{p}$: lever arm/radius of rotation
      \item $\vec{r}$: linear momentum
    \end{itemize}

  \hii{Angular Momentum of a Rigid Body}
    \par For a particle:
    \begin{flalign*}
      & L_i = m_i v_i r_i = m_i (\omega_i r_i) r_i = m_i r_i^2 \omega_i &&
    \end{flalign*}
    \par Then for a rigid body:
    \begin{flalign*}
      & \ra \SUM(L_i) = \SUM(m_i r_i^2 \omega_i) = \SUM(m_i r_i^2) \omega && \\
      & \ra \SUM(L_i) = \SUM(I_i) \omega && \\
      & \ra L = I \omega &&
    \end{flalign*}
    \begin{eqbox}
      L = I \omega
    \end{eqbox}
