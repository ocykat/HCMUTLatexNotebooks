\chapter{Propositional Logic}

\hi{Proposition/Declarative Sentences}
  \hii{Definition}
  \par A \tb{proposition} or a \tb{declarative sentence} is a sentence that is either \tb{True} or \tb{False} but not both.

  \hii{Atomic Declarative Sentence and Notation}
  \par Usually, \tb{atomic} or \tb{indecomposable} sentences are considered.
  \par Each \tb{atomic declarative sentence} is assign a symbol $p$, $q$, $r$, etc.

\hi{Logical Connectives}
  \par \tb{Logical Connective }
  \par Let $p$, and $q$ be propositions (or declarative sentences).

  \hii{Negation}
    \hiii{Definition}
      \begin{itemize}
        \item The \tb{negation} of $p$, denoted by $\lnot p$, is the statement ``It is not the case that $p$".
        \item $\lnot p$ is read ``\tb{NOT} p".
        \item The truth value of $\lnot p$ is the opposite of the truth value of $p$.
      \end{itemize}
    \hiii{Truth Table}
      \begin{center}
        \begin{tabular}{|c|c|}
          \hline
          $p$ & $\lnot p$ \\
          \hline
          $\lT$ & $\lF$ \\
          \hline
          $\lF$ & $\lT$ \\
          \hline
        \end{tabular}
      \end{center}

  \hii{Conjunction}
    \hiii{Definition}
      \begin{itemize}
        \item The \tb{conjunction} of $p$ and $q$, denoted by $p \land q$, is the proposition ``$p$ and $q$".
        \item The conjunction $p \land q$ is \tb{True} when both $p$ and $q$ are \tb{True}
        and is \tb{False} otherwise.
      \end{itemize}

    \hiii{Truth Table}
      \begin{center}
        \begin{tabular}{|c|c|c|}
          \hline
          $p$ & $q$ & $p \land q$ \\
          \hline
          $\lT$ & $\lT$ & $\lT$ \\
          \hline
          $\lT$ & $\lF$ & $\lF$ \\
          \hline
          $\lF$ & $\lT$ & $\lF$ \\
          \hline
          $\lF$ & $\lF$ & $\lF$ \\
          \hline
        \end{tabular}
      \end{center}

  \hii{Disjunction}
    \hiii{Definition}
      \begin{itemize}
        \item Let $p$ and $q$ be propositions. The \tb{disjunction}
        of $p$ and $q$, denoted by $p \lor q$, is the proposition ``$p$ or $q$".
        \item The disjunction $p \lor q$ is \tb{False} when both $p$ and $q$ are \tb{False} and is \tb{True} otherwise.
      \end{itemize}

    \hiii{Truth Table}
      \begin{center}
        \begin{tabular}{|c|c|c|}
          \hline
          $p$ & $q$ & $p \lor q$ \\
          \hline
          $\lT$ & $\lT$ & $\lT$ \\
          \hline
          $\lT$ & $\lF$ & $\lT$ \\
          \hline
          $\lF$ & $\lT$ & $\lT$ \\
          \hline
          $\lF$ & $\lF$ & $\lF$ \\
          \hline
        \end{tabular}
      \end{center}

  \hii{Exclusive Disjunction}
    \hiii{Definition}
      \begin{itemize}
        \item The \tb{exclusive disjunction} of $p$ and $q$, denoted by $p \lxor q$, is the proposition that is \tb{True} when
        exactly one of $p$ and $q$ is \tb{True} and is \tb{False} otherwise.
      \end{itemize}

    \hiii{Truth Table}
    \begin{center}
      \begin{tabular}{|c|c|c|}
        \hline
        $p$ & $q$ & $p \lxor q$ \\
        \hline
        $\lT$ & $\lT$ & $\lF$ \\
        \hline
        $\lT$ & $\lF$ & $\lT$ \\
        \hline
        $\lF$ & $\lT$ & $\lT$ \\
        \hline
        $\lF$ & $\lF$ & $\lF$ \\
        \hline
      \end{tabular}
    \end{center}

  \hii{Implication}
    \hiii{Definition}
      \begin{itemize}
        \item Let $p$ and $q$ be propositions. The conditional statement $p \limpl q$ is the proposition ``if p, then q".
        \item The conditional statement $p \limpl q$ is \tb{False} when $p$
        is \tb{True} and $q$ is \tb{False}, and \tb{True} otherwise.
        \item $p$ is called the \tb{assumption} and $q$ is called the \tb{conclusion} of the \tb{implication} $p \limpl q$. \fnmark
        \fntext{This definition is according to Huth and Ryan's book. In Rosen's book, $p$ is called the \tb{hypothesis} (or antedecent or premise) and $q$ is called the \tb{conclusion} (or consequence).}
      \end{itemize}

    \hiii{Truth Table}
      \begin{center}
        \begin{tabular}{|c|c|c|}
          \hline
          $p$ & $q$ & $p \limpl q$ \\
          \hline
          $\lT$ & $\lT$ & $\lT$ \\
          \hline
          $\lT$ & $\lF$ & $\lF$ \\
          \hline
          $\lF$ & $\lT$ & $\lT$ \\
          \hline
          $\lF$ & $\lF$ & $\lT$ \\
          \hline
        \end{tabular}
      \end{center}

  \hii{Biconditional}
    \hiii{Definition}
      \begin{itemize}
        \item Let $p$ and $q$ be propositions. The \tb{biconditional statement}
        $p \liff q$ is the proposition ``$p$ if and only if $q$".
        \item The \tb{biconditional} is \tb{True} when $p$ and $q$ have the same
        truth value, and is \tb{False} otherwise.
      \end{itemize}

    \hiii{Truth Table}
    \begin{center}
      \begin{tabular}{|c|c|c|}
        \hline
        $p$ & $q$ & $p \liff q$ \\
        \hline
        $\lT$ & $\lT$ & $\lT$ \\
        \hline
        $\lT$ & $\lF$ & $\lF$ \\
        \hline
        $\lF$ & $\lT$ & $\lF$ \\
        \hline
        $\lF$ & $\lF$ & $\lT$ \\
        \hline
      \end{tabular}
    \end{center}

  \hii{Logical Equivalences}
    \begin{center}
      \begin{tabular}{|c|c|}
        \hline
        \textbf{Equivalence} & \textbf{Name} \\
        \hline

        $p \land \lT \equiv p$ & \multirow{2}{*}{Identity laws} \\
        $p \lor \lF \equiv p$ & \\
        \hline

        $p \land \lF \equiv \lF$ & \multirow{2}{*}{Domination laws} \\
        $p \lor \lT \equiv \lT$ & \\
        \hline

        $p \land p \equiv p$ & \multirow{2}{*}{Idempotent laws} \\
        $p \lor p \equiv p$ & \\
        \hline

        $\lnot (\lnot p) \equiv p$ & Double negation laws \\
        \hline

        $p \land q \equiv q \land p$ & \multirow{2}{*}{Commutative laws} \\
        $p \lor q \equiv q \lor p$ & \\
        \hline

        $(p \land q) \land r \equiv p \land (q \land r)$
          & \multirow{2}{*}{Associative laws} \\
        $(p \lor q) \lor r \equiv p \lor (q \lor r)$ & \\
        \hline

        $\lnot (p \land q) \equiv \lnot p \lor \lnot q$
          & \multirow{2}{*}{De Morgan's laws} \\
        $\lnot (p \land q) \equiv \lnot p \lor \lnot q$ & \\
        \hline

        $p \land (p \lor q) \equiv p$
          & \multirow{2}{*}{Absorption laws} \\
        $p \lor (p \land q) \equiv p$ & \\
        \hline

        $p \lor \lnot p \equiv \lT$ & \multirow{2}{*}{Negation laws} \\
        $p \land \lnot p \equiv \lF$ & \\
        \hline

        $p \land (q \lor r) \equiv (p \land q) \lor (p \land r)$
          & \multirow{2}{*}{Distributed laws} \\
        $p \lor (q \land r) \equiv (p \lor q) \land (p \lor r)$ & \\
        \hline

        $p \limpl q \equiv \lnot p \lor q$
        & \multirow{5}{*}{Logical Equivalences Involving Conditional Statements} \\
        $(p \limpl q) \land (p \limpl r) \equiv p \limpl (q \land r)$
        & \\
        $(p \limpl q) \lor (p \limpl r) \equiv p \limpl (q \lor r)$
        & \\
        $(p \limpl q) \land (q \limpl r) \equiv (p \lor q) \limpl r$
        & \\
        $(p \limpl q) \lor (q \limpl r) \equiv (p \land q) \limpl r$
        & \\
        \hline
      \end{tabular}
    \end{center}

\hi{Binding Priority and Association}
  \par The binding priority from high to low is as follows:
    \begin{itemize}
      \item Negation $\lnot$
      \item Conjunction $\land$ and Disjunction $\lor$
      \item Implication $\land$
    \end{itemize}
  \par Implication is right-associative, meeaning that $p \limpl q \limpl r$ is equivalent to $p \limpl (q \limpl r)$.

\hi{Premises, Conclusion and Sequent}
    \par Suppose we have:
    \begin{itemize}
      \item a set of formulas $\phi_1, \ldots, \phi_n$ called \tb{premises}.
      \item a formula $\psi$ called \tb{conclusion}.
    \end{itemize}
    \par The intention is by applying \tb{proof rules} to the premises, we finally obtain the conclusion. This intention can be denoted by:
    \begin{align*}
      \phi_1, \ldots, \phi_n \lprove \psi
    \end{align*}
    \par The expression is called a \tb{sequent}.
    \par A \tb{sequent} is \tb{valid} if a proof for it can be found.

\hi{Natural Deduction}
  \par A natural deduction can be denoted by:
  \begin{center}
    \AxiomC{$\phi_1$}
    \AxiomC{$\phi_2$}
    \AxiomC{$\ldots$}
    \AxiomC{$\phi_n$}
    \RightLabel{[Rule Name]}
    \QuaternaryInfC{$\psi$}
    \DisplayProof
  \end{center}

  \hii{Rules for Conjunction (AND)}
    \begin{enumerate}[a.]
      \item \tb{Introduction Rule} is denoted by $\land i$ and read \tb{``and-introduction"}.
        \begin{center}
          \AxiomC{$\phi$}
          \AxiomC{$\psi$}
          \RightLabel{$\land i$}
          \BinaryInfC{$\phi \land \psi$}
          \DisplayProof
        \end{center}

      \item \tb{Elimination Rule}: is denoted by $\land e$ and read \tb{``and-elimination"}.
        \begin{center}
          \AxiomC{$\phi \land \psi$}
          \RightLabel{$\land e_1$}
          \UnaryInfC{$\phi$}
          \DisplayProof
          \hskip 2cm
          \AxiomC{$\phi \land \psi$}
          \RightLabel{$\land e_2$}
          \UnaryInfC{$\psi$}
          \DisplayProof
        \end{center}
    \end{enumerate}
  
    \par \tb{Example}: Prove that $p \land q, r \lprove q \land r$.
    \snote{Example 1.4, page 6}
      \begin{logicproof}{1} % 1 = maximum number of subproofs
        p \land q & premise \\
        r         & premise \\
        q         & $\land e_2$ 1 \\
        q \land r & $\land i$ 3, 2
      \end{logicproof}

  \hii{Rules for Double Negation}
    \begin{enumerate}[a.]
      \item \tb{Introduction Rule} is denoted by $\lnotnot i$.
        \begin{center}
          \AxiomC{$\phi$}
          \RightLabel{$\lnotnot i$}
          \UnaryInfC{$\lnotnot \phi$}
          \DisplayProof
        \end{center}

      \item \tb{Elimination Rule} is denoted by $\lnotnot e$.
        \begin{center}
          \AxiomC{$\lnotnot \phi$}
          \RightLabel{$\lnotnot e$}
          \UnaryInfC{$\phi$}
          \DisplayProof
        \end{center}
    \end{enumerate}

    \par \tb{Example}: Prove the sequent $p, \lnotnot (q \land r) \lprove \lnotnot p \land r$.
    \snote{Example 1.5, page 8}
      % =================================== [[[
      \begin{logicproof}{1} % 1 = maximum number of subproofs
        p                    & premise \\
        \lnotnot (q \land r) & premise \\
        \lnotnot p           & $\lnotnot i$ 1 \\
        q \land r            & $\lnotnot e$ 2 \\
        r                    & $\land e_2$ 4 \\
        \lnotnot p \land r   & $\land i$ 3, 5
      \end{logicproof}
      % =================================== ]]]

  \hii{Rules for Implication}
    \begin{enumerate}[a.]
      \item \tb{Elimination Rule} is denoted by $\limpl e$ and read \tb{``implies-elimination"}. A, also known as \tb{modus ponens}.
        % ================================= [[[
        \begin{center}
          \AxiomC{$\phi$}
          \AxiomC{$\phi \limpl \psi$}
          \RightLabel{$\limpl e$}
          \BinaryInfC{$\psi$}
          \DisplayProof
        \end{center}
        % ================================= ]]]

      \par A well-known related rule is modus tollens, denoted by MT. It is also an elimination rule.
        % ================================= [[[
        \begin{center}
          \AxiomC{$\phi \limpl \psi$}
          \AxiomC{$\lnot \psi$}
          \RightLabel{MT.}
          \BinaryInfC{$\lnot \phi$}
          \DisplayProof
        \end{center}
        % ================================= ]]]

        \par \tb{Example}: Prove that $p \limpl (q \limpl r), p, \lnot r \lprove \lnot q$

      \item \tb{Introduction Rule} is denoted by $\limpl i$ and read \tb{``implies-introduction"}.
        % ================================= [[[
        \newsavebox\ImplIntroAssump
        \sbox\ImplIntroAssump{
          \fbox{
            \AxiomC{$\phi$}
            \noLine
            \UnaryInfC{$\vdots$}
            \noLine
            \UnaryInfC{$\psi$}
            \DisplayProof
          }
        }

        \begin{center}
          \AxiomC{\usebox\ImplIntroAssump}
          \RightLabel{$\limpl i$}
          \UnaryInfC{$\phi \limpl \psi$}
          \DisplayProof
        \end{center}
        % ================================= ]]]

        \par Here, to prove $\phi \limpl \psi$, we need to make a temporary \tb{assumption} $\theta$. \tb{A box} is opened when \tb{an assumption} is made. When the proof no longer depends on the assumption, the box is closed.
    \end{enumerate}

  \hii{Rules for Disjunction (OR)}

    \begin{enumerate}[a.]
      \item \tb{Introduction Rule} is denoted by $\lor i$.
        \begin{center}
          \AxiomC{$\phi$}
          \RightLabel{$\lor i_1$}
          \UnaryInfC{$\phi \lor \psi$}
          \DisplayProof
          \hskip 2cm
          \AxiomC{$\psi$}
          \RightLabel{$\lor i_2$}
          \UnaryInfC{$\phi \lor \psi$}
          \DisplayProof
        \end{center}

      \item \tb{Elimination Rule} is denoted by $\lor e$.
        \begin{center}
          \AxiomC{$\phi \lor \psi$}
            \alwaysNoLine
            \AxiomC{$\phi$}
            \UnaryInfC{$\vdots$}
            \UnaryInfC{$\chi$}
            \alwaysNoLine
            \AxiomC{$\psi$}
            \UnaryInfC{$\vdots$}
            \UnaryInfC{$\chi$}
          \RightLabel{$\lor e$}
          \alwaysSingleLine
          \TrinaryInfC{$\chi$}
          \DisplayProof
        \end{center}
    \end{enumerate}

\hii{Rules for Contradiction}
  \par A \tb{contradiction} is an expression of the form
  \[
  \phi \land \lnot \phi
  \]
  where $\phi$ is any formula. A contradiction is denoted by $\lcontrad$.
  \par \tb{Elimination Rule}: This rule says that: \ti{a contradiction can prove anything}.
    \begin{center}
      \AxiomC{$\lcontrad$}
      \UnaryInfC{$\phi$}
      \DisplayProof
    \end{center}

\hii{Rules for Negation}
    \begin{enumerate}[a.]
      \item \tb{Introduction Rule} is denoted by $\lnot i$.
        \begin{center}
          \alwaysNoLine
          \AxiomC{$\phi$}
          \UnaryInfC{$\ldots$}
          \UnaryInfC{$\psi$}
          \RightLabel{$\lnot i$}
          \alwaysSingleLine
          \UnaryInfC{$\phi \lnot \psi$}
          \DisplayProof
        \end{center}

      \item \tb{Elimination Rule} is denoted by $\lnot e$.
        \begin{center}
          \AxiomC{$\phi$}
          \AxiomC{$\lnot \phi$}
          \RightLabel{$\lnot e$}
          \BinaryInfC{$\lcontrad$}
          \DisplayProof
        \end{center}
    \end{enumerate}

\hi{Derived Rules}
  \hii{Modus Tollens}
    \begin{center}
      \AxiomC{$\phi \limpl \psi$}
      \AxiomC{$\lnot \psi$}
      \RightLabel{MT}
      \BinaryInfC{$\lnot \phi$}
      \DisplayProof
    \end{center}

  \hii{Proof By Contradiction (PCB)}
    \begin{center}
      \alwaysNoLine
      \AxiomC{$\lnot \phi$}
      \UnaryInfC{$\vdots$}
      \UnaryInfC{$\lcontrad$}
      \RightLabel{PBC}
      \alwaysSingleLine
      \UnaryInfC{$\phi$}
      \DisplayProof
    \end{center}

  \hii{Law of the Excluded Middle (LEM)}
    \par This rule says that: \ti{the statement $\phi \lor \lnot \phi$ is always \tb{True}}.
    \begin{center}
      % \AxiomC{$\phi$}
      % \AxiomC{$\lnot \phi$}
      \AxiomC{}
      \RightLabel{LEM}
      \UnaryInfC{$\phi \lor \lnot \phi$}
      \DisplayProof
    \end{center}

\hi{Provable Equivalence}
  \par Let $\phi$ and $\psi$ be formulas of propositional logic. We say that $\phi$ and $\psi$ are provably equivalent iff (we write ``iff" for ``if, and only if" in the sequel) the sequents $\phi \lprove \psi$ and $\psi \lprove \phi$ are valid; that is, there is a proof of $\psi$ from $\phi$ and another one going the other way around. As seen earlier, we denote that $\phi$ and $\psi$ are provably equivalent by $\phi \lproveeq \psi$.
