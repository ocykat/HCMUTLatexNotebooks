\documentclass[12pt, a4paper]{report}


% MARGINS
\usepackage[margin=0.75in]{geometry}


% HEADINGS
\setcounter{secnumdepth}{4}
\newcommand{\hi}{\section}
\newcommand{\hii}{\subsection}
\newcommand{\hiii}{\subsubsection}
\newcommand{\hiiiBEGIN}[1]{\subsubsection \begin{enumerate}}
\newcommand{\hiiiEND}{\end{enumerate}}
\newcommand{\hiv}{\item\textbf}


% INDENTATION
\usepackage{indentfirst}


% TABLE OF CONTENTS
% links
\usepackage{hyperref}
\hypersetup{
  colorlinks,
  citecolor=black,
  filecolor=black,
  linkcolor=black,
  urlcolor=black
}


% APPENDICES
\usepackage[toc,page]{appendix}


% TEXT
% Bold, italic, underlined text
\newcommand{\tb}[1]{\textbf{#1}}
\newcommand{\ti}[1]{\textit{#1}}
\newcommand{\tbi}[1]{\textbf{\textit{#1}}}
\newcommand{\tu}[1]{\underline{#1}}
\newcommand{\tbu}[1]{\textbf{\underline{#1}}}
\newcommand{\smf}[1]{\small #1 \normalsize}
\newenvironment{smfont}{\small}{\normalsize}


% MATH
\usepackage{amsmath}
\usepackage{amssymb}
\usepackage{gensymb}

% Equation box
\usepackage{empheq}
\newenvironment{eqbox}
  {\setkeys{EmphEqEnv}{align}\setkeys{EmphEqOpt}{box=\fbox}\EmphEqMainEnv}
  {\endEmphEqMainEnv}

% Arrows
\newcommand{\ra}{\Rightarrow}
\newcommand{\lra}{\Leftrightarrow}

% Sum, product, limit
\newcommand{\SUM}[1]{\sum\limits #1}
\newcommand{\PROD}[1]{\prod\limits #1}

% Bold in math mode
\usepackage{bm}

% Absolute value
\DeclarePairedDelimiter\abs{\lvert}{\rvert}%
\DeclarePairedDelimiter\norm{\lVert}{\rVert}%
%   Swap the definition of \abs* and \norm*, so that \abs
%   and \norm resizes the size of the brackets, and the 
%   starred version does not.
\makeatletter
\let\oldabs\abs
\def\abs{\@ifstar{\oldabs}{\oldabs*}}
\let\oldnorm\norm
\def\norm{\@ifstar{\oldnorm}{\oldnorm*}}
\makeatother

% Floor and ceiling
\usepackage{mathtools}
\DeclarePairedDelimiter\ceil{\lceil}{\rceil}
\DeclarePairedDelimiter\floor{\lfloor}{\rfloor}

% Inner Product
\newcommand{\iprod}[1]{\langle #1 \rangle}

% Derivatives
\newcommand{\dif}[2]{\dfrac{d #1}{d #2}}
\newcommand{\ddif}[2]{\dfrac{d^2 #1}{d #2^2}}
\newcommand{\difi}[2]{d #1/d #2}
\newcommand{\diff}[2]{\dfrac{d}{d #2} #1}
\newcommand{\pd}[2]{\dfrac{\partial #1}{\partial #2}}
\newcommand{\pdds}[2]{\dfrac{\partial^{2} #1}{\partial #2^{2}}}
\newcommand{\pdd}[3]{\dfrac{\partial^{2} #1}{\partial #2 \partial #3}}
\newcommand{\pddsi}[2]{{\partial^{2} #1} / {\partial #2^{2}}}
\newcommand{\pddi}[3]{{\partial^{2} #1} / {\partial #2 \partial #3}}

% Integrals
\usepackage{esint}
\newcommand{\INT}{\int \limits}
\newcommand{\OINT}{\oint \limits}
\newcommand{\IINT}{\iint \limits}
\newcommand{\IIINT}{\iiint \limits}


% FOOTNOTE
%   one foot note stays on one page
\interfootnotelinepenalty=10000

\newcommand{\fnmark}{\footnotemark}
\newcommand{\fnmarksame}{\footnotemark[\value{footnote}]}
\newcommand{\fntext}{\footnotetext}


% PSEUDOCODE
\usepackage{algorithm}
\usepackage{algorithmicx}
\usepackage[noend]{algpseudocode}
\usepackage{caption}

\renewcommand{\thealgorithm}{\arabic{chapter}.\arabic{algorithm}}

\newcommand*\Let[2]{\State #1 $\gets$ #2}
\algrenewcommand\algorithmicrequire{\textbf{Input:}}
\algrenewcommand\algorithmicensure{\textbf{Output:}}
\newcommand{\INPUT}[1]{\Require{#1} \Statex}
\newcommand{\OUTPUT}[1]{\Ensure{#1} \Statex}
\newcommand{\INPUTOUTPUT}[2]{\Require{#1} \Ensure{#2} \Statex}
\newcommand{\LET}[2]{\Let{$#1$}{$#2$}}
\newcommand{\FOR}[2]{\For{$#1 \gets #2$}}
\newcommand{\ENDFOR}{\EndFor}
\newcommand{\TO}{\textrm{ \tb{to} }}
\newcommand{\DOWNTO}{\textrm{ \tb{downto} }}
\newcommand{\AND}{\textrm{ \tb{and} }}
\newcommand{\OR}{\textrm{ \tb{or} }}
\newcommand{\XOR}{\textrm{ \tb{xor} }}
\newcommand{\GETS}{\gets}
\newcommand{\IF}[1]{\If{$#1$}}
\newcommand{\ELSEIF}[1]{\ElsIf{$#1$}}
\newcommand{\ELSE}{\Else}
\newcommand{\ENDIF}{\EndIf}
\newcommand{\WHILE}[1]{\While{$#1$}}
\newcommand{\ENDWHILE}{\EndWhile}
\newcommand{\FUNCTION}[2]{\Function{#1}{$#2$}}
\newcommand{\ENDFUNCTION}{\EndFunction}
\newcommand{\PROCEDURE}[2]{\Procedure{#1}{$#2$}}
\newcommand{\ENDPROCEDURE}{\EndProcedure}
\newcommand{\CALLFUNC}[2]{\State \Call{#1}{$#2$}}
\newcommand{\CALLPROC}[2]{\State \Call{#1}{$#2$}}
\newcommand{\RETURN}[1]{\State \Return{#1}}


\begin{document}
\tableofcontents

\setcounter{chapter}{8}
\chapter{MSI Circuits}

\hi{Decoder}
  \hii{Introduction}
    \par A decoder is a logic circuit that accepts a set of inputs that
      represents a binary number and activates only the output that
      corresponds to that input number. All other outputs remain inactive.

  \hii{Inputs}
    \par For a decoder which has $N$ inputs, it may not utilize all $2^{N}$
      possible input codes but only certain ones. (BCD-to-decimal decoder is
      one example). For decoders of this type, if any of the unused code is
      applied to the input, none of the outputs will be activated.

  \hii{Referring decoder}
    \par Some examples:
    \begin{itemize}
      \item 3-line-to-8-line decoder: 3 input lines, 8 output lines
      \item binary-to-octal decoder: convert a 3bit binary input and activates
        one of the eight outputs corresponding to that code.
      \item 1-of-8 decoder: only 1 of the 8 outputs is activated at one time
    \end{itemize}

  \hii{ENABLE Inputs}
    \par Some decoders have one or more ENABLE inputs to control the decoder.

  \hii{BCD-to-Decimal Decoders - Driver}
    \par Difference between a decoder and a driver: the driver has
      \footnotemark{open-collector} outputs that can operate at higher current
      and voltage limits than a normal decoder output.
      \footnotetext{Open-collector output: the output is connected to the base (B)
        of a NPN transistor. If B is triggered, then there is electric current going
        from collector (C) to emitter (E).}

\hi{Encoder}
  \hii{Introduction}
    \par An \tb{encoder} has a number of input lines, only \tb{one} of which is
    activated at a given time, and produces an $N$-bit output code, depending on
    which input is activated.
    \par Example: octal-to-binary encoder.
  \hii{Priority Encoder}
    \par Drawback of simple encoder: when more than one input is activated at one time,
      the output is unexpected.
    \par A \tb{priority encoder} ensure that when two or more inputs are activated, the
    output code will correspond to the highest-numbered input.
    \par Example: 74147 Decimal-to-BCD Priority Encoder
  \hii{Switch Encoder}
  \par Example: 74147: switch priority decoder
    \par \tb{Switch encoder} is used whenever BCD data must be entered manually into a
      digital system.
    \par When a key is pressed, the BCD code for the corresponding digit is sent to a
      four-bit FF register;
    \par When another key is pressed, the BCD code for the corresponding digit is sent to
      \ti{another} four-bit FF register;
    \par A calculator that can handle eight digits will have eight four-bit registers to
    store the BCD. Each four-bit register drives a decoder/driver and a numerical display. 

\hi{Multiplexers (Data selectors)}
  \hii{Introduction}
    \par A \tb{digital multiplexer} or \tb{data selector} is a logic circuit that accepts
      several digital data inputs and selects one of them at any given time to pass on to
      the output.
    \par The routing of the desired data input is controlled by SELECT (ADDRESS) inputs.
  \hii{Two-Input Multiplexer}
  \hii{Quad Two-Input MUX (74ALS157)}
  \hii{Four-Input Multiplexer}
    \par Another approach is to use tristate buffers.
    \par The decoder is used to make sure that only one input is selected.
  \hii{Eight-Input Multiplexer}
    \par Example: 74ALS151

\hi{Demultiplexers (Data Distributors)}
  \hii{Introduction}
    \par A \tb{demultiplexer} takes a single input and distributes it over several outputs.
    \par 1 DATA input signal is transmitted to \tb{only 1 output} at the time. The output is
      selected by SELECT inputs.
  \hii{1-Line-to-8-Line Demultiplexer}
    \par A single data input line $I$ is connected to all eight AND gates.
    \par At any moment, only one AND gate is enabled.
    \par Data input $I$ will appear at output of the only enabled AND gate.

\chapter{Dynamics of Rotational Motion}
\hi{Torque}
  \begin{itemize}
    \item Vector form:
    \begin{eqbox}
      \tau_z = \vec{F} \times \vec{r}
    \end{eqbox}

    \item Magnitude:
    \begin{eqbox}
      \tau_z = Fr\sin(\theta) = F_{tan}r
    \end{eqbox}
    where
    \begin{itemize}
      \item $tau$: torque
      \item $\vec{F}$: force
      \item $\vec{r}$: lever arm
    \end{itemize}

  \end{itemize}


\hi{Torque and Angular Acceleration for a Rigid Body}
  \par According to Newton's Second Law, for a rotating particle under a force $F$:
    \begin{align*}
      F_{i.tan} = m_i a_{i.tan}
    \end{align*}
    where
    \begin{itemize}
      \item $F_{i.tan}$: tangential component of the force $F$
      \item $m$: mass of the particle
      \item $a_{i.tan}$: tangential component of acceleration
    \end{itemize}
  \par Relationship between tangential acceleration and rotational acceleration:
  \begin{align*}
    a_{i.tan} = r_i \alpha_z
  \end{align*}
  \par Therefore:
  \begin{flalign*}
    & \qquad F_{i.tan} = m_i r_i \alpha_z && \\
    & \ra F_{i.tan} r_i = m_i r_i^2 \alpha_z && \\
    & \ra \SUM (F_{i.tan} r_i) = \SUM (m_i r_i^2) \alpha_z && \\
    & \ra \SUM (\tau_{z.i.tan}) = \SUM (I_i) \alpha_z && \\
    & \ra \tau_z =  I \alpha_z &&
  \end{flalign*}
  \begin{eqbox}
    \tau_z = I \alpha_z
  \end{eqbox}
  where
  \begin{itemize}
    \item $\tau_z$: torque
    \item $I$: moment of inertia
    \item $\alpha_z$: angular acceleration
  \end{itemize}


\hi{Angular Momentum}
  \hii{Definition}
    \BOOKSECTION{10.5}
    \begin{align*}
      \vec{L} = \vec{p} \times \vec{r} = m\vec{v} \times \vec{r}
    \end{align*}
    where:
    \begin{itemize}
      \item $\vec{L}$: angular momentum
      \item $\vec{p}$: lever arm/radius of rotation
      \item $\vec{r}$: linear momentum
    \end{itemize}

  \hii{Angular Momentum of a Rigid Body}
    \par For a particle:
    \begin{flalign*}
      & L_i = m_i v_i r_i = m_i (\omega_i r_i) r_i = m_i r_i^2 \omega_i &&
    \end{flalign*}
    \par Then for a rigid body:
    \begin{flalign*}
      & \ra \SUM(L_i) = \SUM(m_i r_i^2 \omega_i) = \SUM(m_i r_i^2) \omega && \\
      & \ra \SUM(L_i) = \SUM(I_i) \omega && \\
      & \ra L = I \omega &&
    \end{flalign*}
    \begin{eqbox}
      L = I \omega
    \end{eqbox}

\setcounter{chapter}{13}
\chapter{Functional Dependencies and Normalization for Relational Databases}

\hi{Informal Design Guidelines for Relation Schemas}

  \par Four design guidelines discussed are:

  \begin{itemize}
    \item Making sure that the semantics of the attributes is clear in the schema
    \item Reducing the redundant information in tuples
    \item Reducing the NULL values in tuples
    \item Disallowing the possibility ofgenerating spurious tuples
  \end{itemize}

  \hii{Imparting Clear Semantics to Attributes in Relations}
    \par The \tb{semantics} of a relation refers to its meaning resulting from the interpretation of attribute values in a tuple.
    \par The easier it is to explain the semantics of the relation, the better the relation schema design will be.
    \par \tb{Guideline 1}:
    \begin{itemize}
      \item Design a relation schema so that it is easy to explain its meaning.
      \item Do not combine attributes from different entity types into a single relation.
      \item Do not combine attributes from different relationship types into a single relation.
      \item Do not combine attributes from different entity types and relationship types into a single relation.
    \end{itemize}

  \hii{Redundant Information in Tuples and Update Anomalies}
    \par Storing natural joins of base relations (relations of entity types and relationship types) leads to an additional problem refered to as \tb{update anomalies}. There are three types of update anomalies:
    \begin{itemize}
      \item \tb{Insertion Anomalies}
      \item \tb{Deletion Anomalies}
      \item \tb{Modification Anomalies}
    \end{itemize}
    \par \tb{Guideline 2}: Design the base relation schemas so that no insertion, deletion, or modification anomalies are present in the relations. If any anomalies are present, note them clearly and make sure that the programs that update the database will operate correctly.

  \hii{NULL Values in Tuples}
    \par \tb{Guideline 3}: As far as possible, avoid placing attributes in a base relation whose values may frequently be NULL. If NULLs are unavoidable, make sure that they apply in exceptional cases only and do not apply to a majority of tuples in the relation.

  \hii{Generation of Spurious Tuples}
    \par \tb{Guideline 4}: Design relation schemas so that they can be joined with equality conditions on attributes that are appropriately related (primary key, foreign key) pairs in a way that guarantees that no spurious tuples are generated.


\hi{Functional Dependencies}
  \hii{Definition}
    \par Suppose that the relational database schema has $n$ attributes $A_1 , \ldots, A_n$. The database can be described by one single \tb{universal relation schema} $R = \{A_1, \ldots, A_n\}$. Define:
    \begin{itemize}
      \item $X$ and $Y$ as subsets of attributes of $R$.
      \item $r$ as a relation state of $R$ (relation state is the set of all tuples in the relation schema).
      \item $t_1$ and $t_2$ as two tuples in $r$.
    \end{itemize}
    \par A \tb{functional dependency} $X \to Y$ specifies the constraint that: for any two tuples $t_1$ and $t_2$ in $r$, if $t_1[X] = t_2[X]$, then $t_1[Y] = t_2[Y]$.
    \par The value of the $Y$ component of a tuple in $r$ is determined by the value of the $X$ component; alternatively, the values of the $X$ component of a tuple uniquely/functionally determine the value of the $Y$ component. (with 1 $X$ there can only be 1 corresponding $Y$)
    \par If $X$ is a \tb{candidate key} of $R$, then $X \to R$.
    \par If $X \to Y$, it does not guarantee whether $Y \to X$ or not.
    \par Functional dependency is a property of the \tb{semantics} or \tb{meaning of the attributes}.
    \par The purpose of functional dependency is to further describe a relation schema $R$ by specifying constraints on its attributes that must hold at all time. Relation states $r(R)$ that satisfy the functional dependency constraints are called \tb{legal relation states}.

\hi{Normal Forms Based on Primary Keys}
  \hii{Normalization of Relations}
    \par \tb{Normalization} process takes a relation schema through a series of tests to certify whether it satisfies a certain \tb{normal form}.
    \par \tb{Normalization of data}: a process of analyzing the given relation schemas based on their FDs and primary keys to
    \begin{itemize}
      \item Minimize redundancy
      \item Minimize insertion, deletion, update anomalies
    \end{itemize}
    \par In this process, a relation schema that does not meet the condition for a normal form is decomposed into smaller relation schemas that meet the \tb{normal form test} (condition for a normal form) that is not met by the original relation.
    \par The normalization process through decomposition must confirm the existence of some properties including two properties:
    \begin{itemize}
      \item The nonadditive join or loseless join property: there is no spurious tuple generated with respect to the relation schemas created after decomposition.
      \item The dependency preservation property: each functional dependency is represented in some individual relation resulting after decomposition.
    \end{itemize}
    \par The first property is extremely critical and \tb{must be achieved at any cost}. The second property, although desirable, is sometimes sacrified.

  \hii{Practical Use of Normal Forms}
    \par In practice, normalization up to 3NF, BCNF, or at most 4NF is used. Higher normal forms are hard to understand or to detect for database designers and users.
    \par Database designers \tb{need not} normalize to the highest possible normal form. Relations may be left in a lower normalization status for performance reasons.
    \par \tb{Definition}: \tb{Denormalization} is the process of storing the join of higher normal form relations as a base relation which is in a lower normal form.

  \hii{Definition of Keys and Attributes Participating in Keys}
    \par We revisit some definition in the previous chapter. Given a relation schema $R = \{A_1, \ldots, A_n\}$.
    \begin{itemize}
      \item \tb{superkey}: a super key is a set of attributes $S \subseteq R$ with the property that no two tuples $t_1$ and $t_2$ in a legal relation state $r$ of $R$ will have $t_1[S] = t_2[S]$.
      \item \tb{key}: a key $K$ is a super key with the additional property that removal of any attribute from $K$ will cause $K$ not to be a superkey anymore. We can say that a key is a minimal superkey.
      \item \tb{candidate key}: if a relation schema has more than one key, each is called a candidate key.
      \item \tb{primary key}: if a relation schema has more than one key, then one can be chosen arbitrarily to be the primary key. If the relation schema only has one key, then that only key is the primary key.
      \item \tb{secondary key}: any candidate key that is not a primary key
    \end{itemize}
  \par \tb{Definition}: An attribute of relation schema $R$ is called a \tb{prime attribute} of $R$ if it is a member of some candidate key of $R$. An attribute is called \tb{nonprime} if it is not a prime attribute.

  \hii{First Normal Form}
    \par First Normal Form (1NF) disallow relations within relations or relations as attribute values within tuples. \tb{The only attribute values permitted by 1NF are \tu{single atomic} (or \tu{indivisible}) values.}

  \hii{Second Normal Form (based on Primary Key)}
    \par Second Normal Form (2NF) is based on the concept of \tb{full functional dependency}.
    \par A functional dependency $X \to Y$ is a \tb{full functional dependency} if removal of any attribute $A$ from $X$ means that the dependency does not hold anymore.
    \par A functional dependency $X \to Y$ is a \tb{partial functional dependency} if some attribute $A$ can be removed from $X$ and the dependency still holds.
    \par \tb{Definition of 2NF}: A relation schema $R$ is in 2NF if every nonprime attribute $A$ in $R$ is \tb{fully functional dependent} on the primary key of $R$.
    \par \tb{Example}:
    \img[width=12cm]{img/2nf-01.png}{}
    \par In the figure,
      \begin{itemize}
        \item FD2 violates 2NF: \ilc{Ename} can be functionally determined by \ilc{Ssn} only.
        \item FD3 violates 2NF: both \ilc{Pname} and \ilc{Plocation} can be functionally determined by \ilc{Pnumber} only.
      \end{itemize}

  \hii{Third Normal Form (based on Primary Key)}
    \par Third Normal Form (3NF) is based on the concept of \tb{transitive dependency}.
    \par A functional dependency $X \to Y$ is a \tb{transitive dependency} if $\exists Z \subseteq R$:
    \begin{itemize}
      \item $Z$ is neither a candidate key nor subset of any key of $R$
      \item $X \to Z$ and $Z \to Y$ holds.
    \end{itemize}
    \par \tb{Definition of 3NF}: A relation schema $R$ is in 3NF if it satisfies 2NF \tb{and} no nonprime attribute of $R$ is transitively dependent on the primary key.
    \img[width=12cm]{img/3nf-01.png}{}
    \par In the figure, the relation schema \ilc{EMP_DEPT} is in 2NF but not in 3NF because of the transitive dependency of \ilc{Dmgr_ssn} and \ilc{Dname} on \ilc{Ssn} via \ilc{Dnumber}.

\hi{General Definitions of Second and Third Normal Forms}
    \par In general, partial and transitive dependencies should be avoided because they cause the update anomalies.
    \par General definition of \tb{prime attribute}: an attribute that is part of any candidate key.
    \par Partial and full functional dependencies and transitive dependencies is considered with respect to all candidate keys of a relation in this section.

  \hii{General Definition of 2NF}
    \par \tb{General definition of 2NF}: A relation schema $R$ is in 2NF if every nonprime attribute $A$ in $R$ is not partially dependent on any key of $R$.
    \par \tb{Example}:
    \img[width=12cm]{img/2nf-02.png}{}
    \par The \ilc{LOTS} relation schema violates the general definition of 2NF because \ilc{Tax_rate} is partially dependent on the candidate key
    \par Solution: Decompose the relation as follows:
    \img[width=14cm]{img/2nf-03.png}{}

  \hii{General Definition of Third Normal Form}
    \par \tb{General definition of 3NF}: A relation schema $R$ is in 3NF if: whenever a \tb{nontrivial functional dependency} $X \to R$ holds in $R$, either:
    \begin{itemize}
      \item $X$ is a superkey of $R$.
      \item $A$ is a prime attribute of $R$.
    \end{itemize}
    \par \tb{Example}:
    \img[width=14cm]{img/3nf-02.png}{}
    \par Here, \ilc{LOTS1} violates 3NF because:
    \begin{itemize}
      \item \ilc{Area} is not a superkey
      \item \ilc{Price} is not a prime attribute
    \end{itemize}
    \par Solution:
    \img[width=14cm]{img/3nf-03.png}{}
    \par Here, both \ilc{LOTS1a} and \ilc{LOTS1b} are in 3NF.
    \par Note that:
    \begin{itemize}
      \item \ilc{LOTS1} violates 3NF because \ilc{Price} is transitively dependent on each of the candidate keys of \ilc{LOTS1} via the nonprime attribute \ilc{Area}.
      \item If a schema is in 3NF, it is certainly in 2NF.
    \end{itemize}

  \hii{Interpreting the General Definition of Third Normal Form}
    \par \tb{Alternative Definition of 3NF}: A relation schema $R$ is in 3NF if every nonprime attribute of $R$ meets both of the following conditions:
    \begin{itemize}
      \item It is fully functionally dependent on every key of $R$.
      \item It is nontransitively dependent on every key of $R$.
    \end{itemize}

  \hii{Boyce-Codd Normal Form}
    \par Boyce-Codd normal form (BCNF) is simpler but stricter than 3NF.
    \par Every relation that is in BCNF is also in 3NF, but a relation in 3NF does not necessary in BCNF.
    \par \tb{Definition of BCNF}: A relation schema $R$ is in BCNF if: whenever a nontrivial functional dependency $X \to A$ holds in $R$, then $X$ is a superkey of $R$.
    \par (The second condition in 3NF is excluded).

\chapter{Relational Database Design Algorithms and Further Dependencies}

\hi{Further Topics in Functional Dependencies: Inference Rules, Equivalence, and Minimal Cover}
  \hii{Inference Rules}
    \par Let $F$ be the set of functional dependencies that are specified on a relation schema $R$.

    \hiii{Inferred Functional Dependencies}
      \par An FD $X \to Y$ is \tb{inferred from} or \tb{implied by} a set of dependencies $F$ specified on $R$ if $X \to Y$ holds in \tb{every} legal relation state $r$ of $R$.

    \hiii{Closure of Functional Dependencies}
      \par The set of all dependencies that include $F$ as well as dependencies that can be inferred from $F$ is called the \tb{closure} of $F$ and is denoted by $F^{+}$.

    \hiii{Some Notations}
      \begin{itemize}
        \item $F |= X \to Y$: the FD $X \to Y$ is inferred from $F$
        \item $\{X, Y\} \to Z$ can be abbreviated to $XY \to Z$
        \item $\{X, Y, Z\} \to \{U, V\}$ can be abbreviated to $XYZ \to UV$
      \end{itemize}

    \hiii{Inference Rules}
      \par Aarmstrong's Axioms:
      \begin{itemize}
        \item \tb{IR1: Reflexive Rule}:
          $X \supseteq Y \Rightarrow X \to Y$
        \item \tb{IR2: Augmentation Rule}:
          $\{ X \to Y \} |= XZ \to YZ$
        \item \tb{IR3: Transitive Rule}:
          $\{ X \to Y, Y \to Z \} |= X \to Z$
      \end{itemize}
      \par Other rules:
      \begin{itemize}
        \item \tb{IR4: Decomposition Rule/Projective Rule}:
          $\{ X \to YZ \} |= X \to Y$
        \item \tb{IR5: Union Rule/Additive Rule}:
          $\{ X \to Y, X \to Z \} |= X \to YZ$
        \item \tb{IR6: Pseudotransitive Rule}:
          $\{ X \to Y, WY \to Z \} |= WX \to Z$
      \end{itemize}

    \hiii{Closure of Attributes under The FD Set}
      \par Let $X$ be a set of attributes of $R$. Then \tb{the closure of $X$ under $F$}, denoted by $X^{+}$, are all attributes that are functional determined by $X$ based on $F$.
      \par \tb{Algorithm}:
        \begin{algorithm}[H]
          \caption{Determine $X^{+}$ of $X \subseteq R$ under $F$}
          \begin{algorithmic}[1]
            \State $X^{+} := X$
            \While{\texttt{True}}
              \State $oldX^{+}$ := $X^{+}$
              \For{each $Y \to Z \in F$}
                \If{$X^{+} \supseteq Y$}
                  \State $X^{+}$ := $X^{+} \cup Z$
                \EndIf
              \EndFor
              \If{$oldX^{+} = X^{+}}$
                \State break
              \EndIf
            \EndWhile
          \end{algorithmic}
        \end{algorithm}

  \hii{Equivalence of Sets of Functional Dependencies}
      \hiii{Cover Relationship}
        \par A set of functional dependencies $F$ is said to cover another set of functional dependencies $E$ if every FD in $E$ is also in $F^+$ - meaning that every dependencies in $E$ can be \tb{inferred} from $F$.

      \hiii{Equivalent Set of FDs}
        \par Two sets of FDs $E$ and $F$ are \tb{equivalent} if $E^+ = F^+$.

  \hii{Minimal Sets of Functional Dependencies}
    \hiii{Extraneous Attribute}
      \par An attribute in a functional dependency is considered an extraneous attribute if we can remove it without changing the closure of the set of dependencies.

    \hiii{Canonical Form of Set of FDs}
      \par An FD is said to be in \tb{canonical form} if it only has a single attribute on the right-hand side.
      \par A set of FDs $F$ is said to be in \tb{canonical form} if for every FD in $F$, there is only a single attribute on the right-hand side.
      \par Example: If $X \to YZ \in F$, then $F$ is not in canonical form.

    \hiii{Minimal Set of FDs}
      \par A set of FDs $F$ is said to be \tb{minimal} if it satisfies the following conditions:
      \begin{itemize}
        \item $F$ is in canonical form
        \item Given that $X \supset Y$, we cannot replace $X \to A$ in $F$ with $Y \to A$ and still have a set of dependencies that is equivalent to $F$.
        \item We cannot remove any dependency from $F$ and still have a set of dependencies that is equivalent to $F$.
      \end{itemize}
      \par In short, a minimal set of dependencies must be in \tb{standard} or \tb{canonical form} (condition 1) and has \tb{no redundancies} (conditions 2 and 3).

    \hiii{Minimal Cover}
      \par A \tb{minimal cover} of a set of functional dependencies $E$ is a \tb{minimal set of dependencies} that is equivalent to $E$. We can always find \tb{at least one} minimal cover $F$ for any set of dependencies $E$ using the following algorithm:
        \begin{algorithm}[H]
          \caption{Determine the minimal cover $F$ for a set of FDs $E$}
          \Input{set of FDs $E$}
          \begin{algorithmic}[1]
            \State $F := E$
            \For{each FD $X \to \{ A_1, \ldots, A_n \} \in F$}
              \State Replace with $X \to A_1, \ldots, X \to A_n$
              \Comment{Place the FD in a canonical form}
            \EndFor
            \For{each FD $X \to A \in F$}
              \For{each attribute $B \in X$}
                \If{$ F - \{ X \to A \} + \{ (X - \{B\}) \to A \}$ is equivalent to $F$}
                  \State Replace $X \to A$ with $(X - \{B\}) \to A$ in $F$
                  \Comment{Remove all extraneous attributes}
                \EndIf
              \EndFor
            \EndFor
            \For{each remaining FD $X \to A \in F$}
              \If{$F - \{X \to A\}$ is equivalent to $F$}
                \State Remove $X \to A$ from $F$
                \Comment{Remove all remaining redundant FDs}
              \EndIf
            \EndFor
          \end{algorithmic}
        \end{algorithm}
      \par Note that on line 6 I used the plus ($+$) symbol instead of the union ($\cup$) symbol (like in the book) for easy-reading.

    \hiii{Determine the Key of a Relation}
      \begin{algorithm}[H]
        \caption{Determine the key $K$ of a relation $R$}
        \Input{$R$ and the set of FDs $F$ of $R$}
        \begin{algorithmic}[1]
          \State $K := R$
          \For{each attribute $A \in K$}
            \State Compute $(K - A)^+$ with respect to $F$
            \If{$(K - A)^+$ contains all attributes in $R$}
              \State $K := K - \{A\}$
            \EndIf
          \EndFor
        \end{algorithmic}
      \end{algorithm}

\hi{Properties of Relational Decompositions}
  \hii{Relation Decomposition and Insufficiency of Normal Forms}
    \par The relation schema $D = \{R_1, \ldots, R_m\}$ resulting from decomposing the relation schema $R$ is called a \tb{decomposition} of $R$.
    \par \tb{Attribute preservation condition of a decomposition}: each attribute in $R$ will appear in at least one relation schema $R_i$ in the decomposition so that no attributes are lost:
    \[
      \bigcup\limits_{i = 1}^{m} R_i = R
    \]
    \par Apart from the attribute preservation condition, another goal is to have each individual relation $R_i$ in $D$ be in BCNF or 3NF. However, this condition does not guarantee a good database design on its own.

  \hii{Dependency Preservation Property of a Decomposition}



\begin{appendices}

\end{appendices}

\end{document}
