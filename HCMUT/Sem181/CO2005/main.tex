\documentclass[9pt, landscape, a4paper]{article}
% MARGINS
\usepackage{geometry}
\geometry{top=0.75cm,bottom=0.75cm,left=1cm,right=1cm,includehead,includefoot}

% FONT
\renewcommand{\familydefault}{\sfdefault}

% LINE SPACING
% \setlength{\parskip}{6pt}

% HEADER % FOOTER
\usepackage{fancyhdr}
\pagestyle{fancy}
\fancyhead[L]{HCM City University of Technology}
\fancyhead[R]{Dept. of Computer Science \& Technology}
\fancyfoot[L]{Nhat M. Nguyen - 1752039}
\fancyfoot[R]{\thepage}

\fancypagestyle{plain}
{
\fancyhead[L]{HCM City University of Technology}
\fancyhead[R]{Dept. of Computer Science \& Technology}
\fancyfoot[L]{Nhat M. Nguyen}
\fancyfoot[C]{}
\fancyfoot[R]{\thepage}
}

% == Multi-column ==
\usepackage{multicol}
\setlength{\columnsep}{0.5cm}
\setlength{\columnseprule}{0.2pt}
\newcommand{\colbreak}{\vfill\null\columnbreak}

% == TOC ==
\usepackage{tocloft}
\renewcommand*{\ttdefault}{pcr}
\renewcommand\cftsecfont{\fontsize{8}{9}\bfseries}
\renewcommand\cftsecpagefont{\fontsize{8}{9}\mdseries}
\renewcommand\cftsubsecfont{\fontsize{5}{6}\mdseries}
\renewcommand\cftsubsecpagefont{\fontsize{5}{6}\mdseries}
\renewcommand\cftsecafterpnum{\vspace{-1ex}}
\renewcommand\cftsubsecafterpnum{\vspace{-1ex}}

% HEADINGS
\setcounter{secnumdepth}{4}
\newcommand{\hi}{\section}
\newcommand{\hii}{\subsection}
\newcommand{\hiii}{\subsubsection}
\newcommand{\hiiiBEGIN}[1]{\subsubsection \begin{enumerate}}
\newcommand{\hiiiEND}{\end{enumerate}}
\newcommand{\hiv}{\item\textbf}

% == Indentation ==
\usepackage{indentfirst}

% TEXT
% Bold, italic, underlined text
\newcommand{\tb}[1]{\textbf{#1}}
\newcommand{\ti}[1]{\textit{#1}}
\newcommand{\tbi}[1]{\textbf{\textit{#1}}}
\newcommand{\tu}[1]{\underline{#1}}
\newcommand{\tbu}[1]{\textbf{\underline{#1}}}

% FOOTNOTE
% + one footnote stays on one page
\interfootnotelinepenalty=10000
\newcommand{\FNM}{\footnotemark}
\newcommand{\SFNM}{\footnotemark[\value{footnote}]}
\newcommand{\FNT}{\footnotetext}

% IMAGES
\usepackage{graphicx}
\usepackage{subcaption}
\usepackage{float}
\newcommand{\img}[2][]
  {
    \begin{figure}[H]
      \centering
      \includegraphics[#1]{#2}
    \end{figure}
  }

% MATH
\usepackage{amsmath}
\usepackage{amssymb}
\usepackage{gensymb}
\newcommand{\sqbr}[1]{[#1]}
\newcommand{\ls}{<}
\newcommand{\gr}{>}
% equation boxes
\usepackage{empheq}
\newenvironment{eqbox}
  {\setkeys{EmphEqEnv}{align}\setkeys{EmphEqOpt}{box=\fbox}\EmphEqMainEnv}
  {\endEmphEqMainEnv}
% bold in math env
\usepackage{amsbsy}
% word above equal sign
\usepackage{mathtools}
% \newcommand\eqsign[1]{\stackrel{\mathclap{\normalfont\mbox{#1}}}{=}}
\newcommand\eqsign[1]{\mathrel{\overset{\makebox[0pt]{\mbox{\normalfont\tiny\sffamily #1}}}{=}}}
% macros
\newcommand{\vt}{\overrightarrow}
\newcommand{\avg}{\overline}
\newcommand{\ra}{\Rightarrow}
\newcommand{\lra}{\Leftrightarrow}
\newcommand{\dt}{\Delta}
\newcommand{\dif}[2]{\frac{d #1}{d #2}}
\newcommand{\pd}[2]{\frac{\partial #1}{\partial #2}}
\newcommand{\pdd}[3]{\frac{\partial #1}{\partial #2 \partial #3}}
\newcommand{\INT}{\int \limits}
\newcommand{\IINT}{\iint \limits}
\newcommand{\IIINT}{\iiint \limits}
\newcommand{\OINT}{\oint \limits}
\newcommand{\SUM}{\sum \limits}
\newcommand{\PROD}{\prod \limits}

% PSEUDOCODE
\usepackage{algpseudocode,algorithm,algorithmicx}
\usepackage{caption}

% \renewcommand{\thealgorithm}{\arabic{chapter}.\arabic{algorithm}}
\newcommand*\Let[2]{\State #1 $\gets$ #2}
\algrenewcommand\algorithmicrequire{\textbf{Precondition:}}
\algrenewcommand\algorithmicensure{\textbf{Postcondition:}}
\newcommand{\TO}{\textrm{ to }}
\newcommand{\AND}{\textrm{ and }}
\newcommand{\OR}{\textrm{ or }}
\newcommand{\LET}[2]{\Let{$#1$}{$#2$}}
\newcommand{\FOR}[1]{\For{$#1$}}
\newcommand{\ENDFOR}{\EndFor}
\newcommand{\IF}[1]{\If{$#1$}}
\newcommand{\ELSEIF}[1]{\Elseif{$#1$}}
\newcommand{\ELSE}{\Else}
\newcommand{\ENDIF}{\EndIf}
\newcommand{\WHILE}[1]{\While{$#1$}}
\newcommand{\ENDWHILE}{\EndWhile}
\newcommand{\FUNCTION}[2]{\Function{#1}{$#2$}}
\newcommand{\ENDFUNCTION}{\EndFunction}
\newcommand{\RETURN}[1]{\State \Return{$#1$}}
\newcommand{\CALLFUNC}[2]{\Call{#1}{$#2$}}
\newcommand{\CALLPROC}[2]{\Call{#1}{$#2$}}

% === CODE ===
\usepackage{listings}
\usepackage{inconsolata}
\usepackage{color}

\definecolor{dkgreen}{rgb}{0,0.6,0}
\definecolor{gray}{rgb}{0.5,0.5,0.5}
\definecolor{mauve}{rgb}{0.58,0,0.82}

\lstdefinestyle{cpp} {
  language=C++,
  aboveskip=1mm,
  belowskip=1mm,
  basicstyle=\ttfamily,
  numberstyle=\tiny\color{gray},
  commentstyle = \color{dkgreen},
  keywordstyle = \color{blue},
  stringstyle = \color{mauve},
  breaklines=true,
  postbreak=\mbox{\textcolor{red}{$\hookrightarrow$}\space},
}

\lstdefinestyle{java} {
  language=Java,
  aboveskip=1mm,
  belowskip=1mm,
  basicstyle=\ttfamily,
  numberstyle=\tiny\color{gray},
  commentstyle = \color{dkgreen},
  keywordstyle = \color{blue},
  stringstyle = \color{mauve},
  breaklines=true,
  postbreak=\mbox{\textcolor{red}{$\hookrightarrow$}\space},
}

% INFO
\title{\vspace{-4ex}\Large{OOP Notebook}}
\author{Nhat M. Nguyen - ID: 1752039}
\date{January 2019}

\begin{document}

  \maketitle
  \img[width=3cm]{logo.jpeg}
  \setcounter{tocdepth}{2}
  \begin{multicols}{2}
    \tableofcontents
  \end{multicols}
  \begin{center}
  \ti{Allowed in examination room}
  \end{center}
\clearpage
\begin{multicols}{2}
\chapter{Graph}
  \hi{Single-Source Shortest Paths}
    \hii{Dijkstra Algorithm}

  \par \ti{A lot of people can pronounce the name Van \tb{Dijk} (a football
    player) without any problem, while they would pronounce the name
    \tb{Dijk}stra incorrectly any day of the week}.

  \hiii{Introduction}

    \par The \tb{Dijkstra Algorithm} solves the single-source shortest path
      problem: Given a weighted graph and a source vertex. Find the shortest
      path from source to all other vertices in the graph.

    \par Dijkstra \tb{does not work on graph with negative weights}.

  \hiii{Notations}
    \par Given a graph $G = (V, E)$, that $\forall e = (u, v, w) \in E: w \geq
    0$, and a source vertex $s$.
    \par We denote:
    \begin{itemize}
      \item $(u, v)$ as the edge from vertex $u$ to vertex $v$.
      \item $w_{uv}$ as the weight of the edge from vertex $u$ to vertex $v$.
      \item $d[u]$ as the distance from vertex $s$ to vertex $u$.
      \item $\delta[u]$ as the \tb{minimum} distance from vertex $s$ to vertex
        $u$.
      \item $Q$ as the min-priority-queue storing the vertices keyed by the
        $d$ values. \ti{Note that all vertices are distinct}.
      \item $S$ as the set of vertices that have their min distance from source
        determined already.
    \end{itemize}

  \hiii{Pseudocode}
    \begin{algorithm}[H]
      \caption{Dijkstra}
      \begin{algorithmic}[1]
        \Function{Dijkstra}{$G$, $s$}
          \For{$u \in G$}
            \Let{$d[u]$}{$\infty$}
          \EndFor
          \State\Call{Push}{$Q$, $s$}
          \Let{$d[s]$}{$0$}
          \While{$Q \neq \emptyset$}
            \Let{$u$}{\Call{ExtractMin}{$Q$}}
            \Let{$S$}{$S \cup \set{u}$}
            \For{all vertices $v \in G.adj[u]$}
              \State\Call{Relax}{$u$, $v$, $w$}
            \EndFor
          \EndWhile
        \EndFunction

        \State

        \Function{Relax}{$u$, $v$, $w$}
          \If{$d[u] + w < d[v]$}
            \Let{$d[v]$}{$d[u] + w$}
            \If{$v \in Q$}
              \State\Call{Push}{$Q$, $v$}
            \Else
              \State\Call{DecreaseKey}{$Q$, $v$, $w$}
            \EndIf
          \EndIf
        \EndFunction
      \end{algorithmic}
    \end{algorithm}

  \hiii{Proof Of Correctness}
    \par \tb{Statement}:
    \par \fbox{
      \begin{fboxenv}
        \par For every vertices $u$ added to $S$, the distance between $u$ and
        $s$ is the shortest possible and will not be changed.
				\[
					\forall u \in S, d[u] = \delta[u]
				\]
      \end{fboxenv}
    }
    \par \tb{Proof}: By mathematical induction:
    \begin{itemize}
      \item \tb{Base case}: $|S| = 1$. This case is trivial.
      \item \tb{Inductive Step}:
        \par \ti{Inductive Hypothesis}: Suppose that the statement is true
          $\forall u \in S$ up to the previous iteration.
        \par Let $v$ be the vertex added to $S$ in the current iteration.
        \par Let $u'$ be the vertex that $v$ is relaxed from. This means that
          $u'$ has been added to $S$ in a previous iteration. According to
          the inductive hypothesis, $d[u'] = \delta[u']$. Based on the
          relaxation step, it is clear that:
          \[
            \not \exists u \in Q, u \not \equiv u':
              \delta[u] + w_{uv} < \delta[u'] + w_{u'v} = d[v]
          \]
        \par We interpret this as follow: if there exists a shorter path to
          $v$, then that path must contain at least 1 vertex not in $S$.
        \par Let $P$ be the shortest path from $s$ to $v$ that is not the
        currently explored path. This means:
				\[
          l(P) = \delta[v] < d[v] \mbox{ and } (u', v) \not \in P \eqnumber{1}
				\]
        \par $P$ can be split into two parts:
          \begin{itemize}
            \item $P_1$ which contains only vertices in $S$ and ends with vertex
              $x$.
            \item $P_2$ starts with vertex $y \not \in S$.
          \end{itemize}
        \par Because $y$ comes before $v$ on the shortest path $P$, and also
        because the graph only contains positive weights:
        \[
          \delta[y] < \delta[v] \eqnumber{2}
        \]
        \par $y$ is on the shortest path from $s$ to $v$, meaning that $P_1 +
        (x, y)$ must also be the shortest path from $s$ to $y$. At the moment,
        since $x \in S$, $d[y]$ must have been updated by relaxation from $x$.
        Therefore:
        \[
          d[y] = \delta[y] = \delta[x] + w_{xy} \eqnumber{3}
        \]
        \par On the other hand, in this iteration, $v$ is chosen from $Q$
        instead of $y$, which means:
        \[
          d[v] < d[y] \eqnumber{4}
        \]
        \par Combining the inequalities (1), (2), (3), and (4), we obtain the
        following result:
        \[
          \bm{d[v]} < d[y] = \delta[y] < \delta[v] < \bm{d[v]}
        \]
        \par By contradiction, such path $P$ does not exists. Therefore,
        $d[v] = \delta[v]$. (Q.E.D)
    \end{itemize}


\end{multicols}
\end{document}