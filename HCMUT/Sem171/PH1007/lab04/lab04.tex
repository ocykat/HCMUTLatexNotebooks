\documentclass[12pt, a4paper]{article}

% Math supporting package
\usepackage{amsmath}
\usepackage{amssymb}
\usepackage{gensymb}
\usepackage{multirow}
\usepackage{array}
\usepackage[fleqn]{mathtools}
\usepackage{relsize}

% Margins
\usepackage[margin=0.5in]{geometry}

% Indent the first paragraph after a chaper/section heading
\usepackage{indentfirst} 

% Macros
\newcommand{\mul}{\times}
\newcommand{\vt}{\overrightarrow}
\newcommand{\avg}{\overline}
\newcommand{\ra}{\Rightarrow}
\newcommand{\Dt}{\Delta}

% Formatting
\newcommand{\mRow}{\multirow}
\newcommand{\impt}[1]{\textbf{\textit{#1}}}

\newcolumntype{M}[1]{>{\centering\arraybackslash}m{#1}}
\newcolumntype{N}{@{}m{0pt}@{}}

% Headings
\newcommand{\hi}{\section}
\newcommand{\hii}{\subsection}
\newcommand{\hiiBEGIN}[1]{\subsection{#1} \begin{enumerate}}
\newcommand{\hiiEND}{\end{enumerate}}
\newcommand{\hiii}{\item\textbf}
\newcommand{\hiiiBEGIN}[1]{\item\textbf{#1} \begin{enumerate}}
\newcommand{\hiiiEND}{\end{enumerate}}
\newcommand{\hiv}{\item\textbf}

% Tables
\newcommand{\tableBEGIN}[1]{\begin{center} \begin{tabular}{#1}}
\newcommand{\tableEND}{\end{tabular} \end{center}}

% Center aligned
\setcounter{secnumdepth}{4}

\setlength{\abovedisplayskip}{0pt}
\setlength{\belowdisplayskip}{0pt}
\setlength{\abovedisplayshortskip}{0pt}
\setlength{\belowdisplayshortskip}{0pt}

% Pictures
\usepackage{graphicx}
\usepackage{float}
\graphicspath{{}}

% Math Macros
\newcommand{\SUM}[1]{\mathlarger{\sum\limits #1}}
\newcommand{\INT}[1]{\mathlarger{\int\limits #1}}

% Physics experiment macros
\newcommand{\relerr}[1]{\dfrac{\Dt #1}{\avg{#1}}}

\begin{document}
\noindent
\begin{tabular}{lll}
    \textbf{School} & : & Ho Chi Minh City University of Technology \\
    \textbf{Class} & : & PH1007 \\
    \textbf{Group} & : & 1 \\
    \textbf{Date} & : & October 21, 2017 \\
\end{tabular}\\
\rule[2ex]{\textwidth}{2pt}

\vspace{2cm}

\begin{center}
    {\scshape\Large Lab Report 04 \par}
    \vspace{1.5cm}
    {\Huge\bfseries DETERMINING THE THERMOCOUPLE CONSTANT \par}
    \vspace{3cm}

    \begin{tabular}{|c|c|c|c|}
        \hline 
        \multicolumn{4}{|c|}{\textbf{GROUP 1 -- MEMBER LIST}} \\ 
        \hline 
        \textbf{No.} &\qquad\qquad \textbf{Name}\qquad\qquad\qquad & \qquad\textbf{ID}\qquad\qquad & \qquad\textbf{Note}\qquad\qquad \\ 
        \hline 
        1 & Nguyen Minh Nhat  & 1752039 & Leader \\ 
        \hline 
        2 & Pham Hoang Minh   & 1752353 &  \\ 
        \hline 
        3 & Huynh Gia An Tien & 1752538 &  \\ 
        \hline 
        4 & Pham Minh Tuan    & 1752595 &  \\ 
        \hline 
        5 & Tran Dang Khoa    & 1752297 &  \\ 
        \hline 
    \end{tabular} 

    \vspace{3cm}

    \begin{table}[ht]
        \centering
        \begin{tabular}{|M{4cm}|M{4cm}|N|}
            \hline
            \multicolumn{2}{|c|}{\textbf{Confirmation of Instructor}} \\
            \hline
             &  \\ [50pt]
            \hline
        \end{tabular}
    \end{table}
\end{center}

\pagebreak

\hi{Objective}

\hi{Measurement Data}
    \begin{itemize}
        \item Electronic milivoltmeter: \\
            $\delta_{V\_elec} = \dfrac{5}{150} = 3.33 \%$ \\
            $U_{elec\_max} = 150 (mV)$
        \item Digital milivoltmeter: \\
            $\delta_{V\_dig} = \dfrac{0.1}{200} = 0.5 \% $ \\
            $n\alpha = \dfrac{200}{2000} = 0.1 (mV)$
        \item Digital thermometer: \\
            $\Dt{T} = 0.1 (\degree C)$
        \item $T_{2} = 27.3 \degree C$
    \end{itemize}

    \begin{center}
        \textbf{Table 1.} \\ \vspace{5mm}
        \begin{tabular}{|c|c|c|c|c|c|c|}
            \hline
                \quad No. \quad &
                \quad $T_{1} (\degree C)$ \quad &
                \quad $T_{1} - T_{2} (\degree C)$ \quad &
                \quad $\Dt (T_{1} - T_{2}) (\degree C)$ \quad &
                \quad $U_{amplified} (mV)$ \quad \\
            \hline
            1       & 95.1 & 67.8 & 0.2 & 124.7 \\
            \hline
            2       & 90.0 & 62.7 & 0.2 & 114.0 \\
            \hline
            3       & 85.0 & 57.7 & 0.2 & 106.7 \\
            \hline
            4       & 80.1 & 52.8 & 0.2 &  95.5 \\
            \hline
            5       & 75.1 & 47.8 & 0.2 &  87.0 \\
            \hline
            6       & 70.2 & 42.9 & 0.2 &  78.2 \\
            \hline
            7       & 65.1 & 37.8 & 0.2 &  67.5 \\
            \hline
            8       & 60.0 & 32.7 & 0.2 &  58.3 \\
            \hline
            9       & 55.1 & 27.8 & 0.2 &  49.6 \\
            \hline
            10      & 50.2 & 22.9 & 0.2 &  41.2 \\
            \hline
            11      & 44.9 & 17.6 & 0.2 &  31.7 \\
            \hline
        \end{tabular}
    \end{center}

\hi{Calculation}
    \hii{Values and Deviations}
    \begin{align*}
        &k = \dfrac{U_{dig}}{U_{elec}} = \dfrac{15.0}{5} = 3.0 \\
        &\Dt U_{elec} = \delta_{V\_elec} \cdot U_{elec\_max} = 3.33 \% \cdot 150 = 5 \quad (mV) \\
        &\Dt U_{dig} = \delta_{V\_dig} \cdot U_{dig\_max} + n \alpha
        = 0.5 \% \cdot 200 + 0.1 = 1.1 \quad (mV) \\
        &\dfrac{\Dt k}{k} = \dfrac{\Dt U_{dig}}{U_{dig}} + \dfrac{\Dt U_{elec}}{U_{elec}}
        = \dfrac{1.1}{200.0} + \dfrac{5}{150} = 3.883 \% \\
        &\Dt (T_{1} - T_{2}) = \Dt T_{1} + \Dt T_{2} = 2 \Dt T = 0.2 (\degree C)
    \end{align*}
    \pagebreak
    \par \textbf{\textit{Formulas for the second table:}} \\
    \begin{align*}
        &E_{thermal\_i} = \dfrac{U_{amplified\_i}}{k} = \dfrac{U_{amplified\_i}}{3}  \\
        &\Dt U_{amplified\_i} = \delta_{V\_dig} \cdot U_{amplified\_i} + n \alpha
        = 0.5 \% \cdot U_{amplified\_i} + 0.1 \\
        &\epsilon_{i} = \dfrac{\Dt E_{thermal\_i}}{E_{thermal\_i}}
        = \dfrac{\Dt{k}}{k} + \dfrac{\Dt U_{amplified\_i}}{U_{amplified}} \\
        &\Dt E_{thermal\_i} = \epsilon_{i} \cdot E_{thermal\_i} \\
    \end{align*}

    \begin{center}
        \textbf{Table 2.} \\ \vspace{5mm}
        \begin{tabular}{|c|c|c|c|c|c|c|}
            \hline
                No. &
                $T_{1} - T_{2} (\degree C)$ &
                $U_{amplified} (mV)$ &
                $\Dt U_{amplified} (mV)$ &
                $\epsilon$ &
                $E_{thermal}$ &
                $\Dt E_{thermal}$ \\
            \hline
            1       & 67.8 & 124.7 & 0.7 & 4.463 & 41.6 & 1.9 \\
            \hline
            2       & 62.7 & 114.0 & 0.7 & 4.471 & 38.0 & 1.7 \\
            \hline
            3       & 57.7 & 106.7 & 0.6 & 4.477 & 35.6 & 1.6 \\
            \hline
            4       & 52.8 &  95.5 & 0.6 & 4.488 & 31.8 & 1.4 \\
            \hline
            5       & 47.8 &  87.0 & 0.5 & 4.498 & 29.0 & 1.3 \\
            \hline
            6       & 42.9 &  78.2 & 0.5 & 4.511 & 26.1 & 1.2 \\
            \hline
            7       & 37.8 &  67.5 & 0.4 & 4.531 & 22.5 & 1.0 \\
            \hline
            8       & 32.7 &  58.3 & 0.4 & 4.555 & 19.4 & 0.9 \\
            \hline
            9       & 27.8 &  49.6 & 0.3 & 4.585 & 16.5 & 0.8 \\
            \hline
            10      & 22.9 &  41.2 & 0.3 & 4.626 & 13.7 & 0.6 \\
            \hline
            11      & 17.6 &  31.7 & 0.3 & 4.698 & 10.6 & 0.5 \\
            \hline
        \end{tabular}
    \end{center}

    \pagebreak

    \hii{Graph}
    \begin{equation*}
        E_{thermal} = C \cdot (T_{1} - T_{2})
    \end{equation*}

    \par \textbf{Scale:} $1 \degree C$ on the $x$ axis has the same length with $1 mV$ on the $y$
    axis.

            \begin{figure}[h]
                \begin{center}
                    \includegraphics{graph.png}
                \end{center}
            \end{figure}[h]

    \pagebreak

    \hiiBEGIN{Determine the thermocouple constant}
        \hiii{Method 1}
        \par In the working range of temperature, let the graph representing the function
        $E = CT$ be the segment $OA$. Then, the average $\avg{C}$ and the absolute error
        $\Dt C$ of constant $C$ are calculated as follows:
        \begin{align*}
            &\avg{C} = \dfrac{\SUM{_{i = 1}^{n} (T_{1i} - T_{2}) \cdot E_{thermal}}}{\SUM{_{i = 1}^{n} (T_{1i} - T_{2})^{2}}} = 0.607 \quad (mV/\degree C) \\
            &\xi_{thermal\_i} = C \cdot (T_{1i} - T_{2}) = 0.607 \cdot (T_{1i} - T_{2}) \\
            &S^{2} = \dfrac{1}{n - 1} \SUM{_{i = 1}^{n} (E_{thermal\_i} - \xi_{thermal\_i})}
            = 0.104 \quad (mV^{2}) \\
            &(\Dt C)^{2} = \dfrac{S^{2}}{\SUM{_{i = 1}^{n} (T_{1i} - T_{2})^{2}}} \\
            & \ra (\Dt C) = 0.002 \quad (mV/\degree C)
        \end{align*}
        \par \impt{Result: }
        \[
            \boxed{C = \avg{C} + \Dt C = 0.607 \pm 0.002 \quad (mV/\degree C)}
        \]
    
        \hiiiBEGIN{Method 2}
            \hiv{Calculation of $C$}
                \par Define $\alpha$ as the angle of the line $OA$ with respect to the horizontal
                axis. We have:
                \begin{align*}
                    \avg{C} = \tan{\alpha} = \dfrac{K_{y} \cdot AH}{K_{x} \cdot OH}
                   % = \dfrac{}{}
                \end{align*}
            \hiv{Calculation of $\Dt C$}
                \par Applying differential, we obtain:
                \begin{align*}
                    dC = d(\tan{\alpha}) = \dfrac{d\alpha}{cos^{2}\alpha} = (1 + \tan^{2}\alpha)d(\alpha)
                \end{align*}
                \par Since $C^{2} = \tan^{2}\alpha << 1$, we can deduce that:
                \begin{align*}
                    \Dt C = (1 + C^{2})\Dt \alpha \approx \Dt \alpha
                \end{align*}
                \par We can consider $\Dt \alpha$ as the angle between two lines starting from the
                origin $O$ and going through the two endpoints of the error segment $2 \Dt E$.
                $\Dt \alpha$ is approximated as follow:
                \begin{align*}
                    \Dt \alpha \approx \dfrac{2\Dt E}{OA}
                \end{align*}
%                with
                %\begin{align*}
                    %OA = \sqrt{\avg{AH}^{2} + \avg{OH}^2} = \\
                    %2 \Dt E = \\
                %\end{align*}
        \hiiiEND
            

    \hiiEND

\end{document}
