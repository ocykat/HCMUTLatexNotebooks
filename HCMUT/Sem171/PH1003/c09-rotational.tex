\chapter{Rotation of a Rigid Object About a Fixed Axis}
    \hi{Rigid Objects}
        \par A \impt{rigid object} is one that is \impt{nondeformable} - that is, the relative
        locations of all particles of which the object is composed remain constant.
        \par In dealing with a rotating object, analysis is greatly simplified by assuming that
        the object is rigid.
        \par When a rigid object rotates about a fixed axis, every particle of the object
        moves in a circle whose center is the axis of rotation.

    \hi{Angular Position, Velocity, and Acceleration}
        \hiiBEGIN{Angular Position}
            \hiii{Angular position}
            \par The \impt{angular position} of the rigid object is the angle $\theta$ between
            \impt{a reference line} on the object and a fixed reference line in space.

            \hiii{Angular displacement}:
            \begin{equation}
                \Delta \theta = \theta_{f} - \theta_{i}
            \end{equation}
        \hiiEND

        \hiiBEGIN{Angular Speed \& Velocity}
            \hiii{Average Angular Speed}
                \begin{equation}
                    \avg{\omega} = \dfrac{\Delta \theta}{\Delta t}
                \end{equation}
            \hiii{Instantaneous Angular Speed}
                \begin{equation}
                    \omega = \dfrac{d \theta}{dt}
                \end{equation}
            \hiii{Angular Velocity}
                \par Angular velocity is a vector quantity. Its magnitude is the angular speed
                and its direction is determined using the \impt{right hand rule}.
        \hiiEND

        \hiiBEGIN{Angular Acceleration}
            \hiii{Average Angular Acceleration}
                \begin{equation}
                    \avg{\alpha} = \dfrac{\Delta \omega}{\Delta t}
                \end{equation}
            \hiii{Instantaneous Angular Speed}
                \begin{equation}
                    \alpha = \dfrac{d \omega}{dt}
                \end{equation}
        \hiiEND

    \hi{Rotational Motion with Constant Angular Acceleration}
        \begin{equation}
            \omega_{f} = \omega_{i} + \alpha t
        \end{equation}
        \begin{equation}
            \theta_{f} = \theta_{i} + \omega_{i}t + \dfrac{1}{2}\alpha t^{2}
        \end{equation}
        \begin{equation}
            \omega_{f}^2 = \omega_{i}^2 + 2\alpha(\theta_{f} - \theta_{i})
        \end{equation}
        \begin{equation}
            \theta_{f} = \theta_{i} + \dfrac{1}{2}(\omega_{i} + \omega_{f})t
        \end{equation}

    \hi{Angular and Linear Quantities}
        \hii{Velocity}
            \begin{equation}
                v = \dfrac{ds}{dt} = r \dfrac{d \theta}{dt} = r \omega
            \end{equation}
            in which:
            \begin{itemize}
                \item $r$: perpendicular distance to the specified axis
            \end{itemize}

        \hiiBEGIN{Acceleration}
            \hiii{Tangential Acceleration}
                \begin{equation}
                    a_{t} = \dfrac{dv}{dt} = r \dfrac{d \omega}{dt} = r \alpha
                \end{equation}
            \hiii{Centripetal Acceleration}
                \begin{equation}
                    a_{c} = \dfrac{v^{2}}{r} = r \omega^{2}
                \end{equation}
            \hiii{Total Linear Acceleration}
                \begin{equation}
                    a = \sqrt{a_{t}^2 + a_{c}^2} = \sqrt{(r \alpha)^{2} + (r \omega^{2})^{2}} = r \sqrt{\alpha^{2} + \omega^{4}}
                \end{equation}
        \hiiEND

    \hi{Rotational Kinetic Energy}
        \par When an object is rotating, each of its particle has kinetic energy determined by its
        mass and tangential speed.
        \begin{equation}
            K_{i} = \dfrac{1}{2} m_{i}v_{i}^{2}
        \end{equation}
        \par The total kinetic energy of the object:
        \begin{equation} \label{eq:rke}
            \sum K = \SUM{_{i} K_{i}} = \SUM{\dfrac{1}{2} m_{i}v_{i}^{2}}
            = \SUM{\dfrac{1}{2} m_{i}r_{i}^{2} \omega_{i}^{2}}
            = \dfrac{1}{2} \omega_{i}^{2} \SUM{m_{i}r_{i}^{2}}
        \end{equation}
        \par The equation \eqref{eq:rke} can be simplify by defining a new quantity: the
        \impt{moment of inertia}.
        \begin{equation}
            I = \SUM{_{i} m_{i}r_{i}^{2}}
        \end{equation}
        \begin{equation}
            \sum K = \dfrac{1}{2} I \omega^{2}
        \end{equation}

    \hi{Calculation of Moments of Inertia}
        \hii{The General Way}
            \par If the extended rigid object is divided into many small volume elements:
            \begin{equation}
                I = \lim_{\Delta m_{i} \to 0} \SUM{r_{i}^{2} \Delta m_{i}}
                = \INT{r^{2} dm}
            \end{equation}
            \par We also have:
            \begin{flalign*}
                & \rho = \dfrac{m}{V} \\
                \ra & \rho = \dfrac{dm}{dV} \\
                \ra & dm = \rho dV 
            \end{flalign*}
            in which: $\rho$ is the \impt{volumetric mass density}
            \par Therefore:
            \begin{equation}
                I = \INT{\rho r^{2} dV}
            \end{equation}
        \hii{More Complicated Cases}
            \par It is relatively easy to calculate the moments of inertia of objects with:
                \begin{itemize}
                    \item Simple geometry.
                    \item The \impt{rotation axis} coincide with an \item{axis of symmetry}.
                \end{itemize}
            \par To calculate the moment of inertia about an arbitrary axis, one can use
            the \impt{parallel-axis theorem}, also known as \impt{Huygens - Steiner} theorem:
            \par \textit{Suppose the moment of inertia about an axis through the center of
            mass of an object is $I_{CM}$. The moment of inertia about any axis parallel to
            and a distance $D$ away from this axis is:}
            \begin{equation}
                I = I_{CM} + MD^{2}
            \end{equation}
            in which:
            \begin{itemize}
                \item $I$: moment of inertia about the arbitrary axis
                \item $I_{CM}$: moment of inertia about the parallel axis containing
                    the center of mass
                \item $M$: total mass of the object
                \item $D$: distance between the two axes
            \end{itemize}
                
    \hi{Torque}
        \hii{Torque - Definition \& Magnitude}
            \begin{itemize}
                \item \impt{Torque}, \impt{moment}, or \impt{moment of force} is
                rotational force.
                \item Notation: $\tau$ or $M$.
                \item The \impt{magnitude} of torque depends on three quantities:
                    \begin{itemize}
                        \item The force applied $F$
                        \item The length $r$ of the lever arm connecting the axis to
                            the point of force application $r$
                        \item The angle $\theta$ between the force vector and the lever arm
                    \end{itemize}
                \begin{equation}
                    \tau = r \times F = |r||F|sin \theta
                \end{equation}
            \end{itemize}

        \hii{Torque and Angular Acceleration}
            \par Consider a particle of mass $m$ rotating in a circle of radius $r$
            under the influence of a tangential force $F_{t}$ and a radial force
            $F_{r}$.
            \par The magnitude of torque about the center of the circle due to $F_{t}$:
            \begin{align*}
                \tau = F_{t} \mul r = ma_{t}r
            \end{align*}
            \par The magnitude of tangential acceleration:
            \begin{align*}
                a_{t} = \omega r
            \end{align*}
            \par Therefore:
            \begin{equation} \label{eq:torq_n_momentOfInertia}
                \tau = m \omega r^{2} = I \alpha
            \end{equation}

    \hi{Work, Power, and Energy in Rotational Motion}
        \hii{Power}
        \par \textit{Note that only the $F_{t}$ component does work and the $F_{c}$ component
        does not.}
        \par The work done by a force $F$ on an object as it rotates through a very small
        distance $ds = rd\theta$ is:
        \begin{align*}
            dW & = F_{t} \mul ds = (F \sin \theta) r d \theta
        \end{align*}
        \begin{equation}
            \ra dW = \tau d \theta 
        \end{equation}
        \begin{align*}
            \ra \dfrac{dW}{dt} = \tau \dfrac{d \theta}{dt}
        \end{align*}
        \begin{equation}
            \ra P = \tau \omega 
        \end{equation}

        \hii{Work - Kinetic Energy Theorem for Rotational Motion}
            \par According to the \eqref{eq:torq_n_momentOfInertia} equation:
            \begin{align*}
                \tau = I \omega \\
                \ra \tau & = I \dfrac{d\omega}{dt} \\
                & = I \dfrac{d \theta}{dt} \dfrac{d \omega}{d \theta} \\
                & = I \omega \dfrac{d \omega}{d \theta} \\
                \ra \tau d \theta & = I \omega d \omega \\
                \ra dW & = I \omega d \omega
            \end{align*}
            \begin{equation}
                \ra W = \INT{_{\omega_{i}}^{\omega_{f}} I \omega d \omega}
                = \dfrac{1}{2} I \omega_{f}^{2} - \dfrac{1}{2} I \omega_{i}^{2}
            \end{equation}
            \par This is the equation of the \impt{Work - Kinetic energy theorem for rotational
            motion}.
