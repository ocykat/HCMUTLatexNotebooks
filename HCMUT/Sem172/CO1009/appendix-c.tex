\chapter{Verilog}

\hi{Terminology}
  \begin{itemize}
    \item \tb{Hardware Registers}: circuits typically composed of flip-flops,
      often with many characteristics similar to memory, such as:
      \begin{itemize}
        \item The ability to read and write multiples bits at a time.
        \item Using an address to select a particular register in a manner
          similar to a memory address.
      \end{itemize}
    \item \tb{Register-transfer Level - RTL}: a design abstraction which
      models a synchronous digital circuit in terms of the flow of digital
      signals (data) between hardware registers, and the logical operations
      performed on those signals.
  \end{itemize}

\hi{ASIC/FPGA Design Stages}
  \begin{itemize}
    \item \tb{Specification}: Define what are the important parameters of the
      system that one is planning to design.
    \item \tb{High-level Design}: Define various blocks in the design and how
      they communicate.
    \item \tb{Low-level Design}: Describe how each block is implemented.
    \item \tb{RTL Coding}: Low-level design is converted into HDL code, using
      synthesizable constructs of the language.
    \item \tb{Simulation}: Verifying the functional characteristics of models at
      any level of abstraction.
    \item \tb{Synthesis}: The abstract form of circuit design is turned into a
      design implementation in terms of logic gates, typically by a computer
      program called a \ti{synthesis tool}.
    \item \tb{Place and Route}
    \item \tb{Post Silicon Validation}

  \end{itemize}

\hi{Module}
  \par The module name must match the file name. For example, if the module name
    is ``module\textunderscore name", then the file name must be
    ``module\textunderscore name.v".
\begin{lstlisting}[language=Verilog]
  module module_name()
  endmodule
\end{lstlisting}

\hi{Output and Input Ports}
  \par A module takes ports as parameters.
  \par Ports are then specified as ``output", ``input", or ``inout" in the
    module.
  \par The order of parameters is not important. However, a good convention is
    to include the output ports first, then the input ports.
\begin{lstlisting}[language=Verilog]
  module module_name(out_port, in_port_1, in_port_2);
    output out_port;
    input in_port_1, in_port_2;
  endmodule
\end{lstlisting}

\hi{Vectors}
  \par Verilog allows vectors for input and output bus.
\begin{lstlisting}[language=Verilog]
  module module_name(out, in);
    output [3:0] out; // 4 bits bus
    input  [7:0] in;  // 8 bits bus
  endmodule
\end{lstlisting}
  \par It is a convention that vector of ports is indexed from right to left,
    starting from the LSB and ending with the MSB.

\hi{Reg and Wire}
  \par In Verilog, ``\tb{reg}" and ``\tb{wire}" are driver types.
  \par In simple words, in a physical circuit, a driver would be anything that
    electrons can move through/into
  \begin{itemize}
    \item A ``reg" is used to store a value.
    \item A ``wire" is used to connect two points in the circuit.
  \end{itemize}

\hi{Gate-Level Modeling}
  \par Available gate primitives:
  \begin{itemize}
    \item and
    \item nand
    \item or
    \item nor
    \item xor
    \item xnor
    \item not
  \end{itemize}
  \par A gate can take any arbitrary number inputs, but only returns 1 output.
  \par Syntax:
\begin{lstlisting}
  <gate primitive> <gate_name(optional)>(output, input_1, input_2, ...);
\end{lstlisting}

\hi{Logic Values and Signal Strength}
  \begin{itemize}
    \item 0: zero, low, false
    \item 1: one, high, true
    \item z or Z: high impedance, floating
    \item x or X: unknown, uninitialized, contention, don't-care
  \end{itemize}

\hi{Example 1: Gate-Level Modeling - Majority Circuit}

\begin{lstlisting}[language=Verilog]
module majority(
    out,
    a, b, c
);
    output out;
    input  a, b, c;
    
    wire w1, w2, w3;

    and and_1(w1, a, b);
    and and_2(w2, b, c);
    and and_3(w3, c, a);
    or  or_1(out, w1, w2, w3);

endmodule
\end{lstlisting}

\chapter{Adder Circuits}

\hi{Half-Adder Circuit}
  \hii{Truth table}
    \begin{tabular}{|c|c|c|c|}
      \hline 
      IN1 & IN2 & SUM & CARRY \\ 
      \hline 
      0 & 0 & 0 & 0 \\ 
      \hline 
      0 & 1 & 1 & 0 \\ 
      \hline 
      1 & 0 & 1 & 0 \\ 
      \hline 
      1 & 1 & 0 & 1 \\ 
      \hline 
    \end{tabular} 


\hi{Full-Adder Circuit}
  \hii{Truth table}
  \begin{tabular}{|c|c|c|c|c|}
      \hline 
      CARRY-IN & IN1 & IN2 & SUM & CARRY-OUT \\ 
      \hline 
      0 & 0 & 0 & 0 & 0 \\ 
      \hline 
      0 & 0 & 1 & 1 & 0 \\ 
      \hline 
      0 & 1 & 0 & 1 & 0 \\ 
      \hline 
      0 & 1 & 1 & 0 & 1 \\ 
      \hline 
      1 & 0 & 0 & 1 & 0 \\ 
      \hline 
      1 & 0 & 1 & 0 & 1 \\ 
      \hline 
      1 & 1 & 0 & 0 & 1 \\ 
      \hline 
      1 & 1 & 1 & 1 & 1 \\ 
      \hline 
  \end{tabular} 

