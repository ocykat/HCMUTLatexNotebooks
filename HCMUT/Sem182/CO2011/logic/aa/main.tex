\documentclass[12pt, a4paper]{report}


% === MARGINS ===
\usepackage[margin=0.75in]{geometry}


% === HEADINGS ===
\setcounter{secnumdepth}{4}
\newcommand{\hi}{\section}
\newcommand{\hii}{\subsection}
\newcommand{\hiii}{\subsubsection}
\newcommand{\hiiiBEGIN}[1]{\subsubsection \begin{enumerate}}
\newcommand{\hiiiEND}{\end{enumerate}}
\newcommand{\hiv}{\item\textbf}


% === INDENTATION ===
\usepackage{indentfirst}


% === TABLE OF CONTENTS ===
% links
\usepackage{hyperref}
\hypersetup{
  colorlinks,
  citecolor=black,
  filecolor=black,
  linkcolor=black,
  urlcolor=black
}


% === APPENDICES ===
\usepackage[toc, page]{appendix}


% === TEXT ===
% Bold, italic, underlined text
\newcommand{\tb}[1]{\textbf{#1}}
\newcommand{\ti}[1]{\textit{#1}}
\newcommand{\tbi}[1]{\textbf{\textit{#1}}}
\newcommand{\tu}[1]{\underline{#1}}
\newcommand{\tbu}[1]{\textbf{\underline{#1}}}
\newcommand{\smf}[1]{\small #1 \normalsize}
\newenvironment{smfont}{\small}{\normalsize}


% === ENUMERATE ===
\usepackage[shortlabels]{enumitem}
% Usage: \begin{enumerate}[a.] or [a)] or [(A)] etc.


% === MATH ===
% Basic packages
\usepackage{amsmath}
\usepackage{amssymb}
\usepackage{gensymb}

% Equation box
\usepackage{empheq}
\newenvironment{eqbox}
  {\setkeys{EmphEqEnv}{align}\setkeys{EmphEqOpt}{box=\fbox}\EmphEqMainEnv}
  {\endEmphEqMainEnv}

% Arrows
\newcommand{\ra}{\Rightarrow}
\newcommand{\lra}{\Leftrightarrow}

% Sum, product, limit
\newcommand{\SUM}[1]{\sum\limits #1}
\newcommand{\PROD}[1]{\prod\limits #1}

% Bold in math mode
\usepackage{bm}

% Absolute value
\DeclarePairedDelimiter\abs{\lvert}{\rvert}%
\DeclarePairedDelimiter\norm{\lVert}{\rVert}%
%   Swap the definition of \abs* and \norm*, so that \abs
%   and \norm resizes the size of the brackets, and the
%   starred version does not.
\makeatletter
\let\oldabs\abs
\def\abs{\@ifstar{\oldabs}{\oldabs*}}
\let\oldnorm\norm
\def\norm{\@ifstar{\oldnorm}{\oldnorm*}}
\makeatother

% Floor and ceiling
\usepackage{mathtools}
\DeclarePairedDelimiter\ceil{\lceil}{\rceil}
\DeclarePairedDelimiter\floor{\lfloor}{\rfloor}

% Inner Product
\newcommand{\iprod}[1]{\langle #1 \rangle}

% Derivatives
\newcommand{\dif}[2]{\dfrac{d #1}{d #2}}
\newcommand{\ddif}[2]{\dfrac{d^2 #1}{d #2^2}}
\newcommand{\difi}[2]{d #1/d #2}
\newcommand{\diff}[2]{\dfrac{d}{d #2} #1}
\newcommand{\pd}[2]{\dfrac{\partial #1}{\partial #2}}
\newcommand{\pdds}[2]{\dfrac{\partial^{2} #1}{\partial #2^{2}}}
\newcommand{\pddsi}[2]{{\partial^{2} #1} / {\partial #2^{2}}}
\newcommand{\pdd}[3]{\dfrac{\partial^{2} #1}{\partial #2 \partial #3}}
\newcommand{\pddi}[3]{{\partial^{2} #1} / {\partial #2 \partial #3}}

% Integrals
\usepackage{esint}
\newcommand{\INT}{\int \limits}
\newcommand{\OINT}{\oint \limits}
\newcommand{\IINT}{\iint \limits}
\newcommand{\IIINT}{\iiint \limits}

% Ratios
\newcommand{\ratio}[3]{\dfrac{#1_{#2}}{#1_{#3}}}

% Logical operators
\newcommand{\lnotnot}{\lnot \lnot}
\newcommand{\lxor}{\oplus}
\newcommand{\limpl}{\rightarrow}
\newcommand{\liff}{\leftrightarrow}
\newcommand{\lT}{\mathbf{T}}
\newcommand{\lF}{\mathbf{F}}
\newcommand{\lyeild}{\vdash}
\newcommand{\deduce}{\Rightarrow}
\newcommand{\lcontrad}{\perp}
\newcommand{\lprove}{\vdash}
\newcommand{\lproverev}{\dashv}
\newcommand{\lproveeq}{\vdash \dashv}

% Logic related
\usepackage{bussproofs}


% === FOOTNOTE ===
%   one foot note stays on one page
\interfootnotelinepenalty=10000

\newcommand{\fnmark}{\footnotemark}
\newcommand{\fnmarksame}{\footnotemark[\value{footnote}]}
\newcommand{\fntext}{\footnotetext}


% === IMAGES ===
\usepackage{graphicx}
\usepackage{caption}
\usepackage{subcaption}
\usepackage{float}
\newcommand{\img}[3][]
{
  \begin{figure}[H]
    \centering
    \includegraphics[#1]{#2}
    \caption*{#3}
  \end{figure}
}
% Usage: \img[width=...]{<path_to_img>}{<caption>}


% === PSEUDOCODE ===
\usepackage{algorithm}
\usepackage{algorithmicx}
\usepackage[noend]{algpseudocode}
\usepackage{caption}

\renewcommand{\thealgorithm}{\arabic{chapter}.\arabic{algorithm}}

\newcommand*\Let[2]{\State #1 $\gets$ #2}
\algrenewcommand\algorithmicrequire{\textbf{Input:}}
\algrenewcommand\algorithmicensure{\textbf{Output:}}
\newcommand{\INPUT}[1]{\Require{#1} \Statex}
\newcommand{\OUTPUT}[1]{\Ensure{#1} \Statex}
\newcommand{\INPUTOUTPUT}[2]{\Require{#1} \Ensure{#2} \Statex}
\newcommand{\LET}[2]{\Let{$#1$}{$#2$}}
\newcommand{\FOR}[2]{\For{$#1 \gets #2$}}
\newcommand{\ENDFOR}{\EndFor}
\newcommand{\TO}{\textrm{ \tb{to} }}
\newcommand{\DOWNTO}{\textrm{ \tb{downto} }}
\newcommand{\AND}{\textrm{ \tb{and} }}
\newcommand{\OR}{\textrm{ \tb{or} }}
\newcommand{\XOR}{\textrm{ \tb{xor} }}
\newcommand{\GETS}{\gets}
\newcommand{\IF}[1]{\If{$#1$}}
\newcommand{\ELSEIF}[1]{\ElsIf{$#1$}}
\newcommand{\ELSE}{\Else}
\newcommand{\ENDIF}{\EndIf}
\newcommand{\WHILE}[1]{\While{$#1$}}
\newcommand{\ENDWHILE}{\EndWhile}
\newcommand{\FUNCTION}[2]{\Function{#1}{$#2$}}
\newcommand{\ENDFUNCTION}{\EndFunction}
\newcommand{\PROCEDURE}[2]{\Procedure{#1}{$#2$}}
\newcommand{\ENDPROCEDURE}{\EndProcedure}
\newcommand{\CALLFUNC}[2]{\State \Call{#1}{$#2$}}
\newcommand{\CALLPROC}[2]{\State \Call{#1}{$#2$}}
\newcommand{\RETURN}[1]{\State \Return{#1}}

% === TABLE ===
\usepackage{multirow}

% === CODE ===
\usepackage{listings}
\usepackage{inconsolata}
\lstset
{
  basicstyle=\ttfamily,
}


% === BOOK SECTION MARKER ===
\newcommand{\booktitle}{}
\newcommand{\booksection}[1]{\smf{\ti{Section #1 - \booktitle}}}


\begin{document}

\chapter{Graph}
  \hi{Single-Source Shortest Paths}
    \hii{Dijkstra Algorithm}

  \par \ti{A lot of people can pronounce the name Van \tb{Dijk} (a football
    player) without any problem, while they would pronounce the name
    \tb{Dijk}stra incorrectly any day of the week}.

  \hiii{Introduction}

    \par The \tb{Dijkstra Algorithm} solves the single-source shortest path
      problem: Given a weighted graph and a source vertex. Find the shortest
      path from source to all other vertices in the graph.

    \par Dijkstra \tb{does not work on graph with negative weights}.

  \hiii{Notations}
    \par Given a graph $G = (V, E)$, that $\forall e = (u, v, w) \in E: w \geq
    0$, and a source vertex $s$.
    \par We denote:
    \begin{itemize}
      \item $(u, v)$ as the edge from vertex $u$ to vertex $v$.
      \item $w_{uv}$ as the weight of the edge from vertex $u$ to vertex $v$.
      \item $d[u]$ as the distance from vertex $s$ to vertex $u$.
      \item $\delta[u]$ as the \tb{minimum} distance from vertex $s$ to vertex
        $u$.
      \item $Q$ as the min-priority-queue storing the vertices keyed by the
        $d$ values. \ti{Note that all vertices are distinct}.
      \item $S$ as the set of vertices that have their min distance from source
        determined already.
    \end{itemize}

  \hiii{Pseudocode}
    \begin{algorithm}[H]
      \caption{Dijkstra}
      \begin{algorithmic}[1]
        \Function{Dijkstra}{$G$, $s$}
          \For{$u \in G$}
            \Let{$d[u]$}{$\infty$}
          \EndFor
          \State\Call{Push}{$Q$, $s$}
          \Let{$d[s]$}{$0$}
          \While{$Q \neq \emptyset$}
            \Let{$u$}{\Call{ExtractMin}{$Q$}}
            \Let{$S$}{$S \cup \set{u}$}
            \For{all vertices $v \in G.adj[u]$}
              \State\Call{Relax}{$u$, $v$, $w$}
            \EndFor
          \EndWhile
        \EndFunction

        \State

        \Function{Relax}{$u$, $v$, $w$}
          \If{$d[u] + w < d[v]$}
            \Let{$d[v]$}{$d[u] + w$}
            \If{$v \in Q$}
              \State\Call{Push}{$Q$, $v$}
            \Else
              \State\Call{DecreaseKey}{$Q$, $v$, $w$}
            \EndIf
          \EndIf
        \EndFunction
      \end{algorithmic}
    \end{algorithm}

  \hiii{Proof Of Correctness}
    \par \tb{Statement}:
    \par \fbox{
      \begin{fboxenv}
        \par For every vertices $u$ added to $S$, the distance between $u$ and
        $s$ is the shortest possible and will not be changed.
				\[
					\forall u \in S, d[u] = \delta[u]
				\]
      \end{fboxenv}
    }
    \par \tb{Proof}: By mathematical induction:
    \begin{itemize}
      \item \tb{Base case}: $|S| = 1$. This case is trivial.
      \item \tb{Inductive Step}:
        \par \ti{Inductive Hypothesis}: Suppose that the statement is true
          $\forall u \in S$ up to the previous iteration.
        \par Let $v$ be the vertex added to $S$ in the current iteration.
        \par Let $u'$ be the vertex that $v$ is relaxed from. This means that
          $u'$ has been added to $S$ in a previous iteration. According to
          the inductive hypothesis, $d[u'] = \delta[u']$. Based on the
          relaxation step, it is clear that:
          \[
            \not \exists u \in Q, u \not \equiv u':
              \delta[u] + w_{uv} < \delta[u'] + w_{u'v} = d[v]
          \]
        \par We interpret this as follow: if there exists a shorter path to
          $v$, then that path must contain at least 1 vertex not in $S$.
        \par Let $P$ be the shortest path from $s$ to $v$ that is not the
        currently explored path. This means:
				\[
          l(P) = \delta[v] < d[v] \mbox{ and } (u', v) \not \in P \eqnumber{1}
				\]
        \par $P$ can be split into two parts:
          \begin{itemize}
            \item $P_1$ which contains only vertices in $S$ and ends with vertex
              $x$.
            \item $P_2$ starts with vertex $y \not \in S$.
          \end{itemize}
        \par Because $y$ comes before $v$ on the shortest path $P$, and also
        because the graph only contains positive weights:
        \[
          \delta[y] < \delta[v] \eqnumber{2}
        \]
        \par $y$ is on the shortest path from $s$ to $v$, meaning that $P_1 +
        (x, y)$ must also be the shortest path from $s$ to $y$. At the moment,
        since $x \in S$, $d[y]$ must have been updated by relaxation from $x$.
        Therefore:
        \[
          d[y] = \delta[y] = \delta[x] + w_{xy} \eqnumber{3}
        \]
        \par On the other hand, in this iteration, $v$ is chosen from $Q$
        instead of $y$, which means:
        \[
          d[v] < d[y] \eqnumber{4}
        \]
        \par Combining the inequalities (1), (2), (3), and (4), we obtain the
        following result:
        \[
          \bm{d[v]} < d[y] = \delta[y] < \delta[v] < \bm{d[v]}
        \]
        \par By contradiction, such path $P$ does not exists. Therefore,
        $d[v] = \delta[v]$. (Q.E.D)
    \end{itemize}



\end{document}
