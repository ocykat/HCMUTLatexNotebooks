\begin{multicols}{2}
\hi{Computer Abstraction}

  \hii{Terminology}
    \begin{itemize}
      \item \tb{execution time/response time}: the time between the start and completion of a task
      \item \tb{throughput/bandwidth}: the total amount of work done in a given
        time
    \end{itemize}

  \hii{Performance}
    \par Performance is defined as:
    \begin{eqbox}
      \text{Performance}_x = \frac{1}{\text{Execution time}_x}
    \end{eqbox}
    \par Relative performance:
    \begin{eqbox}
      \ratio{\text{Performance}}{a}{b} = \ratio{\text{Execution time}}{b}{a} 
    \end{eqbox}

  \hii{CPU execution time}
    \begin{itemize}
      \item \tb{CPU execution time/CPU time}: the time the CPU spends computing
        for one task and does not include time spent waiting for I/O or running
        other programs.
      \item \tb{Clock cycle}: clock period
      \item \tb{Clock rate}: clock frequency
    \end{itemize}
    
    \begin{eqbox}
      \text{CPU time} = \text{\# Clock cycles} \times \text{Clock cycle} \\
      t = n_{cycles} \times T
    \end{eqbox}

    \begin{eqbox}
      \text{CPU time} = \frac{\text{\# Clock cycles}}{\text{Clock cycle}} \\
      t = \frac{n_{cycles}}{f}
    \end{eqbox}

  \hii{Number of Cycles for a program}
    \begin{eqbox}
      \text{\# Clock cycles}
      = \text{\# Instructions} \times \text{(average) \# Cycles per instruction}
    \end{eqbox}
    \begin{eqbox}
      \text{\# Clock cycles}
      = \text{Instruction count} \times \text{CPI}
    \end{eqbox}
    \par Composed formula: 
    \begin{eqbox}
      \text{CPU time}
      = \frac{\text{Instruction count} \times \text{CPI}}{\text{Clock rate}}
    \end{eqbox}

\end{multicols}
\noindent\makebox[\linewidth]{\rule{\paperwidth}{0.4pt}}