\chapter{Geometric Intuition}

\hi{Linear Dependence and Linear Independence}
  \par $n$ vectors form a linear independent set if they can span the
    $n$-dimensional space.
  \par $n$ vectors form a linear dependent set if the cannot span the
    $n$-dimensional space.

\hi{Matrix Multiplication and Linear Transformation}
  \hii{Transformation}
    \par Here, the word ``\tb{transformation}" means ``\tb{function}";
    \par The \tb{transformation} of a vector is the ``movement" of that
      vector to a new coordinate.
  \hii{Linear Transformation}
    \par Visually, a transformation is \tb{linear} if it has 2 properties:
    \begin{itemize}
      \item All lines must remain lines without getting curved.
      \item The origin remains fixed.
    \end{itemize}
  \hii{Describe Linear Transformation Numerically}
    \par To create a linear transformation, we just need to keep track of
      the \tb{unit vectors}.
    \par \tb{Property}:
    \par Let $v$ be a vector and $v'$ is the result of the linear
    transformation of $v$.
    \par If we call the unit vectors along the $x$- and $y$-axes $i$ and $j$,
    respectively, and their result of linear transformation $i'$ and $j'$,
    then
    \begin{equation}
      v = ai + bj \Rightarrow v' = ai' + bj'
    \end{equation}
