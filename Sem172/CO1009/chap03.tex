\chapter{Describing Logic Circuits}

\hi{Boolean Constants and Variables}

\hi{Truth Table}

\hi{OR operation and OR gates}
    \hii{Notation}
        \begin{eqbox}
            A + B
        \end{eqbox}
    \hii{Truth table}
        \begin{center}
            \begin{tabular}{|c|c|c|}
                \hline
                A & B & A + B \\
                \hline
                0 & 0 & 0 \\
                \hline
                0 & 1 & 1 \\
                \hline
                1 & 0 & 1 \\
                \hline
                1 & 1 & 1 \\
                \hline
            \end{tabular}
        \end{center}
    \hii{OR gate}
        \begin{tikzpicture}
            \node (A) at (0, 2) {$A$};
            \node (B) at (0, 0) {$B$};

            \node[OR, logic gate inputs=nn] at (2.5, 1) (or) {};

            \draw (A) -| ($(A) + (1, 0)$) |- (or.input 1);
            \draw (B) -| ($(B) + (1, 0)$) |- (or.input 2);
            \draw (or.output) -- ($(or.output) + (2, 0)$);
        \end{tikzpicture}

\hi{AND operation and AND gates}
    \hii{Notation}
        \begin{eqbox}
            A \cdot B
        \end{eqbox}
    \hii{Truth table}
        \begin{center}
            \begin{tabular}{|c|c|c|}
                \hline
                A & B & A $\cdot$ B \\
                \hline
                0 & 0 & 0 \\
                \hline
                0 & 1 & 0 \\
                \hline
                1 & 0 & 0 \\
                \hline
                1 & 1 & 1 \\
                \hline
            \end{tabular}
        \end{center}
    \hii{AND gate}
        \begin{tikzpicture}
            \node (A) at (0, 2) {$A$};
            \node (B) at (0, 0) {$B$};

            \node[AND, logic gate inputs=nn] at (2.5, 1) (and) {};

            \draw (A) -| ($(A) + (1, 0)$) |- (and.input 1);
            \draw (B) -| ($(B) + (1, 0)$) |- (and.input 2);
            \draw (and.output) -- ($(and.output) + (2, 0)$);
        \end{tikzpicture}

\hi{NOT operation}
    \hii{Notation}
        \begin{eqbox}
            \bar{A} \mbox{ or } A'
        \end{eqbox}
    \hii{NOT gate}
        \begin{tikzpicture}
            \node (A) at (0, 0) {$A$};
            \node[NOT, logic gate inputs=n] at (2, 0) (not) {};
            \draw (A) -- (not.input);
            \draw (not.output) -- ($(A) + (4, 0)$);
        \end{tikzpicture}

\hi{Describing Logic Circuits Algebraically}
    \hii{Operator Precedence}
        \begin{itemize}
            \item parentheses
            \item AND operator
            \item OR operator
        \end{itemize}

\hi{Implementing Circuits From Boolean Expressions}

\hi{Laws of Logic}
    \hii{Identity Laws}
        \begin{align*}
            x + 0 = x \\
            x \cdot 1 = x
        \end{align*}
    \hii{Domination Laws}
        \begin{align*}
            x + 1 = 1 \\
            x \cdot 0 = 0
        \end{align*}
    \hii{Idempotent Laws}
        \begin{align*}
            x + x = x \\
            x \cdot x = x
        \end{align*}
    \hii{Double Negation Law}
        \begin{align*}
            \bar{\bar{x}} = x
        \end{align*}
    \hii{Negation Laws}
        \begin{align*}
            x + \bar{x} &= 1 \\
            x \cdot \bar{x} &= 0
        \end{align*}
    \hii{Commutative Laws}
        \begin{align*}
            x + y &= y + x \\
            x \cdot y &= y \cdot x
        \end{align*}
    \hii{Associative Laws}
        \begin{align*}
            x + y + z &= x + (y + z) \\
            x \cdot y \cdot z &= x \cdot (y \cdot z)
        \end{align*}
    \hii{Absorption Laws}
        \begin{align*}
            x + (x \cdot y) &= x \\
            x \cdot (x + y) &= x
        \end{align*}
    \hii{Distributed Laws}
        \begin{align*}
            x \cdot (y + z) &= xy + xz \\
            x + (y \cdot z) &= (x + y) \cdot (x + z)
        \end{align*}
    \hii{Unnamed Laws}
        \begin{eqbox}
            x + \bar{x} \cdot y &= x + y \\
            x \cdot (\bar{x} + y) \cdot y &= x \cdot y
        \end{eqbox}
    \hii{De Morgan's Laws}
        \begin{eqbox}
            \overline{x + y} &= \bar{x} \cdot \bar{y} \\
            \overline{x \cdot y} &= \bar{x} + \bar{y}
        \end{eqbox}
    \hii{Consensus Laws}
        \begin{eqbox}
            xy + \bar{x} z + yz &= xy + \bar{x} z \\
            (x + y)(\bar{x} + z)(y + z) + yz &= (x + y)(\bar{x} + z)
        \end{eqbox}
        \textbf{Proof:}
        \begin{flalign*}
            & xy + \bar{x} z + yz \\
            & = xy + \bar{x} z + (x + \bar{x}) yz & \mbox{Negation Laws} & \\
            & = xy + \bar{x} z + xyz + \bar{x} yz & \mbox{Distributed Laws} & \\
            & = xy(1 + z) + \bar{x}z(1 + y)       & \mbox{Associative Laws} &\\
            & = xy + \bar{x}z                     & \mbox{Domination Laws} &
        \end{flalign*}

\hi{NOR gates and NAND gates}
    \hii{NOR gates}
        \begin{tikzpicture}
            \node (A) at (0, 2) {$A$};
            \node (B) at (0, 0) {$B$};

            \node[NOR, logic gate inputs=nn] at (2.5, 1) (nor) {};

            \draw (A) -| ($(A) + (1, 0)$) |- (nor.input 1);
            \draw (B) -| ($(B) + (1, 0)$) |- (nor.input 2);
            \draw (nor.output) -- ($(A) + (4, -1)$);
        \end{tikzpicture}

    \hii{NAND gates}
        \begin{tikzpicture}
            \node (A) at (0, 2) {$A$};
            \node (B) at (0, 0) {$B$};

            \node[NAND, logic gate inputs=nn] at (2.5, 1) (nand) {};

            \draw (A) -| ($(A) + (1, 0)$) |- (nand.input 1);
            \draw (B) -| ($(B) + (1, 0)$) |- (nand.input 2);
            \draw (nand.output) -- ($(A) + (4, -1)$);
        \end{tikzpicture}

\hi{Universality of NAND gates and NOR gates}
    \par All boolean expressions consist of various combinations of the basic operations of OR, AND
    and INVERT.
    \par NAND and NOR gates can be used to perform all basic operations.

\hi{Alternate Logic-gate Representation}
    \hii{Conversion Procedure}
        \par The alternate symbol for each gate is obtained from the standard symbol by doing the
        following:
        \begin{enumerate}
            \item Invert each input and output of the standard symbol, by adding or removing bubbles.
            \item Change the operation symbol from AND to OR or from OR to AND.
                (INVERTER stays unchanged).
        \end{enumerate}
    \hii{Notes}
        \begin{itemize}
            \item The equivalences can be extended to gates with \textbf{any} number of inputs.
            \item None of the standard symbols have bubbles on their inputs, and all the alternate
                symbols do.
            \item The standard and alternate symbols for each gate represent the same physical
                circuit.
            \item NAND and NOR gates are inverting gates, so both the standard and the alternate
                symbols for each will have a bubble on \textbf{either} the input or the output.
            \item AND and OR gates are noninverting gates, so the alternate symbols for each will
                have bubbles on \textbf{both} inputs and output.
        \end{itemize}
    \hii{Logic-Symbol Interpretation}
        \par To interpret the logic-gate operation:
        \begin{enumerate}
            \item Note which logic state is the active state for the input.
            \item Note which logic state is the active state for the output.
            \item There are two cases:
                \begin{itemize}
                    \item AND gate: Output goes ... when \textbf{all} input are ...
                    \item OR gate: Output goes ... when \text{any} input is ...
                \end{itemize}
        \end{enumerate}

\hi{Notes on Gate Representation}
    \hii{Standard vs Alternate}
        \par If properly used, alternate logic gates make the circuit much clearer than standard
        logic gates.
    \hii{Bubble Placement}
        \par Whenever possible, bhoose gate symbols so that bubble outputs are connected to bubble
        inputs, and nonbubble outputs to nonbubble inputs.
    \hii{Asserted Levels}
        \par When a logic signal is in its active state, it is said to be \textbf{asserted}.
        \par When a logic signal is in its inactive state, it is said to be \textbf{unasserted}.

