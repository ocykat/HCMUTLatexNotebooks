\documentclass[12pt, a4paper]{article}

% Math supporting package
\usepackage{amsmath}
\usepackage{gensymb}
\usepackage{relsize}

% Margins
\usepackage[margin=0.75in]{geometry}

% Indent the first paragraph after a chaper/section heading
\usepackage{indentfirst} 

% Macros
\newcommand{\mul}{\cdot}
\newcommand{\vt}{\overrightarrow}
\newcommand{\avg}{\overline}
\newcommand{\ra}{\Rightarrow}
\newcommand{\SUM}[1]{\mathlarger{\sum\limits #1}}
\newcommand{\INT}[1]{\mathlarger{\int\limits #1}}

% Formatting
\newcommand{\mRow}{\multirow}
\newcommand{\impt}[1]{\textbf{\textit{#1}}}

% Headings
\newcommand{\hi}{\section}
\newcommand{\hii}{\subsection}
\newcommand{\hiiBEGIN}[1]{\subsection{#1} \begin{enumerate}}
\newcommand{\hiiEND}{\end{enumerate}}
\newcommand{\hiii}{\item\textbf}
\newcommand{\hiiiBEGIN}[1]{\item\textbf{#1} \begin{enumerate}}
\newcommand{\hiiiEND}{\end{enumerate}}
\newcommand{\hiv}{\item\textbf}

% Tables
\newcommand{\tableBEGIN}[1]{\begin{center} \begin{tabular}{#1}}
\newcommand{\tableEND}{\end{tabular} \end{center}}

% Center aligned
\newenvironment{nscenter}
    {\parskip=0pt\par\nopagebreak\centering}
    {\par\noindent\ignorespacesafterend}
\newcommand{\CENTER}[1]{\begin{center} #1 \end{center}}
\newcommand{\NSCENTER}[1]{\begin{nscenter} #1 \end{nscenter}}

% Horizontal line for rule of inference
\newcommand{\roiLine}[1]{\overline{\quad #1 \quad}}

% Set section numbering
\setcounter{secnumdepth}{4}

\begin{document}

\hi{Average}
    \par The \impt{average}, or \impt{mean} of $n$ independent measurements:
    \begin{equation}
        \avg{x} = \dfrac{\SUM{_{i}^{n} x_{i}}}{n}
    \end{equation}

\hi{Average Deviation}
    \par \impt{Average deviation} tells on average (with ~50\% confidence) how
    much the individual measurements vary from the mean.
    \begin{equation}
        \avg{d} = \dfrac{\SUM{_{i}^{n} |x_{i} - \avg{x}|}}{n}
    \end{equation}

\hi{Standard Deviation}
    \par \impt{Standard deviation} is the most common way to characterize
    the spread of a data set. The standard deviation is always slightly greater
    than the average deviation.
    \par Steps to find the standard deviation:
    \begin{itemize}
        \item Find the \impt{average}.
        \item Subtract the average from the $n$ measurements to obtain
            \impt{$n$ deviations}.
        \item Square each deviation and add all up.
        \item Divide the result by $(n - 1)$ and take the square root.
    \end{itemize}
    \par Let $\delta x_{i} = x_{i} - \avg{x}$. The following equation is
    obtained:
    \begin{equation}
        s = \sqrt{\dfrac{\SUM{_{i}^{n} \delta x_{i}^{2}}}{n - 1}}
    \end{equation}

\end{document}
