\documentclass[9pt, landscape, a4paper]{article}

% === MARGINS ===
\usepackage{geometry}
\geometry{top=0.75cm,bottom=0.75cm,left=1cm,right=1cm,includehead,includefoot}

% === FONT ===
% \renewcommand{\familydefault}{\sfdefault}

% LINE SPACING
% \setlength{\parskip}{6pt}

% === HEADER % FOOTER ===
\usepackage{fancyhdr}
\pagestyle{fancy}
\fancyfoot[L]{HCM City University of Technology}
\fancyhead[R]{Chapter \thesection}
\fancyhead[L]{Probability \& Statistics}
% \fancyfoot[R]{Dept. of Computer Science \& Technology}
% \fancyfoot[R]{\thepage}

% \fancypagestyle{plain}
% {
% \fancyhf{} % clear all header and footer fields
% \fancyhead[L]{HCM City University of Technology}
% \fancyhead[R]{Dept. of Computer Science \& Technology}
% \fancyfoot[L]{Nhat M. Nguyen}
% % \fancyfoot[C]{}
% % \fancyfoot[R]{\thepage}
% }

% === MULTICOLUMN ===
\usepackage{multicol}
\setlength{\columnsep}{0.5cm}
\setlength{\columnseprule}{0.2pt}
\newcommand{\colbreak}{\vfill\null\columnbreak}

% === TOC ===
\usepackage{tocloft}
% \renewcommand*{\ttdefault}{pcr}
% \renewcommand\cftsecfont{\fontsize{8}{9}\bfseries}
% \renewcommand\cftsecpagefont{\fontsize{8}{9}\mdseries}
% \renewcommand\cftsubsecfont{\fontsize{5}{6}\mdseries}
% \renewcommand\cftsubsecpagefont{\fontsize{5}{6}\mdseries}
% \renewcommand\cftsecafterpnum{\vspace{-1ex}}
% \renewcommand\cftsubsecafterpnum{\vspace{-1ex}}

% === HEADINGS ===
\setcounter{secnumdepth}{4}
\newcommand{\hi}{\section}
\newcommand{\hii}{\subsection}
\newcommand{\hiii}{\subsubsection}
\newcommand{\hiiiBEGIN}[1]{\subsubsection \begin{enumerate}}
\newcommand{\hiiiEND}{\end{enumerate}}
\newcommand{\hiv}{\item\textbf}

% === INDENTATION ===
\usepackage{indentfirst}

% === ENUMERATE ===
\usepackage[shortlabels]{enumitem}

% === TEXT ===
% Bold, italic, underlined text
\newcommand{\tb}[1]{\textbf{#1}}
\newcommand{\ti}[1]{\textit{#1}}
\newcommand{\tbi}[1]{\textbf{\textit{#1}}}
\newcommand{\tu}[1]{\underline{#1}}
\newcommand{\tbu}[1]{\textbf{\underline{#1}}}

% === TEXTBOX ===
\usepackage{varwidth}
\newenvironment{fboxenv}
  {
    \begin{minipage}{\dimexpr\textwidth-2\fboxsep-2\fboxrule\relax}
  }
  {
    \end{minipage}
  }

\newcommand{\textbox}[1]{\framebox{\parbox{\dimexpr\linewidth-2\fboxsep-2\fboxrule}{\itshape #1 }}}


% === MATH ===
% Basic packages
\usepackage{amsmath}
\usepackage{amssymb}
\usepackage{gensymb}

% Equation box
\usepackage{empheq}
\newenvironment{eqbox}
  {\setkeys{EmphEqEnv}{align}\setkeys{EmphEqOpt}{box=\fbox}\EmphEqMainEnv}
  {\endEmphEqMainEnv}

% Arrows
\newcommand{\ra}{\Rightarrow}
\newcommand{\lra}{\Leftrightarrow}

% Sum, product, limit
\newcommand{\SUM}[1]{\sum\limits #1}
\newcommand{\PROD}[1]{\prod\limits #1}

% Bold in math mode
\usepackage{bm}

% Absolute value
\DeclarePairedDelimiter\abs{\lvert}{\rvert}%
\DeclarePairedDelimiter\norm{\lVert}{\rVert}%
%   Swap the definition of \abs* and \norm*, so that \abs
%   and \norm resizes the size of the brackets, and the
%   starred version does not.
\makeatletter
\let\oldabs\abs
\def\abs{\@ifstar{\oldabs}{\oldabs*}}
\let\oldnorm\norm
\def\norm{\@ifstar{\oldnorm}{\oldnorm*}}
\makeatother

% Floor and ceiling
\usepackage{mathtools}
\DeclarePairedDelimiter\ceil{\lceil}{\rceil}
\DeclarePairedDelimiter\floor{\lfloor}{\rfloor}

% Inner Product
\newcommand{\iprod}[1]{\langle #1 \rangle}

% Derivatives
\newcommand{\dif}[2]{\dfrac{d #1}{d #2}}
\newcommand{\ddif}[2]{\dfrac{d^2 #1}{d #2^2}}
\newcommand{\difi}[2]{d #1/d #2}
\newcommand{\diff}[2]{\dfrac{d}{d #2} #1}
\newcommand{\pd}[2]{\dfrac{\partial #1}{\partial #2}}
\newcommand{\pdds}[2]{\dfrac{\partial^{2} #1}{\partial #2^{2}}}
\newcommand{\pddsi}[2]{{\partial^{2} #1} / {\partial #2^{2}}}
\newcommand{\pdd}[3]{\dfrac{\partial^{2} #1}{\partial #2 \partial #3}}
\newcommand{\pddi}[3]{{\partial^{2} #1} / {\partial #2 \partial #3}}

% Integrals
\usepackage{esint}
\newcommand{\INT}{\int \limits}
\newcommand{\OINT}{\oint \limits}
\newcommand{\IINT}{\iint \limits}
\newcommand{\IIINT}{\iiint \limits}

% Ratios
\newcommand{\ratio}[3]{\dfrac{#1_{#2}}{#1_{#3}}}

% Hand-written-like fonts
\usepackage{eucal}

% Superscript+subscript
\newcommand{\spc}[2]{{#1}^{#2}}
\newcommand{\sbc}[2]{{#1}_{#2}}
\newcommand{\ssc}[3]{{#1}^{#2}_{#3}}

\usepackage{titlesec}
\titlespacing{\section}{0pt}{10pt}{5pt}
\titlespacing{\subsection}{0pt}{10pt}{3pt}
\titlespacing{\subsubsection}{0pt}{10pt}{3pt}
\titlespacing{\subsubsubsection}{0pt}{10pt}{3pt}
% \usepackage[parfill]{parskip}           % For compact enumerations

% INFO
\title{\vspace{-4ex}\Large{Probability \& Statistics - Cheat Sheet}}
\author{Nhat M. Nguyen - nhat.nguyen.cs17@gmail.com}
\date{June 2019}

\begin{document}

  \maketitle

  \par \tb{Disclaimer: I am not 100\% sure that everything in this cheat sheet is correct. If you find a formula suspicious, you should recheck your own materials. :) }
  \begin{multicols}{2}
    \tableofcontents
  \end{multicols}

  \clearpage

  \hi{Logic}

\hii{Natural Deduction Rules}

% AndIntroduction
\newsavebox\AndIntro
\sbox\AndIntro{
  \AxiomC{$\phi$}
  \AxiomC{$\psi$}
  \RightLabel{$\land i$}
  \BinaryInfC{$\phi \land \psi$}
  \DisplayProof
}

% AndElimination
\newsavebox\AndElim
\sbox\AndElim{
  \AxiomC{$\phi \land \psi$}
  \RightLabel{$\land e_1$}
  \UnaryInfC{$\phi$}
  \DisplayProof
  \hskip 1cm
  \AxiomC{$\phi \land \psi$}
  \RightLabel{$\land e_2$}
  \UnaryInfC{$\psi$}
  \DisplayProof
}

% Double-Negation Introduction
\newsavebox\DNegIntro
\sbox\DNegIntro{
  \AxiomC{$\phi$}
  \RightLabel{$\lnotnot i$}
  \UnaryInfC{$\lnotnot \phi$}
  \DisplayProof
}

% Double-Negation Elimination 
\newsavebox\DNegElim
\sbox\DNegElim{
  \AxiomC{$\lnotnot \phi$}
  \RightLabel{$\lnotnot e$}
  \UnaryInfC{$\phi$}
  \DisplayProof
}

% 

\begin{tabular}{|c|c|c|}
  \hline
  Rule              & Introduction      & Elimination      \\ \hline
  Conjunction (AND) & \usebox\AndIntro  & \usebox\AndElim  \\ \hline
  Double Negation   & \usebox\DNegIntro & \usebox\DNegElim \\ \hline
\end{tabular}

  \chapter{Partial Derivatives}

\hi{Function of two variables}
    \hii{Definition}
        \par A function of two variables is a rule that assigns
        to each ordered pair of real numbers $(x, y)$ in a set a unique real number
        denoted by $f(x, y)$. The set $D$ is the \textbf{domain} of $f$ and its
        \textbf{range} is the set of values that $f$ takes on.
        \par We often write:
        \begin{eqbox}
            z = f(x, y)
        \end{eqbox}
        where:
        \begin{itemize}
            \item $x$, $y$: independent variables
            \item $z$: dependent variable
        \end{itemize}
    \hii{Domain}
        \par Given the function: $f(x, y)$. The domain of $f$ is the
        \textbf{Cartesian product} of the set $D_{x}$ and the set $D_{y}$, where
        $D_{x}$ and $D_{y}$ are the set of all possible values of $x$ and $y$,
        respectively. This is true for any multivariable function.
        \begin{eqbox}
            D = D_{x} \times D_{y} \subseteq \mathbb{R}^{2}
        \end{eqbox}
    \hii{Graph}
        \par If $f$ is a function of two variables with domain $D$, then the graph
        of $f$ is the set of all points $(x, y, z)$ in $\mathbb{R}^{3}$ such that
        $z = f(x, y)$ and $(x, y) \in D$.
    \hii{Level Curves}
        \par The \textbf{level curves} of a function $f$ of two variables are the curves
        with equations $f(x, y) = k$, where $k$ is a constant (in the range of $f$).
        \par The graph of level curves is called the \textbf{contour graph}.
        \par To construct a contour graph, the set of $k$ is required.

\hi{Limits and Continuity}
    \hiiBEGIN{Limits}
        \hiii{Definition}
            \par Let $f$ be a function of two variables whose domain $D$ includes points arbitrarily
            close to $(a, b)$. Then we say that the \textbf{limit} of $f(x, y)$ as $(x, y)$ approaches
            $(a, b)$ is $L$ and we write
            \begin{eqbox}
                \lim_{(x, y) \to (a, b)} f(x, y) = L
            \end{eqbox}
            if for every number $\epsilon > 0$ there is a corresponding number $\delta > 0$ such that:
            \begin{center}
                if $(x, y) \in D$ and $0 < \sqrt{(x - a)^{2} - (y - b)^{2})} < \delta$, then
                $|f(x, y) - L| < \epsilon$
            \end{center}
        \hiii{Existence of limit}
            \par For a function of one variable
            \begin{center}
                If $\lim_{x \to a^{-}} f(x) \neq \lim_{x \to a^{+}}$, then $\lim_{x \to a} f(x)$ does
                not exists.
            \end{center}
            \par For a multivariable function, the limit at one point can be approaches from
            infinitely many directions. If there \textbf{exists two different paths} of approach
            along which the function $f(x, y)$ has different limits, then the limit at that point
            does not exists.
    \hiiEND
    \hii{Continuity}
        \par A function $f$ of two variables is called \textbf{continuous} at $(a, b)$ if:
        \begin{eqbox}
            \lim_{(x, y) \to (a, b)} f(x, y) = f(a, b)
        \end{eqbox}
        \par We say $f$ is continuous on $D$ if $f$ is continuous at every point $(a, b)$ in $D$.

\hi{Partial Derivatives}
    \hii{Definition}
        \par Given the function $f(x, y)$.
        \par The \textbf{partial derivative} of $f$ with respect to $x$ at $(a, b)$, denoted by
        $f_{y}(a, b)$, is obtained by keeping $y$ fixed $(y = b)$ and finding the ordinary
        derivative at $a$ of the function $G(x) = f(x, b)$.
        \begin{eqbox}
            f_{x} (x, y) = \lim_{\Delta x \to 0} \frac{f(x + \Delta x, y) - f(x, y)}{\Delta x} \\
            f_{y} (x, y) = \lim_{\Delta y \to 0} \frac{f(x, y + \Delta y) - f(x, y)}{\Delta y}
        \end{eqbox}
    \hii{Notations}
        \begin{eqbox}
            f_{x} (x, y) = f_{x} = \pd{f}{x} = \pd{}{x} f(x, y) = \pd{z}{x} = D_{1}f = D_{x}f \\
            f_{y} (x, y) = f_{y} = \pd{f}{y} = \pd{}{y} f(x, y) = \pd{z}{y} = D_{2}f = D_{y}f
        \end{eqbox}
    \hii{Rule for Finding Partial Derivatives}
        \par Given the function $z = f(x, y)$.
        \begin{itemize}
            \item To find $f_{x}$, regard $y$ as a constant and differentiate $f(x, y)$ with respect
                to $x$.
            \item To find $f_{y}$, regard $x$ as a constant and differentiate $f(x, y)$ with respect
                to $y$.
        \end{itemize}
    \hii{Higher Derivatives}
        \begin{alignat*}{4}
            (f_{x})_{x} (x, y) = f_{xx} = f_{11}
                &= \pd{}{x}\bigg(\pd{f}{x}\bigg)
                &= \pd{^{2}f}{x^{2}}
                &= \pd{^{2}z}{x^{2}} \\
            (f_{x})_{y} (x, y) = f_{xy} = f_{12}
                &= \pd{}{y}\bigg(\pd{f}{x}\bigg)
                &= \pdd{^{2}f}{y}{x}
                &= \pdd{^{2}z}{y}{x} \\
            (f_{y})_{x} (x, y) = f_{yx} = f_{21}
                &= \pd{}{x}\bigg(\pd{f}{y}\bigg)
                &= \pdd{^{2}f}{x}{y}
                &= \pdd{^{2}z}{x}{y} \\
            (f_{y})_{y} (x, y) = f_{yy} = f_{22}
                &= \pd{}{y}\bigg(\pd{f}{y}\bigg)
                &= \pd{^{2}f}{y^{2}}
                &= \pd{^{2}z}{y^{2}}
        \end{alignat*}


\hi{Tangent Plane and Linear Approximation}
    \hii{Differentials}
        \par For a differentiable function of two variables, $z = f(x, y)$, we define the
        differentials $dx$ and $dy$ to be independent variables. Then the differential $dz$,
        also called the total differntial, is defined by:
        \begin{eqbox}
            dz = f_{x}(x, y) dx + f_{y}(x, y) dy = \pd{f}{x} dx + \pd{f}{y} dy
        \end{eqbox}
        \par If we take $dx = \Delta x = x - a$ and $dy = \Delta y = y - b$,
        then the differential of $z$ is:
        \begin{alignat*}{2}
            dz &= f_{x} (x_{1}, y_{1}) \Delta x + f_{y} (x_{1}, y_{1}) \Delta y \\
            &= f_{x} (x_{1}, y_{1}) (x_{2} - x_{1}) + f_{y} (x_{1}, y_{1}) (y_{2} - y_{1})
        \end{alignat*}

  \begin{multicols*}{3}

\hi{Distributions}
  \hii{Discrete Distributions}

  \hiii{Binomial Distribution}
    \begin{itemize}
      \item \tb{Intepretation}:
        \begin{itemize}
          \item The number of winning game in $n$ games, given that the chance to win any one game is $p$.
          \item The number of red balls obtained after picking \tb{with replacement} from $n$ balls from a box of red and blue balls, given that the chance to pick a red ball at one time is $p$.
        \end{itemize}
      \item \tb{Notation}:
        \[
          X \sim B(n, p)
        \]
      \item \tb{P.M.F}:
        \[
          f(k) = P(X = k) = {n \choose k} p^k q^{n - k}
        \]
      \item \tb{Expectation}:
        \[
          \mu = E(X) = np
        \]
      \item \tb{Variance}:
        \[
          \sigma^2 = V(X) = npq = np(1 - p)
        \]
    \end{itemize}

  \hiii{Hypergeometric Distribution}
    \begin{itemize}
      \item \tb{Intepretation}:
        \begin{itemize}
          \item The number of red balls obtained after picking \tb{without replacement} $n$ balls from $N$ balls from a box of red and blue balls, given that the number of red balls in the box is $m$.
        \end{itemize}
      \item \tb{Notation}:
        \[
          X \sim H(N, m, n)
        \]
      \item \tb{P.M.F}:
        \[
          f(k) = P(X = k) = \dfrac{{m \choose k} {N - m \choose n - k}}{{N \choose n}}
        \]
      \item \tb{Expectation}:
        \[
          \mu = E(X) = np, \text{ where } p = \frac{m}{N}
        \]
      \item \tb{Variance}:
        \[
          \sigma^2 = V(X) = npq \frac{N - n}{N - 1}, \text{ where } p = \frac{m}{N}
        \]
    \end{itemize}

  \hiii{Poisson Distribution}
    \begin{itemize}
      \item \tb{Intepretation}:
        \begin{itemize}
          \item The number of events occuring in a fixed period of time.
        \end{itemize}
      \item \tb{Notation}:
        \[
          X \sim Po(\lambda)
        \]
      \item \tb{P.M.F}:
        \[
          f(k) = P(X = k) = \frac{\lambda^k}{k!} e^{-\lambda}
        \]
      \item \tb{Expectation}:
        \[
          \mu = E(X) = \lambda
        \]
      \item \tb{Variance}:
        \[
          \sigma^2 = V(X) = \lambda
        \]
    \end{itemize}

  \hii{Continuous Distributions}

  \hiii{Uniform Distribution}
    \begin{itemize}
      \item \tb{Notation}:
        \[
          X \sim U(a, b)
        \]
      \item \tb{P.D.F}:
        \[
          f(x) =
            \begin{cases}
              \dfrac{1}{b - a} & x \in [a, b]\\
              0 & \text{otherwise}
            \end{cases}
        \]
      \item \tb{C.D.F}:
        \[
          F(x) =
            \begin{cases}
              0 & x < a \\
              \dfrac{x - a}{b - a} & x \in [a, b)\\
              1 & x \geq b
            \end{cases}
        \]
      \item \tb{Expectation}:
        \[
          \mu = E(X) = \frac{a + b}{2}
        \]
      \item \tb{Variance}:
        \[
          \sigma^2 = V(X) = \frac{(b - a)^2}{12}
        \]
    \end{itemize}

  \hiii{Normal Distribution}
    \begin{itemize}
      \item \tb{Notation}:
        \[
          X \sim N(\mu, \sigma^2)
        \]
      \item \tb{Standardizing}:
        \[
          Y = \frac{X - \mu}{\sigma} \sim N(0, 1)
        \]
      \item \tb{C.D.F}:
        \par For $X \sim N(0, 1)$:
        \[
          \Phi(x) = \int\limits_{-\infty}^{x} f(u) du
        \]
      \item \tb{Expectation}:
        \[
          \mu = E(X) = \mu
        \]
      \item \tb{Variance}:
        \[
          \sigma^2 = V(X) = \sigma^2
        \]
      \item \tb{Properties}
        \begin{itemize}
          \item If $X \sim N(\mu, \sigma^2)$ and $Y = aX + b$:
            \[
              Y \sim N(a\mu + b, a^2 \sigma^2)
            \]
          \item If $X \sim N(\mu, \sigma^2)$:
            \[
              \sum\limits_{i = 1}^{n} X_i \sim N\bigg( \sum\limits_{i = 1}^{n} \mu_i, \sum\limits_{i = 1}^{n} \sigma_i^2 \bigg)
            \]
        \end{itemize}
    \end{itemize}


\end{multicols*}

  \begin{multicols*}{3}

\hi{Appendix}

  \hii{Integrals}
    \begin{itemize}
      \item \[\int x^n dx = \frac{1}{n + 1} x^{n + 1} + C \]
      \item \[\int \frac{1}{x} dx = \ln|x| + C \]
      \item \[\int \frac{1}{t^2} dx = - \frac{1}{t} + C\]
      \item \[\int e^x dx = e^x + C\]
      \item \[\int a^x dx = \frac{a^x}{\ln(a)} + C\]
      \item \[\int e^{ax} dx = \frac{e^{ax}}{a} + C\]
      \item \[\int x e^x dx = (x - 1)e^x + C\]
      \item \tb{Integration by parts}:
        \[
          \int udv = uv - v \int du
        \]
        \par Choose $u$ in this order: log - algebraic - trig - exp
    \end{itemize}

\end{multicols*}

\begin{multicols}{3}
\hi{Confidence Intervals}

  \hii{Definitions and Terminology}
    \hiii{Terminology}
      \begin{itemize}
        \item A \tb{population} is a collection of objects about which information is sought.
        \item A \tb{sample} is a part of the population that is observed.
        \item A \tb{parameter} is a numerical characteristic of a population.
        \item A \tb{statistics} is a numerical function of the sampled data, used to estimate an unknown parameter
      \end{itemize}

    \hiii{Mean}
      \begin{itemize}
      \item \tb{Population Mean}: denoted $\mu$, is the average of all values in the \tb{entire population}.
      \item \tb{Sample Mean}: denoted $\hat{\mu} = \bar{X}$, is the average value of a \tb{sample}. For a sample with $n$ values:
        \[
          \hat{\mu} = \bar{X} = \frac{x_1 + \ldots + x_n}{n}
        \]
      \end{itemize}

    \hiii{Median}
      \par List all data values in sorted order from smallest to largest. For a sample with $n$ values:
      \begin{itemize}
        \item If $n \not \mid 2$: the mean is one single value at the middle position.
        \item If $n \mid 2$: the mean is the average of two values at the middle positions.
      \end{itemize}

    \hiii{Variance}
      \par The \tb{sample variance}, denoted $s^2$, is the approximation of the \tb{population variance} $\sigma^2$.
      \[
        s^2 = \frac{\sum (x_i - \bar{X})^2}{n - 1} = \frac{\sum (x_i - \hat{\mu})^2}{n - 1}
      \]
      where $n - 1$ is called the \tb{degree of freedom}.

    \hiii{Confidence Level and Significance Level}
      \begin{itemize}
        \item \tb{Confidence level} $\gamma$ is the measure of the degree of reliability of the interval.
        \item \tb{Significance level} $\alpha$ is the probability we allow ourselves to be wrong when we are estimating a parameter with a confidence interval.
      \end{itemize}
      \[
        \gamma + \alpha = 1
      \]

  \hii{Case 1: Normal Population with Known Standard Deviation}
    \hiii{Confidence Interval of the Population Mean}
    \par If $x_1, \ldots, x_n$ are independent and identically distributed (i.i.d.), following a normal distribution $N(\mu, \sigma^2)$ and $\alpha = 1 - \gamma$, then the \tb{confidence interval of the population mean} is:
    \[
      \mu = \hat{\mu} \pm z_{\alpha / 2} \times \frac{\sigma}{\sqrt{n}}
    \]

    \hii{Sample Size According to Maximum Error}
      \par If $\bar{X}$ is used to estimate $\mu$, then we can be $100(1 - a)\%$ confident that the error will not exceed a specified amount $\epsilon$ when the sample size is:
      \[
        n = \bigg\lceil \frac{\sigma z_{\alpha/2}}{\epsilon} \bigg\rceil^2
      \]

    \hii{One-Sided Convidence Interval of the Population Mean}
      \begin{itemize}
        \item A $100(1 - \alpha)\%$ upper-confidence bound for $\mu$ is
        \[
          \mu \leq \hat{\mu} + z_{\alpha} \times \frac{\sigma}{\sqrt(n)}
        \]
        \item A $100(1 - \alpha)\%$ lower-confidence bound for $\mu$ is
        \[
          \mu \geq \hat{\mu} - z_{\alpha} \times \frac{\sigma}{\sqrt(n)}
        \]
      \end{itemize}

  \hii{Case 2: Normal Population with Unknown Standard Deviation}
    \hiii{Confidence Interval of the Population Mean}
     \par If $x_1, \ldots, x_n$ are independent and identically distributed (i.i.d.) then the \tb{confidence interval of the population mean} is:
     \[
      \mu = \hat{\mu} \pm t_{n - 1, \alpha / 2} \times \frac{s}{\sqrt(n)}
     \]
    \hiii{Confidence Interval of the Population Variance}
      \begin{itemize}
        \item \tb{Two-sided Confidence Interval}
        \par Choose $c_1$ and $c_2$ so that the area in each tail of $\chi_{n - 1}^2$ distribution is $\alpha / 2$. Then the $\gamma$-confidence interval
        for the unknown variance $\sigma^2$ is:
        \[
          \frac{(n - 1)s^2}{c_2} \leq \sigma^2 \leq \frac{(n - 1)s^2}{c_1}
        \]
        \item \tb{One-sided Confidence Interval}
        \par Choose $c_1$ and $c_2$ so that the area in each tail of $\chi_{n - 1}^2$ distribution is $\alpha$. Then the $\gamma$-confidence interval
        for the unknown variance $\sigma^2$ are:
        \[
          \sigma^2 \geq \frac{(n - 1)s^2}{c_2}
        \]
        and
        \[
          \sigma^2 \leq \frac{(n - 1)s^2}{c_1}
        \]
      \end{itemize}

    \hii{Case 3: Large Sample Size}
      \hiii{Confidence Interval of the Population Mean}
      \par If $x_1, \ldots, x_n$ are independent and identically distributed (i.i.d.) and $n$ is large, then:
        \[
          \mu = \hat{\mu} \pm z_{\alpha / 2} \times \frac{s}{\sqrt(n)}
        \]

\end{multicols}

\clearpage

  \begin{multicols*}{3}
  
\hi{Hypothesis Testing for One Sample}

\hii{Terminology}
  \begin{itemize}
    \item \tb{null hypothesis}:  the claim that is initially assumed to be true, denoted by $H_0$
    \item \tb{alternative hypothesis}: the assertion that is contradictory to $H_0$, denoted by $H_1$
    \item \tb{test of hypotheses}: a method for using sample data to decide \tb{whether the null hypothesis should be rejected}.
    \item \tb{significant level}: the probability of a type I error, denoted by $\alpha$
  \end{itemize}

\hii{Tips}
  \begin{itemize}
    \item The \tb{equal sign} only appears in the \tb{null hypothesis} $H_0$.
    \item The thing we want to prove appears in the \tb{alternative hypothesis} $H_1$. 
  \end{itemize}

\hii{Case 1: Normal Population with Known Standard Deviation}
  \hiii{Hypothesis Testing}
  \begin{itemize}
    \item \tb{Step 1}: Compute the statistic:
      \[
        z = \frac{\bar{x} - \mu_0}{\sigma / \sqrt{n}}
      \]
    \item \tb{Step 2}: Apply the decision rule:
      \begin{center}
        \begin{tabular}{|c|c|}
          \hline
          \textbf{$H_1$}   & \textbf{Rejection Region} \\ \hline
          $\mu \neq \mu_0$ & $|z| > z_{\alpha/2}$      \\ \hline
          $\mu < \mu_0$    & $z < -z_{\alpha}$         \\ \hline
          $\mu > \mu_0$    & $z > z_{\alpha}$          \\ \hline
        \end{tabular}
      \end{center}
  \end{itemize}

    \hiii{Hypothesis Testing with Propotion}
    \begin{itemize}
      \item \tb{Step 1}: Compute the statistic:
        \[
          z = \frac{\hat{p} - p_0}{\sqrt{\hat{p}\hat{q}/n}}
        \]
      \item \tb{Step 2}: Apply the decision rule:
        \begin{center}
          \begin{tabular}{|c|c|}
            \hline
            \textbf{$H_1$}   & \textbf{Rejection Region} \\ \hline
            $p \neq p_0$ & $|z| > z_{\alpha/2}$      \\ \hline
            $p < p_0$    & $z < -z_{\alpha}$         \\ \hline
            $p > p_0$    & $z > z_{\alpha}$          \\ \hline
          \end{tabular}
        \end{center}
    \end{itemize}

\hii{Case 2: Normal Population with Unknown Standard Deviation}

\begin{itemize}
  \item \tb{Step 1}: Compute the statistic:
    \[
      t = \frac{\bar{x} - \mu_0}{s / \sqrt{n}}
    \]
  \item \tb{Step 2}: Apply the decision rule:
      \begin{center}
        \begin{tabular}{|c|c|}
          \hline
          \textbf{$H_1$}   & \textbf{Rejection Region}   \\ \hline
          $\mu \neq \mu_0$ & $|t| > t_{\alpha/2, n - 1}$ \\ \hline
          $\mu < \mu_0$    & $t < -t_{\alpha, n - 1}$    \\ \hline
          $\mu > \mu_0$    & $t > t_{\alpha, n - 1}$     \\ \hline
        \end{tabular}
      \end{center}
\end{itemize}

\hii{Case 3: Any Distribution - Large Sample Size}

\begin{itemize}
  \item \tb{Step 1}: Compute the statistic:
    \[
      z = \frac{\bar{x} - \mu_0}{s / \sqrt{n}}
    \]
  \item \tb{Step 2}: Apply the decision rule:
      \begin{center}
        \begin{tabular}{|c|c|}
          \hline
          \textbf{$H_1$}   & \textbf{Rejection Region} \\ \hline
          $\mu \neq \mu_0$ & $|z| > z_{\alpha/2}$      \\ \hline
          $\mu < \mu_0$    & $z < -z_{\alpha}$         \\ \hline
          $\mu > \mu_0$    & $z > z_{\alpha}$          \\ \hline
        \end{tabular}
      \end{center}
\end{itemize}

\par \tb{Testing with Population Proportion - Large Sample Size}

\begin{itemize}
  \item \tb{Step 1}: Compute the statistic:
    \[
      z = \frac{\hat{p} - p_0}{\sqrt{p_0 q_0 / n}}
    \]
  \item \tb{Step 2}: Apply the decision rule:
      \begin{center}
        \begin{tabular}{|c|c|}
          \hline
          \textbf{$H_1$}   & \textbf{Rejection Region} \\ \hline
          $p \neq p_0$ & $|z| > z_{\alpha/2}$      \\ \hline
          $p < p_0$    & $z < -z_{\alpha}$         \\ \hline
          $p > p_0$    & $z > z_{\alpha}$          \\ \hline
        \end{tabular}
      \end{center}
\end{itemize}


\end{multicols*}

  \chapter{Basic SQL}

\hi{SQL Data Definition and Data Types}
  \hii{Terminology}
    \begin{itemize}
      \item A \tb{table} is equivalent to a \tb{relation}.
      \item A \tb{row} is equivalent to a \tb{tuple}.
      \item A \tb{column} is equivalent to an \tb{attribute}.
    \end{itemize}

  \hii{Schema}
    \hiii{Schema}
      \par A \tb{schema} is a group of related tables.
      \par An \tb{SQL schema}:
        \begin{itemize}
          \item is identified by a \tb{schema name}
          \item includes an authorization identifier to indicate the user or account who owns the schema
          \item includes descriptors for each element in the schema.
        \end{itemize}

      \par \tb{Creating a schema}:
      % \begin{lstlisting}[style=SQL]
% \begin{minted}[linenos,tabsize=2,breaklines]{SQL}
\begin{lstlisting}
CREATE SCHEMA `schema_name' AUTHORIZATION `author_name'
\end{lstlisting}

  \hii{Catalog}
    \par A \tb{catalog} is a named collection of schemas.

  \hii{CREATLE TABLE command in SQL}
    \par The CREATE TABLE command is used to specify a new relation by
      \begin{itemize}
      \item giving it a name
      \item specifying its attributes and initial constraints.
      \end{itemize}
    \par The attributes are specified first, and each attribute is given a name, a data type to specify its domain of values, and possibly attribute constraints, such as NOT NULL. The key, entity integrity, and referential integrity constraints can be specified within the CREATE TABLE statement
after the attributes are declared, or they can be added later using the ALTER TABLE command.
    \par The relations declared through CREATE TABLE statements are called base tables (or base relations); this means that the table and its rows are actually created and stored as a file by the DBMS. Base relations are distinguished from virtual relations, created through the CREATE VIEW statement (see Chapter 7), which may or may not correspond to an actual physical file.
    \par In SQL, columns are ordered while rows are not.

  \hii{Attribute Data Types and Domains in SQL}

\hi{Specifying Constraints in SQL}
  \hii{Specifying Attribute Constraints and Attribute Defaults}
    \par SQL allows NULLs as attribute values, a constraint NOT NULL may be specified if NULL is not permitted for a particular attribute.
    \par Primary keys are always implicitly specified as NOT NULL.
    \par It is also possible to define a default value for an attribute by appending the clause
      \lstinline{DEFAULT <value>} to an attribute definition.

  \hii{Specifying Key and Referential Integrity Constraints}
    \par The PRIMARY KEY clause specifies one or more attributes that make up the primary
key of a relation. If a primary key has a single attribute, the clause can follow the
attribute directly.
    \par The UNIQUE clause specifies alternate (unique) keys, also known as candidate keys.
    \par Referential integrity is specified via the FOREIGN KEY clause.
    \par The default action that SQL takes for an integrity violation is to reject the update operation that will cause a violation, which is known as the RESTRICT option. However, the schema designer can specify an alternative action to be taken by attaching a referential triggered action clause to any foreign key constraint. The options include SET NULL, CASCADE, and SET DEFAULT. An
option must be qualified with either ON DELETE or ON UPDATE.
    \par A constraint may be given a constraint name, following the keyword CONSTRAINT .

  \hii{Specifying Constraints on Tuples Using CHECK}
    \par In addition to key and referential integrity constraints, which are specified by special keywords, other table constraints can be specified through additional CHECK clauses at the end of a CREATE TABLE statement. These can be called row-based constraints because they apply to each row individually and are checked whenever a row is inserted or modified.

\hi{Basic Retrieval Queries in SQL}

  \hii{Important Notes on SQL}
    \par SQL allows a table (relation) to have two or more tuples that are identical in all their attribute values. Hence, in general, an SQL table is not a set of tuples, because a set does not
allow two identical members; rather, it is a multiset (sometimes called a bag) of tuples. Some SQL relations are constrained to be sets because a key constraint has been declared or because the DISTINCT option has been used with the SELECT statement.

    \par The SELECT statement of SQL \ti{is not the same} as the SELECT operation of relational algebra.

  \hii{The SELECT-FROM-WHERE Structure of Basic SQL Queries}
    % \begin{minted}[linenos,tabsize=2,breaklines]{SQL}
    \begin{lstlisting}
SELECT <attribute_list>
FROM <table_list>
WHERE <condition>;
    \end{lstlisting}
      where
    \begin{itemize}
      \item \lstinline{<attribute list>} is a list of attribute names whose values are to be retrieved by the query.
      \item \lstinline{<table list>} is a list of the relation names required to process the query.
      \item \lstinline{<condition>} is a conditional (Boolean) expression that identifies the tuples to be retrieved by the query.
    \end{itemize}

    \par \tb{Logical comparison operators}: \lstinline{=, <, <=, >, >=, <>}

    \begin{itemize}
      \item The \lstinline{SELECT} clause specifies the attributes whose values are to be retrieved, which are called the \tb{projection attributes}.
      \item The \lstinline{WHERE} clause specifies the Boolean condition that must be true for any retrieved tuple, which is known as the \tb{selection condition}.
      \item A selection condition that joins two different tuples is called a \tb{join condition}.
      \item A query that involves only selection and join conditions plus projection attributes is known as a \tb{select-project-join} query.
    \end{itemize}

  \hii{Ambiguous Attribute Names, Aliasing, Renaming, and Tuple Variables}
    \par In SQL, the same name can be used for many attributes as long as they are in different tables. If this is the case, and a multitable query refers to two or more attributes with the same name, we must qualify the attribute name with the relation name to prevent ambiguity. This is done by prefixing the relation name to the attribute name and separating the two by a period.
    \par The ambiguity of attribute names also arises in the case of queries that refer to the
same relation twice (recursive ?). In this case, we are required to declare alternative relation names E and S , called aliases or tuple variables, for the EMPLOYEE relation. An alias can follow the keyword AS.
      % \begin{minted}[linenos,tabsize=2,breaklines]{SQL}
      \begin{lstlisting}
SELECT E.Fname, E.Lname, S.Fname, S.Lname
FROM EMPLOYEE AS E, EMPLOYEE AS S
WHERE E.Super_ssn = S.Ssn;
      \end{lstlisting}
    \par It is also possible to rename the relation \tb{attributes} within the query in SQL by giving them aliases.

  \hii{Unspecified WHERE Clause and Use of the Asterisk}
    \par A missing WHERE clause indicates no condition on tuple selection; hence, all tuples of the relation specified in the FROM clause qualify and are selected for the query result.

    \par If more than one relation is specified in the FROM clause and there is no WHERE clause, then the CROSS PRODUCT —all possible tuple combinations—of these relations is selected.

    \par To retrieve all the attribute values of the selected tuples, we do not have to list the attribute names explicitly in SQL; we just specify an asterisk (*), which stands for all the attributes. The * can also be prefixed by the relation name or alias; for example, EMPLOYEE.* refers to all attributes of the EMPLOYEE table.

  \hii{Tables as Sets in SQL}
    \par If we do want to eliminate duplicate tuples from the result of an SQL query, we use the keyword DISTINCT in the SELECT clause, meaning that only distinct tuples should remain in the result. In general, a query with SELECT DISTINCT eliminates duplicates, whereas a query with SELECT ALL does not. Specifying SELECT with neither ALL nor DISTINCT —as in our previous examples—is equivalent to SELECT ALL. If we do want to eliminate duplicate tuples from the result of an SQL query, we use the keyword DISTINCT in the SELECT clause, meaning that only distinct tuples should remain in the result. In general, a query with SELECT DISTINCT eliminates duplicates, whereas a query with SELECT ALL does not. Specifying SELECT with neither ALL nor DISTINCT is
equivalent to SELECT ALL .
    \par SQL has directly incorporated some of the set operations from mathematical set
theory, which are also part of relational algebra: set union (UNION), set difference (EXCEPT), and set intersection (INTERSECT) operations. The relations resulting from these set operations are sets of tuples; that is, duplicate tuples are eliminated from the result. These set operations apply only to \tb{type compatible relations}, so we must make sure that the two relations on which we apply the operation have the same attributes and that the attributes appear in the same order in both relations.
  \par SQL also has corresponding multiset operations, which are followed by the keyword ALL (UNION ALL, EXCEPT ALL, INTERSECT ALL). Their results are multisets (duplicates are not eliminated). Basically, each tuple—whether it is a duplicate or not— is considered as a different tuple when applying these operations.

  \hii{Substring Pattern Matching and Arithmetic Operators}
    \par \tb{Example}: Retrieve all employees whose address is in Houston, Texas.
      % \begin{minted}[linenos,tabsize=2,breaklines]{SQL}
      \begin{lstlisting}
SELECT Fname, Lname
FROM   EMPLOYEE
WHERE  Address LIKE ‘%Houston,TX%’;
      \end{lstlisting}
    \par \tb{Example}: Find all employees who were born during the 1950s.
      % \begin{minted}[linenos,tabsize=2,breaklines]{SQL}
      \begin{lstlisting}
SELECT Fname, Lname
FROM EMPLOYEE
WHERE Bdate LIKE ‘_ _ 5 _ _ _ _ _ _ _’;
      \end{lstlisting}

  \hii{Ordering of Query Results}
    \par SQL allows the user to order the tuples in the result of a query by the values of one or more of the attributes that appear in the query result.
    \par \tb{Example}: Retrieve a list of employees and the projects they are working on, ordered by department and, within each department, ordered alphabetically by last name, then first name.
      % \begin{minted}[linenos,tabsize=2,breaklines]{SQL}
      \begin{lstlisting}
SELECT D.Dname, E.Lname, E.Fname, P.Pname
FROM DEPARTMENT AS D, EMPLOYEE AS E, WORKS_ON AS W, PROJECT AS P
WHERE  D.Dnumber = E.Dno AND E.Ssn = W.Essn AND W.Pno = P.Pnumber
ORDER BY D.Dname, E.Lname, E.Fname;
      \end{lstlisting}
    \par Ascending and Descending:
       % \begin{minted}[linenos,tabsize=2,breaklines]{SQL}
       \begin{lstlisting}
    ORDER BY D.Dname DESC, E.Lname ASC, E.Fname ASC
      \end{lstlisting}

\hi{INSERT, DELETE, and UPDATE Statements in SQL}
  \hii{The INSERT Command}

    \par In its simplest form, INSERT is used to add a single tuple (row) to a relation (table). We must specify the relation name and a list of values for the tuple. The values should be listed in the same order in which the corresponding attributes were specified in the CREATE TABLE command.

      % \begin{minted}[linenos,tabsize=2,breaklines]{SQL}
      \begin{lstlisting}
INSERT INTO EMPLOYEE
VALUES ( 'Richard', 'K', 'Marini', '653298653', '1962-12-30', '98 Oak Forest, Katy, TX', 'M', 37000, ‘653298653’, 4 );
      \end{lstlisting}

    \par A second form of the INSERT statement allows the user to specify explicit attribute
    names that correspond to the values provided in the INSERT command. This is use-
    ful if a relation has many attributes but only a few of those attributes are assigned
    values in the new tuple. However, the values must include all attributes with NOT
    NULL specification and no default value. Attributes with NULL allowed or DEFAULT
    values are the ones that can be left out.

      % \begin{minted}[linenos,tabsize=2,breaklines]{SQL}
      \begin{lstlisting}
INSERT INTO EMPLOYEE(Fname, Lname, Dno, Ssn)
VALUES (‘Richard’, ‘Marini’, 4, ‘653298653’);
      \end{lstlisting}

    \par A DBMS that fully implements SQL should support and enforce all the integrity
constraints that can be specified in the DDL.

  \hii{The DELETE Command}
    \par The DELETE command removes tuples from a relation. It includes a WHERE clause, similar to that used in an SQL query, to select the tuples to be deleted. Tuples are explicitly deleted from only one table at a time. However, the deletion may propagate to tuples in other relations if referential triggered actions are specified in the referential integrity constraints of the DDL.
    \par A missing WHERE clause specifies that all tuples in the relation are to be deleted; however, the table remains in the database as an empty table. We must use the DROP TABLE command to remove the table definition.
    % \begin{minted}[linenos,tabsize=2,breaklines]{SQL}
    \begin{lstlisting}
DELETE FROM EMPLOYEE
WHERE Lname = ‘Brown’;
    \end{lstlisting}

  \hii{The UPDATE Command}
    \par The UPDATE command is used to modify attribute values of one or more selected tuples. As in the DELETE command, a WHERE clause in the UPDATE command selects the tuples to be modified from a single relation. However, updating a primary key value may propagate to the foreign key values of tuples in other relations if such a referential triggered action is specified in the referential integrity constraints of the DDL.
    \par An additional SET clause in the UPDATE command specifies the attributes to be modified and their new values.
    % \begin{minted}[linenos,tabsize=2,breaklines]{SQL}
    \begin{lstlisting}
UPDATE PROJECT
SET Plocation = ‘Bellaire’, Dnum = 5
WHERE Pnumber = 10;

UPDATE EMPLOYEE
SET Salary = Salary * 1.1
WHERE Dno = 5;
    \end{lstlisting}

\hi{Aggregate Functions}
  \par \tb{Purpose}: summarize information from multiple tuples into a single-tuple summary.

  \par \tb{List of built-in aggregate functions}:
    \begin{itemize}
      \item SUM
      \item AVG
      \item COUNT
      \item MAX
      \item MIN
    \end{itemize}

\par These functions can be used in the SELECT clause or in a HAVING clause

  \hii{Built-in Aggregate Functions}

    \hiii{SUM and AVG}
      \begin{itemize}
        \item set/multiset of numeric values
      \end{itemize}

      % \begin{minted}[linenos,tabsize=2,breaklines]{SQL}
      \begin{lstlisting}
        SELECT SUM(attribute_name)
        FROM   table_name;
      \end{lstlisting}

    \hiii{MAX and MIN}
      \begin{itemize}
        \item Comparable attributes (the domain values have a total ordering among one another)
      \end{itemize}

    \hiii{COUNT}
      \par Returns the \ti{number of tuples or values} that satisfy a condition.
      \par The asterisk refers to \ti{all rows (tuples)}.
      \par Examples:
        \begin{itemize}
          \item Simple count all
          \item Count non-distinct
          \item Count distinct
        \end{itemize}

\hi{Grouping: GROUP BY and HAVING Clauses}
  \par \tb{Objective}: Partition in to non-overlapping subsets (or groups) and apply aggregate functions.

  \hii{GROUP BY}
    \par Each group consist of the tuples that have the same value of some attributes, called \tb{grouping attributes}

    \par The grouping attributes can be defined by the \lstinline{GROUP BY} clause. The GROUP BY clause specifies the grouping attributes, which should also appear in the SELECT clause, so that the value resulting from applying each aggregate function to a group of tuples appears along with the value of the grouping attribute(s).

  \hii{HAVING}
    \par Retrieve the values of aggregate functions only for groups that satisfy certain conditions.

  \chapter{More SQL: Complex Queries, Triggers, Views, and Schema Modification}

\hi{More Complex SQL Retrieval Queries}
  \hii{Comparisons Involving NULL and Three-Valued Logic}
    \par Meanings of NULL:
      \begin{itemize}
        \item Unknown value
        \item Unavailable or withheld value
        \item Not applicable attribute
      \end{itemize}
    \par In general, each individual NULL value is considered to be different from every other NULL value in the various database records. When a record with NULL in one of its attributes is involved in a comparison operation, the result is considered to be UNKNOWN.
    \par SQL allows queries that check whether an attribute value is NULL. Rather than using = or <> to compare an attribute value to NULL , SQL uses the comparison operators IS or IS NOT . This is because SQL considers each NULL value as being distinct from every other NULL value, so equality comparison is not appropriate.
      \begin{minted}{SQL}
SELECT Fname, Lname
FROM EMPLOYEE
WHERE Super_ssn IS NULL;
      \end{minted}

  \hii{Nested Queries, Tuples, and Set/Multiset Comparisons}
    \par Some queries require that existing values in the database be fetched and then used in a comparison condition. Such queries can be conveniently formulated by using nested queries, which are complete select-from-where blocks within another SQL query. That other query is called the outer query. These nested queries can also appear in the WHERE clause or the FROM clause or the SELECT clause or other SQL clauses as needed.
    \par If a nested query returns a single attribute and a single tuple, the query result will be a single (scalar) value. In such cases, it is permissible to use = instead of IN for the comparison operator. In general, the nested query will return a table (relation), which is a set or multiset of tuples. SQL allows the use of tuples of values in comparisons by placing them within
    parentheses and allowing the IN operator.
    \begin{minted}{SQL}
SELECT DISTINCT Essn
FROM WORKS_ON
WHERE (Pno, Hours) IN
                  (
                    SELECT Pno, Hours
                    FROMWORKS_ON
                    WHERE Essn = ‘123456789’
                  );
    \end{minted}
    \par In addition to the IN operator, a number of other comparison operators can be used to compare a single value v (typically an attribute name) to a set or multiset v (typically a nested query). The = ANY (or = SOME ) operator returns TRUE if the value v is equal to some value in the set V and is hence equivalent to IN . The two keywords ANY and SOME have the same effect. Other operators that can be combined with ANY (or SOME ) include >, >=, <, <=, and <>. The keyword ALL can also be combined with each of these operators.

  \hii{Correlated Nested Queries}
    \par Whenever a condition in the WHERE clause of a nested query references some attribute of a relation declared in the outer query, the two queries are said to be correlated.
    \par In general, a query written with nested select-from-where blocks and using the = or IN comparison operators can always be expressed as a single block query.

  \hii{The EXISTS and UNIQUE Functions in SQL}
    \par EXISTS and UNIQUE are Boolean functions that return TRUE or FALSE; hence, they can be used in a WHERE clause condition.
    \par The EXISTS function in SQL is used to check whether the result of a nested query is empty (contains no tuples) or not. The result of EXISTS is a Boolean value TRUE if the nested query result contains at least one tuple, or FALSE if the nested query result contains no tuples.
    \par EXISTS and NOT EXISTS are typically used in conjunction with a correlated nested
query.
      \begin{minted}{SQL}
SELECT Fname, Lname
FROM EMPLOYEE
WHERE NOT EXISTS (
                  SELECT *
                  FROM DEPENDENT
                  WHERE Ssn = Essn
                  );
      \end{minted}
\par There is another SQL function, UNIQUE ( Q ), which returns TRUE if there are no
duplicate tuples in the result of query Q ; otherwise, it returns FALSE . This can be
used to test whether the result of a nested query is a set (no duplicates) or a multiset
(duplicates exist).

\hi{Explicit Sets and Renaming in SQL}
\par We have seen several queries with a nested query in the WHERE clause. It is also possible to use an explicit set of values in the WHERE clause, rather than a nested query. Such a set is enclosed in parentheses in SQL.

\begin{minted}{SQL}
SELECT DISTINCT Essn
FROM WORKS_ON
WHERE Pno IN (1, 2, 3);
\end{minted}

\hi{Aggregate Functions in SQL}
  \hii{GROUP BY}
    \par \lstinline{GROUP BY} is followed by \tb{an attribute}.
    \par \tb{Example}: For each department, retrieve the department number, the number of employees in the department, and their average salary.
    \begin{minted}[linenos,tabsize=2,breaklines]{SQL}
SELECT   Dno, COUNT (*), AVG (Salary)
FROM     EMPLOYEE
GROUP BY Dno;
    \end{minted}
  \hii{HAVING}

  \hii{HAVING}
  \par \lstinline{HAVING} is followed by \tb{a condition}.
  \par \tb{Example}: For each project on which more than two employees work, retrieve the project number, the project name, and the number of employees who work on the project.
    \begin{minted}[linenos,tabsize=2,breaklines]{SQL}
SELECT     Pnumber, Pname, COUNT (*)
FROM       PROJECT, WORKS_ON
WHERE      Pnumber = Pno
GROUP BY   Pnumber, Pname
HAVING     COUNT (*) > 2;
    \end{minted}


\hi{Assertions and Triggers}
  \hii{Specifying Constraints as Assertions}
    \begin{minted}[linenos,tabsize=2,breaklines]{SQL}
CREATE ASSERTION SALARY_CONSTRAINT
CHECK (
  NOT EXISTS (
    SELECT * 
    FROM EMPLOYEE E, EMPLOYEE M, DEPARTMENT D
    WHERE E.Salary > M.Salary
          AND E.Dno = D.Dnumber 
          AND D.Mgr_ssn = M.Ssn
  )
);
    \end{minted}

  \hii{Specifying Actions as Triggers}
  \par A \tb{trigger} has 3 components:
  \begin{itemize}
    \item The \tb{event}
    \item The \tb{condition}
    \item The \tb{action}
  \end{itemize}
    \begin{minted}[linenos,tabsize=2,breaklines]{SQL}  
CREATE TRIGGER SALARY_VIOLATION
BEFORE INSERT OR UPDATE OF SALARY, SUPERVISOR_SSN ON EMPLOYEE
FOR EACH ROW
  WHEN (
    NEW.SALARY > (
      SELECT SALARY
      FROM EMPLOYEE
      WHERE SSN = NEW.SUPERVISOR_SSN
    )
  )
  INFORM_SUPERVISOR(NEW.Supervisor_ssn, NEW.Ssn);
    \end{minted}
  \par \lstinline{INFORM_SUPERVISOR} is the action.
  \clearpage
\begin{multicols}{3}

\hi{MSI Logic Circuits}

  \hii{Terminology}
    \begin{itemize}
      \item \tb{Decoder}:
        \begin{itemize}
          \item \tb{Input}: $n$-bit
          \item \tb{Output}: only $\pmb{1}$ corresponding output
        \end{itemize}
      \item \tb{7-seg Decoder}:
        \begin{itemize}
          \item \tb{Common-Anode LED Display}: outputs are active-LOW
          \item \tb{Common-Cathode LED Display}: outputs are active-HIGH
        \end{itemize}
      \item \tb{Encoder}
        \begin{itemize}
          \item \tb{Input}: only 1 is active at a time
          \item \tb{Output}: $n$-bit
        \end{itemize}
      \item \tb{Priority Encoder}: the output is corresponding to the
        \tb{largest} active input.
      \item \tb{Multiplexer}: (MUX, also Data Selector)
        \begin{itemize}
          \item \tb{Input}: $n$ SELECT inputs and $\leq 2^{N}$ DATA inputs
          \item \tb{Output}: only 1 output for all
        \end{itemize}
      \item \tb{Demultiplexer}: (DEMUX, also Data Distributor)
        \begin{itemize}
          \item \tb{Input}: $n$ SELECT inputs and 1 DATA input
          \item \tb{Output}: $\leq 2^{N}$ outputs, only 1 receives the DATA input
        \end{itemize}
      \item \tb{Magnitude Comparator}: circuit to compare two $n$-bit numbers
        If the two $n$-bit input numbers are equal, the comparison will also
        based on the \tb{cascading input}.
        \par \ti{Connecting MCs: LSB first}
      \item \tb{Code Converter}: circuit to convert one type to another type
        of binary code
      \item \tb{Data Busing}: multiple device have their outputs connected to
        a common set of bus lines
        \par \ti{Only outputs of one device can be transmitted on the bus lines
        at a time. To achieve that, we use \tb{tristate registers}}.
  \end{itemize}

\end{multicols}
  % 
\hi{Terms}
  \begin{itemize}
    \item {R}
      \begin{itemize}
        \item
      \end{itemize}
  \end{itemize}



  % \begin{multicols*}{3}

\hi{Appendix 1}

  \hii{Integrals}
    \begin{itemize}
      \item \[\int x^n dx = \frac{1}{n + 1} x^{n + 1} + C \]
      \item \[\int \frac{1}{x} dx = \ln|x| + C \]
      \item \[\int \frac{1}{t^2} dx = - \frac{1}{t} + C\]
      \item \[\int e^x dx = e^x + C\]
      \item \[\int a^x dx = \frac{a^x}{\ln(a)} + C\]
      \item \[\int e^{ax} dx = \frac{e^{ax}}{a} + C\]
      \item \[\int x e^x dx = (x - 1)e^x + C\]
      \item \tb{Integration by parts}:
        \[
          \int udv = uv - v \int du
        \]
        \par Choose $u$ in this order: log - algebraic - trig - exp
    \end{itemize}

\end{multicols*}

\end{document}
