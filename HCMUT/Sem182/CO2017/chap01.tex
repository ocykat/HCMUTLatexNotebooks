\chapter{Introduction}

\hi{What Operating System Do?}

  \hii{Computer System}
    \par A \tb{computer system} consists of 4 components:
    \begin{itemize}
      \item \tb{hardware}
      \item \tb{operating system}
      \item \tb{application programs}
      \item \tb{users}
    \end{itemize}

\hi{Computer System Organization}

  \hii{Computer System Operations}
    \begin{itemize}
      \item \tb{Bootstrap program}: the initial program that run when the computer
        starts up, initializes all aspects of the system, from CPU registers to
          device controllers to memory content. It locates and loads into the
          memory the operating system kernel. It is usually stored in the ROM
          or EEPROM, known as firmware.
        \begin{itemize}
          \item ROM: read-only memory
          \item EEPROM: electrical erasable programmable read-only memory
        \end{itemize}
      \item \tb{Interrupt}: a signal from the hardware or the software.
        Hardware may trigger an interrupt any time by sending the signal to the
        CPU. Software may trigger interrupt by executing a special operation
        called a \tb{system call} (or \tb{monitor call}). When the CPU is
        interrupt, it stops what it is doing and immediately transfer execution
        to a fixed location. This location contains the starting address where
        the service routine for the interrupt is located.
      \item \tb{Interrupt vector}: a table of pointers stored in low memory
        holding the addresses of the interrupt service routine for various
        devices. The technique of using interrupt vector is to speed up
        interrupt handling.
    \end{itemize}

  \hii{Storage Structure}
    \par All forms of memory provide an \tb{array of words}. Each word has its
      own address. Interaction is achived through a sequence of \tb{load} and
      \tb{store} instructions to specific memory address.

\hi{Operating System Definition}
  \par The \tb{Operating System} is a:
  \begin{itemize}
    \item \tb{resource allocator}
    \item \tb{control program}
  \end{itemize}
