\chapter{Matrix algebra}

\hi{Matrix}

  \hii{Matrix}
    \par An $m \times n$ matrix is a table of numbers that contains
    $m$ rows and $n$ columns.
    \begin{eqbox}
      A = 
      \begin{pmatrix}
        a_{11} & a_{12} & ... & a_{1n} \\
        a_{21} & a_{22} & ... & a_{2n} \\
        ...    & ...    & ... & ...    \\
        a_{m1} & a_{m2} & ... & a_{mn}
      \end{pmatrix}
      = (a_{ij})_{m \times n}
    \end{eqbox}

  \hii{Zero Matrix}
    \par If $a_{ij} = 0 \forall i, j$, then $A$ is called a zero matrix
    and is denoted by $0$.

  \hii{Transpose}
    \par A matrix $A^{T}$ that is form from $A$ by interchaging rows by
    columns is called a \tb{transpose} of $A$.

\hi{Square Matrix}

  \hii{Square Matrix}
    \par If the number of rows equals the number of columns, then the
    matrix is called \tb{a square matrix of order $n$}.
  
  \hii{Diagonal}
    \par All elements $a_{ij}$ where $i = j$ form a \tb{diagonal} in $A$.

  \hii{Trace}
    \par The sum $\SUM{_{i = 1}^{n} a_{ii}}$ is called a \tbi{trace}
      of $A$.

  \hii{Diagonal Matrix}
    \par If $a_{ij} = 0 \quad \forall i \neq j$, then $A$ is called a
      \tb{diagonal matrix}.

  \hii{Identity Matrix}
    \par If
    \[
      a_{ij} =
      \begin{cases}
      0 \mbox{ if } i \neq j \\
      1 \mbox{ if } i = j
      \end{cases}
    \]
      then $A$ is called an \tbi{identity matrix}, denoted by $I$.
  
  \hii{Triangular Matrix}
    \par If $\forall i > j, a_{ij} = 0$, then $A$ is called an \tb{upper
      triangular matrix}.
    \par \ti{Example}:
    \[
      A =
      \begin{pmatrix}
        1 & 2 &  3 \\
        0 & 3 & -1 \\
        0 & 0 &  5 \\
      \end{pmatrix}
    \]
    \par If $\forall i < j, a_{ij} = 0$, then $A$ is called a \tb{lower
      triangular matrix}.


\hi{Matrix Operations}

  \hii{Equality}
    \[
      A = B \iff
      \begin{cases}
        \mbox{A \& B have the same size} \\
        \mbox{each corresponding entries are equal}
      \end{cases}
    \]

  \hii{Sum}
    \par The operation can only be carry out if $A$ \& $B$ have the same
    size.
    \[
      A + B = (a_{ij} + b_{ij})_{m \times n}
    \]

  \hii{Scalar multiplication}
    \[
      \alpha \times A = (\alpha \times a_{ij})_{m \times n}
    \]

  \hii{Matrix multiplication}
    \par The operation can only be carry out if the number of columns of 
    $A$ equals the number of rows of $B$.
    \par Given:
    \[
        A = (a_{ij})_{m \times n} \mbox{ \& }
        B = (b_{ij})_{n \times p}
    \]
    then
    \[
      AB = C = (c_{ij})_{m \times p}
    \]
    where each entry $c_{ij}$ in $C$ is the result of the \tb{dot product} of
    row $i$ of $A$ and column $j$ of $B$.
    \[
      c_{ij} = (i^{th} \mbox{ row of A}) . (j^{th} \mbox{ column of B})
    \]

  \hii{Matrix Exponential for Square Matrices}
    \par Let $A$ be a \tb{square matrix}.
    \[
      A^{n} =
      \begin{cases}
        I \qquad \mbox{ if } n = 0 \\
        \PROD{_{i = 1}^{n} A} \quad \mbox{ if } n \in \mathbb{N*}
      \end{cases}
    \]

  \hii{Polynomial Function of Square Matrix}
    \par Let $f(x)$ be a \tb{polynomial function}.
    \[
      f(x) = \alpha_{n}x^{n} + \alpha_{n - 1}x^{n - 1}
        + \ldots + \alpha_{1}x^{1} + \alpha{0}x^{0}
    \]
    \par Given a \tb{square matrix} A. then
    \begin{alignat*}{2}
      f(A) &= \alpha_{n}A^{n} + \alpha_{n - 1}A^{n - 1}
        + \ldots + \alpha_{1}A^{1} + \alpha_{0}A^{0} \\
      &= \alpha_{n}A^{n} + \alpha_{n - 1}A^{n - 1}
        + \ldots + \alpha_{1}A + \alpha_{0}I
    \end{alignat*}

\hi{Properties of Matrix Operations}
  \hiiBEGIN{Properties of Summation}
    \hiii{Commutative Rule}
      \[
        A + B = B + A
      \]

    \hiii{Associative Rule}
      \[
        (A + B) + C = A + (B + C)
      \]

    \hiii{Identity Rule}
      \[
        A + 0 = A
      \]

  \hiiEND

  \hiiBEGIN{Properties of Scalar Multiplication}
    \hiii{Commutative Rule}
      \[
        \alpha (A \times B) = (\alpha A) B = A (\alpha B)
      \]
    \hiii{Distributive Rule}
      \[
        \alpha (A + B) = \alpha A + \alpha B
      \]
  \hiiEND

  \hiiBEGIN{Properties of Matrix Multiplication}
    \hiii{Associative Rule}
      \[
        (A \times B) \times C = A \times (B \times C)
      \]

    \hiii{Distributive Rule}
      \begin{alignat*}{2}
        A \times (B + C) = AB + AC \\
        (A + B) \times C = AC + BC
      \end{alignat*}

    \hiii{Identity Rule}
      \par Given $A = (a_{ij})_{m \times n}$. Then:
      \[
        I_{m} \times A = A \times I_{n} = A
      \]
      or
      \[
        I \times A = A \times I = A
      \]

    \hiii{Note}
      \begin{itemize}
        \item There is no commutative rule for multiplication.
          \[
            A \times B \neq B \times A
          \]
        \item
          \[
            A \times B = 0 \not \Rightarrow A = 0 \lor B = 0
          \]
      \end{itemize}

  \hiiEND

  \hiiBEGIN{Properties of the Transpose of a Matrix}
    \hiii{Transpose Property}
      \[
        (A^{T})^{T} = A
      \]

    \hiii{Summation}
      \[
        (A + B)^{T} = A^{T} + B^{T}
      \]

    \hiii{Scalar Multiplication}
      \[
        r(A^{T}) = (rA)^{T}
      \]     

    \hiii{Matrix Multiplication}
      \[
        (A + B)^{T} = B^{T} + A^{T}
      \]
  \hiiEND


\hi{Echelon form \& Elementary Row Operations}
  \hii{Leading entry}
    \par In a matrix, the \tb{leading entry} of a row is the first non-zero
    element of that row.
  \hii{Elementary Row Operations}
    \par There are 3 types of elementary operations for matrices:
    \begin{itemize} 
      \item Scaling: multiply a row by a non-zero number.
      \item Adding: add a constant $c$ times one row to another.
      \item Interchanging: interchange two rows.
    \end{itemize} 
  \hii{Echelon form}
    \par A matrix is said to be in \tb{Echelon form} if:
    \begin{enumerate}
      \item All non-zero rows are above any rows of all zeros.
      \item Each leading entry of a row is on the right of the leading entry
        of the row above it.
    \end{enumerate}

\hi{Gaussian Elimination}
  \hii{System of Linear Equations (Linear System)}
    \par A system of $n$ linear equations and $n$ unknowns has the form of:
%    \[
      %\begin{cases}
        %\begin{aligned}
          %a_{11} x_{1} &+& a_{12} x_{2} &+& \ldots &+& a_{1n} x_{n} &=& b_{1}  \\
          %a_{21} x_{1} &+& a_{22} x_{2} &+& \ldots &+& a_{2n} x_{n} &=& b_{2}  \\
          %\vdots       &+& \vdots       &+& \vdots &+& \vdots       &=& \vdots \\
          %a_{n1} x_{1} &+& a_{n2} x_{2} &+& \ldots &+& a_{nn} x_{n} &=& b_{n}  \\
        %\end{aligned}
      %\end{cases}
    %\]
    \[
      \left\{
        \begin{aligned}
          a_{11} x_{1} &+& a_{12} x_{2} &+& \ldots &+& a_{1n} x_{n} &=& b_{1}  \\
          a_{21} x_{1} &+& a_{22} x_{2} &+& \ldots &+& a_{2n} x_{n} &=& b_{2}  \\
          \vdots       &+& \vdots       &+& \vdots &+& \vdots       &=& \vdots \\
          a_{n1} x_{1} &+& a_{n2} x_{2} &+& \ldots &+& a_{nn} x_{n} &=& b_{n}  \\
        \end{aligned}
      \right.
    \]
    \par There are 3 different senarios when a system is solved:
      \begin{itemize}
        \item The system has no solution.
        \item The system has exactly one solution.
        \item The system has more than one solution.
      \end{itemize}
    \par If the system has at least one solution, it is said to be \tb{consistent}.
    Otherwise, it is \tb{inconsistent}.

  \hii{Matrix form of a System of Linear Equations}
    \par A system of linear equations has the matrix form:
    \[
      \begin{pmatrix}
        a_{11} & a_{12} & \ldots & a_{1n} = b_{1} \\
        a_{21} & a_{22} & \ldots & a_{2n} = b_{2} \\
        \vdots & \vdots & \vdots & \vdots = \vdots \\
        a_{n1} & a_{n2} & \ldots & a_{nn} = b_{n} \\
      \end{pmatrix}
      \begin{pmatrix}
        x_{1} \\
        x_{2} \\
        \vdots \\
        x_{n} \\
      \end{pmatrix}
      =
      \begin{pmatrix}
        b_{1} \\
        b_{2} \\
        \vdots \\
        b_{n} \\
      \end{pmatrix}
    \]
    \[
      AX = B
    \]
    where
    \begin{itemize}
      \item $A$: coefficient matrix
      \item $X$: column vector of unknowns
      \item $B$: column vector of free coefficients
    \end{itemize}
    
    \par The matrix $(A|B)$ is called an \tb{augmented matrix}.

  \hii{Gaussian Elimination}
    \begin{itemize}
      \item Step 1: Write the augmented matrix $(A|B)$.
      \item Step 2: Reduce the matrix to \tb{reduced Echelon form} using only
        row elementary operations.
      \item Step 3: Solve in order from right to left: $x_{n} \to ... \to x_{1}$.
    \end{itemize}


\hi{Rank}
  \hii{Rank of a Matrix}
    \par Given matrix $A = (a_{ij})_{m \times n}$.
    \par The rank of $A$, denoted by $r(A)$, is the number of non-zero rows of $A$
    after it has been transformed to \tb{reduced Echelon form}.

  \hii{Rouche - Capelli theorem}
    \par Given a linear system $AX = B$ of $n$ unknowns.
    \begin{itemize}
      \item If $r(A|B) \neq r(A)$, then the system is inconsistent.
      \item If $r(A|B) = r(A)$, then the system is consistent.
        \begin{itemize}
          \item $r(A|B) = r(A) = n$: the system has exactly one
            solution.
          \item $r(A|B) = r(A) < n$: the system has infinitely
            many solutions.
        \end{itemize}
    \end{itemize}

\hi{Determinant of Square Matrix}
  \hii{Notation}
    \par The \tb{determinant} of a square matrix is denoted by $det(A)$ or $|A|$.
  \hii{Cofactor}
    \par Let $A$ be a square matrix and $a_{ij}$ be an element of $A$. A cofactor of
    the element $a_{ij}$ is the quantity:
    \[
      A_{ij} = (-1)^{i + j}
      \begin{bmatrix}
        \mbox{from A by deleting} \\
        \mbox{the $i^{th}$ row and} \\
        \mbox{the $j^{th}$ column}
      \end{bmatrix}
    \]

  \hii{Basic Determinant Formula}
    \par Let $A$ be a square matrix of order $n$.
    \[
      det(A) = \SUM{_{j = 1}^{n} a_{1j} A_{1j}}
    \]
    \par Note that if $n = 1$, then $det(A) = 0$.

  \hiiBEGIN{Properties of Determinant}
    \hiii{Calculating Determinant}
      \par A determinant can be computed by expanding along any row $r$ or
      down any column $c$.
      \[
        det(A) = \SUM{_{i = 1}^{n} a_{ic} A_{ic}}
               = \SUM{_{j = 1}^{n} a_{rj} A_{rj}}
      \]

    \hiii{Determinant of Triangular Matrix}
      \par A determinant of a triangular matrix is the product of all diagonal
      elements.
      \[
        A \text{ is a triangular matrix }
          \Rightarrow det(A) = \PROD{_{i = 1}^{n} a_{ii}}
      \]

    \hiii{Determinant and Elementary Row Operations}
      \begin{itemize}
        \item $A \xrightarrow[r_{i} \to \alpha r_{i}]{} B \quad \quad
          \rightarrow det(B) = \alpha det(A)$
        \item $A \xrightarrow[r_{i} \to r_{i} + \beta r_{j}]{} B \quad
          \rightarrow det(B) = det(A)$
        \item $A \xrightarrow[r_{i} \leftrightarrow r_{j}]{} B \quad \quad
          \rightarrow det(B) = - det(A)$
      \end{itemize}
      \par The same result is obtained for elementary column operations.

    \hiii{Determinant of Scalar Multiplication}
      \[
        det(\alpha \times A) = \alpha^{n} det(A)
      \]

    \hiii{Determinant of Matrix Multiplication}
      \[
        det(A \times B) = det(A) \times det(B)
      \]
      \par \tb{Note}:
      \[
        det(A \times B) = det(A) \times det(B)
      \]

    \hiii{Determinant of Transpose of a Matrix}
      \[
        det(A^{T}) = det(A)
      \]

    \hiii{Cases where determinant equals zero}
      \begin{itemize}
        \item $A$ has a zero row (column).
        \item $A$ has two identical row (column).
        \item $A$ has two propotional rows (column).
      \end{itemize}

  \hiiEND


\hi{Inverse of a Square Matrix}
  \hii{Invertible Matrix}
    \par A square matrix $A$ is invertible if there exists a square matrix $B$
    such that:
    \[
      AB = BA = I
    \]

  \hii{The Inverse of a Matrix}
    \par Matrix $B$ is called the inverse of $A$ and is denoted by $A^{-1}$.
    \par Thus: $A \times A^{-1} = I$.

  \hii{Invertibility Theorem}
    \par A square matrix $A$ is invertible if and only if $det(A) \neq 0$.

  \hiiBEGIN{Calculating the Inverse Matrix}
    \hiii{Method 1}
      \[
        A = \frac{1}{det(A)} \times P_{A}
      \]
      where
      \[
        (P_{A})^{T} =
        \begin{pmatrix}
          A_{11} & A_{12} & \ldots & A_{1n} \\
          A_{21} & A_{22} & \ldots & A_{2n} \\
          \vdots & \vdots & \vdots & \vdots \\
          A_{n1} & A_{n2} & \ldots & A_{nn} \\
        \end{pmatrix}
      \]

    \hiii{Method 2}
    \[ 
      (A|I) \xrightarrow[\text{only row operations}]{} (I|A^{-1})
    \] 
  \hiiEND

  \hiiBEGIN{Properties of Invertible Matrix}
    \hiii{The Inverse of Inverse Matrix}
      \[
        (A^{-1})^{-1} = A
      \]

    \hiii{The Inverse of a Tranpose Matrix}
      \[
        (A^{-1})^{T} = (A^{T})^{-1}
      \]

    \hiii{Scalar Multiplication}
      \[
        (\alpha A)^{-1} = \frac{1}{\alpha} \times A^{-1}
      \]
    
    \hiii{Matrix Multiplication}
      \[
        (A \times B)^{-1} = A^{-1} \times B^{-1}
      \]

    \hiii{Determinant}
      \[
        det(A^{-1}) = \frac{1}{det(A)}
      \]

    \hiii{P Matrix}
      \[
        det(P_{A}) = (det(A))^{n - 1}
      \]
      where $n$ is the order of $A$.

    \hiii{Rank}
      \begin{align*}
        det(A) \neq 0 \Rightarrow r(A) = n \\
        det(A) = 0 \Rightarrow r(A) < n \\
      \end{align*}

  \hiiEND
