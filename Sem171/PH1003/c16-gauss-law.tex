\chapter{Gauss's Law}
\hi{Electric Flux}
    \hii{Electric Flux for Uniform Electric Field}
        \par \impt{Electric flux} is propotional to the number of electric field lines
        of a uniform electric field penetrating some surface \textit{perpendicularly}.
        \begin{equation}
            \Phi_{E} = \vt{E} \cdot \vt{A} = EA\cos(90 \degree) = EA
        \end{equation}
        where:
        \begin{itemize}
            \item $\Phi_{E}$: electric flux $[N \cdot m^{2} / C]$
            \item $E$: electric field $[N / C]$
            \item $A$: the area of the surface $[m^{2}]$
        \end{itemize}
        \par If the surface is not perpendicular to the electric field, the flux through it
        equals:
        \begin{equation}
            \Phi_{E} = \vt{E} \cdot \vt {A} = EA\cos(\vt{E}, \vt{A}) = EA\cos(\theta)
        \end{equation}
    \hii{Electric Flux for Nonuniform Electric Field}
        \par In a general case where the electric field is not uniform, consider a general surface
        divided up into a large number of small elements each of area $\Dt A$. The electric flux
        through each element is:
        \begin{align*}
            \Dt \Phi_{E} = E_{i} \Dt A_{i} \cos(\theta_{i})
        \end{align*}
        \par If we let the number of elements approaches infinity, the sum of electric flux is the
        integral:
        \begin{equation}
            \sum \Phi_{E} = \lim_{\Dt A_{i} \to 0} \sum E_{i} \cdot \Dt A_{i} = \INT{E \cdot dA}
        \end{equation}
    \hii{Electric Flux through a closed surface}
        \par A \impt{closed surface} (or \impt{gaussian surface}) is defined as one that divides
        space into an inside and outside region, so that one cannot move from one region to the
        other without crossing the surface.
        \par The \impt{net flux} through the surface is proportional to the net number of lines
        \impt{leaving} the surface.
        \begin{equation} \label{eq:eflux}
            \Phi_{E} = \OINT{E \cdot dA} = \OINT{E_{n}dA}
        \end{equation}
        \par Sign:
        \begin{itemize}
            \item $\Phi_{E leave} > 0$
            \item $\Phi_{E enter} < 0$
        \end{itemize}

\hi{Gauss's Law}
    \hii{Mathematical Analysis}
        \par \textit{Consider a positive point charge $q$ located at the center of a sphere of
        radius $r$}.
        \par The field lines are directed radially outward and hence are perpendicular to the
        surface at every point on the surface. Thus, the magnitude of the electric field everywhere
        on the surface of the sphere is:
        \begin{align*}
            E = \frac{k_{e}q}{r^{2}}
        \end{align*}
        \par Also, according to the equation \eqref{eq:eflux}:
        \begin{equation} \label{eq:eflux}
            \Phi_{E} = \OINT{E \cdot dA} = E \OINT{dA}
        \end{equation}
        \par Because $E$ is the same for every point, it can be moved out of the integral.
        \par The surface is spherical:
        \begin{equation}
            \Phi_{E} = E \OINT{dA} = \frac{kq}{r^{2}} \cdot (4 \pi r^{2})
            = 4 \pi kq = \frac{q}{\epsilon_{0}}
        \end{equation}
        \par The net flux through any closed surface surrounding a point charge $q$ is given by
            $q / \epsilon_{0}$ and is \textit{independent of the shape of the surface.}
        \par The electric field due to many charges is the vector sum of the electric fields produced
        by the individual charges.
        \begin{align*}
            \OINT{E \cdot dA} = \OINT{\SUM{_{i = 1}^{n} E_{i}} \cdot dA}
        \end{align*}
    \hii{Gauss's Law}
        \par \textit{The net flux through any closed surface is}
        \begin{eqbox}
            \Phi_{E} = \OINT{E \cdot dA} = \frac{q_{in}}{\epsilon_{0}}
        \end{eqbox}
