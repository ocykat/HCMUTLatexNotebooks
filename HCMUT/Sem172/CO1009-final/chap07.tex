\begin{multicols}{3}

\hi{Counters \& Registers}
  \hii{Terminology}
    \begin{itemize}
      \item \tb{Asynchronous/Ripple Counter}: input $CLK$ connects to only
        one FF
      \item \tb{Synchronous Counter}: input $CLK$ connects to all FFs
      \item \tb{Decade Counter}: any MOD10 counter
      \item \tb{BCD Counter}: decade counter that counts from 0 to 9
      \item \tb{Ring Counter}: in a ring counter, value of $Q_{(i - 1) \mod MOD}$
        is passed to $Q_{i \mod MOD}$ after a transition. Simplest version is using
        $D$ FFs (each $Q$ is connected to $D$ of the next FF).
      \item \tb{Johnson Counter}: constructed exactly like a ring counter, except that
        $D$ of the first FF is connected to $\bar{Q}$ of the last FF.
    \end{itemize}

  \hii{Asynchronous Counter Formulas}
    \par Let $N$ be the number of FFs.
    \hiii{MOD Number}
      \begin{eqbox}
        MOD = 2^{N}
      \end{eqbox}
    \hiii{Frequency}
      \begin{eqbox}
        f_{out} = \frac{f_{in}}{MOD} = \frac{f_{in}}{2^{N}}
      \end{eqbox}
    \hiii{Propagation Delay}
      \begin{eqbox}
        T_{clock} \geq N \times t_{PD} \\
        f_{max} = \frac{1}{N \times t_{PD}}
      \end{eqbox}

  \hii{Designing Counter Problems}
    \hiii{Asynchronous Counters}
      \begin{itemize}
        \item \tb{CLK inputs}
          \begin{itemize}
            \item \tb{UP} counter: $Q$ of the previous FF is connected to
              $CLK$ of the next FF.
            \item \tb{DOWN} counter: $\bar{Q}$ of the previous FF is connected to
                $CLK$ of the next FF.
          \end{itemize}
        \item \tb{JK FF}
          \par Both $J$ input and $K$ input are connected to $1$. (toggle state)
        \item \tb{D FF}
          \par The $D$ input is connected to the $\bar{Q}$ output of the same FF.    
      \end{itemize}
      
    \hiii{Synchronous Counters}
      \begin{itemize}
        \item \tb{CLK inputs}
          \par All $CLK$ inputs of all FFs are connected to \tb{one} common
            clock signal.
        \item \tb{JK FF}
          \par Both $J$ input and $K$ input of the $i_{th}$ FF are connected to
            $Q_{0} \cdot Q_{1} \cdot \ldots \cdot Q_{i - 1}$.
            \begin{equation}
              J_{i} = K_{i} = \PROD{_{j = 0}^{i - 1} Q_{j}}
            \end{equation}
        \item \tb{D FF}
          \par The $D$ input of the $i_{th}$ FF is connected to
            $Q_{0} \cdot Q_{1} \cdot \ldots \cdot Q_{i - 1} \cdot \bar{Q}$.
            \begin{equation}
              D_{i} = \bigg( \PROD{_{j = 0}^{i - 1} Q_{j}} \bigg) \cdot \bar{Q_{i}}
            \end{equation}
        \item \tb{$\pmb{MOD < 2^{N}}$}
          \par All $Q$ outputs that are 1 if COUNTER $= MOD$ are inputs of a NAND gate.
            The output of the NAND gate is connected to all asynchronous $CLR$ input
            of all $FF$.
        \item \tb{Special Counting Sequence}
          \begin{itemize}
            \item \tb{Step 1}: Create a previous state - next state table for all
              state from $0$ to $2^{N} \geq MAX$ \ti{in binary order}.
              \par \ti{All undefined state should be the previous state of state $0$}.
            \item \tb{Step 2}: Determine the value of $J$s and $K$s (or $D$s) for
              each transition.
                \begin{tabular}{|c|c|c|}
                  \hline
                  \textbf{Transition} & \textbf{J} & \textbf{K} \\ \hline
                  $0 \to 0$           & 0          & X          \\ \hline
                  $0 \to 1$           & 1          & X          \\ \hline
                  $1 \to 0$           & X          & 1          \\ \hline
                  $1 \to 1$           & X          & 0          \\ \hline      
                \end{tabular}
            \item \tb{Step 3}: Use K-map to simplify each $J$ and $K$ (or $D$).
            \item \tb{Step 4}: Construct the circuit.
          \end{itemize}
      \end{itemize}


\end{multicols}