\chapter{Disk Storage, Basic Structures, Hashing, and Modern Storage
Architectures}

\hi{Introduction}
%  \hii{Storage Hierarchy}
    %\begin{itemize}
      %\item{Primary storage}:
      %\begin{itemize}
        %\item Storage media that can be operated on directly by the computer’s
        %central processing unit (CPU).
        %\item Provides fast access to data but is of limited storage capacity.
      %\end{itemize}
      %\item \tb{Secondary storage}
      %\item \tb{Tertiary storage}
        %\par Optical disks (CD-ROMs, DVDs, and other similar storage media) and
        %tapes.
    %\end{itemize}

  \hii{Memory Hierarchies and Storage Devices}
    \hiii{Memory Hierarchies}
      \begin{itemize}
        \item \tb{Primary Storage Level}
          \begin{itemize}
            \item Cache memory (SRAM): used to speed up execution
            \item Main memory (DRAM): low cost but volatile and lower speed
              compared to cache, where a program resides and executes.
          \end{itemize}
        \item \tb{Second and Tertiary Storage Level}: includes
          \begin{itemize}
            \item magnetic disks
            \item mass storage in the form of CD-ROM (compact disk read-only
              memory) and DVD (digital video disk) devices
            \item tapes
          \end{itemize}
      \end{itemize}
      \par \tb{Main memory database}: the case where the entire database can be kept in main memory
  (with a backup copy on magnetic disk).

    \hiii{Other Storage Devices}
      \begin{itemize}
        \item Flash Memory
        \item Optical Drives
        \item Magnetic Tapes
      \end{itemize}

  \hii{Storage Organization and Databases}
    \begin{itemize}
      \item \tb{Persistent data}: data that must persist over long periods of
        time
      \item \tb{Transient data}: data that persist for only a limited time
        during program execution
    \end{itemize}
    \par \tb{Most databases are stored permanently (or persistently) on
    magnetic disk secondary storage} for the following reasons:
    \begin{itemize}
      \item Databases are too large to fit entirely in main memory.
      \item Data loss occurs less often thanks to the \tb{nonvolatile} nature
        of magnetic disks.
      \item Cost of storage per unit of secondary storage is less than primary
        storage.
    \end{itemize}
    \par Typical database applications need only a small portion of the database at a time for
        processing. Whenever a certain portion of the data is needed, it must be located on
        disk, copied to main memory for processing, and then rewritten to the disk if the
        data is changed. The data stored on disk is organized as files of records. Each record
        is a collection of data values that can be interpreted as facts about entities, their
        attributes, and their relationships.
    \par There are several primary file organizations, which determine how the file records are physically placed on the disk, and hence how the records can be accessed. 
    \begin{itemize}
      \item A heap file (or unordered file): places the records on disk in no
        particular order by appending new records at the end of the file.
      \item A sorted file (or sequential file) keeps the records ordered by the value of a particular field (called the sort key). A hashed file uses a hash function applied to a particular field (called the hash key) to determine a record’s placement on disk. Other primary file organizations, such as B-trees, use tree structures.
    \par \tb{Primary file organization} determines how the file records are 
