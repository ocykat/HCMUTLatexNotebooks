\documentclass[12pt, a4paper]{report}

% Math supporting package
\usepackage{amsmath}
\usepackage{gensymb}
\usepackage{relsize}

% Margins
\usepackage[margin=0.75in]{geometry}

% Indent the first paragraph after a chaper/section heading
\usepackage{indentfirst} 

% Macros
\newcommand{\mul}{\cdot}
\newcommand{\vt}{\overrightarrow}
\newcommand{\avg}{\overline}
\newcommand{\ra}{\Rightarrow}
\newcommand{\Dt}{\Delta}
\newcommand{\SUM}[1]{{\sum\limits #1}}
\newcommand{\INT}[1]{{\int\limits #1}}
\newcommand{\OINT}[1]{{\oint\limits #1}}

% Formatting
\newcommand{\mRow}{\multirow}
\newcommand{\mCol}{\multicolumn}
\newcommand{\impt}[1]{\textbf{\textit{#1}}}

% Headings
\newcommand{\hi}{\section}
\newcommand{\hii}{\subsection}
\newcommand{\hiiBEGIN}[1]{\subsection{#1} \begin{enumerate}}
\newcommand{\hiiEND}{\end{enumerate}}
\newcommand{\hiii}{\item\textbf}
\newcommand{\hiiiBEGIN}[1]{\item\textbf{#1} \begin{enumerate}}
\newcommand{\hiiiEND}{\end{enumerate}}
\newcommand{\hiv}{\item\textbf}

% Tables
\newcommand{\tableBEGIN}[1]{\begin{center} \begin{tabular}{#1}}
\newcommand{\tableEND}{\end{tabular} \end{center}}

% Center aligned
\newenvironment{nscenter}
    {\parskip=0pt\par\nopagebreak\centering}
    {\par\noindent\ignorespacesafterend}
\newcommand{\CENTER}[1]{\begin{center} #1 \end{center}}
\newcommand{\NSCENTER}[1]{\begin{nscenter} #1 \end{nscenter}}

% Horizontal line for rule of inference
\newcommand{\roiLine}[1]{\overline{\quad #1 \quad}}

% Set section numbering
\setcounter{secnumdepth}{4}

% Link for TOC
\usepackage{hyperref}
\hypersetup{
    colorlinks,
    citecolor=black,
    filecolor=black,
    linkcolor=black,
    urlcolor=black
}

% Equation box
\usepackage{empheq}
\newenvironment{eqbox}{%
    \setkeys{EmphEqEnv}{align}
    \setkeys{EmphEqOpt}{box=\fbox}
    \EmphEqMainEnv%
}{%
    \endEmphEqMainEnv%
}

% Numbering for Math environments
\newcounter{mitemize}
\newcounter{mitem}[mitemize]
\newcommand{\mitemize}{\refstepcounter{mitemize}}
\newcommand{\mitem}{\refstepcounter{mitem} \themitem. \quad}

% Endline for left-align in Math environments
\newcommand{\mendl}{&&\\}

% Images
\usepackage{graphicx}
\usepackage{subcaption}
\graphicspath{ {img/} }


\begin{document}
\tableofcontents

\chapter{Motion in One Dimension}
    \hi{Position, Velocity and Speed} %1.1
        \hiiBEGIN{Position}
            \hiii{Position}
                \par A particle's position is the location of the particle with respect to a
                chosen reference point that we can consider to be the origin of a
                coordinate system.
            \hiii{Displacement}
                \par The displacement of a particle is defined as its change in position in some
                time interval.
                \begin{center}
                    \begin{equation}
                        \Delta x = x_{f} - x_{i}
                    \end{equation}
                \end{center}
                \textit{in which:}
                \begin{itemize}
                    \item $\Delta x$: displacement
                    \item $x_{i}$: initial position
                    \item $x_{f}$: final position
                \end{itemize}
        \hiiEND


\chapter{Motion in Two Dimensions}
    \hi{The Position, Velocity and Acceleration Vectors}
        \hiiBEGIN{Position}
            \hiii{Position Vector}
                \par Position vector is the vector drawn from the origin of some
                coordinate system to the particle located in the $xy$ plane.

            \hiii{Displacement Vector}
                \par Displacement vector is the vector indicating the change in position
                of a particle in some time interval.
                \begin{center}
                    \begin{equation}
                        \vt{\Delta r} = \vt{r_{f}} - \vt{r_{i}}
                    \end{equation}
                \end{center}
                \textit{in which:}
                \begin{itemize}
                    \item $\vt{\Delta r}$: displacement
                    \item $\vt{r_{i}}$: initial position
                    \item $\vt{r_{f}}$: final position
                \end{itemize}
        \hiiEND

        \hiiBEGIN{Velocity}
            \hiii{Average Velocity}
                The average velocity of a particle during the time interval $\Delta t$is 
                defined as the displacement of the particle divided by the time interval.
                \begin{center}
                    \begin{equation}
                        \avg{\vt{\Delta v}} = \frac{\vt{\Delta r}}{t}
                    \end{equation}
                \end{center}

            \hiii{Instantaneous Velocity}
                The instantaneous velocity of a particle is defined as the limit of the
                average velocity $\frac{\Delta r}{\Delta t}$ as $\Delta t$ approaches zero:
                \begin{center}
                    \begin{equation}
                        v = \lim_{x \to 0} \frac{\Delta r}{\Delta t} = \frac{dx}{dt}
                    \end{equation}
                \end{center}
        \hiiEND

        \hiiBEGIN{Acceleration}
            \hiii{Average Acceleration}
                The average acceleration of a particle is defined as the change in the
                instantaneous velocity vector $\vt{\Delta v}$ divided by the time interval
                $\Delta t$ during which that change occurs.
                \begin{center}
                    \begin{equation}
                        \avg{\vt{\Delta a}} = \frac{\vt{\Delta v}}{t}
                    \end{equation}
                \end{center}

            \hiii{Instantaneous Acceleration}
                The instantaneous acceleration of a particle is defined as the limit of the
                average acceleration $\frac{\Delta v}{\Delta t}$ as $\Delta t$ approaches
                zero:
                \begin{center}
                    \begin{equation}
                        a = \lim_{t \to 0} \frac{\Delta v}{\Delta t} = \frac{dv}{dt}
                    \end{equation}
                \end{center}
        \hiiEND
    
    \pagebreak

    \hi{Two-Dimensional Motion with Constant Acceleration}
        \hii{Position vector}
            \par When a particle moves with constant acceleration, the position vector for it
            in the $xy$ plane can be written as
            \begin{equation}
                r = x \hat{i} + y \hat{j}
            \end{equation}
            where $x$, $y$ and $r$ change with time as the particle moves while the unit
            vector $\hat{i}$ and $\hat{j}$ remain constant.

        \hii{Velocity vector}
            \begin{equation}
                v = \frac{dr}{dt} = \frac{dx}{dt} \hat{i} + \frac{dy}{dt} \hat{j}
                = v_{x} \hat{i} + v_{y} \hat{j}
            \end{equation}
            \par Because $a$ is constant:
            \begin{equation}
                \begin{aligned}
                    v_{f} &= (v_{xi} + a_{x}t) \hat{i} + (v_{yi} + a_{y}t) \hat{j} \\
                    &= (v_{xi} \hat{i} + v_{yi} \hat{j}) + (a_{x} \hat{i} + a_{y} \hat{j})t \\
                    &= v_{i} + at
                \end{aligned}
            \end{equation}

        \hii{Equations of kinematics}
            \begin{equation}
                r_{f} = r_{i} + v_{i}t + \frac{1}{2} at^{2}
            \end{equation}

    \pagebreak

    \hi{Projectile Motion}
        \hii{Definition}
            \par A projectile motion is a motion in which two assumptions are made:
            \begin{itemize}
                \item The free-fall acceleration $g$ is constant over the range of motion
                and is directed downward.
                \item The effect of air resistance is negligible.
            \end{itemize}

        \hii{Velocity}
            \par At the beginning, the velocity components are:
            \begin{equation}
                \begin{aligned}
                v_{xi} = v_{i} \cos \theta_{i} \qquad
                v_{yi} = v_{i} \sin \theta_{i}
                \end{aligned}
            \end{equation}

            \par After some time interval $t$, these components become:
            \begin{equation}
                \begin{aligned}
                v_{xi} = v_{i} \cos \theta_{i} \qquad
                v_{yi} = v_{i} \sin \theta_{i} - gt
                \end{aligned}
            \end{equation}

        \hii{Position}
            \begin{equation} 
                \label{eq:01}
                x_{f} = x_{i} + v_{ix}t
            \end{equation}

            \begin{equation}
                \label{eq:02}
                y_{f} = y_{i} + v_{iy}t - \frac{1}{2} gt^{2}
            \end{equation}

            \par If the particle starts from the origin of the referene frame, which
            means $x_{i} = 0$ and $y_{i} = 0$, then:
            \begin{equation} 
                \label{eq:03}
                x = v_{ix}t = v_{i}\cos\theta_{i}t
            \end{equation}

            \begin{equation}
                \label{eq:04}
                y = v_{iy}t - \frac{1}{2}gt^{2} = v_{i}\sin\theta_{i}t - \frac{1}{2}gt^{2}
            \end{equation}


            \par From the equation \eqref{eq:03}:
            \begin{equation}
                t = \frac{x}{v_{i}\cos\theta_{i}}
            \end{equation}

            \par Substitute $t$ in the equation \eqref{eq:04}:
            \begin{equation}
                \begin{aligned}
                \label{eq:05}
                    y &= v_{i}\sin\theta_{i} \mul \frac{x}{v_{i}\cos\theta_{i}}
                    - \frac{1}{2}g (\,\frac{x}{v_{i}\cos\theta_{i}})\,^{2} \\
                    &= x\tan{\theta_{i}} - \frac{gx^{2}}{2v^{2}_{i}\cos^{2}\theta_{i}}
                \end{aligned}
            \end{equation}

            Because the equation \eqref{eq:05} is in the form of a quadratic equation,
            the path is parabolic.

    \pagebreak

    \hi{Uniform Circular Motion}
        \hii{Definition}
            Uniform circular motion can be described as the motion of an object in a circle
            at a constant speed. As an object moves in a circle, it is constantly changing
            its direction. At all instances, the object is moving tangent to the circle.

        \hii{Centripetal Acceleration}
            \begin{equation}
                a_{c} = \frac{v^{2}}{r}
            \end{equation}

        \hii{Period}
            \begin{equation}
                T = \frac{2\pi r}{v}
            \end{equation}

    \pagebreak
    
    \hi{Tangential and Radial Acceleration}
        \hii{Definition}
            \par When a particle moves along a curved path, the total acceleration $\vt{a}$ 
            changes from point to point. This vector can be resolved into two components:
            \begin{itemize}
                \item A radial component $\vt{a_{r}}$
                \item A tangential component $\vt{a_{r}}$
            \end{itemize}

        \hii{Model circle}
            \par At each point on the path, there exists a model circle on which the radial
            component and a tangential component are defined.

        \hii{Radial Acceleration}
            \par The Radial Acceleration lies along the radius of the model circle, arises from
            the change in direction of the velocity vector.
            \begin{equation}
                a_{r} = -a{c} = -\frac{v^{2}}{r}
            \end{equation}
            \par The negative sign indicates that the direction of the centripetal acceleration
            is toward the center of the circle representing the radius of curvature, which
            is opposite the direction of the radial unit vector $\hat{r}$.

        \hii{Tangential Acceleration}
        \par The Tangential Acceleration is perpendicular to the radius, parallel to the
        instantaneous velocity, and causes the change in the speed of the particle.
        \begin{equation}
            a_{t} = \frac{d|v|}{dt}
        \end{equation}

    \pagebreak 

    \hi{Relative Velocity and Relative Acceleration}
        \par Consider a particle which is being observed by two observers:
        \begin{itemize} 
            \item One in reference frame S, fixed relative to Earth
            \item One in reference frame S', moving relative to S with a constant velocity
                $\vt{v_{0}}$
        \end{itemize} 

        \par Suppose at t = 0, the origins of the two reference frames coincide in space. Thus,
        at time t, the origins of the referene frames will be separated by a distance
        $v_{0}t$.
        \par If we denote:
        \begin{itemize}
            \item $\vt{r}$ as the position vector of the particle related to the origin of S.
            \item $\vt{r'}$ as the position vector of the particle related to the origin of S'.
        \end{itemize} 
        \par then we have:
        \begin{equation}
        \label{eq:06}
            r' = r - v_{0}t
        \end{equation}

        \par By differentiating equation \eqref{eq:06} and note that $v_{0}$ is constant, we obtain:
        \begin{equation}
        \label{eq:07}
            \begin{aligned}
                & \frac{dr'}{dt} = \frac{dr}{dt} - v_{0} \\
                & \ra v' = v - v_{0}
            \end{aligned}
        \end{equation}

        \par The equations \eqref{eq:06} and \eqref{eq:07} are known as \textbf{Galilean
        transformation equations}.

        \par Taking the time derivative of equation \eqref{eq:07}, we obtain:
        \begin{equation}
            \frac{dv'}{dt} = \frac{dv}{dt} - \frac{dv_{0}}{dt}
        \end{equation}
        \par Because $v_{0}$ is constant, $\dfrac{dv_{0}}{dt} = 0$. Therefore, the acceleration of
        the particle measured by an observer in one frame of the reference is the same as
        that measured by any other observer \textit{moving with constant velocity related to
        the first frame}.
        \begin{equation}
            a' = a.
        \end{equation}

    \pagebreak


\chapter{The laws of motion}

    \hi{The concept of Force}

        \hii{Definition}
            \begin{itemize}
                \item A force cause an object to \textit{accelerate}.
                \item The \textbf{net force} acting on an object is defined as the vector
                sum of all forces acting on the object. If the net force exerted on the
                object is zero, the acceleration of the object is zero and its velocity
                remains constant.
                \item When the velocity of an object is constant (including when the object
                is at rest), the object is said to be in \textbf{equilibrium}.
            \end{itemize}
            
        \hii{Two types of forces}
            \par There are two types of forces:
            \begin{itemize}
            \item \textbf{Contact forces} which involve physical contact between two objects.
            \item \textbf{Field forces} which do not involve physical contact between two
                objects.
            \end{itemize}

        \hii{Four fundamental forces}
            \begin{itemize}
                \item Gravitational forces
                \item Electromagnetic forces
                \item Nuclear forces
                \item Weak forces
            \end{itemize}

    \hi{Newton's Laws}
        \hii{Newton's First Law and Inertial Frame}
            \par Newton's First Law of motion, sometimes called the \textit{law of inertia},
            defines a special set of reference frames called \textit{inertial frames}. The
            law can be stated as follow:
            \par \textit{If an object does not interact with other objects, it is possible
            to identify a reference frame in which the object has zero acceleration}.
            \par Such inertial frame is called an \textbf{inertial frame of reference}.
            \par Any reference frame that moves with constant velocity relative to an
            inertial frame is itself an inertia frame.
            \par The law can also be stated in a more practical way:
            \par \textit{In the absence of external forces, when viewed from an inertial
            reference frame, an object at rest remains at rest and an object in motion
            continues in motion with a constant velocity (that is, with a constant speed in
            a straight line)}.

        \hii{Newton's Second Law}
            \par \textit{When viewed from an inertial reference frame, the acceleration of an
            object is directly propotional to the net force acting on it and inversely propotional
            to its mass}.
            \begin{equation}
                \sum F = ma
            \end{equation}
            \par in which:
                \begin{itemize}
                    \item $\sum F$: net force $[N]$ or $[kg \mul m/s]$
                    \item $m$: mass $[kg]$
                    \item $a$: acceleration $[m/s^2]$
                \end{itemize}

        \hii{Newton's Third Law}
            \par \textit{If two objects interact, the force $F_{12}$ exerted by object 1 on object
            2 is equal in magnitude and opposite in direction to the force $F_{21}$ exerted by
            object 2 on object 1.}
            \begin{equation}
                F_{12} = -F_{21}
            \end{equation}

        \hii{Problem Solving Hints}
            \par Procedure:
            \begin{itemize}
                \item Draw a diagram to help conceptualize the problem.
                \item Categorize the problem:
                    \begin{itemize}
                        \item If $a = 0$, the particle is in equilibrium.
                            This means: $\sum F = 0$.
                        \item If $a \neq 0$, the particle is undergoing an acceleration.
                            This means: $\sum F = ma$.
                    \end{itemize}
                \item Establish convenient coordinate axes for each object and find the components
                    of the force in these axes. (making a table is a good idea).
                \item Apply Newton's second law in component form.
                \item Solve the component equations for the unknowns. The number of unknowns and
                    the number of independent equations must be the same.
                \item Point out the direction of the unknown vectors (if any). \textit{(Example:
                    The direction of $\vt{a}$ relative to the positive $x$ axis is: $45 \degree$)}.
            \end{itemize}

    \hi{Force of Friction}
        \hii{Normal Force}
            \par The normal force is defined as the net force compressing two parallel surfaces
            together, and its direction is perpendicular to the surfaces.
        \hii{Force of Friction}
            \par When an object is in motion either on a surface or in a viscous medium such as air
            or water, there is resistance to the motion because the object interacts with its
            surroundings. We call such resistance a \textbf{force of friction}.
        \hii{Static Friction}
            \par Static friction is friction between two or more solid objects that are not
            moving relative to each other.
            \par The static friction force must be overcome by an applied force before an object
            can move. The maximum possible friction force between two surfaces before sliding
            begins is the product of the coefficient of static friction and the normal force.
            \begin{equation}
                F_{s-max} = \mu_{s} F_{n}
            \end{equation}
            in which:
            \begin{itemize}
                \item $F_{max}$: maximum possible friction force $[N]$
                \item $\mu_{s}$: coefficient of static friction
                \item $F_{n}$: normal force $[N]$
            \end{itemize}
        \hii{Kinetic Friction}
            \par Kinetic friction occurs when two objects are moving relative to each other and rub
            togeter.
            \par The kinetic friction force between two surfaces after sliding begins is the product
            of the coefficient of kinetic friction and the normal force.
            \begin{equation}
                F_{k} = \mu_{k} F_{n}
            \end{equation}
            in which:
            \begin{itemize}
                \item $F_{max}$: maximum possible friction force $[N]$
                \item $\mu_{s}$: coefficient of static friction
                \item $F_{n}$: normal force $[N]$
            \end{itemize}

\pagebreak


\chapter{Circular Motion and Other Applications of Newton's Laws}
    \hi{Uniform Circular Motion}
        \hii{Centripetal acceleration}
            \par When a particle moving with uniform speed $v$ in a circular path of radius $r$
            experiences an acceleration that has a magnitude:
            \begin{equation}
                a_{c} = \dfrac{v^2}{r}
            \end{equation}
            \par The acceleration is called \impt{centripetal acceleration} because it is directed
            toward the centre of the circular path.
        \hii{Centripetal force}
            \par Applying Newton's second law along the radial direction, the net force causing
            the centripetal acceleration can be evaluated:
            \begin{equation}
                \sum F = ma_{c} = \dfrac{mv^2}{r}
            \end{equation}
    
    \hi{Nonuniform Circular Motion}
        \par If a particle moves with varying speed in a circular path, in addition to the
        \impt{radial component} of acceleration, there is a \impt{tangential component} having
        the magninude:
        \begin{equation}
            a_{t} = \dfrac{dv}{dt}
        \end{equation}
        and responsible for the change in the speed of the particle with time.
        \par Because:
        \begin{equation}
            \sum a = a_{r} + a_{t}
        \end{equation}
        applying the Newton's second law, we have:
        \begin{equation}
            \sum F = F_{r} + F_{t}
        \end{equation}

    \hi{Motion in Accelerated Frames}
        \hii{Validity of Newton's laws}
            \par Newton's laws are only valid only when observations are made in an inertial
            frame of reference.

        \hii{Non-inertial Frame of Reference}
            \par A \impt{non-inertial} reference frame is a frame of reference that is undergoing
            acceleration with respect to an inertial frame.

        \hii{Fictitous Force}
            \par A \impt{fictitous force} (also called \impt{inertial force}) is an apparent
            force that acts on all masses whose motion is described using a non-inertial frame
            of reference.

            \par In a coordinate system S' which moves by translation relative to an inertial
            system S, the motion of a mechanical system takes place as if the coordinate system
            were inertial, but on every point of mass m an additional "inertial force" acted:
            \begin{equation}
                F = -ma
            \end{equation}
            where a is the acceleration of the system S'.

        \hii{Equilibrium in an Accelerated Frame}
            \par A \impt{fictitous force} (also called \impt{inertial force}) is an apparent
            force that acts on all masses whose motion is described using a non-inertial frame
            of reference.
            \par When the frame accelerates with the acceleration $a$, if the object in the frame
            is in equilibrium, then according to an inertial observer, there is a force $F$
            exerting on the object with the magnitude:
            \begin{equation}
                \sum F = ma
            \end{equation}

    \hi{Motion in Presence of Resistive Forces}

\chapter{Energy and Energy Transfer}
    \hi{System and Environment}
        \par In physics, a \impt{physical system} is a portion of the physical universe chosen for
        analysis. Everything outside the system is known as the environment. The environment
        is ignored except for its effects on the system.

    \hi{The Scalar Product of Two Vectors}
        \par The \impt{scalar product}, or \impt{dot product} of any two vectors $\vt{a}$ and
        $\vt{b}$ is a scalar quantity equal to the product of the magnitudes of the two vectors
        and the cosine of the angle $\theta$ between them.
        \begin{equation*}
            \vt{a} \mul \vt{b} = |\vt{a}||\vt{b}|\cos \theta
        \end{equation*}

    \hi{Work done by a Constant Force}
        \par The \impt{work} $W$ done on a system by an agent exerting a constant force on the system
        is the dot product of the magnitude $F$ of the force, the magnitude $\Delta r$ of the
        displacement of the point of application of the force, and $\cos \theta$, where $\theta$ is
        the angle between the force and displacement vectors:
        \begin{equation}
            W = \vt{F} \mul \vt{r} = F \Delta r \cos \theta
        \end{equation}
        in which:
        \begin{itemize}
            \item $W$: work done by the agent $[J]$ or $[N \mul m]$
            \item $F$: force exerted by the agent [N]
            \item $r$: magnitude of the displacement [m]
            \item $\theta$: angle between the force and displacement vector [rad]
        \end{itemize}

    \hi{Work Done by a Varying Force}
        \par The total work done by the net force $ \sum F_{x}$ as the particle moves from $r_{i}$
        to $r_{f}$:
        \begin{equation} \label{eq:work01}
            \sum W = \INT{_{x_{i}}^{x_{f}} \sum \vt{F}d\vt{r}}
        \end{equation}

    \hi{Kinetic Energy}
        \par \impt{Kinetic energy} $K$ of a particle of mass $m$ moving with a speed $v$ is defined
        as:
        \begin{equation} \label{eq:kinetic01}
            K = \dfrac{1}{2}mv^{2}
        \end{equation}

    \hi{Kinetic Energy and the Work-Kinetic Energy Theorem}
        \par Based on the equation \eqref{eq:work01}, we have:
        \begin{align*}
            \begin{split}
                \sum W & = \INT{_{x_{i}}^{x_{f}} \sum Fdx} \\
                & = \INT{_{x_{i}}^{x_{f}} \sum madx} \mbox{\qquad (the Second Newton's law)} \\
                & = \INT{_{x_{i}}^{x_{f}} \sum m \dfrac{dv}{dt}dx} \\
                & = \INT{_{v_{i}}^{v_{f}} \sum mvdv} \\
                & = \dfrac{1}{2}mv_{f}^{2} - \dfrac{1}{2}mv_{i}^{2} \\
            \end{split}
        \end{align*}
        \par In combination with the equation \eqref{eq:kinetic01}, we conclude that:
        \begin{equation}
            \sum W = K_{f} - K_{i} = \Delta K \\
        \end{equation}

    \hi{Isolated and Nonisolated System}
        \par If a system does not interact with its environment, it is an \impt{isolated system}.
        \par On the other hand, if the system is acted on by various forces, resulting in a change
        in its kinetic energy, it is a \impt{nonisolated system}. 
        \par The energy associated with an object's temperature is called \impt{internal energy}. 
        If positive work has been done on an object but there is no increase in the object's
        kinetic energy, then the energy has been transformed into internal energy. In this case, the
        work-kinetic energy theorem cannot be applied.

    \hi{Power}
        \hii{Average Power}
            \begin{equation}
                \avg{P} = \dfrac{W}{t}
            \end{equation}
        \hii{Instantaneous Power}
            \begin{equation}
                P = \dfrac{dW}{dt} = F \mul \dfrac{dr}{dt} = F \mul v
            \end{equation}

\chapter{Potential Energy}
    \hi{Conservative and Nonconservative Forces}
        \par \impt{Conservative forces} have these two equivalent properties:
        \begin{itemize}
            \item The work done by a conservative force on a particle moving between any two
                points is independent of the path taken by the particle.
            \item The work done by a conservative force on a particle moving through any closed
                path is zero.
        \end{itemize}

   \hi{Gravitational Potential Energy}
        \begin{equation}
            U_{g} = mgy
        \end{equation}
        in which:
        \begin{itemize}
            \item $U_{g}$: gravitational potential energy $[J]$
            \item $m$: the mass of the object $[kg]$
            \item $g$: gravitational acceleration $[m/s^{2}]$
            \item $y$: the height of the object according to the reference configuration $[m]$
        \end{itemize}
        \par \impt{Note:} When dealing with problems related to gravitational energy, one
        must choose a reference configuration for which the gravitational potential energy
        is set equal to some value, which is normally zero. The choice of reference configuration
        is completely arbitrary because the important quantity is the \textit{difference} in
        potential energy and this difference is independent of the choice of reference configuration.

    \hi{Relationship between Work done and The Change in Gravitational Potential Energy}
        \begin{equation}
            W = \Delta U = U_{f} - U_{i}
        \end{equation}

    \hi{Isolated System}
        \par In an isolated system, the only type of forces allowed is \impt{conservative forces}.

    \hi{Mechanical Energy}
        \par \impt{Mechanical Energy} is the sum of $kinetic$ and $potential$ energies.
        \begin{equation}
            E_{mech} = K + U_{g}
        \end{equation}

    \hi{Conservation of Mechanical Energy}
        \par In an \impt{isolated system}:
        \begin{equation*}
            E_{i} = E_{f}
        \end{equation*}
        \begin{equation}
            \ra K_{i} + U_{i} = K_{f} + U_{f} 
        \end{equation}

    \hi{Elastic Potential Energy}
        \begin{equation}
            U_{s} = \dfrac{1}{2}kx^{2}
        \end{equation}

\chapter{Linear Momentum and Collisions}
    \hi{Linear Momentum}
        \hii{Linear Momentum}
            \par The \impt{linear momentum} of a particle or an object that can be modeled as a
            particle of mass $m$ moving with a velocity $v$ is defined to be the product of the mass
            and velocity:
            \begin{equation}
                \vt{p} = m\vt{v}
            \end{equation}
            \par Using Newton's Second law of motion, we can relate the linear momentum of a particle
            to the resultant force acting on the particle:
            \begin{align*}
                \sum F = ma = m \dfrac{dv}{dt}
            \end{align*}
            \par Since the mass $m$ is assumed to be constant:
            \begin{align*}
                \sum F = ma = \dfrac{d(mv)}{dt} = \dfrac{dp}{dt}
            \end{align*}
            \par This shows that the \textit{time rate of change of the linear momentum of a particle
            is equal to the net force acting on the particle}. This is the way Newton originally
            showed his Second law of motion.
        \hii{Conservation of Linear Momentum}
            \par Suppose that in an isolated system, there are two particles with mass $m_{1}$ and
            $m_{2}$ moving with velocity $v_{1}$ and $v_{2}$ at an instant of time. When the two
            particles collide, according to the Newton's Third law of motion:
            \begin{align*}
                F_{12} + F_{21} = 0
            \end{align*}
            \par In combination with the Newton's Second law, we have:
            \begin{flalign*}
                & m_{1}a_{1} + m_{2}a_{2} = 0 \\
                \ra & m_{1} \dfrac{dv_{1}}{dt} + m_{2} \dfrac{dv_{2}}{dt} = 0 \\
                \ra & \dfrac{d}{dt} (m_{1}v_{1} + m_{2}v_{2}) = 0 \\
                \ra & \dfrac{d}{dt} (p_{1} + p_{2}) = 0 \\
                \ra & \dfrac{dp_{system}}{dt} = 0
            \end{flalign*}
            \par The rate of change in the total momentum of the system is $0$ over time. Therefore,
            the total momentum is conserved.
            \begin{equation}
                \Delta p = 0
            \end{equation}
            or
            \begin{equation}
                p_{1i} + p_{2i} = p_{1f} + p_{2f} 
            \end{equation}
    \hi{Impulse}
        \par The \impt{impulse} of the force $F$ acting on a particle equals the change in the
        momentum of the particle.
        \par According to the Newton's Second law:
        \begin{flalign*}
            & dp = Fdt \\
            \ra & \Delta p = p_{f} - p_{i} = \INT{_{t_{i}}^{t_{f}} Fdt}
        \end{flalign*}
        \par According to the definition of impulse:
        \begin{equation}
            I = \Delta p = \INT{_{t_{i}}^{t_{f}} Fdt}
        \end{equation}
        \par If the impulse is constant:
        \begin{equation}
            I = F \Delta t
        \end{equation}

    \hi{Collisions in One Dimension}
        \hii{Elastic Collision}
            \par An \impt{elastic collision} between two objects is one in which \impt{the total
            kinetic energy} of the system \impt{is the same} before and after the collision.
        \hii{Inelastic Collision}
            \par An \impt{inelastic collision} between two objects is one in which \impt{the
            total kinetic energy} of the system \impt{is not the same} before and after the collision.
            \par When the colliding objects stick together after the collision, the collision is
            called \impt{perfectly inelastic}.

    \hi{Collisions in Two Dimensions}

    \hi{The Center of Mass}
        \par In a \impt{system of particles}, the position of the \impt{center of mass} is given
        by the equation:
        \begin{equation}
            r_{CM} = \dfrac{\SUM{_{i} m_{i}r_{i}}}{M}
        \end{equation}
        in which:
        \begin{itemize}
            \item $r_{CM}$: the position vector of the center of mass
            \item $m_{i}$: the mass of each particle
            \item $r_{i}$: the position vector of each particle
            \item $M$: the total mass of the system
        \end{itemize}
        \par In the case of an \impt{extended object}, the particle separation is very small, and
        so each particle has a very small mass.
        \begin{equation}
            r_{CM} = \lim_{\Delta m_{i} \to 0} \dfrac{\SUM{_{i} r_{i} \Delta m_{i}}}{M}
            = \dfrac{1}{M} \INT{rdm} \\
        \end{equation}

    \hi{Motion of a System of Particles}
        \hii{The Momentum of the System}
            \begin{equation} \label{eq:sysMomentum01}
                v_{CM} = \dfrac{dr_{CM}}{dt} = \dfrac{1}{M} \SUM{_{i} m_{i}} \dfrac{dr_{i}}{dt}
                = \dfrac{\SUM{_{i} m_{i}v_{i}}}{M}
            \end{equation}
            \begin{equation}
                \ra Mv_{CM} = \SUM{_{i} m_{i}v_{i}} = \SUM{_{i} p_{i}} = \SUM{p}
            \end{equation}
            \par \textit{Conclusion}: The \impt{total linear momentum} of a system equals the
            \impt{total mass} multiplied by the \impt{velocity of the center of mass}.
        \hii{The Force of the System}
            \par By differentiating the equation \eqref{eq:sysMomentum01}, we obtain:
            \begin{equation}
                a_{CM} = \dfrac{dv_{CM}}{dt} = \dfrac{1}{M} \SUM{_{i} m_{i}} \dfrac{dv_{i}}{dt}
                = \dfrac{\SUM{_{i} m_{i}a_{i}}}{M}
            \end{equation}
            \begin{equation}
                \ra Ma_{CM} = \SUM{_{i} m_{i}a_{i}} = \SUM{_{i} F_{i}} = \SUM{F}
            \end{equation}
            \par \textit{Conclusion}: The \impt{net external force} on a system of particles
            equals the \impt{total mass of the system} multiplied by the \impt{acceleration of
            the center of mass}.
        \hii{Motion of the Center of Mass}
            \par The \impt{Center of Mass} of a system of particles of combined mass $M$ moves
            like an equivalent particle of mass $M$ would move under the influence of the net
            external force on the system.

\chapter{Rotation of a Rigid Object About a Fixed Axis}
    \hi{Rigid Objects}
        \par A \impt{rigid object} is one that is \impt{nondeformable} - that is, the relative
        locations of all particles of which the object is composed remain constant.
        \par In dealing with a rotating object, analysis is greatly simplified by assuming that
        the object is rigid.
        \par When a rigid object rotates about a fixed axis, every particle of the object
        moves in a circle whose center is the axis of rotation.

    \hi{Angular Position, Velocity, and Acceleration}
        \hiiBEGIN{Angular Position}
            \hiii{Angular position}
            \par The \impt{angular position} of the rigid object is the angle $\theta$ between
            \impt{a reference line} on the object and a fixed reference line in space.

            \hiii{Angular displacement}:
            \begin{equation}
                \Delta \theta = \theta_{f} - \theta_{i}
            \end{equation}
        \hiiEND

        \hiiBEGIN{Angular Speed \& Velocity}
            \hiii{Average Angular Speed}
                \begin{equation}
                    \avg{\omega} = \dfrac{\Delta \theta}{\Delta t}
                \end{equation}
            \hiii{Instantaneous Angular Speed}
                \begin{equation}
                    \omega = \dfrac{d \theta}{dt}
                \end{equation}
            \hiii{Angular Velocity}
                \par Angular velocity is a vector quantity. Its magnitude is the angular speed
                and its direction is determined using the \impt{right hand rule}.
        \hiiEND

        \hiiBEGIN{Angular Acceleration}
            \hiii{Average Angular Acceleration}
                \begin{equation}
                    \avg{\alpha} = \dfrac{\Delta \omega}{\Delta t}
                \end{equation}
            \hiii{Instantaneous Angular Speed}
                \begin{equation}
                    \alpha = \dfrac{d \omega}{dt}
                \end{equation}
        \hiiEND

    \hi{Rotational Motion with Constant Angular Acceleration}
        \begin{equation}
            \omega_{f} = \omega_{i} + \alpha t
        \end{equation}
        \begin{equation}
            \theta_{f} = \theta_{i} + \omega_{i}t + \dfrac{1}{2}\alpha t^{2}
        \end{equation}
        \begin{equation}
            \omega_{f}^2 = \omega_{i}^2 + 2\alpha(\theta_{f} - \theta_{i})
        \end{equation}
        \begin{equation}
            \theta_{f} = \theta_{i} + \dfrac{1}{2}(\omega_{i} + \omega_{f})t
        \end{equation}

    \hi{Angular and Linear Quantities}
        \hii{Velocity}
            \begin{equation}
                v = \dfrac{ds}{dt} = r \dfrac{d \theta}{dt} = r \omega
            \end{equation}
            in which:
            \begin{itemize}
                \item $r$: perpendicular distance to the specified axis
            \end{itemize}

        \hiiBEGIN{Acceleration}
            \hiii{Tangential Acceleration}
                \begin{equation}
                    a_{t} = \dfrac{dv}{dt} = r \dfrac{d \omega}{dt} = r \alpha
                \end{equation}
            \hiii{Centripetal Acceleration}
                \begin{equation}
                    a_{c} = \dfrac{v^{2}}{r} = r \omega^{2}
                \end{equation}
            \hiii{Total Linear Acceleration}
                \begin{equation}
                    a = \sqrt{a_{t}^2 + a_{c}^2} = \sqrt{(r \alpha)^{2} + (r \omega^{2})^{2}} = r \sqrt{\alpha^{2} + \omega^{4}}
                \end{equation}
        \hiiEND

    \hi{Rotational Kinetic Energy}
        \par When an object is rotating, each of its particle has kinetic energy determined by its
        mass and tangential speed.
        \begin{equation}
            K_{i} = \dfrac{1}{2} m_{i}v_{i}^{2}
        \end{equation}
        \par The total kinetic energy of the object:
        \begin{equation} \label{eq:rke}
            \sum K = \SUM{_{i} K_{i}} = \SUM{\dfrac{1}{2} m_{i}v_{i}^{2}}
            = \SUM{\dfrac{1}{2} m_{i}r_{i}^{2} \omega_{i}^{2}}
            = \dfrac{1}{2} \omega_{i}^{2} \SUM{m_{i}r_{i}^{2}}
        \end{equation}
        \par The equation \eqref{eq:rke} can be simplify by defining a new quantity: the
        \impt{moment of inertia}.
        \begin{equation}
            I = \SUM{_{i} m_{i}r_{i}^{2}}
        \end{equation}
        \begin{equation}
            \sum K = \dfrac{1}{2} I \omega^{2}
        \end{equation}

    \hi{Calculation of Moments of Inertia}
        \hii{The General Way}
            \par If the extended rigid object is divided into many small volume elements:
            \begin{equation}
                I = \lim_{\Delta m_{i} \to 0} \SUM{r_{i}^{2} \Delta m_{i}}
                = \INT{r^{2} dm}
            \end{equation}
            \par We also have:
            \begin{flalign*}
                & \rho = \dfrac{m}{V} \\
                \ra & \rho = \dfrac{dm}{dV} \\
                \ra & dm = \rho dV 
            \end{flalign*}
            in which: $\rho$ is the \impt{volumetric mass density}
            \par Therefore:
            \begin{equation}
                I = \INT{\rho r^{2} dV}
            \end{equation}
        \hii{More Complicated Cases}
            \par It is relatively easy to calculate the moments of inertia of objects with:
                \begin{itemize}
                    \item Simple geometry.
                    \item The \impt{rotation axis} coincide with an \item{axis of symmetry}.
                \end{itemize}
            \par To calculate the moment of inertia about an arbitrary axis, one can use
            the \impt{parallel-axis theorem}, also known as \impt{Huygens - Steiner} theorem:
            \par \textit{Suppose the moment of inertia about an axis through the center of
            mass of an object is $I_{CM}$. The moment of inertia about any axis parallel to
            and a distance $D$ away from this axis is:}
            \begin{equation}
                I = I_{CM} + MD^{2}
            \end{equation}
            in which:
            \begin{itemize}
                \item $I$: moment of inertia about the arbitrary axis
                \item $I_{CM}$: moment of inertia about the parallel axis containing
                    the center of mass
                \item $M$: total mass of the object
                \item $D$: distance between the two axes
            \end{itemize}
                
    \hi{Torque}
        \hii{Torque - Definition \& Magnitude}
            \begin{itemize}
                \item \impt{Torque}, \impt{moment}, or \impt{moment of force} is
                rotational force.
                \item Notation: $\tau$ or $M$.
                \item The \impt{magnitude} of torque depends on three quantities:
                    \begin{itemize}
                        \item The force applied $F$
                        \item The length $r$ of the lever arm connecting the axis to
                            the point of force application $r$
                        \item The angle $\theta$ between the force vector and the lever arm
                    \end{itemize}
                \begin{equation}
                    \tau = r \times F = |r||F|sin \theta
                \end{equation}
            \end{itemize}

        \hii{Torque and Angular Acceleration}
            \par Consider a particle of mass $m$ rotating in a circle of radius $r$
            under the influence of a tangential force $F_{t}$ and a radial force
            $F_{r}$.
            \par The magnitude of torque about the center of the circle due to $F_{t}$:
            \begin{align*}
                \tau = F_{t} \mul r = ma_{t}r
            \end{align*}
            \par The magnitude of tangential acceleration:
            \begin{align*}
                a_{t} = \omega r
            \end{align*}
            \par Therefore:
            \begin{equation} \label{eq:torq_n_momentOfInertia}
                \tau = m \omega r^{2} = I \alpha
            \end{equation}

    \hi{Work, Power, and Energy in Rotational Motion}
        \hii{Power}
        \par \textit{Note that only the $F_{t}$ component does work and the $F_{c}$ component
        does not.}
        \par The work done by a force $F$ on an object as it rotates through a very small
        distance $ds = rd\theta$ is:
        \begin{align*}
            dW & = F_{t} \mul ds = (F \sin \theta) r d \theta
        \end{align*}
        \begin{equation}
            \ra dW = \tau d \theta 
        \end{equation}
        \begin{align*}
            \ra \dfrac{dW}{dt} = \tau \dfrac{d \theta}{dt}
        \end{align*}
        \begin{equation}
            \ra P = \tau \omega 
        \end{equation}

        \hii{Work - Kinetic Energy Theorem for Rotational Motion}
            \par According to the \eqref{eq:torq_n_momentOfInertia} equation:
            \begin{align*}
                \tau = I \omega \\
                \ra \tau & = I \dfrac{d\omega}{dt} \\
                & = I \dfrac{d \theta}{dt} \dfrac{d \omega}{d \theta} \\
                & = I \omega \dfrac{d \omega}{d \theta} \\
                \ra \tau d \theta & = I \omega d \omega \\
                \ra dW & = I \omega d \omega
            \end{align*}
            \begin{equation}
                \ra W = \INT{_{\omega_{i}}^{\omega_{f}} I \omega d \omega}
                = \dfrac{1}{2} I \omega_{f}^{2} - \dfrac{1}{2} I \omega_{i}^{2}
            \end{equation}
            \par This is the equation of the \impt{Work - Kinetic energy theorem for rotational
            motion}.


\chapter{Angular Momentum}

\hi{The Vector Product and Torque}
    \hii{The Vector Product}
    \par The Torque vector, $\vt{\tau}$, is related to two vectors:
    \begin{itemize}
        \item The force vector $\vt{F}$
        \item The position vector $\vt{r}$
    \end{itemize}
    \begin{equation}
        \vt{\tau} = \vt{F} \times \vt{r}
    \end{equation}

\hi{Angular Momentum}
    \hii{Angular Momentum and Linear Momentum}
        \begin{equation}
            \vt{L} = \vt{r} \times \vt{p}
        \end{equation}
    \hii{Angular Momentum and Torque}
    \begin{equation}
        \tau = \dfrac{dL}{dt}
    \end{equation}

\hi{Angular Momentum of a Rotational Rigid Object}
    \begin{equation}
        L = I \omega
    \end{equation}

\chapter{Temperature}

\hi{Temperature and the Zeroth Law of Thermodynamics}
    \hii{Thermal Equilibrium}
        \par Thermal equilibrium is a situation in which two objects would not exchange
        energy by heat or electromagnetic radiation if they were placed in thermal
        contact.
    \hii{Zeroth Law of Thermodynamics}
        \par If objects A and B are separately in thermal equilibrium with a third object
        C, then A and B are in thermal equilibrium with each other.
    \hii{Temperature}
        \par Two objects in thermal equilibrium with each other are at the same
        temperature.

\hi{Absolute Temperature Scale}
    \hii{Experiment with Gas Thermometers}
        \par Suppose that temperartures are measured with gas thermometers containing
        dfferent gases at different initial pressures.
        \par If we visualize the result on a pressure - temperature $P-T$ graph, the
        results obtained by one particular thermometer would form a straight line.
        In order words, $P$ and $T$ have a linear relationship.
        \par Experiments show that in every case, the pressure is zero when the
        temperature is about $-273 \degree C$.

\hi{Thermal Expansion of Solid and Liquids}
    \hii{Thermal Expansion}
        \par As temperature increases, its volume increases. This phenomenon is known as
        \impt{thermal expansion}.
        \par Thermal expansion is a consequence of the change in the \textit{average
        separation} between the atoms in an object.
    \hii{Linear Expansion}
        \begin{equation}
            \alpha = \frac{\Dt L / L_{i}}{\Dt T}
        \end{equation}
        or
        \begin{eqbox}
            \Dt L = \alpha L_{i}\Dt T
        \end{eqbox}
        or
        \begin{equation}
            L_{f} - L_{i} = \alpha L_{i} (T_{f} - T_{i})
        \end{equation}
        where:
        \begin{itemize}
            \item $\alpha$: average coefficient of linear expansion
        \end{itemize}
    \hii{Volume Expansion}
        \begin{equation}
            \beta = \frac{\Dt V / V_{i}}{\Dt T}
        \end{equation}
        or
        \begin{eqbox}
            \Dt V = \beta V_{i}\Dt T
        \end{eqbox}
        or
        \begin{equation}
            V_{f} - V_{i} = \beta V_{i} (T_{f} - T_{i})
        \end{equation}
        where:
        \begin{itemize}
            \item $\beta$: average coefficient of volume expansion
        \end{itemize}

\hi{Macroscopic Description of an Ideal Gas}
    \hii{Equation of State for an Ideal Gas}
        \begin{eqbox}
            PV = nRT
        \end{eqbox}
        where
        \begin{itemize}
            \item $P$: pressure $[Pa]$ or $[N/m^{2}]$
            \item $V$: volume $[m^{3}]$
            \item $n$: moles of gas $[mol]$
            \item $R$: ideal gas constant/universal gas constant
                $(R = 8.314 J/mol \cdot K)$
            \item $T$: temperature $[K]$
        \end{itemize}
        \par The ideal gas law can also be expressed in terms of the total number of
        molecules $N$:
        \begin{eqbox}
            PV = Nk_{B}T
        \end{eqbox}
        where $k_{B}$ is the Boltzmann's constant.

\chapter{Heat and the First Law of Thermodynamics}

\hi{Terminology}

\hi{Heat and Internal Energy}
    \hii{Internal Energy}
        \impt{Internal Energy} is all the energy of a system that is associated with
        its microscopic components - atoms and molecules - when viewed from a
        reference frame at rest with respect to the center of mass of the system.

    \hiiBEGIN{Heat}
        \hiii{Definition}
            \par \impt{Heat} is defined as the transfer of energy across the
            \textit{boundary} of a system due to a temperature difference between
            the system and its surroundings.
        \hiii{Units}
            \par $1 cal$ is defined as the amount of energy transfer necessary to raise 
            the temperature of $1g$ of water from $14.5 \degree C$ to $15.5 \degree C$.
            \begin{align*}
                1 Cal = 1000 cal
            \end{align*}
            \par Heat is also measured in \impt{joules}.
            \begin{align*}
                1 cal = 4.186 J
            \end{align*}
        \hiii{The Mechanical Equivalent of Heat}
            \par We have already found that whenever friction is present in a
            mechanical system, some mechanical energy is lost. In other words,
            mechanical energy is not conserved in the presence of nonconservative force.
            \par The lost energy is transformed into internal energy.
            \par Read the \textit{Joule's experiment}.
    \hiiEND

\hi{Specific Heat and Calorimetry}
    \hii{Heat capacity}
        \par The \impt{heat capacity} $C$ of a particular sample of a substance is
        defined as the amount of energy needed to raise the temperature of that sample by
        $1 \degree C$.
        \begin{equation}
            Q = C \Dt T
        \end{equation}
    \hii{Specific heat}
        \par The \impt{specific heat} $c$ of a substance is the heat capacity per unit mass.
        \begin{equation}
            Q = mc \Dt T
        \end{equation}
    \hiiBEGIN{Calorimetry}
        \hiii{Technique of Calorimetry}
            \par One technique for measuring specific heat involves heating a sample to 
            some known mass $m_{x}$ and temperature $T_{x}$, placing it in a vessel 
            containing water of known mass $m_{w}$ and temperature $T_{w} < T_{x}$, and
            measuring the temperature of the water after equilibrium has been reached.
            \par This technique is called \impt{calorimetry}, and the devices in which
            this energy tranfer occurs are called \impt{calorimeters}.
        \hiii{Applying Conservation of Energy to Calorimetry}
            \par Applying the theory of Conservation of Energy:
            \begin{equation}
                Q_{hot} + Q_{cold} = 0
            \end{equation}
            or
            \begin{equation} \label{eq:calorimetry}
                Q_{hot} = - Q_{cold}
            \end{equation}
            \par From the equation \eqref{eq:calorimetry}, we have:
            \begin{align*}
                m_{x}c_{x}(T_{x} - T_{f}) = m_{w}c_{w}(T_{f} - T_{w})
            \end{align*}
            in which $T_{f}$ is the final temperature of the equilibrium.
    \hiiEND

\hi{Latent Heat}
    \par \impt{Latent heat} is thermal energy released or absorbed, by a body or a
    thermodynamic system, during a constant-temperature process - usually a first-order
    phase transition.
    \par When the physical characteristics of a substance change from one form to
    another, such change is commonly referred to as \impt{phase change}. Common phase
    changes are:
    \begin{itemize}
        \item Solid to liquid and vice versa. 
        \item Liquid to gas and vice versa. 
        \item Change in the crystalline structure of a solid.
    \end{itemize}
    \par The energy required to change the phase of a given mass $m$ of a pure substance is:
    \begin{equation}
        Q = \pm mL
    \end{equation}
    in which: $L$ is the latent heat.
    \par \impt{Latent heat of fusion} $L_{f}$ is the term used when the phase change is
    from solid
    to liquid.
    \par \impt{Latent heat of vaporization} $L_{v}$ is the term used when the phase
    change is from liquid to gas.

\hi{Work and Heat in Thermodynamic Processes}
    \hiiBEGIN{Work}
        \hiii{Experiment}
            \par Consider a gas container in a cylinder fitted with a movable piston. At
            equilibrium, the gas occupies a volume $V$ and exerts a uniform pressure $P$
            on the cylinder's walls and on the piston.
            \par If the piston has a cross-sectional area $A$, the force exerted by the
            gas on the piston is $F = PA$.
            \par Assume that the piston is pushed inward and and the gas is compressed
            \impt{quasi-statically} (slowly enough to allow the system to remain in
            thermal equilibrium all the time). As the piston is pushed downward, the 
            instantaneous work done is:
            \begin{equation}
                dW = \vt{F} \cdot dr = -F \hat{j} \cdot dy \hat{j} = -Fdy = -PAdy = -PdV
            \end{equation}
            \par Therefore:
            \begin{eqbox}
                W = - \INT{_{V_{i}}^{V_{f}} PdV}
            \end{eqbox}
            where $W$ is the \textbf{work done on the system}.
        \hiii{Result}
            \par \textit{The work done on a gas in a quasi-static process that takes the
            gas from an initial state to a final state is the negative of the area under 
            the curve on a $PV$ diagram, evaluated between the initial and final states.}
            \par Therefore, $W$ \impt{depends on the particular path taken}.
    \hiiEND
    \hii{Heat}
        \par Like work done, energy transfer by heat also \impt{depends on the particular
        path taken}
        \par Read page 617.

\hi{The First Law of Thermodynamics}
    \par The change in internal energy can be expressed as:
    \begin{eqbox}
        \Dt E_{int} = Q + W
    \end{eqbox}
    \par While $Q$ and $W$ are dependent of the path, $E_{int}$ is not dependent of the
    path. It is a state variable.

\hi{Some Applications of the First Law of Thermodynamics}
    \hii{Adiabatic process}
        \par An \impt{adiabatic process} is one during which \impt{no energy enters or
        leaves the system by heat}.
        \begin{equation}
            \Dt E_{int} = W
        \end{equation}
        \par An adiabatic process can be achieved through:
        \begin{itemize}
            \item insulating the walls of the system
            \item performing the process rapidly, so that there is negligible time for energy to tranfer by heat.
        \end{itemize}
    \hii{Isobaric process}
        \par A process that \impt{occurs at a constant pressure} is called an
        \impt{isobaric process}.
        \begin{equation}
            \Dt E_{int} = Q + W = Q - W_{sys} = Q - P \Dt V
        \end{equation}
    \hii{Isovolumetric process}
        \par A process that takes place at a constant volume is called an
        \impt{isovolumetric process}.
        \par In such a process, the value of the work done is zero.
        \begin{equation}
            \Dt E_{int} = Q
        \end{equation}
    \hii{Isothermal process}
        \par A process that occurs at a constant temperature is called an \impt{isothermal
        process}.
        \par Since the internal energy of an ideal gas is a function of temperature only, 
        in a isothermal process:
        \begin{equation}
            \Dt E = Q + W = 0
        \end{equation}
        or
        \begin{equation}
            Q = -W
        \end{equation}
    \hii{Isothermal Expansion of an Ideal Gas}
        \begin{flalign*}
            W = - \INT{_{V_{i}}^{V_{f}} PdV}
            = - \INT{_{V_{i}}^{V_{f}} \frac{nRT}{V} dV}
            = - nRT \INT{_{V_{i}}^{V_{f}} \frac{dv}{V}}
            = nRT\ln\Big(\frac{V_{i}}{V_{f}}\Big)
        \end{flalign*}


\chapter{Heat Engines, Entropy, and the Second Law of Thermodynamics}

\hi{Heat Engines and the Second Law of Thermodynamics}
    \hii{Heat Engines}
        \par A \impt{heat engine} is a device that item takes in energy by heat and, operating in a cyclic process, expels a fraction of that energy by means of work.
        \par \impt{The cyclic process}:
        \begin{enumerate}
            \item The working substance absorbs energy by heat from a high-temperature
            energy reservoir
            \item Work is done by the engine
            \item Energy is expelled by heat to a lower-temperature reservoir.
        \end{enumerate}
    \hii{Work done by an engine}
        \begin{eqbox}
            W_{eng} = |Q_{h}| - Q_{c}
        \end{eqbox}
        \begin{itemize}
            \item $W_{eng}$: work done by the engine
            \item $Q_{h}$: heat absorbed from the hot reservoir
            \item $Q_{c}$: heat given up to the cold reservoir
        \end{itemize}
    \hii{Thermal Efficiency}
        \begin{eqbox}
            e = \frac{W_{eng}}{|Q_{h}|}
            = \frac{|Q_{h}| - |Q_{c}|}{|Q_{h}|}
            = 1 - \frac{|Q_{c}|}{|Q_{h}|}
        \end{eqbox}
    \hii{Kelvin-Planck form of the Second Law of Thermodynamics}
        \par It is impossible to construct a heat engine that, operating in a cycle,
        produces no effect other than the input of energy by heat from a reservoir and the
        performance of an equal amount of work.
        \par In order words:
        \begin{eqbox}
            W_{eng} < |Q_{h}| \mbox{ for all engines}
        \end{eqbox}

\hi{Heat Pumps and Refrigerators}
    \hii{Heat Pumps and Refrigerators Definition}
        \par The device that transfer energy from a cold reservoir to a hot reservoir is a
        \impt{heat pump} or \impt{refrigerator}.
        \par In a refrigerator, the engine takes in energy $|Q_{c}|$ from a cold reservoir
        and expels energy $|Q_{h}|$ to a hot reservoir. This can be accomplished only if
        \textit{work is done \textbf{on} the engine}.
    \hii{Clausius form of the Second Law of Thermodynamics}
        \par It is impossible to construct a cyclical machine whose sole effect is to
        transfer energy continuously by heat from one object to another object at a higher
        temperature without the input of energy by work.
        \par In simpler terms, \textit{energy does not transfer spontaneously by heat
        from a cold object to a hot object}.
    \hii{Coefficient of performance}
        \begin{eqbox}
            \mbox{COP (heating mode) } = \frac{|Q_{h}|}{W_{ref}}
        \end{eqbox}
        \begin{eqbox}
            \mbox{COP (cooling mode) } = \frac{|Q_{c}|}{W_{ref}}
        \end{eqbox}

\hi{Reversible and Irreversible Processes}
    \hii{Reversible and Irreversible Processes}
        \par In a \impt{reversible} process, the system undergoing the process can be
        returned to its initial conditions along the same path on a $PV$ diagram, and every
        point along this path is an equilibrium state.
        \par A process that does not sastisfy these requirements is \impt{irreversible}.
        \par All natural processes are known to be \textit{irreversible}.

\hi{The Carnot Engine}
    \hii{The Carnot Cycle}
        \par Assume that the working substance is an ideal gas contained in a cylinder
        fitted with a movable piston at one end. The cylinder's walls and the piston
        are thermally nonconducting.
        \par There are 4 stages in an Carnot cycle:
        \begin{enumerate}
            \item Process $A \to B$: \impt{isothermal expansion} at $T_{h}$.
                \par The gas is placed in thermal contact with an energy reservoir at
                temperature $T_{h}$.
                \par During the expansion, the gas:
                \begin{itemize}
                    \item absorbs energy $|Q_{h}|$ from the reservoir through the base
                    \item does work $W_{AB}$ in raising the piston.
                \end{itemize}
            \item Process $B \to C$: \impt{adiabatic expansion}
                \par The base of the cylinder is replaced by a thermally nonconducting wall,
                and the gas expands \impt{adiabatically} - no energy enters or leaves the
                system by heat. During the expansion, the temperature of the gas
                \impt{decreases} from $T_{h}$ to $T_{c}$ and the gas the work $W_{BC}$
                in raising the piston.
            \item Process $C \to D$: \impt{isothermal compression} at $T_{c}$
                \par The gas is placed in thermal contact with an energy reservoir at
                temperature $T_{c}$ and is compressed isothermally at temperature
                $T_{c}$.
                \par During the compression, the gas:
                \begin{itemize}
                    \item expels energy $|Q_{c}|$ to the reservoir
                    \item does work $W_{CD}$ on the piston
                \end{itemize}
            \item Process $D \to A$: \impt{adiabatic compression}
                \par The base of the cylinder is replaced by a nonconducting wall.
                \par During this process:
                \begin{itemize}
                    \item The temperature of the gas increases to $T_{h}$
                    \item The work done by the gas on the piston is $W_{CD}$.
                $W_{DA}$.
                \end{itemize}
        \end{enumerate}
        \begin{figure}[h!]
            \centering
            \includegraphics[width=5cm]{img/Carnot-cycle.jpg}    
            \caption{Carnot cycle}
        \end{figure}
    \hii{Carnot's theorem}
        \par No real heat engine operating between two energy reservoirs can be more
        efficient than a Carnot engine operating between the same two reservoirs.
    \hii{Efficiency of Carnot engine}
        \begin{eqbox}
            e_{C} = 1 - \frac{T_{c}}{T_{h}}
        \end{eqbox}
        \par \textbf{Proof}:
        \begin{flalign*}
            %
            &\bullet A \to B \mbox{ is an isothermal process} \mendl
            &\ra \Dt E_{int} = 0 \mendl
            &\ra |Q_{h}| = |-W_{AB}|
                = - \INT{_{V_{i}}^{V_{f}} PdV}
                = nRT_{h}\ln \Big(
                    \frac{V_{B}}{V_{A}}
                \Big) \quad (V_{B} > V_{A}) \quad (1) \mendl
            %
            &\bullet C \to D \mbox{ is an isothermal process} \mendl
            &\ra \Dt E_{int} = 0 \mendl
            &\ra |Q_{h}| = |-W_{CD}|
                = - \INT{_{V_{i}}^{V_{f}} PdV}
                = nRT_{c} \ln \Big(
                    \frac{V_{C}}{V_{D}}
                \Big) \quad (V_{C} > V_{D}) \quad (2) \mendl
            &\bullet \mbox{Divide (2) by (1)} \mendl
            &\ra \frac{|Q_{c}|}{|Q_{h}|}
                = \frac{T_{c}}{T_{h}} \cdot \frac{\ln(V_{C}/V_{D})}{\ln(V_{B}/V_{A})}
                \quad (3) \mendl
            %
        \end{flalign*}
        \begin{flalign*}
            &\bullet \mbox{ In an adiabatic process} \mendl
            &\Dt E_{int} = W \mendl
            &\ra \Dt E_{int} - W = 0 \mendl
            &\ra \frac{3}{2}nR \Dt T - P \Dt V = 0 \mendl
            &\ra \frac{3}{2}nRdT - nRTdV = 0 \mendl
            &\mbox{Divide both side for } nRT: \mendl
            &\ra \frac{3}{2}\frac{dT}{T} - \frac{dV}{V} = 0 \mendl
            &\ra \frac{3}{2}\INT{_{T_{i}}^{T_{f}} \frac{dT}{T}}
                - \INT{_{V_{i}}^{V_{f}} \frac{dV}{V}} = 0 \mendl
            &\ra \frac{3}{2} \ln\Big(\frac{T_{f}}{T_{i}}\Big)
                - \ln\Big(\frac{V_{f}}{V_{i}}\Big) = 0 \mendl
            &\ra \ln\Big(\frac{T_{f}^\frac{3}{2}}{T_{i}^\frac{3}{2}}\Big)
                - \ln\Big(\frac{V_{f}}{V_{i}}\Big) = 0 \mendl
            &\ra \ln\Big(
                    \frac{T_{f}^\frac{3}{2}}{T_{i}^\frac{3}{2}}
                    \cdot \frac{V_{i}}{V_{f}}
                    \Big) = 0 \mendl
            &\ra \frac{T_{f}^\frac{3}{2}}{T_{i}^\frac{3}{2}}
                \cdot \frac{V_{i}}{V_{f}} = 1 \quad \mendl
        \end{flalign*}
        \begin{flalign*}
            &\bullet B \to C \mbox{ is an adiabatic process } \mendl
            &\ra \frac{T_{C}^\frac{3}{2}}{T_{B}^\frac{3}{2}}
                \cdot \frac{V_{B}}{V_{C}} = 
                \frac{T_{c}^\frac{3}{2}}{T_{h}^\frac{3}{2}}
                \cdot \frac{V_{B}}{V_{C}} = 1 \quad (4) \mendl
            &\bullet D \to A \mbox{ is an adiabatic process } \mendl
            &\ra \frac{T_{A}^\frac{3}{2}}{T_{D}^\frac{3}{2}}
                \cdot \frac{V_{D}}{V_{A}} = 
                \frac{T_{h}^\frac{3}{2}}{T_{c}^\frac{3}{2}}
                \cdot \frac{V_{D}}{V_{A}} = 1 \quad (5) \mendl
            %
            &\bullet \mbox{ Multiply (4) and (5) } \mendl
            &\ra \frac{V_{B}}{V_{C}} \cdot \frac{V_{D}}{V_{A}} = 1 \mendl
            &\ra \frac{V_{B}}{V_{C}} = \frac{V_{A}}{V_{D}} \mendl
            &\ra \frac{V_{B}}{V_{A}} = \frac{V_{C}}{V_{D}} (6) \mendl
        \end{flalign*}

\hi{Entropy}
    \hii{Statement on Entropy of the Second Law of Thermodynamics}
        \par The entropy of the Universe increases in all real processes.
    \hii{Formula}
        \par Consider any infinitesimal process in which a system changes from
        one equilibrium state to another. If $dQ_{r}$ is the amount of energy
        transferred by heat when the system follows a reversible path between
        the states, then the change in entropy $dS$ is equal to this amount of
        energy for the reversible process divided by the absolute temperature
        of the system.
        \begin{eqbox}
            dS = \frac{dQ_{r}}{T}
        \end{eqbox}
    \hii{Entropy as a State Variable}
        \par The change in entropy during a process depends only on the end
        points and therefore is independent of the actual path followed.
        Consequently, the entropy change for an irreversible process can be
        determined by calculating the entropy change for a reversible process
        that connects the same initial and final states.
    \hii{Entropy in the Carnot cycle}
        \par In a Carnot cycle, there are two processes where the engine exchanges
            energy with the reservoirs:
        \begin{itemize}
            \item Process $A \to B$ where the engine absorbs heat from the
                hot reservoir
            \item Process $C \to D$ where the engine expels heat to the
                cold reservoir
        \end{itemize}
        \begin{flalign*}
            & \Dt S = \Dt S_{AB} + \Dt S_{CD} \mendl
            & \ra \Dt S = \INT{\frac{dQ_{h}}{T_{h}}} + \INT{\frac{dQ_{c}}{T_{c}}} \mendl
        \end{flalign*}
        \par As we have mentioned earlier about the efficiency of a Carnot engine:
        \begin{flalign*}
            & \frac{Q_{h}}{Q_{c}} = \frac{T_{h}}{T_{c}} \mendl
        \end{flalign*}
        \par Therefore:
        \begin{flalign*}
            & \Dt S = 0 \mendl
        \end{flalign*}
        \par In general:
        \begin{flalign*}
            & \OINT{\frac{dQ_{r}}{T}} = 0 \mendl
        \end{flalign*}


\chapter{Properties of Electric Charges}

\hi{Properties of Electric Charges}
    \begin{itemize}
        \item There are two kinds of electric charges:
            \begin{itemize}
                \item Positive
                \item Negative
            \end{itemize}
            The name \impt{positive} and \impt{negative} were given by Benjamin Franklin.
        \item Charges of the same sign repel one another and charges of the opposite signs attract
            one another.
        \item In an isolated system, electric charge is always conserved.
        \item Electric charge always occurs as some integral multiple of a fundamental amount $e$.
            In modern terems, the electric charge is is said to be \impt{quantized}.
    \end{itemize}

\hi{Charging an Object}
    \par{There are three ways to charge an object}
    \begin{itemize}
        \item by \impt{friction}
        \item by \impt{contact}
        \item by \impt{induction}
    \end{itemize}

\hi{Coulomb's Law}
    \par According to Coulomb's experiments, the \impt{electric force} between two stationary
    charged particles has the following properties:
    \begin{itemize}
        \item is inversely proportional to the square of the separation $r$ between the particles
            and directed along the line joining them.
        \item is proportional to the product of the charges $q_{1}$ and $q_{2}$ on the two
            particles.
        \item is attractive if the charges are of opposite sign and repulsive if the charges have
            the same sign.
        \item is a conservative force.
    \end{itemize}
    \par A \impt{point charge} is a particle of zero size that carries an electric charge.
    \par Coulomb's law:
    \begin{equation}
        F_{e} = k_{e} \frac{|q_{1}||q_{2}|}{r^{2}}
    \end{equation}
    where
    \begin{itemize}
        \item $k_{e} = 9 \cdot 10^{9} Nm^{2}/C^{2}$: the Coulomb constant
    \end{itemize}
    \par The Coulomb constant is also written in the form:
    \begin{equation}
        k_{e} = \frac{1}{4 \pi \epsilon_{0}}
    \end{equation}
    where $\epsilon_{0}$ is known as the \impt{permittivity of free space}.

\hi{The Electric Field}
    \hii{Definition}
        \par \impt{The electric field vector $E$} at a point in space is defined as the electric
        force $F_{e}$ acting on a positive test charge $q_{0}$ placed at that point divided by
        the test charge.
        \begin{equation}
            E = \frac{F_{e}}{q_{0}}
        \end{equation}
        \par An electric field exists at a point if a test charge at that point experiences an
        electric force.
    \hii{Direction of an electric field}
        \par The electric field at position $P$ is:
        \begin{itemize}
            \item \impt{directed away} from $q$ if $q > 0$.
            \item \impt{directed toward} $q$ if $q < 0$.
        \end{itemize}
    \hii{Electric field of a group of charges}
        \par At any point $P$, the total electric field due to a group of source charges equals the
        vector sum of the electric fields of all the charges.

\hi{Electric Field of a Continuous Charge Distribution}
    \hii{Method of Evaluation}
        \par To evaluate the electric field of a continuous charge distribution, we divide the
        distribution into many parts, each with charge $\Dt q$.
        \begin{align*}
            E = k \frac{\Dt q}{r^{2}} \hat{r}
        \end{align*}
        \par Suppose each part is infinitely small:
        \begin{align*}
            E = k \SUM{_{\Dt q_{i} \to 0} \frac{\Dt q}{r^{2}} \hat{r}} \\
            = k \INT{\frac{dq}{r^{2}} \hat{r}}
        \end{align*}
    \hiiBEGIN{Uniform Distribution of Charge on an object}
        \hiii{Linear Density of Charge}
            \begin{equation}
                \lambda = \frac{Q}{L} = \frac{dQ}{dL}
            \end{equation}
        \hiii{Surface Density of Charge}
            \begin{equation}
                \sigma = \frac{Q}{S} = \frac{dQ}{dS}
            \end{equation}
        \hiii{Volume Density of Charge}
            \begin{equation}
                \rho = \frac{Q}{V} = \frac{dQ}{dV}
            \end{equation}
    \hiiEND

\hi{Electric Field Lines}
    \par \impt{Electric Field Lines}, first introduced by Faraday, are related to the electric
    field in a region of space in the following manner:
    \begin{itemize}
        \item The electric field vector $E$ is tangent to the electric field line at each point.
            The line has a direction, indicated by an arrowhead, that is the same as that of the
            electric field vector.
        \item The number of lines per unit area through a surface perpendicular to the lines is
            proportional to the magnitude of the electric field in that region. Thus, the field
            lines are close together where the electric field is strong and far apart where the
            field is weak.
    \end{itemize}
    \par The rules for drawing electric field lines:
    \begin{itemize}
        \item The lines must begin on a positive charge and terminate on a negative charge.
            In the case of an excess of one type of charge, some lines will begin or end infinitely
            far away.
        \item The number of lines drawn leaving a positive charge or approaching a negative charge
            is proportional to the magnitude of the charge.
        \item No two field lines can cross.
    \end{itemize}

\hi{Motion of Charged Particles in a Uniform Electric Field}


\chapter{Gauss's Law}
\hi{Electric Flux}
    \hii{Electric Flux for Uniform Electric Field}
        \par \impt{Electric flux} is propotional to the number of electric field lines
        of a uniform electric field penetrating some surface \textit{perpendicularly}.
        \begin{equation}
            \Phi_{E} = \vt{E} \cdot \vt{A} = EA\cos(90 \degree) = EA
        \end{equation}
        where:
        \begin{itemize}
            \item $\Phi_{E}$: electric flux $[N \cdot m^{2} / C]$
            \item $E$: electric field $[N / C]$
            \item $A$: the area of the surface $[m^{2}]$
        \end{itemize}
        \par If the surface is not perpendicular to the electric field, the flux through it
        equals:
        \begin{equation}
            \Phi_{E} = \vt{E} \cdot \vt {A} = EA\cos(\vt{E}, \vt{A}) = EA\cos(\theta)
        \end{equation}
    \hii{Electric Flux for Nonuniform Electric Field}
        \par In a general case where the electric field is not uniform, consider a general surface
        divided up into a large number of small elements each of area $\Dt A$. The electric flux
        through each element is:
        \begin{align*}
            \Dt \Phi_{E} = E_{i} \Dt A_{i} \cos(\theta_{i})
        \end{align*}
        \par If we let the number of elements approaches infinity, the sum of electric flux is the
        integral:
        \begin{equation}
            \sum \Phi_{E} = \lim_{\Dt A_{i} \to 0} \sum E_{i} \cdot \Dt A_{i} = \INT{E \cdot dA}
        \end{equation}
    \hii{Electric Flux through a closed surface}
        \par A \impt{closed surface} (or \impt{gaussian surface}) is defined as one that divides
        space into an inside and outside region, so that one cannot move from one region to the
        other without crossing the surface.
        \par The \impt{net flux} through the surface is proportional to the net number of lines
        \impt{leaving} the surface.
        \begin{equation} \label{eq:eflux}
            \Phi_{E} = \OINT{E \cdot dA} = \OINT{E_{n}dA}
        \end{equation}
        \par Sign:
        \begin{itemize}
            \item $\Phi_{E leave} > 0$
            \item $\Phi_{E enter} < 0$
        \end{itemize}

\hi{Gauss's Law}
    \hii{Mathematical Analysis}
        \par \textit{Consider a positive point charge $q$ located at the center of a sphere of
        radius $r$}.
        \par The field lines are directed radially outward and hence are perpendicular to the
        surface at every point on the surface. Thus, the magnitude of the electric field everywhere
        on the surface of the sphere is:
        \begin{align*}
            E = \frac{k_{e}q}{r^{2}}
        \end{align*}
        \par Also, according to the equation \eqref{eq:eflux}:
        \begin{equation} \label{eq:eflux}
            \Phi_{E} = \OINT{E \cdot dA} = E \OINT{dA}
        \end{equation}
        \par Because $E$ is the same for every point, it can be moved out of the integral.
        \par The surface is spherical:
        \begin{equation}
            \Phi_{E} = E \OINT{dA} = \frac{kq}{r^{2}} \cdot (4 \pi r^{2})
            = 4 \pi kq = \frac{q}{\epsilon_{0}}
        \end{equation}
        \par The net flux through any closed surface surrounding a point charge $q$ is given by
            $q / \epsilon_{0}$ and is \textit{independent of the shape of the surface.}
        \par The electric field due to many charges is the vector sum of the electric fields produced
        by the individual charges.
        \begin{align*}
            \OINT{E \cdot dA} = \OINT{\SUM{_{i = 1}^{n} E_{i}} \cdot dA}
        \end{align*}
    \hii{Gauss's Law}
        \par \textit{The net flux through any closed surface is}
        \begin{eqbox}
            \Phi_{E} = \OINT{E \cdot dA} = \frac{q_{in}}{\epsilon_{0}}
        \end{eqbox}


\chapter{Electric Potential}

\hi{Potential Difference and Electric Potential}
    \hii{Potential Difference}
        \par For an infinitesimal displacement $ds$ of a charge, the work done by the electric field
        on the charge is:
        \begin{align*}
            dW = F \cdot ds = q_{0} E \cdot ds
        \end{align*}
        \par The change in potential energy would be:
        \begin{align*}
            dU = -dW = - q_{0} E \cdot ds
        \end{align*}
        \par If the electric charge is moved from $A$ to $B$:
        \begin{align*}
            \Delta U = - \Delta W = - q_{0} \INT{_{A}^{B} E \cdot ds}
        \end{align*}
        \par Since the force $q_{0}E$ is \impt{conservative}, \textit{the integral does not depend on
        the path taken from $A$ to $B$}.
    \hii{Electric Potential}
        \par The electric potential is independent of the value of $q_{0}$ and has a value at every
        point in an electric field.
        \begin{eqbox}
            V = \frac{U}{q_{0}}
        \end{eqbox}
        \begin{eqbox}
            \Dt V = \frac{\Dt U}{q_{0}} = \frac{- q_{0} \INT{_{A}^{B} E \cdot ds}}{q_{0}}
            = - \INT{_{A}^{B} E \cdot ds}
        \end{eqbox}
    \hii{Work done by the Electric Field}
        \begin{eqbox}
            W = q \Dt V
        \end{eqbox}


\hi{Magnetic Fields and Forces}
    \hii{Magnetic Fields}
        \par A \impt{magnetic field} is a region around a magnetic material or a moving electric
        charge within which the force of magnetism acts.
        \par A magnetic field $B$ at some point in space can be defined in terms of the magnetic
        force $F_{B}$ that the field exerts on a charged particle moving with a velocity $v$.
        \begin{equation}
            \vt{F_{B}} = q \vt{v} \times \vt{B}
        \end{equation}
    \hii{Magnetic Force}
        \begin{itemize}
            \item Magnitude:
                \begin{equation}
                    F_{B} = |q|vB \sin(\vt{v}, \vt{B}) = |q|vB \sin(\theta)
                \end{equation}
            \item Unit: tesla (T)
                \begin{equation}
                    1 T = 1 \frac{N}{C \cdot m/s} = 1 \frac{N}{A \cdot m}
                \end{equation}
        \end{itemize}

\hi{Magnetic Force Acting on a Current-Carrying Conductor}
    \par Consider a \impt{straight} segment of wire of length $L$ and cross-sectional area $A$,
    carrying a current $I$ in a uniform magnetic field $B$.
    \par The magnetic force acting on a charge $q$:
    \begin{align*}
        \vt{F_{B/1q}} = q \vt{v} \cdot \vt{B}
    \end{align*}
    \par Define $n$ as the number of charges per unit volume. The magnetic forces acts on the
    wire:
    \begin{align*}
        \vt{F_{B}} = nALq \vt{v} \cdot \vt{B} 
    \end{align*}
    \par The current in the wire $I$ can be written as:
    \begin{align*}
        I = nqvA
    \end{align*}
    \par Therefore:
    \begin{align*}
        \vt{F_{B}} = I \vt{L} \times \vt{B}
    \end{align*}
    \par Now consider an arbitrarily shaped wire segment of uniform cross section in a magnetic
    field. The lenght of the wire is from $a$ to $b$.
    \begin{align*}
        F_{B} = I \INT{_{a}^{b}ds} \cdot B
    \end{align*}



\chapter{Sources of magnetic field}

\hi{The Biot-Savart Law}
    \par While performing an experiment on the force exerted by an electric current on a nearby
    magnet, Biot and Savart came up with the Biot-Savart Law:
    \par \textit{The magnetic field $dB$ at a point $P$ associated with a length $ds$ of a wire
    carrying a steady current $I$}:
    \begin{equation}
        dB = \frac{\mu_{0}}{4\pi} \frac{Id\textbf{s}\hat{r}}{r^{2}}
    \end{equation}
    where
    \begin{itemize}
        \item $B$: magnetic field \quad $[T]$
        \item $\mu_{0}$: permeability of free space \quad $(4 \pi \cdot 10^{-7} Tm/A)$
        \item $I$: the electric current \quad $[A]$
        \item $s$: the length of the conductor \quad $[m]$
        \item $r$: the distance from $ds$ to $P$ \quad $[m]$
    \end{itemize}
    \par By integration, we obtain:
    \begin{equation}
        B = \frac{\mu_{0}I}{4\pi} \INT{\frac{d\textbf{s}\hat{r}}{r^{2}}}
    \end{equation}
    \par \textbf{Example:} Consider a thin straight wire carrying a constant current $I$ and placed
    along the $x$ axis. Determine the magnitude and direction of the magnetic field at point P due to
    this current.
    \begin{flalign*}
        & \bullet d\textbf{s} \hat{r} = dx \sin(\vt{s}, \vt{r}) \hat{B} \mendl
        & \mbox{We also have:} \mendl
        & d\textbf{B} = \frac{\mu_{0}I}{4\pi} \frac{Id\textbf{s}\hat{r}}{r^{2}}\mendl
        & \ra d\textbf{B} = dB\hat{B}
            = \frac{\mu_{0}I}{4\pi} \frac{dx \sin(\vt{s}, \vt{r}) \hat{B}}{r^{2}} \mendl
        & \ra dB = \frac{\mu_{0}I}{4\pi} \frac{dx \sin(\vt{s}, \vt{r})}{r^{2}} \mendl
        &  = \frac{\mu_{0}I}{4\pi} \frac{dx \sin(\theta)}{r^{2}} \mendl
    \end{flalign*}
    \begin{flalign*}
        & \bullet \mbox{Let a = d(P, Ox)}\mendl
        & x = -a \cot(\theta) \mendl
        & \ra dx = \frac{a}{\sin^{2}(\theta)} d\theta \mendl
        & r = \frac{a}{\sin(\theta)}\mendl
        & \ra dB = \frac{\mu_{0} I}{4\pi}
            \frac{
                \frac{a}{sin^{2}(\theta)}
            }
            {
                \frac{a^{2}}{sin^{2}(\theta)}
            }
            \sin(\theta) d\theta \mendl
        & \ra dB = \frac{\mu_{0} I}{4\pi} \sin(\theta) d\theta \mendl
        & \ra B = \frac{\mu_{0} I}{4\pi a} \INT{_{\theta_{1}}^{\theta_{2}}\sin(\theta) d\theta} \mendl
        & \ra B = \frac{\mu_{0} I}{4\pi a} (\cos\theta_{1} - \cos\theta_{2}) \mendl
    \end{flalign*}
    \begin{flalign*}
        & \bullet \mbox{ If the wire is infinitely long }\mendl
        & \ra B = \frac{\mu_{0} I}{4\pi a} (\cos(0) - \cos(\pi)) \mendl
        & \ra B = \frac{\mu_{0} I}{2\pi a} \mendl
    \end{flalign*}

\hi{Ampere's Law}
    \begin{eqbox}
        B \Dt l = \mu_{0}I
    \end{eqbox}


%Newton's Cradle
%Simple Pendulum
%sth sth pendulum
\end{document}
