\chapter{Combinational Logic Circuits}

\hi{Sum-of-Product form}
    \hii{Sum-of-Product form and Product-of-Sum form}
        \par A \textbf{sum-of-product (SOP)} form consists of two or more AND terms
        that are ORed together.
        \par A \textbf{product-of-sum (POS)} form consists of two or more OR terms
        that are ANDed together.
    \hii{Sum-of-Product form in Simplyfying Logic Circuits}
        \par Methods of logic-circuit simplification require the logic expression
        to be in a \textbf{sum-of-product} form.

\hi{Simplyfying Logic Circuits}
    \par There are two methods for simplyfying logic circuits:
    \begin{itemize}
        \item Utilizing Boolean algebra theorems
        \item Karnaugh mapping
    \end{itemize}

\hii{Designing Combinational Logic Circuits}
    \par The complete design procedure for a combinational logic circuit
        consists of these steps:
    \begin{itemize}
        \item Intepret the problem and set up a truth table to describe its
            operations.
        \item Write the AND (product) term for each case where the output is 1.
        \item Write the sum-of-product (SOP) expression for the output.
        \item Simplify the output expression if possible.
        \item Implement the circuit for the final, simplify expression.
    \end{itemize}

\hi{Karnaugh Map Method}
    \hi{Introduction}
        \par K map, like truth table, is a means for showing the relationship between
        the logic inputs and the desired output.
    \hi{Karnaugh Map Format}
        \begin{itemize}
            \item Each case in the truth table corresponds to a square in the K map.
            \item The K-maps squares a labeled so that horizontally adjacent squares differ
                only in one variable. Note that: each square in the top row is considered to be
                adjacent to a corresponding square in the bottom row; each square in the leftmost
                column are adjacent to corresponding squares in the rightmost column.
            \item Top-to-bottom and left-to-right labelling must be done following the rule of Grey
                code.
            \item Once a K map has been filled with 0s and 1s
        \end{itemize}
    \hii{Complete Simplification Process}
        \begin{enumerate}
            \item Construct the K map according to the truth table.
            \item Loop isolated 1s (1s that are not adjacent to any other 1s).
            \item Look for 1s that are adjacent to only one other 1 and loop
            those pairs.
            \item Loop any octet even if it contains some 1s that have been looped.
            \item 
        \end{enumerate}
