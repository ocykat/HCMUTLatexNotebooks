\chapter{Flip-flops and Related Devices}

\hi{Introduction}
  \par Most digital systems consist of both combinational circuits and
  memory elements.
  \par The most important memory element is the \textbf{flip-flop} (FF),
  which is made up of an assembly of logic gates.
  \par Even though a logic gate, by itself, has no storage capability,
  several can be connected together in ways that permit information
  to be stored.
  \begin{itemize}
    \item $Q$: \textit{normal} FF output
    \item $\bar{Q}$: \textit{inverted} FF output, which is always in the
    opposite state with $Q$.
    \item $Q = 1$: the \textbf{SET} state. If the input cause $Q$ to go to 1,
    we call it "setting the FF/the FF has been set".
    \item $Q = 0$: the \textbf{CLEAR/RESET} state. If the input cause $Q$ to
    go to 0, we call it "clearing or reseting the FF/the FF has been cleared
    or reset".
    \item A FF can have one or more inputs. These inputs are used to cause the
    FF to switch.
    \item Most FF inputs need only to be momentarily activated (pulsed) to switch
    the FF, and the output will remain in that new state even after the input
    pulse is over (FF's \textit{memory} characteristic).
  \end{itemize}
  \par The most basic FF circuit can be constructed from either two NAND gates or
  two NOR gates.

\hi{NAND Gate Latch}
  \hii{Description}
    \begin{itemize}
      \item The two NAND gates are cross-coupled so that the \textbf{output} of
      NAND-1 is connected to one \textbf{input} of NAND-2, and vice versa
      \item There are two latch inputs:
      \begin{itemize}
        \item the SET input is the input that sets Q to the 1 state
        \item the RESET input is the input that resets Q to the 0 state.
      \end{itemize}
      \item The SET and RESET inputs are both normally resting in the HIGH state,
      and one of them will be pulsed LOW whenever we want to change the latch
      outputs.
      \item There are two different possibilities, which \textbf{depends on what
      has occurred previously at the inputs}:
      \begin{itemize}
        \item $Q = 0$ and $\bar{Q} = 1$
        \item $Q = 1$ and $\bar{Q} = 0$
      \end{itemize}
    \end{itemize}

  \hii{Setting and Resetting the FF}
    \begin{itemize}
      \item If the SET input is pulsed LOW while RESET is kept HIGH, then $Q = 1$.
      We call this \textbf{setting the FF}.
      \item If the RESET input is pulsed LOW while RESET is kept HIGH, then $Q = 0$.
      We call this \textbf{clearing/resetting the FF}.
      \item If simultaneous setting and resetting happens, the result is unpredictable.
      Therefore, SET = RESET = 0 condition is normally not used for the NAND latch.
    \end{itemize}

  \hii{Summary of NAND Latch}
    \begin{itemize}
      \item SET = RESET = 1: normal resting state, having no effect on the output.
      $Q$ and $\bar{Q}$ remain in whatever state they were in prior to this input
      condition.
      \item SET = 0, RESET = 1: Q = 1
      \item SET = 1, RESET = 0: Q = 0
      \item SET = 0, RESET = 0: unpredictable behaviour
    \end{itemize}

  \hii{Alternate Representations}
  \par The NAND latch is active-LOW:
  \begin{itemize}
    \item The SET input will set $Q = 1$ when SET goes LOW.
    \item The RESET input will reset $Q = 0$ when RESET goes LOW.
  \end{itemize}
  \par Because the NAND latch is active-LOW, it is often drawn using the alternate
  representation:
  \begin{figure}[H]
    \centering
    \includegraphics{c05/nand-latch.JPG}
    \caption{Alternate Representation of NAND latch}
  \end{figure}

\hi{Summary of NOR Latch}
  \begin{itemize}
    \item SET = RESET = 0: normal resting state, having no effect on the output.
    $Q$ and $\bar{Q}$ remain in whatever state they were in prior to this input
    condition.
    \item SET = 1, RESET = 0: Q = 1
    \item SET = 0, RESET = 1: Q = 0
    \item SET = 1, RESET = 1: unpredictable behaviour
  \end{itemize}

\hi{Pulse}
  \par A pulse that performs its intended function when it goes HIGH is called
  a positive pulse.
  \par A pulse that performs its intended function when it goes LOW is called
  a negative pulse.
  \par In actual circuits it takes time for a pulse waveform to change from one
  level to the other. These transition times are called the rise time ($t_{r}$) and
  the fall time ($t_{f}$) and are defined as the time it takes the voltage to change
  between 10\% and 90\% of the HIGH level voltage.
  \par The transition at the beginning of the pulse is called the leading edge.
  \par The transition at the end of the pulse is the trailing edge.
  \par The duration (width) of the pulse ($t_{w}$) is defined as the time between
  the points when the leading and trailing edges are at 50\% of the HIGH level voltage.

\hi{Clock Signals and Clocked Flip-flops}
  \hii{Asynchronous and Synchronous Systems}
    \begin{itemize}
      \item In asynchronous systems, the outputs can change state any time one
        or more of the inputs change.
      \item In synchronous systems, the exact times at which any output can change
        states are determined by a signal commonly called the \textbf{clock}.
    \end{itemize}

  \hii{Clock}
    \par Most of the system outputs can change state only when the clock makes
    a \textbf{transition} (also called \textbf{edge}).
    \begin{itemize}
      \item When the clock changes from 0 to 1: \textbf{positive-going
        transition (PGT)} or \textbf{rising edge}.
      \item When the clock changes from 1 to 0: \textbf{negative-going
        transition (NGT)} or \textbf{falling edge}.
    \end{itemize}
    \par One \textbf{cycle} is measured from one PGT to the next one, or from
    one NGT to the next one.

  \hii{Clocked Flip-flops}
    \par Clocked FFs have a clock input (often labeled CLK, CK or CP).
    \par In most clocked FFs, the CLK input is edge-triggered, which means
    that it is activated by a signal transition.
    This contrasts with the latches, which are level-triggered.
    \begin{itemize}
      \item A FF with \ti{a small triangle} on its CLK input to indicate
        that this input is activated only when a \textbf{PGT} occurs;
        no other part of the input pulse will have an effect
        on the CLK input.
      \item A FF with \ti{a small triangle with a bubble} on its CLK input
      to indicate that this input is activated only when a \textbf{NGT}
      occurs; no other part of the input pulse will have an effect
      on the CLK input.
    \end{itemize}
    \par Clocked FFs also have one or more control inputs. The control
    inputs will \ti{have no effect on Q until the active clock transition
    occurs.} They are called \textbf{synchronous control inputs}.
    \par In summary, the control inputs get the FF outputs ready to change,
    while the active transition at the CLK input actually triggers
    the change. The control inputs control the WHAT; the CLK input determines the WHEN.

  \hii{Setup and Hold time}
    \par Before and after a transition (either PGT or NGT) the signal
    must be kept at a stable value for the clocked FF to respond reliably.
    \begin{itemize}
      \item The \textbf{setup time} $t_{S}$ is the required time
        \textbf{preceding} a transition.
      \item The \textbf{hold time} $t_{H}$ is the required time
        \textbf{following} a transition.
    \end{itemize}


\hi{Clocked S-R FF}
  \hii{Trigger}
    \par The \tb{clocked S-R FF} triggers on a PGT of the clock signal

  \hii{Operation}
    \par Similar to a NOR gate latch, but does not respond to the inputs until
      the PGT of the clock signal:
    \begin{itemize}
      \item If $S = 0$, $R = 0$, $Q = Q_{0}$ (maintaining previous state).
      \item If $S = 1$, $R = 0$ then $Q = 1$.
      \item If $S = 0$, $R = 1$ then $Q = 0$.
      \item If $S = 1$, $R = 1$, the behaviour is unpredictable.
  \end{itemize}


\hi{Clocked J-K FF}
  \hii{Trigger}
    \par The \tb{clocked J-K FF} triggers on a PGT of the clock signal

  \hii{Operation}
    \par Similar to a NOR gate latch, but does not respond to the inputs until
      the PGT of the clock signal:
    \begin{itemize}
      \item If $J = 0$, $K = 0$, $Q = Q_{0}$ (maintaining previous state).
      \item If $J = 1$, $K = 0$ then $Q = 1$.
      \item If $J = 0$, $K = 1$ then $Q = 0$.
      \item If $J = 1$, $K = 1$, $Q = \bar{Q_{0}}$ (toggled to the opposite
      state).
  \end{itemize}


\hi{Clocked D Flip-flop}
  \hii{Trigger}
    \par The \textbf{clocked D flip-flop} triggers on a PGT.

  \hii{Operation}
    \par This flip-flop has only one synchronous control input, $D$ (data).
    \par Operation: $Q = D$ when a PGT occurs at CLK.

  \hii{Parallel Data Transfer}
    \par One application of the D FF is parallel data transfer.


\hi{D Latch (Transparent Latch)}
  \hii{Description}
    \par The circuit contains a \tb{NAND latch} and two steering NAND gates.
    \par The edge-detector circuit is replaced by the \tb{enable input} (EN)
    as the input of the steering gates.
    \begin{figure}[H]
      \centering
      \includegraphics[width=12cm]{c05/d-latch.jpg}
      \caption{D Latch}
    \end{figure}

  \hii{Operation}
    \begin{itemize}
      \item If $EN = 0$ then $Q = Q_{0}$ (maintaining previous state).
      \item If $EN = 1$, $D = 0$ then $Q = 0$.
      \item If $EN = 1$, $D = 1$ then $Q = 1$.
    \end{itemize}


\hi{Asynchronous Inputs}
  \hii{Description}
    \par Some FFs also have one or more \tb{asynchronous inputs} that operate
      independently of the synchoronous inputs and clock input.
    \par Asynchronous inputs can be used to set the FF to 1 or clear the FF
      to 0 at \ti{any time, regardless of the conditions at the other inputs}.
    \par In order words, asynchronous inputs are \tb{override inputs}, which
      can be used to \ti{override} all the other inputs.

  \hii{Example}
    \par Example: a J-K FF with two asynchronous active-LOW inputs
    $\overline{PRESET}$ and $\overline{CLEAR}$.
    \begin{itemize}
      \item $\overline{PRESET} = \overline{CLEAR} = 1$:
        The asynchronous inputs are \tb{inactive}; the FF operates according to
        the synchronous ones.
      \item $\overline{PRESET} = 0$, $\overline{CLEAR} = 1$, then Q = 1.
      \item $\overline{PRESET} = 1$, $\overline{CLEAR} = 0$, then Q = 0.
    \end{itemize}
  \begin{figure}[H]
    \centering
    \includegraphics[width=12cm]{c05/async-jk-ff.jpg}
    \caption{J-K FF with asynchronous inputs}
  \end{figure}

  \hii{Designations for Asynchronous Inputs}
    \par To distinguish asynchronous inputs from synchronous ones,
      asynchronous inputs are often labeled PRE and CLR instead of SET and
      RESET, respectively.


\hi{FF Timing Considerations}
  \hii{Setup and Hold Times}

  \hii{Propagation Delays}
    \par \tb{Propagation delay} is the delay from the time the signal is
    applied to the time when the output makes its change.
    \par \ti{Note that these delays are measured between the 50 percent points
      on the input and output waveforms}.
    \par For an IC, the maximum values of $t_{PLH}$ (delay from going from
    LOW to HIGH) and $t_{PHL}$ (delay from going from HIGH to LOW) are often
    specified in the manufacturers' datasheet.

  \hii{Maximum Clocking Frequency}
    \par This is the highest frequency that may be applied to the CLK input
    of an FF and still have it trigger reliably.

  \hii{Clock Pulse HIGH and LOW Times}
    \begin{itemize}
      \item $t_W (L)$: the minimum time the CLK signal must remain LOW before
        going HIGH
      \item $t_W (H)$: the minimum time the CLK signal must remain HIGH before
        going LOW
    \end{itemize}
    \par Failure to meet these minimum time requirements can result in
    unreliable triggering.
    \par \ti{These time values are measured between the halfway points on
    the signal transitions.}

  \hii{Asynchronous Active Pulse Width}
    \par PRESET or CLEAR input must also be kept in its active state for
    some minimum duration of time in order to set or clear the FF reliably.
    This duration is called \tb{minimum asynchronous active pulse width}.

  \hii{Clock Transition Times}
    \par For reliable triggering, the \tb{clock waveform transition times}
      (rise and fall times) should be kept \tb{very short}.
    \par If the clock signal takes too long to make the transitions from one
      level to the other, the FF may trigger erratically (unpredictably) or
      not at all.


\hi{Potential Timing Problem in FF Circuits}
  \par \tb{The FF output will go to a state determined by the logic levels
    present at its synchronous control inputs \ti{just prior} to the active
    clock transition}.