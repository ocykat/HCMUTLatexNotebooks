\chapter{Heat and the First Law of Thermodynamics}

\hi{Terminology}

\hi{Heat and Internal Energy}
    \hii{Internal Energy}
        \impt{Internal Energy} is all the energy of a system that is associated with
        its microscopic components - atoms and molecules - when viewed from a
        reference frame at rest with respect to the center of mass of the system.

    \hiiBEGIN{Heat}
        \hiii{Definition}
            \par \impt{Heat} is defined as the transfer of energy across the
            \textit{boundary} of a system due to a temperature difference between
            the system and its surroundings.
        \hiii{Units}
            \par $1 cal$ is defined as the amount of energy transfer necessary to raise 
            the temperature of $1g$ of water from $14.5 \degree C$ to $15.5 \degree C$.
            \begin{align*}
                1 Cal = 1000 cal
            \end{align*}
            \par Heat is also measured in \impt{joules}.
            \begin{align*}
                1 cal = 4.186 J
            \end{align*}
        \hiii{The Mechanical Equivalent of Heat}
            \par We have already found that whenever friction is present in a
            mechanical system, some mechanical energy is lost. In other words,
            mechanical energy is not conserved in the presence of nonconservative force.
            \par The lost energy is transformed into internal energy.
            \par Read the \textit{Joule's experiment}.
    \hiiEND

\hi{Specific Heat and Calorimetry}
    \hii{Heat capacity}
        \par The \impt{heat capacity} $C$ of a particular sample of a substance is
        defined as the amount of energy needed to raise the temperature of that sample by
        $1 \degree C$.
        \begin{equation}
            Q = C \Dt T
        \end{equation}
    \hii{Specific heat}
        \par The \impt{specific heat} $c$ of a substance is the heat capacity per unit mass.
        \begin{equation}
            Q = mc \Dt T
        \end{equation}
    \hiiBEGIN{Calorimetry}
        \hiii{Technique of Calorimetry}
            \par One technique for measuring specific heat involves heating a sample to 
            some known mass $m_{x}$ and temperature $T_{x}$, placing it in a vessel 
            containing water of known mass $m_{w}$ and temperature $T_{w} < T_{x}$, and
            measuring the temperature of the water after equilibrium has been reached.
            \par This technique is called \impt{calorimetry}, and the devices in which
            this energy tranfer occurs are called \impt{calorimeters}.
        \hiii{Applying Conservation of Energy to Calorimetry}
            \par Applying the theory of Conservation of Energy:
            \begin{equation}
                Q_{hot} + Q_{cold} = 0
            \end{equation}
            or
            \begin{equation} \label{eq:calorimetry}
                Q_{hot} = - Q_{cold}
            \end{equation}
            \par From the equation \eqref{eq:calorimetry}, we have:
            \begin{align*}
                m_{x}c_{x}(T_{x} - T_{f}) = m_{w}c_{w}(T_{f} - T_{w})
            \end{align*}
            in which $T_{f}$ is the final temperature of the equilibrium.
    \hiiEND

\hi{Latent Heat}
    \par \impt{Latent heat} is thermal energy released or absorbed, by a body or a
    thermodynamic system, during a constant-temperature process - usually a first-order
    phase transition.
    \par When the physical characteristics of a substance change from one form to
    another, such change is commonly referred to as \impt{phase change}. Common phase
    changes are:
    \begin{itemize}
        \item Solid to liquid and vice versa. 
        \item Liquid to gas and vice versa. 
        \item Change in the crystalline structure of a solid.
    \end{itemize}
    \par The energy required to change the phase of a given mass $m$ of a pure substance is:
    \begin{equation}
        Q = \pm mL
    \end{equation}
    in which: $L$ is the latent heat.
    \par \impt{Latent heat of fusion} $L_{f}$ is the term used when the phase change is
    from solid
    to liquid.
    \par \impt{Latent heat of vaporization} $L_{v}$ is the term used when the phase
    change is from liquid to gas.

\hi{Work and Heat in Thermodynamic Processes}
    \hiiBEGIN{Work}
        \hiii{Experiment}
            \par Consider a gas container in a cylinder fitted with a movable piston. At
            equilibrium, the gas occupies a volume $V$ and exerts a uniform pressure $P$
            on the cylinder's walls and on the piston.
            \par If the piston has a cross-sectional area $A$, the force exerted by the
            gas on the piston is $F = PA$.
            \par Assume that the piston is pushed inward and and the gas is compressed
            \impt{quasi-statically} (slowly enough to allow the system to remain in
            thermal equilibrium all the time). As the piston is pushed downward, the 
            instantaneous work done is:
            \begin{equation}
                dW = \vt{F} \cdot dr = -F \hat{j} \cdot dy \hat{j} = -Fdy = -PAdy = -PdV
            \end{equation}
            \par Therefore:
            \begin{eqbox}
                W = - \INT{_{V_{i}}^{V_{f}} PdV}
            \end{eqbox}
            where $W$ is the \textbf{work done on the system}.
        \hiii{Result}
            \par \textit{The work done on a gas in a quasi-static process that takes the
            gas from an initial state to a final state is the negative of the area under 
            the curve on a $PV$ diagram, evaluated between the initial and final states.}
            \par Therefore, $W$ \impt{depends on the particular path taken}.
    \hiiEND
    \hii{Heat}
        \par Like work done, energy transfer by heat also \impt{depends on the particular
        path taken}
        \par Read page 617.

\hi{The First Law of Thermodynamics}
    \par The change in internal energy can be expressed as:
    \begin{eqbox}
        \Dt E_{int} = Q + W
    \end{eqbox}
    \par While $Q$ and $W$ are dependent of the path, $E_{int}$ is not dependent of the
    path. It is a state variable.

\hi{Some Applications of the First Law of Thermodynamics}
    \hii{Adiabatic process}
        \par An \impt{adiabatic process} is one during which \impt{no energy enters or
        leaves the system by heat}.
        \begin{equation}
            \Dt E_{int} = W
        \end{equation}
        \par An adiabatic process can be achieved through:
        \begin{itemize}
            \item insulating the walls of the system
            \item performing the process rapidly, so that there is negligible time for energy to tranfer by heat.
        \end{itemize}
    \hii{Isobaric process}
        \par A process that \impt{occurs at a constant pressure} is called an
        \impt{isobaric process}.
        \begin{equation}
            \Dt E_{int} = Q + W = Q - W_{sys} = Q - P \Dt V
        \end{equation}
    \hii{Isovolumetric process}
        \par A process that takes place at a constant volume is called an
        \impt{isovolumetric process}.
        \par In such a process, the value of the work done is zero.
        \begin{equation}
            \Dt E_{int} = Q
        \end{equation}
    \hii{Isothermal process}
        \par A process that occurs at a constant temperature is called an \impt{isothermal
        process}.
        \par Since the internal energy of an ideal gas is a function of temperature only, 
        in a isothermal process:
        \begin{equation}
            \Dt E = Q + W = 0
        \end{equation}
        or
        \begin{equation}
            Q = -W
        \end{equation}
    \hii{Isothermal Expansion of an Ideal Gas}
        \begin{flalign*}
            W = - \INT{_{V_{i}}^{V_{f}} PdV}
            = - \INT{_{V_{i}}^{V_{f}} \frac{nRT}{V} dV}
            = - nRT \INT{_{V_{i}}^{V_{f}} \frac{dv}{V}}
            = nRT\ln\Big(\frac{V_{i}}{V_{f}}\Big)
        \end{flalign*}
