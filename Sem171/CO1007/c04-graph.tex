\chapter{GRAPH}
\hi{Graphs and Graph Models}
    \hii{Graphs}
        \par A \impt{graph} $G = (V, E)$ consists of $V$, a \textit{nonempty} set of
        \impt{vertices} (or \impt{nodes}) and $E$, a set of \impt{edges}.
        \par Each edge has either one or two vertices associated with it, called its
        \impt{endpoints}.
        \par An edge is said to \impt{connect} its endpoints.
    \hii{Finite and infinite graph}
        \par A graph with a finite vertex set and a finite edge set is called a \impt{finite graph}.
        \par A graph with an infinite vertex set and an finite edge set is called an
        \impt{infinite graph}.
    \hii{Simple graph}
        \par A simple graph is a graph in which:
        \begin{itemize}
            \item each edge connects two different vertices
            \item no two edges connect the same pair of vertices
        \end{itemize}
    \hii{Multigraph}
        \par Graphs that \textit{may} have \impt{multiple edges} connecting the same vertices are
        called \impt{multigraphs}.
        \par When there are $m$ different edges associated to the same unordered pair of vertices
        ${u, v}$, we also say that ${u, v}$ is an edge of \impt{multiplicity} $m$.
        \par If $G$ is a simple graph, it is definitely a multigraph. The other way around is NOT
        correct.
    \hii{Pseudograph}
        \par Graphs that \textit{may} have \impt{loops} and \textit{may} have multiple edges
        connecting the same pair of vertices to itself are called \impt{pseudograph}.
        \par If $G$ is a multigraph, it is definitely a pseudograph. The other way around is NOT
        correct.
    \hii{Directed graph}
        \par A \impt{directed graph} (or \impt{digraph}) $G = (V, E)$ consists of a nonempty
        set of vertices $V$ and a set of \impt{directed edges} (or \impt{arcs}) $E$.
        \par Each \impt{directed edge} is associated with an \textit{ordered pair} of vertices.
        \par The directed edge associated with the ordered pair $(u, v)$ is said to
        \textit{start} at $u$ and \textit{end} at $v$.
    \hii{Multiple directed graph}
        \par Directed graphs that \textit{may} have \impt{multiple directed edgeds} from a vertex
        to a second (possibly the same) vertex are called \impt{directed multigraphs}.
        \par When there are $m$ different directed edges associated to the same ordered pair
        $(u, v)$, we say that $(u, v)$ is an edge of \impt{multiplicity} $m$.

\hi{Graph Terminology and Special Types of Graphs}
    \hii{Adjacency in an Undirected graph}
        \par Two vertices $u$ and $v$ in an undirected graph $G$ are called \impt{adjacent}
        (or \impt{neighbors}) in $G$ if $u$ and $v$ are endpoints of an edge $e$ of $G$.
        \par Such an edge $e$ is called \impt{incident} with the vertices $u$ and $v$.
        \par $e$ is said to \impt{connect} $u$ and $v$.
        \par The set of all neighbors of a vertex $v$ of $G = (V, E)$, denoted by $N(v)$, is
        called the \impt{neighborhood} of $v$.
        \par If $A$ is a subset of $V$, we denote by $N(A)$ the set of all vertices in $G$
        that are adjacent to at least one vertex in $A$.
        \begin{equation}
            N(A) = \bigcup_{v \in A} N(v)
        \end{equation}

    \hii{Degree of a vertex in an Undirected graph}
        \par The \impt{degree} of a vertex in an \impt{undirected graph} is the number of
        edges incident with it, except that a loop contributes twice to the degree of that
        vertex. The degree of the vertex $v$ is denoted by $deg(v)$.
        \begin{equation}
            deg(v) = \big(\mbox{\# edges incident with } v \big) +
            \big( 2 \cdot \mbox{ \# loops} \big)
        \end{equation}
        \par A vertex of degree zero is called \impt{isolated}.
        \par A vertex of degree one is called \impt{pendant}.

    \hii{The Handshaking Theorem for Undirected graph}
        \par Let $G = (V, E)$ be an undirected graph. Then
        \begin{equation}
            2|E| = \SUM{_{v \in V} deg(v)}
        \end{equation}
        \par An undirected graph has an even number of vertices of odd degree.

    \hii{Adjacency in a Directed graph}
        \par When $(u, v)$ is an edge of the graph $G$ with directed edges:
        \begin{itemize}
            \item $u$ is said to be \impt{adjacent to} $v$
            \item $v$ is said to be \impt{adjacent from} $u$
            \item $u$ is called the \impt{initial vertex} of $(u, v)$
            \item $u$ is called the \impt{terminal vertex} of $(u, v)$
        \end{itemize}
        \par The initial vertex and terminal vertex of a loop are the same.

    \hii{Degree of a vertex in a Directed graph}
        \par In a directed graph:
        \begin{itemize}
            \item the \impt{in-degree} of a vertex $v$, denoted by $deg^{-}(v)$, is the number
                of edges with $v$ as their terminal vertex.
            \item the \impt{out-degree} of a vertex $v$, denoted by $deg^{+}(v)$, is the number
                of edges with $v$ as their initial vertex.
        \end{itemize}
        \par A loop at a vertex contributes 1 to both the in-degree and the out-degree of this
        vertex.

    \hii{The Handshaking Theorem for Directed graph}
        \par Let $G = (V, E)$ be a graph with directed edges. Then:
        \begin{equation}
            \SUM{_{v \in V} deg^{-}(v)} = \SUM{_{v \in V} deg^{+}(v)} = |E|
        \end{equation}

    \hiiBEGIN{Special Simple Graphs}
        \hiii{Complete Graph}
            \par A \impt{complete graph} on $n$ vertices, denoted by $K_{n}$, is a simple graph that
            \textit{contains exactly one edge between each pair of distinct vertices}.

        \hiii{Cycle}
            \par A \impt{cycle} $C_{n} (n \geq 3)$ consists of $n$ vertices $v_{1}, v_{2}, \ldots, v_{n}$
            and edges $\{v_{1}, v_{2}\}, \{v_{2}, v_{3}\}, \ldots, \{v_{n-1}, v_{n}\}$.

        \hiii{Wheel}
            \par A \impt{wheel} $W_{n}$ can be obtained when adding an additional vertex to a cycle
            $C_{n} (n \geq 3)$ and connect this new vertex to each of the $n$ vertices in $C_{n}$, by
            new edges.

        \hiii{$n-$cube}
            \par An \impt{n-dimensional hypercube}, or \impt{$n-$cube}, denoted by $Q_{n}$, is a
            graph that has vertices representing the $2^{n}$ bit strings of length $n$. Two
            vertices are adjacent if and only if the bit strings that they represent differ in
            exactly one bit position.
    \hiiEND

    \hiiBEGIN{Bipartite Graphs}
        \hiii{Definition}
            \par A simple graph $G$ is called \impt{bipartite} if its vertex set $V$ can be
            partitioned into two disjoint set $V_{1}$ and $V_{2}$ such that no edge in the graph
            connects two vertices from the same set.
        \hiii{Theorem}
            \par A simple graph is \impt{bipartite} if and only if it is possible to assign one of
            two different colors to each vertex of the graph so that no two adjacent vertices are
            assigned the same color.
            \par In simple word: A bipartite graph can be colored with 2 colors.
        \hiiiBEGIN{Matchings}
            \hiv{Definition}
                \par A \impt{matching} $M$ in a simple graph $G = (V, E)$ is a subset of the set
                $E$ of edges of the graph such that no two edges are incident with the same vertex.
                In order words, a matching is a subset of edges such that if ${s, t}$ and ${u, v}$
                are distinct edges of the matching, then $s$, $t$, $u$, and $v$ are distinct.
            \hiv{Terminology}
                \begin{itemize}
                    \item A vertex that is the endpoint of an edge of a matching $M$ is said to be
                        \impt{matched} in $M$; otherwise it is said to be \impt{unmatched}.
                    \item A \impt{maximum matching} is a matching with the largest number of edges.
                    \item A matching $M$ is a \impt{complete matching} from $V_{1}$ to $V_{2}$ if
                        every vertex in $V_{1}$ is the endpoint of an edge in the matching, or
                        equivalently, if $|M| = |V_{1}|$.
                \end{itemize}
            \hiv{Hall's Marriage Theorem}
                \par The bipartite graph $G = (V, E)$ with bipartition $(V_{1}, V_{2})$
                has a complete matching from $V_{1}$ to $V_{2}$ if and only if the number 
                of vertices in $V_{1}$ is less than or equals to the total number of
                their neighbors.
                $(|N(A)| \geq |A|)$.
                \par In simple words: The number of men is always less than or equals to
                the number of spouses they have.
        \hiiiEND
    \hiiEND

    \hiiBEGIN{New Graphs from Old}
        \hiii{Subgraph}
            \par A \impt{subgraph} of a graph $G = (V, E)$ is a graph $G_{1} = (V_{1}, E_{1})$ where
                $V_{1} \subseteq V$ and $E_{1} \subseteq E$.
            \par A subgraph $G_{1}$ of $G$ is a \impt{proper subgraph} if $G_{1} \neq G$.
            \par The \impt{subgraph induced} by a subset $V_{1}$ of the vertex set $V$ is the graph
                $(V_{1}, E_{1})$, where the edge set $E_{1}$ contains an edge in $E$ if an only if
                both endpoints of this edge are in $W$.
        \hiii{Graph Union}
            \par The \impt{union} of two simple graphs $G_{1} = (V_{1}, E_{1})$ and
                $G_{2} = (V_{2}, E_{2})$ is the simple graph $G_{1} \cup G_{2}$ with vertex set
                $V_{1} \cup V_{2}$ and edge set $E_{1} \cup E_{2}$.
    \hiiEND

\hi{Representing Graphs and Graph Isomophism}
    \hii{Adjacency Lists}
    
    \hii{Adjacency Matrix}

    \hii{Trade-off between adjacency lists and adjacency matrices}
        \par When the graph is \impt{sparse}, or the number of edges is small, it is preferable
        to use adjacency lists to limit the number of entries.
        \par When the graph is \impt{dense}, or the number of edges is big, it is preferable to
        use adjacency matrix to limit the number of comparisons needed when traversing the lists.

    \hii{Incidence Matrices}

    \hii{Isomophism of Graphs}
        \par The simple graphs $G_{1} = (V_{1}, E_{1})$ and $G_{2} = (V_{2}, E_{2})$ are isomorphic
        if there exists a one-to-one and onto function $f$ from $V_{1}$ to $V_{2}$ with the property
        that $a$ and $b$ are adjacent in $G_{1}$ if and only if $f(a)$ and $f(b)$ are adjacent in
        $G_{2}$, for all $a$ and $b$ in $V_{1}$. Such a function $f$ is called an isomorphism.

\hi{Connectivity}
    \hii{Terminology}
        \par Let $G$ be an undirected graph.
        \begin{itemize}
            \item A \impt{path} of length $n$ from $u$ to $v$ in $G$ is a sequence of $n$ edges
                connecting from $u$ to $v$.
            \item A \impt{circuit} is a path that begins and ends at the same vertex.
            \item A path or circuit is said to \impt{pass through} the vertices or \impt{traverse}
                the edges.
            \item A path or circuit is \impt{simple} if it does not contain the same edge more
                than once.
            \item A \impt{walk} is equivalent to a path.
            \item A \impt{closed walk} is equivalent to a circuit.
            \item A \impt{trail} is equivalent to a simple path.
        \end{itemize}


    \hii{Cut vertices and Cut edges}
        

    \hii{Connectedness in an Undirected Graph}
        \par An undirected graph is called \impt{connected} if there is a \impt{path} between
            every pair of distinct vertices of the graph.

    \hii{Connectedness in an Directed Graph}
        \par A directed graph is \impt{strongly connected} if there is a path from $a$
            to $b$ and from $b$ to $a$ whenever $a$ and $b$ are vertices in the graph.
         \par A directed graph is \impt{weakly connected} if there is a path between
            every two vertices in the underlying undirected graph.

    \hii{Counting Paths between Vertices}
        \par Let $G$ be a graph with adjacency matrix $A$ with respect to the ordering
        $v_{1}, v_{2}, \ldots, v_{n}$ of the the vertices of the graph (with directed or
        undirected edges, with multiple edges and loops allowed). The number of different paths
        of length $r$ from $v_{i}$ to $v_{j}$, where $r$ is a positive integer, equals the
        $(i, j)$th entry of $A^{r}$.

\hi{Euler and Hamilton Paths}
    \hiiBEGIN{Euler paths \& Euler circuits}
        \hiii{Definition}
            \par An \impt{Euler path} in a graph $G$ is a simple path containing every
            \impt{edge} of $G$.
            \par An \impt{Euler circuit} in a graph $G$ is a simple circuit containing
            every \impt{edge} of $G$ 
        \hiii{Theorem on Euler circuit}
            \par A connected multigraph with at least two vertices has an
            \textbf{Euler circuit} if and only if \impt{each} of its vertices has
            \impt{even degree}.
        \hiii{Theorem on Euler path}
            \par A connected multigraph has an \textbf{Euler path} but not an 
            \textbf{Euler circuit} if and only if it has exactly two vertices of
            \impt{odd degree}.
        \hiii{Fleury's algorithm}
            \par Fleury's algorithm can be used to find an Euler circuit in a graph $G$.
    \hiiEND

    \hiiBEGIN{Hamilton paths \& Hamilton circuits}
        \hiii{Definition}
            \par An \impt{Hamilton path} in a graph $G$ is a simple path containing every
            \impt{vertex} of $G$.
            \par An \impt{Hamilton circuit} in a graph $G$ is a simple circuit containing
            every \impt{vertex} of $G$.
        \hiii{Dirac's theorem}
            \par If $G$ is a simple graph with $n$ vertices with $n \geq 3$ such that the
            degree of every vertex in $G$ is at least $n/2$, then $G$ has a Hamilton
            circuit.
        \hiii{Ore's theorem}
            \par If $G$ is a simple graph with $n$ vertices with $n \geq 3$ such that
            $deg(u) + deg(v) \geq n$ for every pair of nonadjacent vertices $u$ and $v$
            in $G$, then $G$ has a Hamilton circuit.
    \hiiEND

\hi{Shortest-Path Problems}

\hi{Planar Graphs}
    \hii{Definition}
        \par A graph is called \impt{planar} if it can be drawn in the plane without any
        edges crossing (where a crossing of edges is the intersection of the lines or 
        arcs representing them at a point other than their common endpoint).
    \hii{Euler's Formula}
        \par Let $G$ be a connected planar simiple graph with $e$ edges and $v$ vertices.
        Let $r$ be the number of regions in a planar representation of $G$. Then
        $r = e - v + 2$.
        \par \textbf{Corollary 1}: If $G$ is a connected planar simple graph with $e$
        edges and $v$ vertices, where $v \geq 3$, then $e \leq 3v - 6$.
        \par \textbf{Corollary 2}: If $G$ is a connected planar simple graph with $e$
        edges and $v$ vertices, where $v \geq 3$ and there is no circuit of length three,
        then $e \leq 2v - 4$.
        \par \textbf{Corollary 3}: If $G$ is a connected planar simple graph, then $G$
        has a vertex of degree not exceeding five.
    \hiiBEGIN{Kuratowski's Theorem}
        \hiii{Theorem}
            \par A graph is nonplanar if and only if it contains a subgraph
            homeomorphic to $K_{3, 3}$ of $K_{5}$.
        \hiii{Homeomorphic}
            \par Given two different graphs. On any edge of this graph, add one or more 
            vertices to it. If after adding, two graphs are isomorphic, then the original
            graphs are homeomorphic.
        \hiiiEND

\hi{Graph Coloring}
    \hii{Definition}
        \par A \impt{coloring} of a simple graph is the assignment of a color to each
        vertex of the graph so that \impt{no two adjacent vertices are assigned the same
        color}.
    \hii{Chromatic number}
        \par The \impt{chromatic number} of a graph is the least number of color needed
        for a coloring of this graph, denoted by $chi(G)$.
    \hii{The Four Color Theorem}
        \par The chromatic number of a planar graph is no greater than four.
