\documentclass[12pt, a4paper]{report}

% MARGINS
\usepackage[margin=0.75in]{geometry}

% FONT
\renewcommand{\familydefault}{\sfdefault}

% LINE SPACING
\setlength{\parskip}{6pt}

% HEADINGS
\setcounter{secnumdepth}{4}
\newcommand{\hi}{\section}
\newcommand{\hii}{\subsection}
\newcommand{\hiii}{\subsubsection}
\newcommand{\hiiiBEGIN}[1]{\subsubsection \begin{enumerate}}
\newcommand{\hiiiEND}{\end{enumerate}}
\newcommand{\hiv}{\item\textbf}

% INDENTATION
\usepackage{indentfirst}

% TABLE OF CONTENTS
% links
\usepackage{hyperref}
\hypersetup{
  colorlinks,
  citecolor=black,
  filecolor=black,
  linkcolor=black,
  urlcolor=black
}

% APPENDICES
\usepackage[toc,page]{appendix}

% TEXT
% Bold, italic, underlined text
\newcommand{\tb}[1]{\textbf{#1}}
\newcommand{\ti}[1]{\textit{#1}}
\newcommand{\tbi}[1]{\textbf{\textit{#1}}}
\newcommand{\tu}[1]{\underline{#1}}
\newcommand{\tbu}[1]{\textbf{\underline{#1}}}
\newcommand{\smf}[1]{\small #1 \normalsize}
\newenvironment{smfont}{\small}{\normalsize}

% MATH
\usepackage{amsmath}
\usepackage{amssymb}
\usepackage{gensymb}

% Equation box
\usepackage{empheq}
\newenvironment{eqbox}
  {\setkeys{EmphEqEnv}{align}\setkeys{EmphEqOpt}{box=\fbox}\EmphEqMainEnv}
  {\endEmphEqMainEnv}

% Arrows
\newcommand{\ra}{\Rightarrow}
\newcommand{\lra}{\Leftrightarrow}

% Sum, product, limit
\newcommand{\SUM}[1]{\sum\limits #1}
\newcommand{\PROD}[1]{\prod\limits #1}

% Bold in math mode
\usepackage{bm}

% Absolute value
\DeclarePairedDelimiter\abs{\lvert}{\rvert}%
\DeclarePairedDelimiter\norm{\lVert}{\rVert}%
%   Swap the definition of \abs* and \norm*, so that \abs
%   and \norm resizes the size of the brackets, and the
%   starred version does not.
\makeatletter
\let\oldabs\abs
\def\abs{\@ifstar{\oldabs}{\oldabs*}}
\let\oldnorm\norm
\def\norm{\@ifstar{\oldnorm}{\oldnorm*}}
\makeatother

% Floor and ceiling
\usepackage{mathtools}
\DeclarePairedDelimiter\ceil{\lceil}{\rceil}
\DeclarePairedDelimiter\floor{\lfloor}{\rfloor}
\DeclarePairedDelimiter\set\{\}

% Inner Product
\newcommand{\iprod}[1]{\langle #1 \rangle}
\newcommand{\pair}[1]{\langle #1 \rangle}

% Derivatives
\newcommand{\dif}[2]{\dfrac{d #1}{d #2}}
\newcommand{\ddif}[2]{\dfrac{d^2 #1}{d #2^2}}
\newcommand{\difi}[2]{d #1/d #2}
\newcommand{\diff}[2]{\dfrac{d}{d #2} #1}
\newcommand{\pd}[2]{\dfrac{\partial #1}{\partial #2}}
\newcommand{\pdds}[2]{\dfrac{\partial^{2} #1}{\partial #2^{2}}}
\newcommand{\pdd}[3]{\dfrac{\partial^{2} #1}{\partial #2 \partial #3}}
\newcommand{\pddsi}[2]{{\partial^{2} #1} / {\partial #2^{2}}}
\newcommand{\pddi}[3]{{\partial^{2} #1} / {\partial #2 \partial #3}}

% Integrals
\usepackage{esint}
\newcommand{\INT}{\int \limits}
\newcommand{\OINT}{\oint \limits}
\newcommand{\IINT}{\iint \limits}
\newcommand{\IIINT}{\iiint \limits}

% Ratios
\newcommand{\ratio}[3]{\dfrac{#1_{#2}}{#1_{#3}}}

% Manual equation number
\newcommand{\eqnumber}[1]{\qquad\mbox{\tb{{(#1)}}}}

% Script fonts
\usepackage[mathscr]{euscript}

% FOOTNOTE
%   one foot note stays on one page
\interfootnotelinepenalty=10000

\newcommand{\fnmark}{\footnotemark}
\newcommand{\fnmarksame}{\footnotemark[\value{footnote}]}
\newcommand{\fntext}{\footnotetext}

% IMAGES
\usepackage{graphicx}
\usepackage{subcaption}
\usepackage{float}
\newcommand{\img}[2][]
  {
    \begin{figure}[H]
      \centering
      \includegraphics[#1]{#2}
    \end{figure}
  }

% PSEUDOCODE
\usepackage{algorithm}
\usepackage{algorithmicx}
\usepackage[noend]{algpseudocode}
\usepackage{caption}

\renewcommand{\thealgorithm}{\arabic{chapter}.\arabic{algorithm}}

\newcommand*\Let[2]{\State #1 $\gets$ #2}
\algrenewcommand\algorithmicrequire{\textbf{Input:}}
\algrenewcommand\algorithmicensure{\textbf{Output:}}
\newcommand{\INPUT}[1]{\Require{#1} \Statex}
\newcommand{\OUTPUT}[1]{\Ensure{#1} \Statex}
\newcommand{\INPUTOUTPUT}[2]{\Require{#1} \Ensure{#2} \Statex}
\newcommand{\LET}[2]{\Let{$#1$}{$#2$}}
\newcommand{\FOR}[2]{\For{$#1 \gets #2$}}
\newcommand{\ENDFOR}{\EndFor}
\newcommand{\TO}{\textrm{ \tb{to} }}
\newcommand{\DOWNTO}{\textrm{ \tb{downto} }}
\newcommand{\AND}{\textrm{ \tb{and} }}
\newcommand{\OR}{\textrm{ \tb{or} }}
\newcommand{\XOR}{\textrm{ \tb{xor} }}
\newcommand{\GETS}{\gets}
\newcommand{\IF}[1]{\If{$#1$}}
\newcommand{\ELSEIF}[1]{\ElsIf{$#1$}}
\newcommand{\ELSE}{\Else}
\newcommand{\ENDIF}{\EndIf}
\newcommand{\WHILE}[1]{\While{$#1$}}
\newcommand{\ENDWHILE}{\EndWhile}
\newcommand{\FUNCTION}[2]{\Function{#1}{$#2$}}
\newcommand{\ENDFUNCTION}{\EndFunction}
\newcommand{\PROCEDURE}[2]{\Procedure{#1}{$#2$}}
\newcommand{\ENDPROCEDURE}{\EndProcedure}
\newcommand{\CALLFUNC}[2]{\State \Call{#1}{$#2$}}
\newcommand{\CALLPROC}[2]{\State \Call{#1}{$#2$}}
\newcommand{\RETURN}[1]{\State \Return{#1}}
\newcommand{\SHIFTLEFT}{\ll}
\newcommand{\SHIFTRIGHT}{\gg}
\newcommand{\Input}[1]{\hspace*{\algorithmicindent} \textbf{Input}: #1}

% FRENCH CONVENTION FOR VERTICAL ARITHMETIC
\usepackage{xlop}

% CODE
\usepackage{listings}
%\makeatletter
%\lst@CCPutMacro\lst@ProcessOther {"2D}{\lst@ttfamily{-{}}{-{}}}
%\@empty\z@\@empty
%\makeatother
\usepackage{inconsolata}
% \lstset{
%   language=[x86masm]Assembler,
%   aboveskip=1mm,
%   belowskip=1mm,
%   numbers=left,
%   frame=tb,
%   % language=[mips]Assembler,
%   basicstyle=\ttfamily
%   \footnotesize,
%   breaklines=true,
%   mathescape=false
% }
% \lstdefinestyle{SQL}{
% \lstset{
%    language=SQL,
%    showspaces=false,
%    basicstyle=\ttfamily,
%    numbers=left,
%    numberstyle=\tiny,
%    commentstyle=\color{gray}
%    frame=tb,
%    breaklines=true,
%    mathescape=false
%    aboveskip=1mm,
%    belowskip=1mm,
%   emph={%
%     AUTHORIZATION%
%     },emphstyle={\color{red}\bfseries\underbar}%
% }%

\newcommand{\ilc}{\lstinline}

\usepackage[cache=false]{minted}


%
\usepackage{varwidth}
\newenvironment{fboxenv}
  {
    \begin{minipage}{\dimexpr\textwidth-2\fboxsep-2\fboxrule\relax}
  }
  {
    \end{minipage}
  }

% SOURCE
%\newcommand{\code}[1]{\texttt{#1}}
\newcommand{\refsource}[1]{Source: {#1}}

\newcommand{\fieldname}[1]{\langle \mbox{#1} \rangle}

\def\ojoin{\setbox0=\hbox{$\bowtie$}%
  \rule[-.02ex]{.25em}{.4pt}\llap{\rule[\ht0]{.25em}{.4pt}}}
\def\leftouterjoin{\mathbin{\ojoin\mkern-5.8mu\bowtie}}
\def\rightouterjoin{\mathbin{\bowtie\mkern-5.8mu\ojoin}}
\def\fullouterjoin{\mathbin{\ojoin\mkern-5.8mu\bowtie\mkern-5.8mu\ojoin}}

\begin{document}
\tableofcontents

\hi{Logic}

\hii{Natural Deduction Rules}

% AndIntroduction
\newsavebox\AndIntro
\sbox\AndIntro{
  \AxiomC{$\phi$}
  \AxiomC{$\psi$}
  \RightLabel{$\land i$}
  \BinaryInfC{$\phi \land \psi$}
  \DisplayProof
}

% AndElimination
\newsavebox\AndElim
\sbox\AndElim{
  \AxiomC{$\phi \land \psi$}
  \RightLabel{$\land e_1$}
  \UnaryInfC{$\phi$}
  \DisplayProof
  \hskip 1cm
  \AxiomC{$\phi \land \psi$}
  \RightLabel{$\land e_2$}
  \UnaryInfC{$\psi$}
  \DisplayProof
}

% Double-Negation Introduction
\newsavebox\DNegIntro
\sbox\DNegIntro{
  \AxiomC{$\phi$}
  \RightLabel{$\lnotnot i$}
  \UnaryInfC{$\lnotnot \phi$}
  \DisplayProof
}

% Double-Negation Elimination 
\newsavebox\DNegElim
\sbox\DNegElim{
  \AxiomC{$\lnotnot \phi$}
  \RightLabel{$\lnotnot e$}
  \UnaryInfC{$\phi$}
  \DisplayProof
}

% 

\begin{tabular}{|c|c|c|}
  \hline
  Rule              & Introduction      & Elimination      \\ \hline
  Conjunction (AND) & \usebox\AndIntro  & \usebox\AndElim  \\ \hline
  Double Negation   & \usebox\DNegIntro & \usebox\DNegElim \\ \hline
\end{tabular}

\chapter{Partial Derivatives}

\hi{Function of two variables}
    \hii{Definition}
        \par A function of two variables is a rule that assigns
        to each ordered pair of real numbers $(x, y)$ in a set a unique real number
        denoted by $f(x, y)$. The set $D$ is the \textbf{domain} of $f$ and its
        \textbf{range} is the set of values that $f$ takes on.
        \par We often write:
        \begin{eqbox}
            z = f(x, y)
        \end{eqbox}
        where:
        \begin{itemize}
            \item $x$, $y$: independent variables
            \item $z$: dependent variable
        \end{itemize}
    \hii{Domain}
        \par Given the function: $f(x, y)$. The domain of $f$ is the
        \textbf{Cartesian product} of the set $D_{x}$ and the set $D_{y}$, where
        $D_{x}$ and $D_{y}$ are the set of all possible values of $x$ and $y$,
        respectively. This is true for any multivariable function.
        \begin{eqbox}
            D = D_{x} \times D_{y} \subseteq \mathbb{R}^{2}
        \end{eqbox}
    \hii{Graph}
        \par If $f$ is a function of two variables with domain $D$, then the graph
        of $f$ is the set of all points $(x, y, z)$ in $\mathbb{R}^{3}$ such that
        $z = f(x, y)$ and $(x, y) \in D$.
    \hii{Level Curves}
        \par The \textbf{level curves} of a function $f$ of two variables are the curves
        with equations $f(x, y) = k$, where $k$ is a constant (in the range of $f$).
        \par The graph of level curves is called the \textbf{contour graph}.
        \par To construct a contour graph, the set of $k$ is required.

\hi{Limits and Continuity}
    \hiiBEGIN{Limits}
        \hiii{Definition}
            \par Let $f$ be a function of two variables whose domain $D$ includes points arbitrarily
            close to $(a, b)$. Then we say that the \textbf{limit} of $f(x, y)$ as $(x, y)$ approaches
            $(a, b)$ is $L$ and we write
            \begin{eqbox}
                \lim_{(x, y) \to (a, b)} f(x, y) = L
            \end{eqbox}
            if for every number $\epsilon > 0$ there is a corresponding number $\delta > 0$ such that:
            \begin{center}
                if $(x, y) \in D$ and $0 < \sqrt{(x - a)^{2} - (y - b)^{2})} < \delta$, then
                $|f(x, y) - L| < \epsilon$
            \end{center}
        \hiii{Existence of limit}
            \par For a function of one variable
            \begin{center}
                If $\lim_{x \to a^{-}} f(x) \neq \lim_{x \to a^{+}}$, then $\lim_{x \to a} f(x)$ does
                not exists.
            \end{center}
            \par For a multivariable function, the limit at one point can be approaches from
            infinitely many directions. If there \textbf{exists two different paths} of approach
            along which the function $f(x, y)$ has different limits, then the limit at that point
            does not exists.
    \hiiEND
    \hii{Continuity}
        \par A function $f$ of two variables is called \textbf{continuous} at $(a, b)$ if:
        \begin{eqbox}
            \lim_{(x, y) \to (a, b)} f(x, y) = f(a, b)
        \end{eqbox}
        \par We say $f$ is continuous on $D$ if $f$ is continuous at every point $(a, b)$ in $D$.

\hi{Partial Derivatives}
    \hii{Definition}
        \par Given the function $f(x, y)$.
        \par The \textbf{partial derivative} of $f$ with respect to $x$ at $(a, b)$, denoted by
        $f_{y}(a, b)$, is obtained by keeping $y$ fixed $(y = b)$ and finding the ordinary
        derivative at $a$ of the function $G(x) = f(x, b)$.
        \begin{eqbox}
            f_{x} (x, y) = \lim_{\Delta x \to 0} \frac{f(x + \Delta x, y) - f(x, y)}{\Delta x} \\
            f_{y} (x, y) = \lim_{\Delta y \to 0} \frac{f(x, y + \Delta y) - f(x, y)}{\Delta y}
        \end{eqbox}
    \hii{Notations}
        \begin{eqbox}
            f_{x} (x, y) = f_{x} = \pd{f}{x} = \pd{}{x} f(x, y) = \pd{z}{x} = D_{1}f = D_{x}f \\
            f_{y} (x, y) = f_{y} = \pd{f}{y} = \pd{}{y} f(x, y) = \pd{z}{y} = D_{2}f = D_{y}f
        \end{eqbox}
    \hii{Rule for Finding Partial Derivatives}
        \par Given the function $z = f(x, y)$.
        \begin{itemize}
            \item To find $f_{x}$, regard $y$ as a constant and differentiate $f(x, y)$ with respect
                to $x$.
            \item To find $f_{y}$, regard $x$ as a constant and differentiate $f(x, y)$ with respect
                to $y$.
        \end{itemize}
    \hii{Higher Derivatives}
        \begin{alignat*}{4}
            (f_{x})_{x} (x, y) = f_{xx} = f_{11}
                &= \pd{}{x}\bigg(\pd{f}{x}\bigg)
                &= \pd{^{2}f}{x^{2}}
                &= \pd{^{2}z}{x^{2}} \\
            (f_{x})_{y} (x, y) = f_{xy} = f_{12}
                &= \pd{}{y}\bigg(\pd{f}{x}\bigg)
                &= \pdd{^{2}f}{y}{x}
                &= \pdd{^{2}z}{y}{x} \\
            (f_{y})_{x} (x, y) = f_{yx} = f_{21}
                &= \pd{}{x}\bigg(\pd{f}{y}\bigg)
                &= \pdd{^{2}f}{x}{y}
                &= \pdd{^{2}z}{x}{y} \\
            (f_{y})_{y} (x, y) = f_{yy} = f_{22}
                &= \pd{}{y}\bigg(\pd{f}{y}\bigg)
                &= \pd{^{2}f}{y^{2}}
                &= \pd{^{2}z}{y^{2}}
        \end{alignat*}


\hi{Tangent Plane and Linear Approximation}
    \hii{Differentials}
        \par For a differentiable function of two variables, $z = f(x, y)$, we define the
        differentials $dx$ and $dy$ to be independent variables. Then the differential $dz$,
        also called the total differntial, is defined by:
        \begin{eqbox}
            dz = f_{x}(x, y) dx + f_{y}(x, y) dy = \pd{f}{x} dx + \pd{f}{y} dy
        \end{eqbox}
        \par If we take $dx = \Delta x = x - a$ and $dy = \Delta y = y - b$,
        then the differential of $z$ is:
        \begin{alignat*}{2}
            dz &= f_{x} (x_{1}, y_{1}) \Delta x + f_{y} (x_{1}, y_{1}) \Delta y \\
            &= f_{x} (x_{1}, y_{1}) (x_{2} - x_{1}) + f_{y} (x_{1}, y_{1}) (y_{2} - y_{1})
        \end{alignat*}

\setcounter{chapter}{4}
\begin{multicols*}{3}
  
\hi{Hypothesis Testing for One Sample}

\hii{Terminology}
  \begin{itemize}
    \item \tb{null hypothesis}:  the claim that is initially assumed to be true, denoted by $H_0$
    \item \tb{alternative hypothesis}: the assertion that is contradictory to $H_0$, denoted by $H_1$
    \item \tb{test of hypotheses}: a method for using sample data to decide \tb{whether the null hypothesis should be rejected}.
    \item \tb{significant level}: the probability of a type I error, denoted by $\alpha$
  \end{itemize}

\hii{Tips}
  \begin{itemize}
    \item The \tb{equal sign} only appears in the \tb{null hypothesis} $H_0$.
    \item The thing we want to prove appears in the \tb{alternative hypothesis} $H_1$. 
  \end{itemize}

\hii{Case 1: Normal Population with Known Standard Deviation}
  \hiii{Hypothesis Testing}
  \begin{itemize}
    \item \tb{Step 1}: Compute the statistic:
      \[
        z = \frac{\bar{x} - \mu_0}{\sigma / \sqrt{n}}
      \]
    \item \tb{Step 2}: Apply the decision rule:
      \begin{center}
        \begin{tabular}{|c|c|}
          \hline
          \textbf{$H_1$}   & \textbf{Rejection Region} \\ \hline
          $\mu \neq \mu_0$ & $|z| > z_{\alpha/2}$      \\ \hline
          $\mu < \mu_0$    & $z < -z_{\alpha}$         \\ \hline
          $\mu > \mu_0$    & $z > z_{\alpha}$          \\ \hline
        \end{tabular}
      \end{center}
  \end{itemize}

    \hiii{Hypothesis Testing with Propotion}
    \begin{itemize}
      \item \tb{Step 1}: Compute the statistic:
        \[
          z = \frac{\hat{p} - p_0}{\sqrt{\hat{p}\hat{q}/n}}
        \]
      \item \tb{Step 2}: Apply the decision rule:
        \begin{center}
          \begin{tabular}{|c|c|}
            \hline
            \textbf{$H_1$}   & \textbf{Rejection Region} \\ \hline
            $p \neq p_0$ & $|z| > z_{\alpha/2}$      \\ \hline
            $p < p_0$    & $z < -z_{\alpha}$         \\ \hline
            $p > p_0$    & $z > z_{\alpha}$          \\ \hline
          \end{tabular}
        \end{center}
    \end{itemize}

\hii{Case 2: Normal Population with Unknown Standard Deviation}

\begin{itemize}
  \item \tb{Step 1}: Compute the statistic:
    \[
      t = \frac{\bar{x} - \mu_0}{s / \sqrt{n}}
    \]
  \item \tb{Step 2}: Apply the decision rule:
      \begin{center}
        \begin{tabular}{|c|c|}
          \hline
          \textbf{$H_1$}   & \textbf{Rejection Region}   \\ \hline
          $\mu \neq \mu_0$ & $|t| > t_{\alpha/2, n - 1}$ \\ \hline
          $\mu < \mu_0$    & $t < -t_{\alpha, n - 1}$    \\ \hline
          $\mu > \mu_0$    & $t > t_{\alpha, n - 1}$     \\ \hline
        \end{tabular}
      \end{center}
\end{itemize}

\hii{Case 3: Any Distribution - Large Sample Size}

\begin{itemize}
  \item \tb{Step 1}: Compute the statistic:
    \[
      z = \frac{\bar{x} - \mu_0}{s / \sqrt{n}}
    \]
  \item \tb{Step 2}: Apply the decision rule:
      \begin{center}
        \begin{tabular}{|c|c|}
          \hline
          \textbf{$H_1$}   & \textbf{Rejection Region} \\ \hline
          $\mu \neq \mu_0$ & $|z| > z_{\alpha/2}$      \\ \hline
          $\mu < \mu_0$    & $z < -z_{\alpha}$         \\ \hline
          $\mu > \mu_0$    & $z > z_{\alpha}$          \\ \hline
        \end{tabular}
      \end{center}
\end{itemize}

\par \tb{Testing with Population Proportion - Large Sample Size}

\begin{itemize}
  \item \tb{Step 1}: Compute the statistic:
    \[
      z = \frac{\hat{p} - p_0}{\sqrt{p_0 q_0 / n}}
    \]
  \item \tb{Step 2}: Apply the decision rule:
      \begin{center}
        \begin{tabular}{|c|c|}
          \hline
          \textbf{$H_1$}   & \textbf{Rejection Region} \\ \hline
          $p \neq p_0$ & $|z| > z_{\alpha/2}$      \\ \hline
          $p < p_0$    & $z < -z_{\alpha}$         \\ \hline
          $p > p_0$    & $z > z_{\alpha}$          \\ \hline
        \end{tabular}
      \end{center}
\end{itemize}


\end{multicols*}

\chapter{Basic SQL}

\hi{SQL Data Definition and Data Types}
  \hii{Terminology}
    \begin{itemize}
      \item A \tb{table} is equivalent to a \tb{relation}.
      \item A \tb{row} is equivalent to a \tb{tuple}.
      \item A \tb{column} is equivalent to an \tb{attribute}.
    \end{itemize}

  \hii{Schema}
    \hiii{Schema}
      \par A \tb{schema} is a group of related tables.
      \par An \tb{SQL schema}:
        \begin{itemize}
          \item is identified by a \tb{schema name}
          \item includes an authorization identifier to indicate the user or account who owns the schema
          \item includes descriptors for each element in the schema.
        \end{itemize}

      \par \tb{Creating a schema}:
      % \begin{lstlisting}[style=SQL]
% \begin{minted}[linenos,tabsize=2,breaklines]{SQL}
\begin{lstlisting}
CREATE SCHEMA `schema_name' AUTHORIZATION `author_name'
\end{lstlisting}

  \hii{Catalog}
    \par A \tb{catalog} is a named collection of schemas.

  \hii{CREATLE TABLE command in SQL}
    \par The CREATE TABLE command is used to specify a new relation by
      \begin{itemize}
      \item giving it a name
      \item specifying its attributes and initial constraints.
      \end{itemize}
    \par The attributes are specified first, and each attribute is given a name, a data type to specify its domain of values, and possibly attribute constraints, such as NOT NULL. The key, entity integrity, and referential integrity constraints can be specified within the CREATE TABLE statement
after the attributes are declared, or they can be added later using the ALTER TABLE command.
    \par The relations declared through CREATE TABLE statements are called base tables (or base relations); this means that the table and its rows are actually created and stored as a file by the DBMS. Base relations are distinguished from virtual relations, created through the CREATE VIEW statement (see Chapter 7), which may or may not correspond to an actual physical file.
    \par In SQL, columns are ordered while rows are not.

  \hii{Attribute Data Types and Domains in SQL}

\hi{Specifying Constraints in SQL}
  \hii{Specifying Attribute Constraints and Attribute Defaults}
    \par SQL allows NULLs as attribute values, a constraint NOT NULL may be specified if NULL is not permitted for a particular attribute.
    \par Primary keys are always implicitly specified as NOT NULL.
    \par It is also possible to define a default value for an attribute by appending the clause
      \lstinline{DEFAULT <value>} to an attribute definition.

  \hii{Specifying Key and Referential Integrity Constraints}
    \par The PRIMARY KEY clause specifies one or more attributes that make up the primary
key of a relation. If a primary key has a single attribute, the clause can follow the
attribute directly.
    \par The UNIQUE clause specifies alternate (unique) keys, also known as candidate keys.
    \par Referential integrity is specified via the FOREIGN KEY clause.
    \par The default action that SQL takes for an integrity violation is to reject the update operation that will cause a violation, which is known as the RESTRICT option. However, the schema designer can specify an alternative action to be taken by attaching a referential triggered action clause to any foreign key constraint. The options include SET NULL, CASCADE, and SET DEFAULT. An
option must be qualified with either ON DELETE or ON UPDATE.
    \par A constraint may be given a constraint name, following the keyword CONSTRAINT .

  \hii{Specifying Constraints on Tuples Using CHECK}
    \par In addition to key and referential integrity constraints, which are specified by special keywords, other table constraints can be specified through additional CHECK clauses at the end of a CREATE TABLE statement. These can be called row-based constraints because they apply to each row individually and are checked whenever a row is inserted or modified.

\hi{Basic Retrieval Queries in SQL}

  \hii{Important Notes on SQL}
    \par SQL allows a table (relation) to have two or more tuples that are identical in all their attribute values. Hence, in general, an SQL table is not a set of tuples, because a set does not
allow two identical members; rather, it is a multiset (sometimes called a bag) of tuples. Some SQL relations are constrained to be sets because a key constraint has been declared or because the DISTINCT option has been used with the SELECT statement.

    \par The SELECT statement of SQL \ti{is not the same} as the SELECT operation of relational algebra.

  \hii{The SELECT-FROM-WHERE Structure of Basic SQL Queries}
    % \begin{minted}[linenos,tabsize=2,breaklines]{SQL}
    \begin{lstlisting}
SELECT <attribute_list>
FROM <table_list>
WHERE <condition>;
    \end{lstlisting}
      where
    \begin{itemize}
      \item \lstinline{<attribute list>} is a list of attribute names whose values are to be retrieved by the query.
      \item \lstinline{<table list>} is a list of the relation names required to process the query.
      \item \lstinline{<condition>} is a conditional (Boolean) expression that identifies the tuples to be retrieved by the query.
    \end{itemize}

    \par \tb{Logical comparison operators}: \lstinline{=, <, <=, >, >=, <>}

    \begin{itemize}
      \item The \lstinline{SELECT} clause specifies the attributes whose values are to be retrieved, which are called the \tb{projection attributes}.
      \item The \lstinline{WHERE} clause specifies the Boolean condition that must be true for any retrieved tuple, which is known as the \tb{selection condition}.
      \item A selection condition that joins two different tuples is called a \tb{join condition}.
      \item A query that involves only selection and join conditions plus projection attributes is known as a \tb{select-project-join} query.
    \end{itemize}

  \hii{Ambiguous Attribute Names, Aliasing, Renaming, and Tuple Variables}
    \par In SQL, the same name can be used for many attributes as long as they are in different tables. If this is the case, and a multitable query refers to two or more attributes with the same name, we must qualify the attribute name with the relation name to prevent ambiguity. This is done by prefixing the relation name to the attribute name and separating the two by a period.
    \par The ambiguity of attribute names also arises in the case of queries that refer to the
same relation twice (recursive ?). In this case, we are required to declare alternative relation names E and S , called aliases or tuple variables, for the EMPLOYEE relation. An alias can follow the keyword AS.
      % \begin{minted}[linenos,tabsize=2,breaklines]{SQL}
      \begin{lstlisting}
SELECT E.Fname, E.Lname, S.Fname, S.Lname
FROM EMPLOYEE AS E, EMPLOYEE AS S
WHERE E.Super_ssn = S.Ssn;
      \end{lstlisting}
    \par It is also possible to rename the relation \tb{attributes} within the query in SQL by giving them aliases.

  \hii{Unspecified WHERE Clause and Use of the Asterisk}
    \par A missing WHERE clause indicates no condition on tuple selection; hence, all tuples of the relation specified in the FROM clause qualify and are selected for the query result.

    \par If more than one relation is specified in the FROM clause and there is no WHERE clause, then the CROSS PRODUCT —all possible tuple combinations—of these relations is selected.

    \par To retrieve all the attribute values of the selected tuples, we do not have to list the attribute names explicitly in SQL; we just specify an asterisk (*), which stands for all the attributes. The * can also be prefixed by the relation name or alias; for example, EMPLOYEE.* refers to all attributes of the EMPLOYEE table.

  \hii{Tables as Sets in SQL}
    \par If we do want to eliminate duplicate tuples from the result of an SQL query, we use the keyword DISTINCT in the SELECT clause, meaning that only distinct tuples should remain in the result. In general, a query with SELECT DISTINCT eliminates duplicates, whereas a query with SELECT ALL does not. Specifying SELECT with neither ALL nor DISTINCT —as in our previous examples—is equivalent to SELECT ALL. If we do want to eliminate duplicate tuples from the result of an SQL query, we use the keyword DISTINCT in the SELECT clause, meaning that only distinct tuples should remain in the result. In general, a query with SELECT DISTINCT eliminates duplicates, whereas a query with SELECT ALL does not. Specifying SELECT with neither ALL nor DISTINCT is
equivalent to SELECT ALL .
    \par SQL has directly incorporated some of the set operations from mathematical set
theory, which are also part of relational algebra: set union (UNION), set difference (EXCEPT), and set intersection (INTERSECT) operations. The relations resulting from these set operations are sets of tuples; that is, duplicate tuples are eliminated from the result. These set operations apply only to \tb{type compatible relations}, so we must make sure that the two relations on which we apply the operation have the same attributes and that the attributes appear in the same order in both relations.
  \par SQL also has corresponding multiset operations, which are followed by the keyword ALL (UNION ALL, EXCEPT ALL, INTERSECT ALL). Their results are multisets (duplicates are not eliminated). Basically, each tuple—whether it is a duplicate or not— is considered as a different tuple when applying these operations.

  \hii{Substring Pattern Matching and Arithmetic Operators}
    \par \tb{Example}: Retrieve all employees whose address is in Houston, Texas.
      % \begin{minted}[linenos,tabsize=2,breaklines]{SQL}
      \begin{lstlisting}
SELECT Fname, Lname
FROM   EMPLOYEE
WHERE  Address LIKE ‘%Houston,TX%’;
      \end{lstlisting}
    \par \tb{Example}: Find all employees who were born during the 1950s.
      % \begin{minted}[linenos,tabsize=2,breaklines]{SQL}
      \begin{lstlisting}
SELECT Fname, Lname
FROM EMPLOYEE
WHERE Bdate LIKE ‘_ _ 5 _ _ _ _ _ _ _’;
      \end{lstlisting}

  \hii{Ordering of Query Results}
    \par SQL allows the user to order the tuples in the result of a query by the values of one or more of the attributes that appear in the query result.
    \par \tb{Example}: Retrieve a list of employees and the projects they are working on, ordered by department and, within each department, ordered alphabetically by last name, then first name.
      % \begin{minted}[linenos,tabsize=2,breaklines]{SQL}
      \begin{lstlisting}
SELECT D.Dname, E.Lname, E.Fname, P.Pname
FROM DEPARTMENT AS D, EMPLOYEE AS E, WORKS_ON AS W, PROJECT AS P
WHERE  D.Dnumber = E.Dno AND E.Ssn = W.Essn AND W.Pno = P.Pnumber
ORDER BY D.Dname, E.Lname, E.Fname;
      \end{lstlisting}
    \par Ascending and Descending:
       % \begin{minted}[linenos,tabsize=2,breaklines]{SQL}
       \begin{lstlisting}
    ORDER BY D.Dname DESC, E.Lname ASC, E.Fname ASC
      \end{lstlisting}

\hi{INSERT, DELETE, and UPDATE Statements in SQL}
  \hii{The INSERT Command}

    \par In its simplest form, INSERT is used to add a single tuple (row) to a relation (table). We must specify the relation name and a list of values for the tuple. The values should be listed in the same order in which the corresponding attributes were specified in the CREATE TABLE command.

      % \begin{minted}[linenos,tabsize=2,breaklines]{SQL}
      \begin{lstlisting}
INSERT INTO EMPLOYEE
VALUES ( 'Richard', 'K', 'Marini', '653298653', '1962-12-30', '98 Oak Forest, Katy, TX', 'M', 37000, ‘653298653’, 4 );
      \end{lstlisting}

    \par A second form of the INSERT statement allows the user to specify explicit attribute
    names that correspond to the values provided in the INSERT command. This is use-
    ful if a relation has many attributes but only a few of those attributes are assigned
    values in the new tuple. However, the values must include all attributes with NOT
    NULL specification and no default value. Attributes with NULL allowed or DEFAULT
    values are the ones that can be left out.

      % \begin{minted}[linenos,tabsize=2,breaklines]{SQL}
      \begin{lstlisting}
INSERT INTO EMPLOYEE(Fname, Lname, Dno, Ssn)
VALUES (‘Richard’, ‘Marini’, 4, ‘653298653’);
      \end{lstlisting}

    \par A DBMS that fully implements SQL should support and enforce all the integrity
constraints that can be specified in the DDL.

  \hii{The DELETE Command}
    \par The DELETE command removes tuples from a relation. It includes a WHERE clause, similar to that used in an SQL query, to select the tuples to be deleted. Tuples are explicitly deleted from only one table at a time. However, the deletion may propagate to tuples in other relations if referential triggered actions are specified in the referential integrity constraints of the DDL.
    \par A missing WHERE clause specifies that all tuples in the relation are to be deleted; however, the table remains in the database as an empty table. We must use the DROP TABLE command to remove the table definition.
    % \begin{minted}[linenos,tabsize=2,breaklines]{SQL}
    \begin{lstlisting}
DELETE FROM EMPLOYEE
WHERE Lname = ‘Brown’;
    \end{lstlisting}

  \hii{The UPDATE Command}
    \par The UPDATE command is used to modify attribute values of one or more selected tuples. As in the DELETE command, a WHERE clause in the UPDATE command selects the tuples to be modified from a single relation. However, updating a primary key value may propagate to the foreign key values of tuples in other relations if such a referential triggered action is specified in the referential integrity constraints of the DDL.
    \par An additional SET clause in the UPDATE command specifies the attributes to be modified and their new values.
    % \begin{minted}[linenos,tabsize=2,breaklines]{SQL}
    \begin{lstlisting}
UPDATE PROJECT
SET Plocation = ‘Bellaire’, Dnum = 5
WHERE Pnumber = 10;

UPDATE EMPLOYEE
SET Salary = Salary * 1.1
WHERE Dno = 5;
    \end{lstlisting}

\hi{Aggregate Functions}
  \par \tb{Purpose}: summarize information from multiple tuples into a single-tuple summary.

  \par \tb{List of built-in aggregate functions}:
    \begin{itemize}
      \item SUM
      \item AVG
      \item COUNT
      \item MAX
      \item MIN
    \end{itemize}

\par These functions can be used in the SELECT clause or in a HAVING clause

  \hii{Built-in Aggregate Functions}

    \hiii{SUM and AVG}
      \begin{itemize}
        \item set/multiset of numeric values
      \end{itemize}

      % \begin{minted}[linenos,tabsize=2,breaklines]{SQL}
      \begin{lstlisting}
        SELECT SUM(attribute_name)
        FROM   table_name;
      \end{lstlisting}

    \hiii{MAX and MIN}
      \begin{itemize}
        \item Comparable attributes (the domain values have a total ordering among one another)
      \end{itemize}

    \hiii{COUNT}
      \par Returns the \ti{number of tuples or values} that satisfy a condition.
      \par The asterisk refers to \ti{all rows (tuples)}.
      \par Examples:
        \begin{itemize}
          \item Simple count all
          \item Count non-distinct
          \item Count distinct
        \end{itemize}

\hi{Grouping: GROUP BY and HAVING Clauses}
  \par \tb{Objective}: Partition in to non-overlapping subsets (or groups) and apply aggregate functions.

  \hii{GROUP BY}
    \par Each group consist of the tuples that have the same value of some attributes, called \tb{grouping attributes}

    \par The grouping attributes can be defined by the \lstinline{GROUP BY} clause. The GROUP BY clause specifies the grouping attributes, which should also appear in the SELECT clause, so that the value resulting from applying each aggregate function to a group of tuples appears along with the value of the grouping attribute(s).

  \hii{HAVING}
    \par Retrieve the values of aggregate functions only for groups that satisfy certain conditions.

\chapter{More SQL: Complex Queries, Triggers, Views, and Schema Modification}

\hi{More Complex SQL Retrieval Queries}
  \hii{Comparisons Involving NULL and Three-Valued Logic}
    \par Meanings of NULL:
      \begin{itemize}
        \item Unknown value
        \item Unavailable or withheld value
        \item Not applicable attribute
      \end{itemize}
    \par In general, each individual NULL value is considered to be different from every other NULL value in the various database records. When a record with NULL in one of its attributes is involved in a comparison operation, the result is considered to be UNKNOWN.
    \par SQL allows queries that check whether an attribute value is NULL. Rather than using = or <> to compare an attribute value to NULL , SQL uses the comparison operators IS or IS NOT . This is because SQL considers each NULL value as being distinct from every other NULL value, so equality comparison is not appropriate.
      \begin{minted}{SQL}
SELECT Fname, Lname
FROM EMPLOYEE
WHERE Super_ssn IS NULL;
      \end{minted}

  \hii{Nested Queries, Tuples, and Set/Multiset Comparisons}
    \par Some queries require that existing values in the database be fetched and then used in a comparison condition. Such queries can be conveniently formulated by using nested queries, which are complete select-from-where blocks within another SQL query. That other query is called the outer query. These nested queries can also appear in the WHERE clause or the FROM clause or the SELECT clause or other SQL clauses as needed.
    \par If a nested query returns a single attribute and a single tuple, the query result will be a single (scalar) value. In such cases, it is permissible to use = instead of IN for the comparison operator. In general, the nested query will return a table (relation), which is a set or multiset of tuples. SQL allows the use of tuples of values in comparisons by placing them within
    parentheses and allowing the IN operator.
    \begin{minted}{SQL}
SELECT DISTINCT Essn
FROM WORKS_ON
WHERE (Pno, Hours) IN
                  (
                    SELECT Pno, Hours
                    FROMWORKS_ON
                    WHERE Essn = ‘123456789’
                  );
    \end{minted}
    \par In addition to the IN operator, a number of other comparison operators can be used to compare a single value v (typically an attribute name) to a set or multiset v (typically a nested query). The = ANY (or = SOME ) operator returns TRUE if the value v is equal to some value in the set V and is hence equivalent to IN . The two keywords ANY and SOME have the same effect. Other operators that can be combined with ANY (or SOME ) include >, >=, <, <=, and <>. The keyword ALL can also be combined with each of these operators.

  \hii{Correlated Nested Queries}
    \par Whenever a condition in the WHERE clause of a nested query references some attribute of a relation declared in the outer query, the two queries are said to be correlated.
    \par In general, a query written with nested select-from-where blocks and using the = or IN comparison operators can always be expressed as a single block query.

  \hii{The EXISTS and UNIQUE Functions in SQL}
    \par EXISTS and UNIQUE are Boolean functions that return TRUE or FALSE; hence, they can be used in a WHERE clause condition.
    \par The EXISTS function in SQL is used to check whether the result of a nested query is empty (contains no tuples) or not. The result of EXISTS is a Boolean value TRUE if the nested query result contains at least one tuple, or FALSE if the nested query result contains no tuples.
    \par EXISTS and NOT EXISTS are typically used in conjunction with a correlated nested
query.
      \begin{minted}{SQL}
SELECT Fname, Lname
FROM EMPLOYEE
WHERE NOT EXISTS (
                  SELECT *
                  FROM DEPENDENT
                  WHERE Ssn = Essn
                  );
      \end{minted}
\par There is another SQL function, UNIQUE ( Q ), which returns TRUE if there are no
duplicate tuples in the result of query Q ; otherwise, it returns FALSE . This can be
used to test whether the result of a nested query is a set (no duplicates) or a multiset
(duplicates exist).

\hi{Explicit Sets and Renaming in SQL}
\par We have seen several queries with a nested query in the WHERE clause. It is also possible to use an explicit set of values in the WHERE clause, rather than a nested query. Such a set is enclosed in parentheses in SQL.

\begin{minted}{SQL}
SELECT DISTINCT Essn
FROM WORKS_ON
WHERE Pno IN (1, 2, 3);
\end{minted}

\hi{Aggregate Functions in SQL}
  \hii{GROUP BY}
    \par \lstinline{GROUP BY} is followed by \tb{an attribute}.
    \par \tb{Example}: For each department, retrieve the department number, the number of employees in the department, and their average salary.
    \begin{minted}[linenos,tabsize=2,breaklines]{SQL}
SELECT   Dno, COUNT (*), AVG (Salary)
FROM     EMPLOYEE
GROUP BY Dno;
    \end{minted}
  \hii{HAVING}

  \hii{HAVING}
  \par \lstinline{HAVING} is followed by \tb{a condition}.
  \par \tb{Example}: For each project on which more than two employees work, retrieve the project number, the project name, and the number of employees who work on the project.
    \begin{minted}[linenos,tabsize=2,breaklines]{SQL}
SELECT     Pnumber, Pname, COUNT (*)
FROM       PROJECT, WORKS_ON
WHERE      Pnumber = Pno
GROUP BY   Pnumber, Pname
HAVING     COUNT (*) > 2;
    \end{minted}


\hi{Assertions and Triggers}
  \hii{Specifying Constraints as Assertions}
    \begin{minted}[linenos,tabsize=2,breaklines]{SQL}
CREATE ASSERTION SALARY_CONSTRAINT
CHECK (
  NOT EXISTS (
    SELECT * 
    FROM EMPLOYEE E, EMPLOYEE M, DEPARTMENT D
    WHERE E.Salary > M.Salary
          AND E.Dno = D.Dnumber 
          AND D.Mgr_ssn = M.Ssn
  )
);
    \end{minted}

  \hii{Specifying Actions as Triggers}
  \par A \tb{trigger} has 3 components:
  \begin{itemize}
    \item The \tb{event}
    \item The \tb{condition}
    \item The \tb{action}
  \end{itemize}
    \begin{minted}[linenos,tabsize=2,breaklines]{SQL}  
CREATE TRIGGER SALARY_VIOLATION
BEFORE INSERT OR UPDATE OF SALARY, SUPERVISOR_SSN ON EMPLOYEE
FOR EACH ROW
  WHEN (
    NEW.SALARY > (
      SELECT SALARY
      FROM EMPLOYEE
      WHERE SSN = NEW.SUPERVISOR_SSN
    )
  )
  INFORM_SUPERVISOR(NEW.Supervisor_ssn, NEW.Ssn);
    \end{minted}
  \par \lstinline{INFORM_SUPERVISOR} is the action.
\clearpage
\begin{multicols}{3}

\hi{MSI Logic Circuits}

  \hii{Terminology}
    \begin{itemize}
      \item \tb{Decoder}:
        \begin{itemize}
          \item \tb{Input}: $n$-bit
          \item \tb{Output}: only $\pmb{1}$ corresponding output
        \end{itemize}
      \item \tb{7-seg Decoder}:
        \begin{itemize}
          \item \tb{Common-Anode LED Display}: outputs are active-LOW
          \item \tb{Common-Cathode LED Display}: outputs are active-HIGH
        \end{itemize}
      \item \tb{Encoder}
        \begin{itemize}
          \item \tb{Input}: only 1 is active at a time
          \item \tb{Output}: $n$-bit
        \end{itemize}
      \item \tb{Priority Encoder}: the output is corresponding to the
        \tb{largest} active input.
      \item \tb{Multiplexer}: (MUX, also Data Selector)
        \begin{itemize}
          \item \tb{Input}: $n$ SELECT inputs and $\leq 2^{N}$ DATA inputs
          \item \tb{Output}: only 1 output for all
        \end{itemize}
      \item \tb{Demultiplexer}: (DEMUX, also Data Distributor)
        \begin{itemize}
          \item \tb{Input}: $n$ SELECT inputs and 1 DATA input
          \item \tb{Output}: $\leq 2^{N}$ outputs, only 1 receives the DATA input
        \end{itemize}
      \item \tb{Magnitude Comparator}: circuit to compare two $n$-bit numbers
        If the two $n$-bit input numbers are equal, the comparison will also
        based on the \tb{cascading input}.
        \par \ti{Connecting MCs: LSB first}
      \item \tb{Code Converter}: circuit to convert one type to another type
        of binary code
      \item \tb{Data Busing}: multiple device have their outputs connected to
        a common set of bus lines
        \par \ti{Only outputs of one device can be transmitted on the bus lines
        at a time. To achieve that, we use \tb{tristate registers}}.
  \end{itemize}

\end{multicols}
\setcounter{chapter}{13}
\chapter{Functional Dependencies and Normalization for Relational Databases}

\hi{Informal Design Guidelines for Relation Schemas}

  \par Four design guidelines discussed are:

  \begin{itemize}
    \item Making sure that the semantics of the attributes is clear in the schema
    \item Reducing the redundant information in tuples
    \item Reducing the NULL values in tuples
    \item Disallowing the possibility ofgenerating spurious tuples
  \end{itemize}

  \hii{Imparting Clear Semantics to Attributes in Relations}
    \par The \tb{semantics} of a relation refers to its meaning resulting from the interpretation of attribute values in a tuple.
    \par The easier it is to explain the semantics of the relation, the better the relation schema design will be.
    \par \tb{Guideline 1}:
    \begin{itemize}
      \item Design a relation schema so that it is easy to explain its meaning.
      \item Do not combine attributes from different entity types into a single relation.
      \item Do not combine attributes from different relationship types into a single relation.
      \item Do not combine attributes from different entity types and relationship types into a single relation.
    \end{itemize}

  \hii{Redundant Information in Tuples and Update Anomalies}
    \par Storing natural joins of base relations (relations of entity types and relationship types) leads to an additional problem refered to as \tb{update anomalies}. There are three types of update anomalies:
    \begin{itemize}
      \item \tb{Insertion Anomalies}
      \item \tb{Deletion Anomalies}
      \item \tb{Modification Anomalies}
    \end{itemize}
    \par \tb{Guideline 2}: Design the base relation schemas so that no insertion, deletion, or modification anomalies are present in the relations. If any anomalies are present, note them clearly and make sure that the programs that update the database will operate correctly.

  \hii{NULL Values in Tuples}
    \par \tb{Guideline 3}: As far as possible, avoid placing attributes in a base relation whose values may frequently be NULL. If NULLs are unavoidable, make sure that they apply in exceptional cases only and do not apply to a majority of tuples in the relation.

  \hii{Generation of Spurious Tuples}
    \par \tb{Guideline 4}: Design relation schemas so that they can be joined with equality conditions on attributes that are appropriately related (primary key, foreign key) pairs in a way that guarantees that no spurious tuples are generated.


\hi{Functional Dependencies}
  \hii{Definition}
    \par Suppose that the relational database schema has $n$ attributes $A_1 , \ldots, A_n$. The database can be described by one single \tb{universal relation schema} $R = \{A_1, \ldots, A_n\}$. Define:
    \begin{itemize}
      \item $X$ and $Y$ as subsets of attributes of $R$.
      \item $r$ as a relation state of $R$ (relation state is the set of all tuples in the relation schema).
      \item $t_1$ and $t_2$ as two tuples in $r$.
    \end{itemize}
    \par A \tb{functional dependency} $X \to Y$ specifies the constraint that: for any two tuples $t_1$ and $t_2$ in $r$, if $t_1[X] = t_2[X]$, then $t_1[Y] = t_2[Y]$.
    \par The value of the $Y$ component of a tuple in $r$ is determined by the value of the $X$ component; alternatively, the values of the $X$ component of a tuple uniquely/functionally determine the value of the $Y$ component. (with 1 $X$ there can only be 1 corresponding $Y$)
    \par If $X$ is a \tb{candidate key} of $R$, then $X \to R$.
    \par If $X \to Y$, it does not guarantee whether $Y \to X$ or not.
    \par Functional dependency is a property of the \tb{semantics} or \tb{meaning of the attributes}.
    \par The purpose of functional dependency is to further describe a relation schema $R$ by specifying constraints on its attributes that must hold at all time. Relation states $r(R)$ that satisfy the functional dependency constraints are called \tb{legal relation states}.

\hi{Normal Forms Based on Primary Keys}
  \hii{Normalization of Relations}
    \par \tb{Normalization} process takes a relation schema through a series of tests to certify whether it satisfies a certain \tb{normal form}.
    \par \tb{Normalization of data}: a process of analyzing the given relation schemas based on their FDs and primary keys to
    \begin{itemize}
      \item Minimize redundancy
      \item Minimize insertion, deletion, update anomalies
    \end{itemize}
    \par In this process, a relation schema that does not meet the condition for a normal form is decomposed into smaller relation schemas that meet the \tb{normal form test} (condition for a normal form) that is not met by the original relation.
    \par The normalization process through decomposition must confirm the existence of some properties including two properties:
    \begin{itemize}
      \item The nonadditive join or loseless join property: there is no spurious tuple generated with respect to the relation schemas created after decomposition.
      \item The dependency preservation property: each functional dependency is represented in some individual relation resulting after decomposition.
    \end{itemize}
    \par The first property is extremely critical and \tb{must be achieved at any cost}. The second property, although desirable, is sometimes sacrified.

  \hii{Practical Use of Normal Forms}
    \par In practice, normalization up to 3NF, BCNF, or at most 4NF is used. Higher normal forms are hard to understand or to detect for database designers and users.
    \par Database designers \tb{need not} normalize to the highest possible normal form. Relations may be left in a lower normalization status for performance reasons.
    \par \tb{Definition}: \tb{Denormalization} is the process of storing the join of higher normal form relations as a base relation which is in a lower normal form.

  \hii{Definition of Keys and Attributes Participating in Keys}
    \par We revisit some definition in the previous chapter. Given a relation schema $R = \{A_1, \ldots, A_n\}$.
    \begin{itemize}
      \item \tb{superkey}: a super key is a set of attributes $S \subseteq R$ with the property that no two tuples $t_1$ and $t_2$ in a legal relation state $r$ of $R$ will have $t_1[S] = t_2[S]$.
      \item \tb{key}: a key $K$ is a super key with the additional property that removal of any attribute from $K$ will cause $K$ not to be a superkey anymore. We can say that a key is a minimal superkey.
      \item \tb{candidate key}: if a relation schema has more than one key, each is called a candidate key.
      \item \tb{primary key}: if a relation schema has more than one key, then one can be chosen arbitrarily to be the primary key. If the relation schema only has one key, then that only key is the primary key.
      \item \tb{secondary key}: any candidate key that is not a primary key
    \end{itemize}
  \par \tb{Definition}: An attribute of relation schema $R$ is called a \tb{prime attribute} of $R$ if it is a member of some candidate key of $R$. An attribute is called \tb{nonprime} if it is not a prime attribute.

  \hii{First Normal Form}
    \par First Normal Form (1NF) disallow relations within relations or relations as attribute values within tuples. \tb{The only attribute values permitted by 1NF are \tu{single atomic} (or \tu{indivisible}) values.}

  \hii{Second Normal Form (based on Primary Key)}
    \par Second Normal Form (2NF) is based on the concept of \tb{full functional dependency}.
    \par A functional dependency $X \to Y$ is a \tb{full functional dependency} if removal of any attribute $A$ from $X$ means that the dependency does not hold anymore.
    \par A functional dependency $X \to Y$ is a \tb{partial functional dependency} if some attribute $A$ can be removed from $X$ and the dependency still holds.
    \par \tb{Definition of 2NF}: A relation schema $R$ is in 2NF if every nonprime attribute $A$ in $R$ is \tb{fully functional dependent} on the primary key of $R$.
    \par \tb{Example}:
    \img[width=12cm]{img/2nf-01.png}{}
    \par In the figure,
      \begin{itemize}
        \item FD2 violates 2NF: \ilc{Ename} can be functionally determined by \ilc{Ssn} only.
        \item FD3 violates 2NF: both \ilc{Pname} and \ilc{Plocation} can be functionally determined by \ilc{Pnumber} only.
      \end{itemize}

  \hii{Third Normal Form (based on Primary Key)}
    \par Third Normal Form (3NF) is based on the concept of \tb{transitive dependency}.
    \par A functional dependency $X \to Y$ is a \tb{transitive dependency} if $\exists Z \subseteq R$:
    \begin{itemize}
      \item $Z$ is neither a candidate key nor subset of any key of $R$
      \item $X \to Z$ and $Z \to Y$ holds.
    \end{itemize}
    \par \tb{Definition of 3NF}: A relation schema $R$ is in 3NF if it satisfies 2NF \tb{and} no nonprime attribute of $R$ is transitively dependent on the primary key.
    \img[width=12cm]{img/3nf-01.png}{}
    \par In the figure, the relation schema \ilc{EMP_DEPT} is in 2NF but not in 3NF because of the transitive dependency of \ilc{Dmgr_ssn} and \ilc{Dname} on \ilc{Ssn} via \ilc{Dnumber}.

\hi{General Definitions of Second and Third Normal Forms}
    \par In general, partial and transitive dependencies should be avoided because they cause the update anomalies.
    \par General definition of \tb{prime attribute}: an attribute that is part of any candidate key.
    \par Partial and full functional dependencies and transitive dependencies is considered with respect to all candidate keys of a relation in this section.

  \hii{General Definition of 2NF}
    \par \tb{General definition of 2NF}: A relation schema $R$ is in 2NF if every nonprime attribute $A$ in $R$ is not partially dependent on any key of $R$.
    \par \tb{Example}:
    \img[width=12cm]{img/2nf-02.png}{}
    \par The \ilc{LOTS} relation schema violates the general definition of 2NF because \ilc{Tax_rate} is partially dependent on the candidate key
    \par Solution: Decompose the relation as follows:
    \img[width=14cm]{img/2nf-03.png}{}

  \hii{General Definition of Third Normal Form}
    \par \tb{General definition of 3NF}: A relation schema $R$ is in 3NF if: whenever a \tb{nontrivial functional dependency} $X \to R$ holds in $R$, either:
    \begin{itemize}
      \item $X$ is a superkey of $R$.
      \item $A$ is a prime attribute of $R$.
    \end{itemize}
    \par \tb{Example}:
    \img[width=14cm]{img/3nf-02.png}{}
    \par Here, \ilc{LOTS1} violates 3NF because:
    \begin{itemize}
      \item \ilc{Area} is not a superkey
      \item \ilc{Price} is not a prime attribute
    \end{itemize}
    \par Solution:
    \img[width=14cm]{img/3nf-03.png}{}
    \par Here, both \ilc{LOTS1a} and \ilc{LOTS1b} are in 3NF.
    \par Note that:
    \begin{itemize}
      \item \ilc{LOTS1} violates 3NF because \ilc{Price} is transitively dependent on each of the candidate keys of \ilc{LOTS1} via the nonprime attribute \ilc{Area}.
      \item If a schema is in 3NF, it is certainly in 2NF.
    \end{itemize}

  \hii{Interpreting the General Definition of Third Normal Form}
    \par \tb{Alternative Definition of 3NF}: A relation schema $R$ is in 3NF if every nonprime attribute of $R$ meets both of the following conditions:
    \begin{itemize}
      \item It is fully functionally dependent on every key of $R$.
      \item It is nontransitively dependent on every key of $R$.
    \end{itemize}

  \hii{Boyce-Codd Normal Form}
    \par Boyce-Codd normal form (BCNF) is simpler but stricter than 3NF.
    \par Every relation that is in BCNF is also in 3NF, but a relation in 3NF does not necessary in BCNF.
    \par \tb{Definition of BCNF}: A relation schema $R$ is in BCNF if: whenever a nontrivial functional dependency $X \to A$ holds in $R$, then $X$ is a superkey of $R$.
    \par (The second condition in 3NF is excluded).

\chapter{Relational Database Design Algorithms and Further Dependencies}

\hi{Further Topics in Functional Dependencies: Inference Rules, Equivalence, and Minimal Cover}
  \hii{Inference Rules}
    \par Let $F$ be the set of functional dependencies that are specified on a relation schema $R$.

    \hiii{Inferred Functional Dependencies}
      \par An FD $X \to Y$ is \tb{inferred from} or \tb{implied by} a set of dependencies $F$ specified on $R$ if $X \to Y$ holds in \tb{every} legal relation state $r$ of $R$.

    \hiii{Closure of Functional Dependencies}
      \par The set of all dependencies that include $F$ as well as dependencies that can be inferred from $F$ is called the \tb{closure} of $F$ and is denoted by $F^{+}$.

    \hiii{Some Notations}
      \begin{itemize}
        \item $F |= X \to Y$: the FD $X \to Y$ is inferred from $F$
        \item $\{X, Y\} \to Z$ can be abbreviated to $XY \to Z$
        \item $\{X, Y, Z\} \to \{U, V\}$ can be abbreviated to $XYZ \to UV$
      \end{itemize}

    \hiii{Inference Rules}
      \par Aarmstrong's Axioms:
      \begin{itemize}
        \item \tb{IR1: Reflexive Rule}:
          $X \supseteq Y \Rightarrow X \to Y$
        \item \tb{IR2: Augmentation Rule}:
          $\{ X \to Y \} |= XZ \to YZ$
        \item \tb{IR3: Transitive Rule}:
          $\{ X \to Y, Y \to Z \} |= X \to Z$
      \end{itemize}
      \par Other rules:
      \begin{itemize}
        \item \tb{IR4: Decomposition Rule/Projective Rule}:
          $\{ X \to YZ \} |= X \to Y$
        \item \tb{IR5: Union Rule/Additive Rule}:
          $\{ X \to Y, X \to Z \} |= X \to YZ$
        \item \tb{IR6: Pseudotransitive Rule}:
          $\{ X \to Y, WY \to Z \} |= WX \to Z$
      \end{itemize}

    \hiii{Closure of Attributes under The FD Set}
      \par Let $X$ be a set of attributes of $R$. Then \tb{the closure of $X$ under $F$}, denoted by $X^{+}$, are all attributes that are functional determined by $X$ based on $F$.
      \par \tb{Algorithm}:
        \begin{algorithm}[H]
          \caption{Determine $X^{+}$ of $X \subseteq R$ under $F$}
          \begin{algorithmic}[1]
            \State $X^{+} := X$
            \While{\texttt{True}}
              \State $oldX^{+}$ := $X^{+}$
              \For{each $Y \to Z \in F$}
                \If{$X^{+} \supseteq Y$}
                  \State $X^{+}$ := $X^{+} \cup Z$
                \EndIf
              \EndFor
              \If{$oldX^{+} = X^{+}}$
                \State break
              \EndIf
            \EndWhile
          \end{algorithmic}
        \end{algorithm}

  \hii{Equivalence of Sets of Functional Dependencies}
      \hiii{Cover Relationship}
        \par A set of functional dependencies $F$ is said to cover another set of functional dependencies $E$ if every FD in $E$ is also in $F^+$ - meaning that every dependencies in $E$ can be \tb{inferred} from $F$.

      \hiii{Equivalent Set of FDs}
        \par Two sets of FDs $E$ and $F$ are \tb{equivalent} if $E^+ = F^+$.

  \hii{Minimal Sets of Functional Dependencies}
    \hiii{Extraneous Attribute}
      \par An attribute in a functional dependency is considered an extraneous attribute if we can remove it without changing the closure of the set of dependencies.

    \hiii{Canonical Form of Set of FDs}
      \par An FD is said to be in \tb{canonical form} if it only has a single attribute on the right-hand side.
      \par A set of FDs $F$ is said to be in \tb{canonical form} if for every FD in $F$, there is only a single attribute on the right-hand side.
      \par Example: If $X \to YZ \in F$, then $F$ is not in canonical form.

    \hiii{Minimal Set of FDs}
      \par A set of FDs $F$ is said to be \tb{minimal} if it satisfies the following conditions:
      \begin{itemize}
        \item $F$ is in canonical form
        \item Given that $X \supset Y$, we cannot replace $X \to A$ in $F$ with $Y \to A$ and still have a set of dependencies that is equivalent to $F$.
        \item We cannot remove any dependency from $F$ and still have a set of dependencies that is equivalent to $F$.
      \end{itemize}
      \par In short, a minimal set of dependencies must be in \tb{standard} or \tb{canonical form} (condition 1) and has \tb{no redundancies} (conditions 2 and 3).

    \hiii{Minimal Cover}
      \par A \tb{minimal cover} of a set of functional dependencies $E$ is a \tb{minimal set of dependencies} that is equivalent to $E$. We can always find \tb{at least one} minimal cover $F$ for any set of dependencies $E$ using the following algorithm:
        \begin{algorithm}[H]
          \caption{Determine the minimal cover $F$ for a set of FDs $E$}
          \Input{set of FDs $E$}
          \begin{algorithmic}[1]
            \State $F := E$
            \For{each FD $X \to \{ A_1, \ldots, A_n \} \in F$}
              \State Replace with $X \to A_1, \ldots, X \to A_n$
              \Comment{Place the FD in a canonical form}
            \EndFor
            \For{each FD $X \to A \in F$}
              \For{each attribute $B \in X$}
                \If{$ F - \{ X \to A \} + \{ (X - \{B\}) \to A \}$ is equivalent to $F$}
                  \State Replace $X \to A$ with $(X - \{B\}) \to A$ in $F$
                  \Comment{Remove all extraneous attributes}
                \EndIf
              \EndFor
            \EndFor
            \For{each remaining FD $X \to A \in F$}
              \If{$F - \{X \to A\}$ is equivalent to $F$}
                \State Remove $X \to A$ from $F$
                \Comment{Remove all remaining redundant FDs}
              \EndIf
            \EndFor
          \end{algorithmic}
        \end{algorithm}
      \par Note that on line 6 I used the plus ($+$) symbol instead of the union ($\cup$) symbol (like in the book) for easy-reading.

    \hiii{Determine the Key of a Relation}
      \begin{algorithm}[H]
        \caption{Determine the key $K$ of a relation $R$}
        \Input{$R$ and the set of FDs $F$ of $R$}
        \begin{algorithmic}[1]
          \State $K := R$
          \For{each attribute $A \in K$}
            \State Compute $(K - A)^+$ with respect to $F$
            \If{$(K - A)^+$ contains all attributes in $R$}
              \State $K := K - \{A\}$
            \EndIf
          \EndFor
        \end{algorithmic}
      \end{algorithm}

\hi{Properties of Relational Decompositions}
  \hii{Relation Decomposition and Insufficiency of Normal Forms}
    \par The relation schema $D = \{R_1, \ldots, R_m\}$ resulting from decomposing the relation schema $R$ is called a \tb{decomposition} of $R$.
    \par \tb{Attribute preservation condition of a decomposition}: each attribute in $R$ will appear in at least one relation schema $R_i$ in the decomposition so that no attributes are lost:
    \[
      \bigcup\limits_{i = 1}^{m} R_i = R
    \]
    \par Apart from the attribute preservation condition, another goal is to have each individual relation $R_i$ in $D$ be in BCNF or 3NF. However, this condition does not guarantee a good database design on its own.

  \hii{Dependency Preservation Property of a Decomposition}


\chapter{Disk Storage, Basic Structures, Hashing, and Modern Storage Architectures}

\hi{Introduction}
  \hii{Memory Hierarchies and Storage Devices}
    \hiii{Memory Hierarchies}
      \begin{itemize}
        \item \tb{Primary Storage Level}:
          \par Primary storage: storage media that can be operated on directly by the computer’s CPU
          \par Properties:
            \begin{itemize}
              \item can be operated on directly by the computer’s CPU
              \item provide fast access
              \item limited storage capacity
              \item violatile (content of memory is lost in case the machine is turned off, power failure, or system crash)
            \end{itemize}
          \par Storage devices:
            \begin{itemize}
              \item Cache memory (SRAM): used to speed up execution
              \item Main memory (DRAM): low cost but volatile and lower speed
                compared to cache, where a program resides and executes.
            \end{itemize}
          \par \tb{Main memory database}: the entire database can be kept in main memory (with a backup copy on magnetic disk).
        \item \tb{Second and Tertiary Storage Level}:
          \par Storage devices:
            \begin{itemize}
              \item Secondary storage level:
              \begin{itemize}
                \item magnetic disks
                \item flash memory (SSD): nonvolatile, high-density high-performance
              \end{itemize}
              \item Tertiary storage level:
              \begin{itemize}
                \item CD-ROM (compact disk read-only memory) and DVD (digital video disk) devices
                \item magnetic tapes: used for archiving and backup storage of data. Tape jukeboxes contains a bank of tapes that are catalogued and can be automatically loaded onto tape drives are popular as tertiary storage.
              \end{itemize}
            \end{itemize}
      \end{itemize}

  \hii{Storage Organization and Databases}
    \hiii{Types of Data}
    \begin{itemize}
      \item \tb{Persistent data}: data that must persist over long periods of time
      \item \tb{Transient data}: data that persist for only a limited time during program execution
    \end{itemize}

    \hiii{The use of Secondary Storage}
      \par \tb{Most databases are stored permanently (or persistently) on magnetic disk secondary storage} for the following reasons:
        \begin{itemize}
          \item Databases are too large to fit entirely in main memory.
          \item Data loss occurs less often thanks to the \tb{nonvolatile} nature of magnetic disks.
          \item Cost of storage per unit of secondary storage is less than primary storage.
        \end{itemize}
      \par Newer technologies such as SSD will be alternatives for magnetic disks in the future. However, magnetic disks will continue to be the primary medium of choice for large database.

    \hiii{The use of Magnetic Tapes}
      \par \tb{Magnetic tapes} are used for \tb{backing up} because storage on tape costs much less than storage on disk.

    \hiii{How data is stored an access on disk}
      \par Typical database applications need only a small portion of the database at a time for processing. Whenever a certain portion of the data is needed, it must be located on disk, copied to main memory for processing, and then rewritten to the disk if the data is changed.
      \par The data stored on disk is organized as \tb{files of records}. Each record is a collection of data values that can be interpreted as facts about entities, their attributes, and their relationships.

    \hiii{File Organization}
      \par There are several primary \tb{file organizations}.
      \par File organizations determine how the file records are physically placed on the disk, and hence, how the records can be accessed.
      \begin{itemize}
        \item A heap file (or unordered file): places the records on disk in no particular order by appending new records at the end of the file.
        \item A sorted file (or sequential file) keeps the records ordered by the value of a particular field (called the sort key).
        \par A hashed file uses a hash function applied to a particular field (called the hash key) to determine a record’s placement on disk.
        \par Other primary file organizations, such as B-trees, use tree structures.
      \end{itemize}

\hi{Secondary Storage Device}
  \par This section discusses some characteristics of magnetic disk and magnetic tape stoage devices.

  \hii{Hardware Description of Disk Devices}
    \hiii{Disk Devices}
      \begin{itemize}
        \item \tb{hard disk drive (HDD)}: the device that holds the disks
        \item \tb{capacity} (of a disk): the number of bytes it can store
        \item \tb{single-side/double-side} disk: a disk that store information on only one surface/both surfaces
        \item \tb{disk pack}: disks are assembled into disk pack, including many disks, to increase storage capacity
        \item \tb{track}: a circle on a disk that has a distinct diameter - information is stored on a disk in these concentric circles
        \item \tb{cylinder}: tracks with the same diameter on the various surfaces
        \item \tb{sector}: arc of a track (arc = part of a circle) - because tracks hold large amount of information, they are divided into sectors. The division of track into sectors is hard-coded on the disk surface and cannot be changed.
        \item \tb{disk-blocks/pages}: also divisions of a track, but is set by the OS during formatting (or initialization), size is fixed during initialization and cannot be changed dynamically, typical size is from 512 to 8192 bytes.
      \end{itemize}
      \par A disk is a \tb{random access addressable device}.
      \begin{itemize}
        \item \tb{hardware address of a block}: combination of cylinder number, track number (within the cyliner), and block number (within the track).
      \end{itemize}

  \hiii{Buffers}
    \begin{itemize}
      \item \tb{buffer}: a contiguous reserved area in main storage that holds one disk block
      \item \tb{cluster}: a group of contiguous disk blocks
      \item \tb{read command}: the contents of disk block/cluster are copied into the buffer
      \item \tb{write command}: the contents of buffer are copied into the disk block/cluster
    \end{itemize}

  \hii{Making Data Access More Efficient on Disk}
    \par Techniques to speed-up data access.
    \begin{itemize}
      \item \tb{Buffering of data}: New data is held waiting in a buffer while old data is processed by an application. This helps dealing with the incompatibility of speeds between CPU and electromechanical devices such as HDD.
      \item \tb{Proper organization of data on disk}: Keep related data on contiguous blocks and contiguous cylinders to minimize unnecessary movement of read/write arm and related seek time.
      \item \tb{Reading data ahead of request}: When a block is read into the buffer, other blocks of the same track is also read although not requested. This strategy works if the application is likely to need consecutive blocks.
      \item \tb{Proper scheduling of I/O requests}
      \item \tb{Use of log disks to temporarily hold writes}
      \item \tb{Use of SSDs or flash memory for recovery purposes}
    \end{itemize}

  \hii{Solid State Device (SSD) Storage}
    \par \tb{SSD storage} is based on \tb{flash memory technology}, therefore it is also known as \tb{flash storage}.
    \par \tb{Properties}
    \begin{itemize}
      \item No moving parts, run silently
      \item Faster access time and higher transfer rates compares to HDD
      \item No restriction on placement of data because any address is directly addressable (unlide HDD where related data from the same relation must be placed on contiguous blocks). This also results in no fragmentation.
    \end{itemize}

  \hii{Magnetic Tape Storage Device}
    \par \tb{Magnetic Tapes} are \tb{sequential access devices}, meaning that, to access the $n$th block on the tape, the preceding $n - 1$ blocks must be scanned first.

\hi{Buffering of Blocks}
  \par While one buffer is being read/written under the control of a disk I/O processor, the CPU can process data in another buffer.

\hi{Placing File Records on Disk}
  \hii{Records and Record Types}
    \par Data is stored in the form of \tb{records}.
    \begin{itemize}
      \item A \tb{record} corresponds to an entity (or a row in the table).
      \item A record consists of a collection of related data \tb{values} or \tb{items}, each value corresponds to a particular \tb{field} of the record.
      \item Each record has a \tb{record type} or \tb{record format} which comprises of field names and their corresponding data types.
      \item The \tb{data type} of a fields specifies the types of values a field can take. There are these types of data type:
      \begin{itemize}
        \item standard data type: number, string, boolean, date, time, etc.
        \item \tb{BLOBs} (binary large objects): complex data items that consist of large unstructured objects, typically stored separately from its record in a pool of disk blocks, and a pointer to the BLOB is included in the record.
      \end{itemize}
    \end{itemize}

\hi{Files, Fixed-Length Records and Variable-Length Records}
  \par A \tb{file} is a sequence of records.
  \begin{itemize}
    \item If every record in the file has exactly the same size (in bytes), the file is said to be made up of \tb{fixed-length records}.
    \item If different records in the file have different size (in bytes), the file is said to be made up of \tb{variable-length records}.
  \end{itemize}

\hi{Record Blocking, Spanned versus Unspanned Records}
  \hii{Record Blocking}
    \par Records of a file must be allocated to \tb{disk blocks} because \tb{a block is the unit of data transfer} between disk and memory.
    \par Let $B$ be the size of a block and $R$ be the size of a record (in bytes).
    \par If $B \geq R$, then we can fit $bfr = \floor{B / R}$ records per block.
    \begin{itemize}
      \item The number $bfr$ is the \tb{blocking factor} for the file
      \item If $R$ not divides $B$, then the size of the unused space is equal to $B - bfr * R$.
    \end{itemize}

  \hii{Spanned vs Unspanned Records}
    \par There are two different record organizations: \tb{spanned} and \tb{unspanned}.

    \img[width=12cm]{img/16-spanned-unspanned}{Unspanned (above) and Spanned (below) organizations}

    \hiii{Spanned Records}
      \par With \tb{spanned} organization, to utilize the unused space at the end of a block, we can store part of a record on one block and the rest on another. A \tb{pointer} at the end of the first block points to the block containing the remainder of the record in case it is not the next consecutive block on disk.
      \par If a record is larger than a block ($R > B$), then we \tb{must} use spanned organization.
      \par For variable-length records using spanned organization, each block store a different number of records. In this case, the blocking factor $bfr$ represents the average number of records per block for the file. We can use $bfr$ to calculate the number of blocks $b$ needed for a file of $r$ records:
      \[
        b = \ceil{r / bfr}
      \]

    \hiii{Unspanned Records}
      \par With \tb{unspanned organization}, records are not allowed to cross block boundaries. This is used with fixed-length records having $B > R$ because it makes each record starts at a known location in the block, simplifying record processing.

  \hii{Allocating File Blocks on Disk}
    \par There are several standard techniques for allocating the blocks of a file on disk:
    \begin{itemize}
      \item In \tb{contiguous allocation}, the file blocks are allocated to consecutive disk blocks. This makes reading the whole file very fast using double buffering, but it makes expanding the file difficult.
      \item In \tb{linked allocation}, each file block contains a pointer to the next file block. This makes it easy to expand the file but makes it slow to read the whole file.
      \item In \tb{indexed allocation}, one or more index blocks contain pointers to the actual file blocks.
    \end{itemize}

  \hii{File Headers}
    \par A \tb{file header} or \tb{file descriptor} contains information about a file that is needed by the system programs that access the file records.


\hi{Operations on Files}
  \par Operations on files are usually grouped into
    \begin{itemize}
      \item \tb{Retrieval operations}: do not change any data in the file, but only locate certain records so that their field values can be examined and processed
      \item \tb{Update operations}: change the file by insertion or deletion of records or by modification of field values.
    \end{itemize}
  \par In either case, we may have to \tb{select} one or more records based on a \tb{selection condition} or \tb{filtering condition}.
  \par A \tb{simple selection condition} involve an \tb{equality comparison} on some field of value. More complex conditions can involved other types of comparison operators, such as $\leq$ or $<$.
  \par \tb{Search operations} on files are generally based on \tb{simple selection conditions}.
  \par Steps of searching with a complex condition:
    \begin{itemize}
      \item A simple condition (with equality comparison) is extracted from the complex condition.
      \item The simple condition is used to locate the records on disk.
      \item Each located record is checked to determine whether it satisfies the full selection condition.
    \end{itemize}
  \par When several file records satisfy a search condition, the first record — with respect to the physical sequence of file records — is initially located and designated the \tb{current record}. Subsequent search operations commence from this record and locate the next record in the file that satisfies the condition.

\hi{Files of Unordered Records (Heap Files)}
  \par In \tb{heap files} or \tb{pile files}, records are placed in the file in the order in which they are inserted, so new records are inserted at the end of the file.
  \par Operations:
  \begin{itemize}
    \item \tb{Inserting a record}: very efficient.
    \item \tb{Searching a record}: linear search, which is expensive. For a file of $b$ block, on average it is required to search $b / 2$ blocks.
    \item \tb{Deleting a record}: a program must first find its block, copy the block into a buffer, delete the record from the buffer, and finally rewrite the block back to the disk. This leaves unused space in the disk block. Another technique used for record deletion is to have \tb{deletion marker} stored with each record. Search programs use the markers to consider only valid records in a block when conducting their search. Both of these deletion techniques require periodic reorganization of the file to reclaim the unused space of deleted records.
    \item \tb{Read all records in order of the values of some field}: expensive, requires creating a sorted copy of the file
  \end{itemize}


\hi{Files of Ordered Records (Sorted Files)}
  \par The records of a file on disk can be ordered based on the values of one of their fields called the \tb{ordering field}. This leads to an \tb{ordered} or \tb{sequential file}.
  \par Operations:
  \begin{itemize}
    \item \tb{Reading the records in order of the ordering key}: extremely efficient because no sorting is required
    \item \tb{Finding the next record from the current one in order of the ordering key}: requires no additional block accesses because the next record is in the same block as the current one (unless the current record is the last one in the block)
    \item \tb{Searching with condition based on the value of an ordering key}: binary search can be used which is faster than linear search
    \item \tb{Accessing records based on values of the nonordering fields}: expensive, requires linear search
    \item \tb{Read all records in order of the values of a nonordering field}: expensive, requires creating a sorted copy of the file
    \item \tb{Inserting and Deleting records}: expensive because the records must remain physically ordered
  \end{itemize}
  \par Ordered files are rarely used in database applications unless an additional access path, called \tb{primary index}, is used. This results in an \tb{indexed-sequential file}.
  \par If the ordering attribute is not a key, the file is called a \tb{clustered file}.


\hi{Hashing Techniques}
  \par A \tb{hash file} provides very fast access to records under certain search conditions through hashing.
  \par The \tb{search condition} \tb{must be an equality condition on a single field} called the \tb{hash field}. In most cases, the hash field is also a key field of the file, in which case it is called the \tb{hash key}.
  \par Hashing requires a \tb{hash function} or \tb{randomized function} $h$, which is applied to the hash field value of a record and yields the address of the disk block in which the record is stored. For most records, we need only \tb{a single-block access} to retrieve that record.

  \hii{Internal Hashing}
    \hiii{Hash Table And Hash Function}
      \par For internal files, hashing is typically implemented as a \tb{hash table} through the use of \tb{an array of records}.
      \par Suppose that the array has $M$ slots, then the hash function transforms the \tb{hash field value} $K$ into an integer between $0$ and $M - 1$.
      \par One common hash function is $h(K) = K mod M$. Noninteger hash field values can be transformed into integers before the mod function is applied.
      \par The problem with most hashing functions is that they do not guarantee that distinct values will hash to distinct addresses, because the \tb{hash field space} - the number of possible values a hash field can take—is usually much larger than the \tb{address space} - the number of available addresses for records.

    \hiii{Collisions}
      \par A \tb{collision} occurs when the hash field value of a record that is being inserted hashes to an address that already contains a different record. In this situation, we must insert the new record in some other position. The process of finding another position is called \tb{collision resolution}.
      \par Methods for collision resolution:
      \begin{itemize}
        \item \tb{Open addressing}: from the occupied position specified by the hash address, the program checks the subsequent positions in order until an unused (empty) position is found.
        \item \tb{Chaining}: 
      \end{itemize}
\chapter{Indexing Structures for Files and Physical Database Design}

\par In this chapter, we assume that a file already exists with some primary organization such as the unordered, ordered, or hashed organizations.
\par \tb{Indexes} are additional auxiliary \tb{access structure} that are used to speed up the retrieval of records under certain search conditioins, providing \tb{secondary access path} that does not affect the physical placement of records in the primary data file on disk.
\par Efficient access to records is based on the \tb{indexing fields} that are used to construct the index. Any field of the file can be used to create an index. Also, multiple indexes on different fields or indexes on multiple fields can be created within a file.

\hi{Types of Single-Level Ordered Indexes}
  \par The idea behind an ordered index is similar to that behind the index used in a textbook, which lists important terms at the end of the book in alphabetical order along with a list of page numbers where the term appears in the book.
  \par An index access structure is usually defined on a single field of a file, called an indexing field (or indexing attribute). The index typically stores each value of the index field along with a list of pointers to all disk blocks that contain records with that field value. The values in the index are ordered so that we can do a binary search on the index. If both the data file and the index file are ordered, and since the index file is typically much smaller than the data file, searching the index using a binary search is a better option.
  \par Types of ordered indexes:
  \begin{itemize}
    \item A \tb{primary index} is specified on the \tb{ordering key field} of an ordered file of records.
    \item If numerous records in the file \tb{have the same value for the ordering field}, then \tb{clustering index} is used.
    \item A \tb{secondary index} can be specified on any \tb{nonordering field} of a file.
  \end{itemize}

  \hii{Primary Indexes}
    \par A \tb{primary index} is an \tb{ordered file} whose records are of fixed length with two fields:
    \begin{itemize}
      \item The first field is of the same data type as the ordering key field - called the \tb{primary key} - of the data file.
      \item The second field is a pointer to a disk block (a block address)
    \end{itemize}
    \par The two field values of index entry $i$ can be denoted as $\pair{K(i), P(i)}$.
    \img[width=14cm]{img/primary-index.png}{}
    \par In the figure, the \ilc{Name} field is the primary key because it is the \tb{ordering key field} of the file with the assumption that each value is unique. Each entry in the index has a \ilc{Name} value and a pointer.
    \par The total number of entries in the index is the same as the number of disk blocks in the ordered data file.
    \par The first record in each block of the data file is called the \tb{anchor record} of the block, or the \tb{block anchor}.
    \par Indexes can also be characterized as dense or sparse. A dense index has an index entry for every search key value (and hence every record) in the data file. A sparse (or nondense) index, on the other hand, has index entries for only some of the search values.

  \hii{Clustering Indexes}
    \par If file records are \tb{physically ordered on a nonkey field} which does not have a distinct value for each record, that field is called the \tb{clustering field} and the data file is called a \tb{clustered file}. We can create a different type of index, called a clustering index, to speed up retrieval of all the records that have the same value for the clustering field.

    \par A clustering index is \tb{an ordered file} with two fields:
    \begin{itemize}
      \item The first field is of the same type as the clustering field of the data file
      \item The second field is a disk block pointer
    \end{itemize}
    \par There is \tb{one entry} in the clustering index \tb{for each distinct value} of the clustering field, and it contains the value and a pointer to the first block in the data file that has a record with that value for its clustering field.
    \par To alleviate the problem of insertion (and deletion), it is common to reserve a whole block (or a cluster of contiguous blocks) for each value of the clustering field.

  \hii{Secondary Indexes}
    \par A \tb{secondary index} provides a secondary means of accessing a data file for which some primary access already exists. The data file records could be ordered, unordered, or hashed. The secondary index may be created on a field that is a candidate key and has a unique value in every record, or on a nonkey field with duplicate values.
    \par The index is an \tb{ordered file} with two fields:
    \begin{itemize}
      \item The first field is of the same data type as some \tb{nonordering field} of the data file that is an indexing field.
      \item The second field is either a block pointer or a record pointer.
    \end{itemize}
    \par Many secondary indexes (and hence, indexing fields) can be created for the same file - each represents an additional means of accessing that file based on some specific field.
    \par A secondary index provides a \tb{logical ordering} on the records by the indexing field. If we access the records in order of the entries in the secondary index, we get them in order of the indexing field. The primary and clustering indexes assume that the field used for \tb{physical ordering} of records in the file is the same as the indexing field.

\hi{Multilevel Indexes}
  \par In the indexing schemas described so far, \tb{binary search} is applied to the index to locate pointers to a disk block or to a record (or records) in the file having a specific index field value. Binary search requires approximately $\log_2(b_i)$ block accesses for an index with $b_i$ blocks.
  \par The idea behind \tb{multilevel index} is to reduce the part of the index that we continue to search by the blocking factor index $bfr_i > 2$, which reduces the search space much faster. The value $bfr_i$ is called the \tb{fan-out} of the multilevel index, denoted by $fo$.
  \par Whereas we divide the record search space into two halves at each step during a binary search, we divide it n-ways (where $n = fo$) at each search step using the multilevel index. Searching a multilevel index requires approximately $log_{fo} b_i$ block accesses
  \img[width=14cm]{img/multilevel-index.png}{}
  \par The first level - the primary index for the data - is an ordered file with a distinct value for each $K(i)$. The second level is the primary index for the first level. The third level is the primary index for the second level, and so on.

\hi{Dynamic Multilevel Indexes Using B-Trees and B$^{+}$ Trees}

\hi{Indexes on Multiple Keys}

\hi{Other Types of Indexes}
  \hii{Hash Indexes}
    \par The \tb{hash index} is a secondary structure to access the file by using hashing on a search key other than the one used for the primary data file organization.

  \hii{Bitmap Indexes}
    \par The \tb{bitmap index} is another popular data structure that facilitates \tb{querying on multiple keys}.
    \par Bitmap indexing is used for relations that contain a \tb{large number of rows}. It creates an index for one or more columns, and each value or value range in those columns is indexed.
    \par Typically, a bitmap index is created for those columns that contain \tb{a fairly small number of unique values}.
    \par To build a bitmap index on a set of records in a relation, the records must be numbered from 0 to $n$ with an id (a record id or a row id) that can be mapped to a physical address made of a block
number and a record offset within the block.
    \par A bitmap index is built on one particular value of a particular field (the column in a relation) and is just an array of bits. Consider a bitmap index for the column $C$ and a value $V$ for that column. For a relation with $n$ rows, it contains $n$ bits. The $i$th bit is set to 1 if the row $i$ has the value $V$ for column $C$; otherwise it is set to a 0. If $C$ contains the valueset $\pair{v_1, v_2, \ldots, v_m}$ with $m$ distinct values, then $m$ bitmap indexes would be created for that column.
    \img[width=12cm]{img/index-bitmap.jpg}{}
    \par In the figure, there are only 2 values of \ilc{Sex} and 3 values of \ilc{Zipcode}. Therefore, using bitmap indexing makes a lot of sense. 2 bitmaps are created for \ilc{Sex} and 3 for \ilc{Zipcode}.

  \hii{Function-Based Indexing}
    \par \tb{Example 1}: A function-based index on the \ilc{EMPLOYEE} table based on an uppercase representation of the \ilc{Lname} column

\begin{minted}[linenos,tabsize=2,breaklines]{SQL}
CREATE INDEX upper_ix ON Employee (UPPER(Lname));
\end{minted}
    
    \par The following query will use the index. Without the function-based index, an Oracle Database might perform a full table scan.

\begin{minted}[linenos,tabsize=2,breaklines]{SQL}
SELECT First_name, Lname
FROM Employee
WHERE UPPER(Lname)= "SMITH"
\end{minted}

    \par \tb{Example 2}: An index is being created on the sum of \ilc{salary} and \ilc{commission_pct}.
\begin{minted}[linenos,tabsize=2,breaklines]{SQL}
CREATE INDEX income_ix
ON Employee(Salary + (Salary*Commission_pct));
\end{minted}

The query does use this index even though the order of attributes is reversed.

\begin{minted}[linenos,tabsize=2,breaklines]{SQL}
SELECT First_name, Lname
FROM Employee
WHERE ((Salary*Commission_pct) + Salary ) > 15000;
\end{minted}


\hi{Some General Issues Concerning Indexing}
  \hii{Logical vs Physical Indexes}
    \par An index entry $\pair{K, Pr}$ where $Pr$ is the physical pointer to a physical record address on disk is called a \tb{physical index}. The disadvantage is that the pointer must be changed if the record is moved to
    another disk location.
    \par A solution is \tb{logical address}, whose index entries are of the form $\pair{K, K_p}$, where $K$ is the secondary indexing field matched with the value $K_p$ of the field used for the primary file organization.
  
  \hii{Index Creation}
    \par General form of the index creation query:
\begin{minted}[linenos,tabsize=2,breaklines]{SQL}
CREATE [ UNIQUE ] INDEX < index name >
ON < table name > ( < column name > [ < order > ] { , < column name > [ < order > ] } )
[ CLUSTER ] ;
\end{minted}
    \begin{itemize}
      \item The keywords \ilc{UNIQUE} and \ilc{CLUSTER} are optional.
      \item The keyword \ilc{CLUSTER} is used when the index to be created should also sort the data file records on the indexing attribute. Thus, specifying \ilc{CLUSTER} on a key (unique) attribute would create some variation of a primary index, whereas specifying \ilc{CLUSTER} on a nonkey (nonunique) attribute would create some variation of a clustering index.
      \item The value for \ilc{<order>} can be either \ilc{ASC} (ascending) or \ilc{DESC} (descending), and it specifies whether the data file should be ordered in ascending or descending values of the indexing attribute. The default is ASC.
    \end{itemize}

    \par \tb{Index Creation Process}
    \par \tb{Indexinng of Strings}

  \hii{Tuning Indexes}

  \hii{Additional Issues Related to Storage of Relations and Indexes}

\hi{Physical Database Design in Relational Databases}
  \hii{Factors That Influence Physical Database Design}
    \begin{itemize}
      \item Analyzing the Database Queries and Transactions
      \item Analyzing the Expected Frequency of Invocation of Queries and
      Transactions
      \item Analyzing the Time Constraints of Queries and Transactions
      \item Analyzing the Expected Frequencies of Update Operations
      \item Analyzing the Uniqueness Constraints on Attributes
    \end{itemize}
  
  \hii{Physical Database Design Decisions}
    \begin{itemize}
      \item Whether to index an attribute
      \item What attribute or attributes to index on
      \item Whether to set up a clustered index
        \par \tb{Range queries} benefit a great deal from clustering.
      \item Whether to use a hash index over a tree index
        \par Hash indexes work well with equality conditions, particularly during joins to find a matching record(s), but they do not
        support range queries.
      \item Whether to use dynamic hashing for the file
        \par \tb{Dynamic hashing} is used when files are very \tb{volatile}.
    \end{itemize}

\begin{appendices}

\end{appendices}

\end{document}
