\chapter{The Relational Algebra and Relational Calculus}

\hi{Unary Relational Operations: SELECT and PROJECT}
  \par \tb{An Unary Operation} is a operation applied to a single relation.

  \hii{The SELECT Operation}
    \hiii{Definition}
      \par The SELECT operation is used to choose a subset of the tuples from a relation that satisfies a selection condition.

    \hiii{Notation}

      \begin{align*}
        \sigma_{\fieldname{selection condition}}(R)
      \end{align*}

    \hiii{Selection Condition}
      \par The Boolean expression specified in $\fieldname{selection condition}$ is made up of a number of clauses of the form

    \begin{align*}
      &\fieldname{attribute name} \fieldname{comparison operator} \fieldname{constant value}
      \qquad \mbox{or} \\
      &\fieldname{attribute name} \fieldname{comparison operator} \fieldname{attribute name}
    \end{align*}

    \par Clauses can be connected by the standard Boolean operators AND, OR, and NOT to form a general selection condition.

    \hiii{Properties}
      \begin{itemize}
        \item Commutative: multiple SELECT operations can be carried out in any order.
          \begin{align}
            \sigma_{\fieldname{cond1}}(\sigma_{\fieldname{cond2}}(R)) =
            \sigma_{\fieldname{cond2}}(\sigma_{\fieldname{cond1}}(R))
          \end{align}
        \item Cascading: multiple SELECT operations can be combined into a single SELECT operation with a conjunction (AND) condition:
          \begin{align}
            \sigma_{\fieldname{cond1}}(\sigma_{\fieldname{cond2}}(R)) =
            \sigma_{\fieldname{cond1 AND cond2}}(R)
          \end{align}
      \end{itemize}

    \hiii{SQL Equivalent}
      \begin{lstlisting}
        SELECT *
        FROM table_name
        WHERE cond_1 AND cond_2
      \end{lstlisting}


  \hii{The PROJECT Operation}
    \hiii{Definition}
      \par The PROJECT operation, selects certain columns from the table and discards the other columns.

    \hiii{Notation}
      \begin{align*}
        \pi_{\fieldname{attribute list}}(R)
      \end{align*}

    \hiii{Degree}
      \par The degree of a PROJECT operation is the number of attributes in the $\fieldname{attribute list}$.

    \hiii{Properties}
      \par \tb{Duplicate elimination}: If the attribute list includes only nonkey attributes of $R$, the PROJECT operation removes any duplicate tuples, so the result of the PROJECT operation is a set of distinct tuples, and hence a valid relation.

    \hiii{SQL Equivalent}

      \begin{lstlisting}
        SELECT DISTINCT attr_1, attr_2
        FROM table_name
      \end{lstlisting}

  \hii{Sequences of Operations and the RENAME Operation}
    \hiii{Sequences of Operations}
      \par For most queries, we need to apply several relational algebra operations one after the other. Either we can write the operations as a single relational algebra expression by nesting the operations, or we can apply one operation at a time and create intermediate result relations. In the latter case, we must give names to the relations that hold the intermediate results.

      \par \tb{Assignment operation notation}
    \begin{align*}
      S &\leftarrow \sigma_{\fieldname{selection condition}}(R)
      S &\leftarrow \pi_{\fieldname{attribute list}}(R) \\
    \end{align*}
    where $S$ is the immediate result relation.

    \hiii{The RENAME Operation}
      \begin{align*}
        \rho_{S(B_1, B_2, \ldots, B_n)}(R)
      \end{align*}
      where $S$ is the new relation name and each attribute name $B_i$ is the new name of the attribute $A_i$.

\hi{Relational Algebra Operations from Set Theory}
  \hii{The UNION, INTERSECTION, and MINUS Operations}
    \hiii{Binary Operations on Relations}

      \par \tb{A Binary Operation} is a operation applied to two sets (of tuples). When binary operations are adapted to relational databases, the two relations on which any of these three operations are applied must have the same type of tuples; this condition has been called \tb{union compatibility} or \tb{type compatibility}.

      \par Two relations $R(A_1, A_2, ..., A_n)$ and $S(B_1 , B_2, \ldots, B_n)$ are said to be union compatible (or type compatible) if:

      \begin{itemize}
        \item They have the same degree $n$
        \item $dom(A_i) = dom(B_i) \forall i: 1 \leq i \leq n$
      \end{itemize}

    \hiii{UNION, INTERSECTION and MINUS}
      \par Given two relations $R$ and $S$. To apply UNION, INTERSECTION and MINUS on $R$ and $S$, $R$ and $S$ must be \tb{union compatibile}.

      \begin{itemize}
        \item \tb{UNION}: $R \cup S$ is the set of all tuples either in $R$ or $S$. Duplicates are eliminated.
        \item \tb{INTERSECTION}: $R \cap S$ is the set of all tuples both in $R$ and $S$.
        \item \tb{SET DIFFERENCE} or \tb{MINUS}: $R - S$ is the set of all tuples that are in $R$ but not in $S$.
      \end{itemize}

    \hiii{Properties}
      \begin{itemize}
        \item Both UNION and INTERSECTION are \tb{commutative}:
          \begin{align*}
            R \cup S &= S \cup R \\
            R \cap S &= S \cap R
          \end{align*}
        \item Both UNION and INTERSECTION are \tb{associative}:
          \begin{align*}
            R \cup (S \cup T) &= (R \cup S) \cup T \\
            R \cap (S \cap T) &= (R \cap S) \cap T
          \end{align*}
        \item INTERSECTION can be expressed in terms of UNION and MINUS:
          \begin{align*}
            R \cap S = ((R \cup S) - (R - S)) - (S - R)
          \end{align*}
      \end{itemize}

  \hii{The CARTESIAN PRODUCT (CROSS PRODUCT) Operation}
    \hiii{Definition}
      \par Given two relations $R(A_1, A_2, \ldots, A_n)$ and $S(B_1, B_2, \ldots, B_m)$. The CARTESIAN PRODUCT $R \times S = Q$ resulting in a new relation $Q$ such that:
      \begin{align*}
        Q(A_1, A_2, \ldots, A_n, B_1, B_2, \ldots, B_m) \qquad \mbox{ and }
        \qquad |Q| = |R| \times |S|
      \end{align*}

    \hiii{SQL Equivalent}
      \begin{itemize}
        \item \lstinline{CROSS JOIN}
        \item \lstinline{SELECT* FROM table1, table2 WHERE *}
      \end{itemize}

\hi{Binary Relational Operations: JOIN AND DIVISION}
  \hii{The JOIN Operation}
    \hiii{Definition}
      \par The \tb{JOIN Operation} is used to combine \ti{related tuples} from two relations into single ``longer" tuples.
      \par Given two relations $R(A_{1}, \ldots, A_{n})$ and $S(B_{1}, \ldots, B_{n})$. Let $Q$ be the result of the JOIN.
      \begin{align*}
        Q \leftarrow R \bowtie_{\fieldname{join condition}} S
      \end{align*}
      \par The number of attributes in $Q$ is $n + m$: $Q(A_1, \ldots, A_n, B_1, \ldots, B_m)$ in that order.
      \par $Q(A_1, \ldots, A_n, B_1, \ldots, B_n)$ has a tuple for each combination of tuples - one from $R$ and one from $S$ - whenever the combination satisfies the join condition.

    \hiii{Variation of JOIN: EQUIJOIN and NATURAL JOIN}
      \par The JOIN operation where the only comparison operator used is equal ($=$) is called an \tb{EQUIJOIN}. The resulting relation of EQUIJOIN always contain one or more pairs of attributes that have \tb{identical values}.
      \par The purpose of the NATURAL JOIN operation is to delete the duplicate attributes.
      \par For the NATURAL JOIN to work, two attributes with the same values but different names must be renamed so that they have the same name first. Only after that can the NATURAL JOIN operation be applied.

    \hiii{More about JOIN}
    \par If there is no join condition, all combinations of tuples qualify and the JOIN degenerates into a CARTESIAN PRODUCT.
    \par A single JOIN operation is used to combine data from two relations so that related information can be presented in a single table. These operations are also known as \tb{inner joins}.

  \hii{Complete Set of Relational Algebra Operations}
    \par The set of relational algebra operations $\{\sigma, \pi, \cup, \rho, -, \times\}$ is a complete set; that is, any of the other original relational algebra operations can be expressed as a sequence of operations from this set.

  \hii{The DIVISION Operation}
    \hiii{Definition}
      \par Given two relations:
      \begin{itemize}
        \item $R(Z)$: a relation $R$ with a set of attribute $Z$
        \item $S(X)$: a relation $S$ with a set of attribute $X$
        \item $X \subseteq Z$
      \end{itemize}
      \par Let $Y$ be the set of attributes of $R$ that are not attributes of $S$, meaning
        $Y = Z - X$.
      \par Given the following DIVISION operation:
      \begin{align*}
        T(Y) = R(Z) \div S(X)
      \end{align*}
      \par The relation $T(Y)$ includes all tuples $t_T$ such that: $\forall t_S \in S$:
      $\forall t_R \in R \mbox{that} t_{R}[X] = t_{S}$: $t_{R}[Y] = t_{T}$.

  \hii{Query Trees}
    \par A \tb{query tree} is a tree structure that corresponds to a relational algebra expression:
    \begin{itemize}
      \item The \tb{input relations} are represented as \tb{leaf nodes}.
      \item The \tb{relational algebra operations} are represented as \tb{internal nodes}.
    \end{itemize}

\hi{Additional Relational Operations}

  \hii{Generalized Pojection}
    \par The \tb{generalized projection operation} extends the projection operation by allowing
functions of attributes to be included in the projection list.
    \begin{align*}
      \pi_{F_1, F_2, \ldots, F_n}(R)
    \end{align*}
    where $F_1, F_2, \ldots, F_n$ are functions over the attributes in relation $R$ and may involve arithmetic operations and constant values.
    \par \tb{Usage}: developing reports where computed values have to be produced in the columns of a query result.

  \hii{Aggregate Functions and Grouping}
    \hiii{Aggregate Functions}
      \par \tb{Aggregate functions} are applied on collections of values from the database. These functions are used in simple statistical queries that summarize information from the database tuples.
      \par List of aggregate functions:
        \begin{itemize}
          \item SUM
          \item AVERAGE
          \item MAXIMUM
          \item MINIMUM
          \item COUNT: counting tuples or values
        \end{itemize}

    \hiii{Grouping}

      \par Sometimes, request involving aggregate functions also involves grouping the tuples in a relation by the value of some of their attributes and then applying an aggregate function \ti{independently to each group}.

      \par \tb{Notation}: An AGGREGATE FUNCTION operation is denoted by the symbol $\mathcal{I}$ (pronounced \ti{script F}):

      \begin{align*}
        \fieldname{grouping attributes} \mathscr{I} \fieldname{function list} (R)
      \end{align*}

      where $\fieldname{grouping attributes}$ is a list of attributes of the relation specified in $R$, and $\fieldname{function list}$ is a list of ($\fieldname{function}$ $\fieldname{attribute}$) pairs. In each such pair:

      \begin{itemize}
        \item $\fieldname{function}$ is one of the allowed functions: SUM, AVERAGE, MAXIMUM, MINIMUM, COUNT
        \item $\fieldname{attribute}$ is an attribute of the relation specified by $R$.
      \end{itemize}

      \par The $\tb{resulting relation}$ has the grouping attributes plus one attribute for each element in the function list.

  \hii{Recursive Closure Operations}
    \par \tb{Recursive Closure Operation} is applied to a \tb{recursive relationship} between tuples of the same type.
    \par To compute the relationship of $n$-level recursive closure, use \tb{transitive closure}.

  \hii{OUTER JOIN}
    \hiii{Inner vs Outer JOIN}
      \par \tb{Inner join} is the type of JOIN that eliminate tuples with no match.
      \par \tb{Outer join} is the type of JOIN that keeps all tuples in $R$ and in $S$ when joining two relations $R$ and $S$.

    \hiii{LEFT OUTER JOIN, RIGHT OUTER JOIN, AND FULL OUTER JOIN}
      \par Given a join operation between two relation $R_{1}$ and $R_{2}$.

      \par The \tb{LEFT OUTER JOIN} keeps every tuple in the first (left) relation $R_1$. If no matching tuple is found in $R_2$, then the attributes of $R_2$ in the join result are filled (padded) with NULL values.

      \par The \tb{RIGHT OUTER JOIN} keeps every tuple in the second (right) relation $R_2$. If no matching tuple is found in $R_1$, then the attributes of $R_1$ in the join result are filled (padded) with NULL values.

      \par The \tb{FULL OUTER JOIN} keeps every tuple in both relations $R_1$ and $R_2$. When no
matching tuples are found, they are padded with NULL values as needed.

  \hii{OUTER UNION operation}
    \par The OUTER UNION operation was developed to take the union of tuples from two relations that have some, but not all, common attributes.
    \begin{itemize}
      \item If two relations have \ti{all} common attributes, they are called \tb{union (type) compatible}.
      \item If two relations have \ti{some} common attributes, they are called \tb{partially compatible}.
    \end{itemize}

    \par Given two relations $R(X, Y)$ and $S(X, Z)$, where $X$ is the subset of attributes that $R$ and $S$ share. $X$ is said to be \ti{union compatible}.

    \par In a OUTER UNION operation between $R$ and $S$:

    \begin{itemize}
      \item The attributes that are union compatible ($X$) are represented only once in the result.
      \item The attributes that are not union compatible from either relation are also kept in the result relation T(X, Y, Z). It is therefore the same as a FULL OUTER JOIN on the common attributes.
    \end{itemize}

    \par Two tuples $t_1 \in R$ and $t_2 \in S$ are said to match if $t_1[X] = t_2[X]$. These will be combined (unioned) into a single tuple in $t$. Tuples in either relation that have no matching tuple in the other relation are padded with NULL values.

\hi{Tuple Relational Calculus}
  \hii{Tuple Variables and Range Relations}
    \hiii{Tuple Variables}
      \par A simple tupele relational calculus query is of the form:
      \begin{align*}
        \{t | COND(t)\}
      \end{align*}
      where $t$ is a tuple variable and $COND(t)$ is a conditional (Boolean) expression involving $t$ that evaluates to either TRUE or FALSE for different assignments of tuples to the variable $t$.

    \hiii{Range Relations}
      \par A \tb{range relation} of tuple variable $t$ is the \ti{relation} $R$ containing that tuple.
