\documentclass[12pt, a4paper]{report}

% Margins
\usepackage[margin=0.75in]{geometry}

% Indent the first paragraph after a chaper/section heading
\usepackage{indentfirst} 

% Table formatting
\usepackage{multirow}

\usepackage[version=4]{mhchem}

% ___MACROS___

% Formatting
\newcommand{\mRow}{\multirow}
\newcommand{\impt}[1]{\textbf{\textit{#1}}}

% Headings
\newcommand{\hi}{\section}
\newcommand{\hii}{\subsection}
\newcommand{\hiiBEGIN}[1]{\subsection{#1} \begin{enumerate}}
\newcommand{\hiiEND}{\end{enumerate}}
\newcommand{\hiii}{\item\textbf}
\newcommand{\hiiiBEGIN}[1]{\item\textbf{#1} \begin{enumerate}}
\newcommand{\hiiiEND}{\end{enumerate}}
\newcommand{\hiv}{\item\textbf}

% Tables
\usepackage{makecell}
\newcommand{\tableBEGIN}[1]{\begin{center} \begin{tabular}{#1}}
\newcommand{\tableEND}{\end{tabular} \end{center}}

% Set section numbering
\setcounter{secnumdepth}{4}

% Pictures
\usepackage{graphicx}
\usepackage{float}
\graphicspath{{img/}}

% Orbitals
\newcommand{\forb}{\framebox{$\uparrow \downarrow$}}
\newcommand{\horb}{\framebox{$\uparrow \mbox{ }$}}

% Mathematical
\usepackage{amsmath}
\usepackage{gensymb}
\usepackage{relsize}
\newcommand{\mul}{\cdot}

% Macros
\newcommand{\Dt}{\Delta}
\newcommand{\SUM}[1]{\sum\limits #1}
\newcommand{\INT}[1]{\int\limits #1}

% Equation box
\usepackage{empheq}
\newenvironment{eqbox}{%
    \setkeys{EmphEqEnv}{align}
    \setkeys{EmphEqOpt}{box=\fbox}
    \EmphEqMainEnv%
}{%
    \endEmphEqMainEnv%
}

% Link for TOC
\usepackage{hyperref}
\hypersetup{
    colorlinks,
    citecolor=black,
    filecolor=black,
    linkcolor=black,
    urlcolor=black
}

\begin{document}
\tableofcontents

\chapter{Atoms, Molecules and Ions}
\hi{The Atomic Theory of Matter}
    \par According to \textbf{Dalton's atomic theory}:
        \begin{itemize}
            \item Each element is composed of extremely small particles called atoms.
            \item All atoms of a given element are identical to one another in properties. Atoms of
            one element are different from the atoms of all other elements.
            \item The atoms of one element cannot be changed into atoms of a different element, nor
            created nor destroyed by chemical reactions;
            \item Compounds are formed when atoms of more than one element combine. A given compound
            always has the same relative number and kind of atoms.
        \end{itemize}

\hi{The Discovery of Atomic Structure}
    \hii{J. J. Thomson's Discovery of Electrons}
        \par J. J. Thomson discovered electrons through the experiment of \textbf{Cathode ray}.
        \par In the experiment, he also discovered the ratio between the charge and the mass of
        an electron.
    \hii{Millikan's oil drop Experiment}
        \par Robert Millikan discovered the charge of the electron through his \textbf{oil-drop
        experiment}.
        In addition, he calculated the mass of the electron using Thomson's charge-to-mass ratio.

    \hii{The nuclear atom}
        \par Thomson proposed the \textbf{plum-pudding model}, in which the atom consisted
        of a uniform positive sphere of matter in which the electrons were embedded.
        \par In 1910, Rutherford disproved Thomson's model with the \textbf{gold foil experiment}.
        In the experiment, almost all $\alpha$ particles passed directly through the gold foid
        without deflection. Rutherford postulated that most of the mass and charge of an atom
        reside in a very small, extremely dense region called the \textbf{nucleus}; most of the
        volume of the atom is empty space in which electrons move around the nucleus.

\hi{The Modern View of Atomic Structure}

\pagebreak

\chapter{Electronic Structure of Atoms}
\hi{The Wave Nature of Light}
    \par Light is a type of {electromagnetic radiation}.
    \par The electromagnetic spectrum:
        \begin{figure}[H]
            \begin{center}
                \includegraphics{ElectromagneticSpectrum.png}
                \caption[width=\textwidth]{Electromagnetic Spectrum}
            \end{center}
        \end{figure}
    \par All types of electromagnetic radiation move through a vacuum at a speed of
    $3 \mul 10^8 m/s$.
    \par The inverses relationship between the frequency and the wavelength of electromagnetic
    radiation:
        \begin{equation}
            c = \lambda f
        \end{equation}

\hi{Quantized Energy and Photons}
    \par Although the wave model of light explains many aspects of its behavior, this model cannot
    explain several phenomena. Three of these are:
    \begin{itemize}
        \item \impt{Blackbody radiation} or \impt{Hot-object radiation}:
            the emmission of light from hot objects.
        \item \impt{The photoelectric effect}: the emmission of electrons from metal surfaces on
            which light shines.
        \item \impt{Emmission spectra}: the emmission of light from electronically excited gas atoms.
    \end{itemize}

    \hii{Blackbody radiation and Planck's theory}
        \textit{When solids are heated, they emit radiation. The wavelength distribution of
            the radiation depends on the temperature: a red-hot object is cooler than a
            white-hot one.}
        \par The relationship between the temperature and the intensity and wavelengths of the
            emitted radiations was solved by Max Planck.
        \par He assumed that: \textit{Energy can be either released or absorbed by atoms only in
            discrete "chunks" of some minimum size called \textbf{quantum}.}
        \par The energy of each quantum can be evaluated with the equation:
        \begin{equation}
            E = hf
        \end{equation}
        \par in which:
            \begin{itemize}
                \item $E$: the energy of a single quantum $[J]$
                \item $h$: Planck's constant $(6.626 \mul 10^{-34} J \mul s)$
                \item $f$: the frequency of the radiation $[Hz]$
            \end{itemize}
        \par According to Planck's theory, matter is allowed to emit and absorb energy only in
        whole-number multiples of $hf$, such as $hf$, $2hf$, $3hf$, and so forth.

    \hii{Photoelectric effect and Photons}
        \par Experiments had shown that light shining on a clean metal surface causes the surface
        to emit electrons. Each metal has a minimum frequency of light below which no electrons
        are emitted.
        \par Einstein assumed that the radiant energy strking the metal surface does not behave
        like a wave but rather as if it were a stream of tiny energy packets called \impt{photons}.
        \par Extending Planck' quantum theory, Einstein deduced that the energy of each photon
        must be quantized:
        \begin{equation}
            E_{photon} = hf
        \end{equation}
        \par \textit{Under the right conditions, a photon can strike a metal surface and be
        absorbed. When this happens, the photon can transfer its energy to an electron in
        the metal. A certain amount of energy - called the work function - is required for
        an electron to overcome the attractive forces that hold it in the metal. If the photons
        of the radiation impinging on the metal have less energy than the work function,
        electons do not acquire sufficient energy to escape from the metal surface, even if the
        light beam is intense. If the photons of radiation have sufficient energy, electrons
        are emitted from the metal. If the photons have more than the minimum energy required
        to free electrons, the excess energy appears as the kinetic energy of the emitted electrons.} 

\hi{Line Spectra and The Bohr's Model}
    \hii{Line Spectra}
        \par Most common radiation sources, including lightbulbs and stars, produce radiation
        \textit{containing many different wavelengths}. A \impt{spectrum} is produced when
        radiation from such sources is separated into its different wavelength components.
        \par There are three types of spectrum:
        \begin{itemize}
            \item Continuous Spectrum
            \item Emmission Spectrum (or \textbf{Line Spectrum})
            \item Absorption Spectrum
        \end{itemize}
    \hii{The Bohr's Model}
        \par The Rutherford's Model of Atom is not feasible since electrons would collapse into the
        nucleus due to the electrostatic force.
        \par Bohr based his model on three postulates:
        \begin{itemize}
            \item Only orbits of certain radii, corresponding to certain definite energies, are
                permitted for the electron in a hydrogen atom.
            \item An electron in a permitted orbit has a specific energy and is in an "allowed"
                energy state. An electron in an allowed energy state will not radiate energy
                and therefore will not spiral into the nucleus.
            \item Energy is emitted or absorbed by the electron only as the electron changes from
                one allowed energy state to another. This energy is emitted or absorbed as a
                photon: $\Delta E = hf$.
        \end{itemize}

    \hii{The Energy States of the Hydrogen Atom}
        \par The energy corresponding to each allowed orbit for the electron in the hydrogen atom:
        \begin{equation} \label{eq:01}
            E_{n} = (-2.18 \mul 10^{-18} J)(\dfrac{1}{n^{2}}) = (-13.6 eV)(\,\dfrac{1}{n^2})\,
        \end{equation}
        \par in which:
        \begin{itemize}
            \item $E_{n}$: the energy corresponding to the $n^{th}$ orbit of the electron
            \item $n$: the \impt{principle quantum number}
        \end{itemize}
        \par $n$ is a whole number values of 1, 2, 3, \ldots to infinity. Each orbit corresponds
            to a different value of $n$. As $n$ increases, the orbit gets larger.
        \par The lower the energy is, the more stable the atom will be. The lowest energy state,
            $n = 1$, is called the \impt{ground state} of the atom. When the electron is in a higher
            senergy, $n \geq 2$, the atom is said to be in a \impt{excited state}.
        \par The state in which the electron is removed from the nucleus is the reference, or
            zero-energy, state of the hydrogen atom. This zero-energy state is \textit{higher} in
            energy than the states with negative energies.
        \par Electron could "jump" from one allowed energy state to another by either absorbing
        or emitting photons whose radiant energy corresponds exactly to the energy difference
        between the two states.
        \begin{equation} \label{eq:02}
            \Delta E = E_{f} - E_{i} = E_{photon} = hf
        \end{equation}
        \par From the equations \eqref{eq:01} and \eqref{eq:02}, we have the formula:
        \begin{equation}
            \Delta E = hf = (-13.6 eV)(\dfrac{1}{n_{f}^2} - \dfrac{1}{n_{i}^2})
        \end{equation}

    \hii{Limitations of the Bohr's Model}
        \par The limitations of the Bohr's Model are:
        \begin{itemize}
            \item It cannot explain the spectra of other atoms, except in a rather crude way.
            \item It cannot fully explain the problem of why the negatively charged electron would not
                just fall into the positively charged nucleus.
        \end{itemize}
        \par Having said that, the Bohr's model introduces two important ideas that are also
        incorporated into our current model:
        \begin{itemize}
            \item Electrons exist only in certain discrete energy levels, which are described by
                quantum numbers.
            \item Energy is involved in moving an electron from one level to another.
        \end{itemize}

\hi{The Wave behavior of Matter}
    \hii{Wave characteristics of particles and matter}
        \par De Broglie suggested that as the electron moves about the nucleus, it is associated
            with a particular wavelength. He went on to propose that the characteristic
            wavelength of the electron, or \impt{of any matter}, depends on its mass, $m$,
            and on its velocity, $v$:
        \begin{equation}
            \lambda = \dfrac{h}{p} = \dfrac{h}{mv}
        \end{equation}
        \par in which:
        \begin{itemize}
            \item $\lambda$: the wavelength of the particle
            \item $h$: Planck's constant
            \item $p$: the momentum of the particle
        \end{itemize}

    \hii{The Uncertainty Principle}
        \par Werner Heisenberg proposed that the dual nature of matter places a fundamental
        limitation on how precisely we can know both the location and the momentum of any object.
        The limitation becomes important only when we deal with matter at the subatomic level.
        \par Heisenberg's principle is called the \impt{Uncertainty Principle}. When applied
        to the elctrons in an atom, this Principle states that it is inherently impossible to
        determine simultaneous both the exact momentum of the electron and its exact location
        in space.
        \par Mathematical relation between the uncertainty of the position, $\Delta x$, and the
        uncertainty in momentum, $\Delta p$:
        \begin{equation} \label{eq:03}
            \Delta x \mul \Delta p \geq \dfrac{h}{4\pi}
        \end{equation}
        \par Example: \textit{An electron has a mass of $9.11 \mul 10^{-31} kg$ and moves at and
        average speed of about $5 \mul 10^{6} m/s$ in a hydrogen atom. Assume that we know the
        speed to an uncertainty of $1\%$. Find the uncertainty of the position}.
        \par Apply the equation \eqref{eq:03}:
        \begin{equation}
            \begin{aligned}
                \Delta x & \geq \dfrac{h}{4 \Delta p \pi} \\
                & = \dfrac{h}{4m \Delta v \pi} \\
                & = \dfrac{6.626 \mul 10^{-34} Js}{4 \pi(9.11 \mul 10^{-31} kg)(5 \mul 10^{4} m/s)}
                & = 1 \mul 10^{-9} m
            \end{aligned}
        \end{equation}
        \par The size of an electron is about $1 \mul 10^{-10} m$. Thus, the uncertainty of the
        position is too high.
        \par Apply the same equation to a real-life object like a tennis ball, we can see that
        the uncertainty would be so small that it would be inconsequential. As a result, the
        principle has no practical consequence.
        \par The De Broglie's hypothesis and Heisenberg's uncertainty principle set the stage for
        a new model of atom that precisely describes the energy of the elctron while describing
        its location not precisely, but in terms of probabilities.

\hi{Quantum Mechanics and Atomic Orbitals}

    \hii{Schrodinger's Wave Equation}
        \par Erwin Schrodinger proposed an equation which incorporates both the wavelike behavior
        and the particle-like behavior of the electron.
        \par Schrodinger treated the electron as a standing circular wave around the nucleus.
        Just as the plucked guitar string produces a fundamental frequency and higher overtones
        (harmonics), there is a lowest-energy standing wave, and higher energy ones, for an
        electron in an atom. Solving Schrodinger's equation leads to a series of mathematical
        functions called \impt{wave functions} that describes the electron in an atom.
        These functions are often denoted with the symbol $\psi$. The square of these functions,
        $\psi^2$, provides information about an electron's location when the elctron is in an
        allowed energy state.
        \par In the quantum mechanical model, according to the uncertainty principle, the exact
        location of an electron cannot be determined. Rather, we determine the \textbf{probability}
        that the electron will be in a certain region of space at a given instant. The square of
        the wave function, $\psi^2$, at a given point in space represents the probability that
        the electron will be found at that location. For this reason, $\psi^2$ is called either
        the \impt{probability density} or the \impt{elecetron density}.

    \hiiBEGIN{Orbitals and Quantum Numbers}
        \hiii{Orbitals}
            \par The solution to Schrodinger's equation for the hydrogen atom yields a set of wave
            functions and corresponding energies. These wave functions are called \impt{orbitals}.
            Each orbital describes a specific distribution of electron density in space, as given
            by the orbital's probability density. Each orbital, therefore, has a
            \textit{characteristic energy and shape}.

        \hiii{Quantum Numbers}
            \par The quantum mechanical model uses \impt{three quantum numbers}: $n$, $l$,
            and $m_{l}$ to describe an orbital:
            \begin{enumerate}
                \item $n$ - the \impt{principle quantum number}:
                    \begin{itemize}
                        \item Values: 1, 2, 3, \ldots
                        \item Properties: \\ As $n$ increases:
                        \begin{itemize}
                            \item The orbital becomes larger.
                            \item The electron spends more time farther from the nucleus.
                            \item The electron has a higher energy and is therefore less tightly bound
                                to the nucleus.
                        \end{itemize}
                   \end{itemize}
                \item $l$ - the \impt{angular momentum quantum number}:
                    \begin{itemize}
                        \item Values: $0, 1, 2, \ldots, n - 1$ for each value of $n$
                        \item Properties:
                            \begin{itemize}
                                \item $l$ defines the shape of the orbital.
                                \item $l$ is generally designated by the letters $s$, $p$, $d$, $f$,
                                    corresponding to the values of $0, 1, 2, 3$, respectively.
                            \end{itemize}
                    \end{itemize}
                    \tableBEGIN{|c|c|c|c|c|}
                        \hline
                        Value of $l$ & 0 & 1 & 2 & 3 \\
                        \hline
                        Letter & $s$ & $p$ & $d$ & $f$ \\
                        \hline
                    \tableEND
                \item $m_{l}$ - the \impt{magnetic quantum number}:
                    \begin{itemize}
                        \item Values: $-l, -l + 1, \ldots, 0, \ldots, l - 1, l$ for each value of $l$
                        \item Properties: $m_{l}$ describes the orientation of the orbital in space.
                    \end{itemize}
            \end{enumerate}

        \hiii{Shell and subshell}
            \begin{itemize}
                \item The collection of orbitals with the same value of $n$ is called an
                    \impt{electorn shell}. Each shell is designated by a number ($n$).
                \item The set of orbitals that have the same $n$ and $l$ values is called a
                    \impt{subshell}. Each subshell is designated by a number ($n$) and a letter
                    (which corresponding to $l$).
            \end{itemize}
    \hiiEND

    \hiiBEGIN{Representations of Orbitals}
        \hiii{The $s$ Orbitals}
            \par The shape of the $s$ orbitals is \impt{spherically symmetric}.
        \hiii{The $p$ Orbitals}
            \par In the $p$ orbitals, the electron density is concentrated in two regions on
            eithter side of the nucleus, separated by a node at the nucleus. We say that this
            \impt{dumbbell-shaped} orbital has two \impt{lobes}.
            \par Beginning with the $n = 2$ shell, each shell has 3 $p$ orbitals, all of which have
            the same size and shape but differ from one another in spatial orientation. These
            orbitals are label $p_{x}$, $p_{y}$ and $p_{z}$ according to the Cartesian axis that
            each lies along.
        \hiii{The $d$ and $f$ Orbitals}
            \par Beginning with the $n = 3$ shell, each shell has 5 $d$ orbitals.
            \begin{itemize}
                \item Four of the $d$-orbital contour representations have a "four-leaf clover"
                    shape, and each lies primarily in a plane.
                    \begin{itemize}
                        \item The $d_{xy}, d_{yz}, d_{yz}$ lie in the $xy$, $xz$ and $yz$ planes,
                            respectively, with the lobes oriented between the axes.
                        \item The lobes of the $d_{x^2 - y^2}$ orbital also lie in the $xy$ plane,
                            but the lobes lie along the $x$ and $y$ axes.
                    \end{itemize}
                \item The $d_{z^2}$ orbital looks very different from the other four: it has two
                    lobes along the $z$ axis and a ``doughnut" in the $xy$ plane. Although it looks
                    different, it has the same energy as the other four $d$ orbitals.
            \end{itemize}
            \par Beginning with the $n = 4$ shell, each shell has 7 $f$ orbitals. The shapes of these
            orbitals are very complicated.
    \hiiEND

\hi{Multiple-electron Atoms}
    \hii{Orbitals and Their Energies}
        \par The electronic structure of multiple-electron atoms can be described in terms of
        orbitals.
        \par In a multiple-electron atom, for a given value of $n$, the energy of an orbital
        increases with increasing value of $l$. Orbitals with the same energy are said to be
        \impt{degenerate}.

    \hiiBEGIN{Electron Spin and the Pauli Exclusion Principle}
        \hiii{Electron Spin}
            \par A puzzling feature of the light spectra of multiple-electron atoms is that: lines
            that were originally thought to be single were actually closely spaced pairs. This meant
            that there were twice as many energy levels as there were supposed to be.
            \par It has been postulated that electrons have an instrinsic property, called
            \impt{electron spin}, that causes each electron to behave as if it were a tiny sphere
            spinning on its own axis.
            \par A new quantum number was assigned: $m_{s}$ - \impt{the spin magnetic quantum number}:
                \begin{itemize}
                    \item Values: $+\dfrac{1}{2}$ or $-\dfrac{1}{2}$
                    \item Properties: The two values indicate the two opposite directions of spin,
                        therefore produce oppositely directed magnetic field. The opposite magnetic
                        fields lead to the splitting of spectral lines into closely spaced pairs.
                \end{itemize}
        \hiii{Pauli Exclusion Principle}
            \par \impt{Principle}: \textit{No two electrons in an atom can have the same set of four
            quantum numbers $n$, $l$, $m_{l}$ and $m_{s}$.}
            \par Result of the principle: an orbital can hold a maximum of two electrons and they
            must have opposite spins.
    \hiiEND

\hi{Electron Configurations}
    \hii{Electron Configurations}
        \par The way in which the electrons are distributed among the various orbitals of an atom is
        called the \impt{electron configuration} of the atom.
        \par The configuration can be represented by writing the symbol for the occupied subshell
        and adding a superscript to indicate the number of electrons in that subshell:
        \begin{center}
            Li: $1s^{2} 2 s^{1}$
        \end{center}
        \par It can also be presented using the \impt{orbital diagram}:
        \begin{center}
            Li: \forb \quad \horb
        \end{center}
        \par In an orbital diagram, a half arrow pointing up represents an electron with a positive
        spin magnetic quantum number $(\,m_{s} = +\dfrac{1}{2})\,$ and a half arrow pointing down
        represents an electron with a negative spin magnetic quantum number
        $(\,m_{s} = +\dfrac{1}{2})\,$.
        \par Electrons having opposite spins are said to be \impt{paired} when they are in the same
        orbital. An \impt{unpaired electron} is one not accompanied by a partner of opposite spin.

    \hii{Hund's Rule}
        \par \impt{Hule's Rule}: For degenerate orbitals, the lowest energy is attained when the
        number of electrons with the same spin is maximized.
        \par This means that electrons will occupy orbitals singly to the maximum extent possible
        and that these single electrons in a given subshell will all have the same spin magnetic
        quantum number. Electron arranged in this way are said to have \impt{parallel spins}.
        \par Hund's rule is based in part on the fact that electrons repel one another. By occupying
        different orbitals, the electron remains as far as possible from another, thus minimizing
        electron-electron repulsions.
        \par Note: \textit{For the first electron which occupies an orbital, the choice of a spin-up
        or spin-down electron is arbitrary. It is customary to show unpaired electrons with their
        spins up}.

    \hii{Condensed Electron Configurations}
        \par In writing the \impt{condensed electron configuration} of an element, the electron
        configuration of the nearest noble-gas element of lower atomic number is represented by
        its chemical symbol in brackets. For example: Li: [He]$2s^{1}$
        \begin{itemize}
            \item We refer to the electrons represented by the symbol for a noble gas as the
                \impt{core electrons}.
            \item The inner-shell electrons are referred to as the \impt{core electrons}.
            \item The electrons given after the noble-gas core are called the \impt{outer-shell
                electrons}.
            \item The outer-shell electrons include the electrons involved in chemical bonding,
                which are called the \impt{valence electrons}.
        \end{itemize}

    \hii{Transition Metals}
        
\hi{Electron Configurations and The Periodic Table}

\chapter{Periodic Properties of the Elements}

\hi{Effective Nuclear Charge}
    \hii{The net attraction on a single atom}
        \par In a multiple-electron atom, each electron is simultaneously:
        \begin{itemize}
            \item attracted to the nucleus.
            \item repelled by the other electrons.
        \end{itemize}
        \par We can view the net electric field as if it results from a single positive charge located
        at the nucleus, called the \impt{effective nuclear charge}, $Z_{eff}$.
        \begin{equation}
            Z_{eff} = Z - S
        \end{equation}
        in which:
        \begin{itemize}
            \item $Z$: number of electrons in the nucleus
            \item $S$: the screening (or shielding) constant. 
        \end{itemize}
        \par The value of S is usually close to the number of core electrons in an atom.

    \hii{Periodic trend in Effective Nuclear Charge}
        \par \textit{The effective nuclear charge \textbf{increases} as we move \textbf{across the
        row} (or \textbf{period}) from left to right.}
        \par This is because the number of core electrons stay the same, while the electric charge of
        the atom increases.
        \par \textit{\textbf{Going down} a column, the effective nuclear charge experienced by valence
        electrons \textbf{increases slightly}, but \textbf{far less} than it does across a row.}
        \par The increase is because the larger the electron cores, the less effective is the
        shielding effect.

\hi{Sizes of Atoms and Ions}
    \hii{Periodic trend in Atomic Radii}
        \par \textit{Within each column (group), atomic radius \textbf{increase from top to bottom}.}
        \par As we go down a column, the outer electrons have a greater probability of being
        farther from the nucleus, causing the atom to increase in size.
        \par \textit{Within each row (period), atomic radius \textbf{decrease} from left to right.}
        \par This is because of the change in the \impt{effective nuclear charge}.

    \hii{Periodic trend in Ionic Radii}
        \par \impt{Cations} are \impt{smaller} than their parent atoms.
        \par \impt{Ations} are \impt{bigger} than their parent atoms.
        \par An \impt{isoelectronic series} is a group of ions with the same number of electrons.
        In an isoelectronic series, the radius \impt{decreases} with \impt{increasing nuclear
        charge}.

\hi{Ionization Energy}
    \hiiBEGIN{Definition of Ionization Energy}
        \hiii{Definition}
        \par The \impt{ionization energy} of an atom or ion is the minimum energy required to remove
        an electron from the ground state of the isolated gaseous atom or ion.
        \par The \textit{first ionization energy}, $I_{1}$, is the energy needed to remove the first
        electron from a neutral atom.
        \begin{center}
            \ce{Na (g) -> Na+ (g) + e-}
        \end{center}
        \par The \textit{second ionization energy}, $I_{2}$, is the energy needed to remove the second
        electron.
        \begin{center}
            \ce{Na+ (g) -> Na^2+ (g) + e-}
        \end{center}
        \hiii{Variations in Successive Ionization Energies}
            \begin{align*}
                I_{1} < I_{2} < I_{3}
            \end{align*}
            \par With each successive removal, an electron is being pulled away from an increasingly
            more positive ion, requiring increasingly more energy.
            \par The ionization energies required for core electrons are much larger than for valence
            electrons.
    \hiiEND
    \hii{Periodic trend in First Ionization Energies}
        \par \textit{Within each \textbf{row}, or \textbf{period}, of the table, $I_{1}$
        \textbf{increases} with increasing atomic number.}
        \par \textit{Within each \textbf{column}, or \textbf{group}, of the table, $I_{1}$
        \textbf{decreases} with increasing atomic number.}
        \par \impt{Irregularities}: Within a row, $I_{1}$ of group IIIA is smaller than group IIA,
        and $I_{1}$ of group VIA is smaller than group VA.
    \hii{Electron Configurations of Ions}
        \par When electrons are removed from an atom to form a cation, they are always removed first
        from the occupied orbitals having the largest principle quantum number, $n$.

\hi{Electron Affinities}
    \hiiBEGIN{Definition of Electron Affinity}
        \hiii{Definition}
            \par Electron affinity is the energy change that occurs when an electron is added
            to a gaseous atom.
        \hiii{Electron Affinity and Ionization Energy}
            \par Ionization energy measures the ease with which an atom \impt{loses} an electron.
            \par Electron Affinity measures the ease with which an atom \impt{gains} an electron.
        \hiii{Sign convention}
            \par For any reaction that \impt{releases} energy, the change $\Dt E$ in total energy
            has a negative value and the reaction is called an exothermic process.
            \begin{equation}
                E_{ea} = - \Dt E
            \end{equation}
    \hiiEND
    \hii{Periodic trend in Electron Affinity}
        \par Similar to the trend in Ionization Energy.

\hi{Metals, Nonmetals, and Metalloids}

\hi{Group Trends}

\chapter{Basic Concepts of Chemical Bonding}

\hi{Chemical Bonds, Lewis Symbols, and the Octet Rule}
    \hiiBEGIN{Type of bonds}
        \hiii{Ionic bond}
            \begin{itemize}
                \item Cause: electrostatic forces between ions of opposite charge.
                \item Interaction: metals with nonmetals.
            \end{itemize}

        \hiii{Covalent bond}
            \begin{itemize}
                \item Cause: the sharing of electrons between two atoms.
                \item Interaction: nonmetals with nonmetals
            \end{itemize}

        \hiii{Metallic bond}
            \begin{itemize}
                \item Cause: the sharing of electrons between two atoms.
                \item Interaction: metals with metals
            \end{itemize}
    \hiiEND

    \hii{Lewis Symbols}
        \par The \impt{Lewis symbol} of an element consists of:
        \begin{itemize}
            \item The chemical symbol for the element.
            \item A dot for each valence electron. The dots are placed on the four sides of the
                atomic symbol: the top, the bottom, the left and the right sides. Each side
                can accommodate up to two electrons.
        \end{itemize}

    \hii{The Octet Rule}
        \par \impt{Atoms tend to gain, lose, or share electrons until they are surrounded by
        eight valence electrons.}

\hi{Ionic Bonding}
    
\hi{Covalent Bonding}

\hi{Bond Polarity and Electronegativity}
    \hii{Bond Polarity}
        \par The concept of \impt{bond polarity} help describe the sharing of electrons between
        atoms.
        \begin{itemize}
            \item A \impt{nonpolar covalent bond} is one in which the electrons are shared equally
                between two atoms.
            \item A \impt{polar covalent bond} is one in which the electron is more greatly
                attracted by one of the two atoms.
        \end{itemize}

    \hii{Electronegativity}
        \par We use a quantity called \impt{electronegativity} to estimate whether a given bond
        is nonpolar covalent, polar covalent or ionic.
        \par \impt{Electronegativity} is defined as the ability of an atom in a molecule to attract
        electrons to itself.
        \par Electronegativity is closely related to ionization energy and electron affinity. An
        atom with a very negative elctron affinity and high ionization energy will both attract
        electrons from other atoms and resist having its electrons attracted away; it will be
        highly \impt{electronegative}.

    \hii{Relationship between Bond Polarity and Electronegativity}
        \par The greater the difference in electronegativity between two atoms, the more polar
        their bond.

\chapter{Molecular Geometry and Bonding Theories}

\hi{Molecular Shapes}
    \par The overall shape of a molecule is determined by its \impt{bond angle}, the angles made by
    the lines joining the nuclei of the atoms in the molecule.
    \par The bond angles, together with the bond lengths, accurately define the shape and size
    of the molecule.
    \par For the molecules with the general formula $AB_{n}$, in which the central atom A is
    bonded to $n$ B atoms, the possible shapes of $AB_{n}$ depend on the value of $n$.
    \begin{itemize}
        \item If $n = 2$: shape is linear (bond angle $= 180 \degree$) or bent
            (bond angle $= 180 \degree$).
        \item If $n = 3$: shape is:
            \begin{itemize}
                \item trigonal planar, which means the A atom lies in the same plane as the B atoms.
                \item trigonal pyramidal, which means the A atom lies above the plane of the B atoms.
            \end{itemize}
        \item If $n = 4$: shape is octahedral.
    \end{itemize}
    \par The shape of these $AB_{n}$ molecules can be predicted using the \impt{valence-shell
    electron-pair repulsion (VSEPR) model}.

\hi{The VSEPR Model}
    \hii{Electron domains}
        \par A \impt{bonding pair} of electrons defines a region in which the electrons will
        most likely be found.
        \par A \impt{nonbonding pair}, or \impt{lone pair} of electron defines a region of electron
        that is located principally on one atom.
        \par Those regions are called the \impt{electron domains}.
        \begin{center}
            \begin{figure}[h]
                    \begin{center}
                        \includegraphics{eDomain.png}
                    \end{center}
            \end{figure}
        \end{center}
        \par For example, \ce{NH3} has four electron domains: three bonding pairs and one nonbonding
        pair.
        \par Each \impt{multiple bond} in a molecule constitutes \impt{a single electron domain}.
        \par For example, \ce{O3} has three electron domains around the central oxygen atom: a
        single bond, a double bond, a nonbonding pair.
        \begin{center}
            \begin{figure}[h]
                    \begin{center}
                        \includegraphics{eDomain02.png}
                    \end{center}
            \end{figure}
        \end{center}
    \hii{The idea of the VSEPR model}
        \par The VSEPR model is based on the idea that electron domains are negatively charged and
        therefore repel one another.
        \par \impt{The best arrangement of a given number of electron domains is the one that
        minimizes the repulsions among them}.
        \par \impt{The shapes of different $AB_{n}$ molecules or ions depend on the number of electron
        domains surrounding the central A atom}.
    \hii{Electron-domain geometry and molecular geometry}
        \par The arrangement of electron domains about the central atom of an $AB_{n}$ molecule or ion
        is called its \textbf{electron-domain geometry}.
        \par In contrast, the \impt{molecular geometry} is the arrangement of only the atoms in a
        molecule or ion - any nonbonding pairs are not part of the description of the molecular
        geometry.
        \par In the VSEPR model, we predict the electron-domain geometry. From knowing how many
        domains are due to nonbonding pairs, we can then predict the molecular geometry of a
        molecule or ion from its electron-domain geometry.
        \par When all the electron domains in a molecule arise from bonds, the molecular geometry is
        identical to the electron-domain geometry. However, when one or more of the domains involve
        nonbonding pairs of electrons, we must remember to ignore those domains when predicting
        molecular shape.
        \par Example: the electron-domain geometry of \ce{NH3} is tetrahedral. Due to a nonbonding
        pair, the molecular geometry of \ce{NH3} is trigonal pyramidal.
    \hii{Predicting the mocular shape}
        \par To predict the shape of molecules or ions, we follow these steps:
        \begin{itemize}
            \item Draw the \impt{Lewis structure}, count the total number of electron domain.
                Each nonbonding electron pair, each single bond, each double bond, and each triple
                bond counts as an electron domain.
            \item Determine the \impt{electron-domain geometry} so that repulsions is minimized.
            \item Determine the \impt{molecular geometry}.
        \end{itemize}
        \begin{center}
            \begin{figure}[h]
                    \begin{center}
                        \includegraphics{moleShapeTable.png}
                    \end{center}
            \end{figure}
        \end{center}
    \hii{Electron Domain Formula}
        \begin{align}
            \mbox{\# domains} &=  \mbox{\# lone pairs} + \mbox{\# bonding pairs} \\
            &= \lceil \dfrac{\mbox{\# valence e} - \mbox{\# bonding e}}{2} \rceil + \mbox{\# bonding pairs}
        \end{align}
        \par We use the ceil function because 1 single electron occupies one elctron domain.
    \hii{The Effect of Nonbonding Electrons and Multiple Bonds on Bond Angles}
        \par The electron domains for \impt{nonbonding electron pairs} exert greater repulsive
        forces on adjacent electron domains and tend to \impt{compress} the bond angle.
        \par Electron domains for multiple bonds exert a greater repulsive force on adjacent
        electron domains than do electron domains for single bonds.
    \hii{Five basic electron-domain geometries predicted by the VSEPR model}
        \par
        \begin{center}
            \begin{figure}[h]
                    \begin{center}
                        \includegraphics[width=5in]{VSEPR.jpg}
                    \end{center}
            \end{figure}
        \end{center}


\hi{Molecular Shape and Molecular Polarity}
    \par A molecule is polar if its central atom has nonbonding electron pairs.

\hi{Covalent Bonding and Orbital Overlap}
    \par The bonding of atoms can be explained by the \impt{valence-bond} theory.
    \par In the Lewis theory, covalent bonding occurs when atoms share electrons, which concentrates
    electron density between the nuclei.
    \par In the \impt{valence-bond} theory, bonding occurs when the orbitals of two nuclei
    \impt{overlaps} with each other. The overlap of orbitals allows two electrons of opposite spin
    to share the space between two atoms, forming the covalent bond.
    \par More concisely, the overlap will occur between the \impt{valence orbital} of each atom.

\hi{Covalent Bonding and Orbital Overlap}

\hi{Hybridization}
    \hii{Relationship between number of groups and type of hybridization}
        \par A group can either be:
        \begin{itemize}
            \item A lone pair
            \item A bond (single or multiple)
        \end{itemize}
        \begin{center}
            \begin{tabular}{|c|c|}
                \hline
                Number of groups & Type of hybridization \\
                \hline
                1 & $s$ \\
                \hline
                2 & $sp$ \\
                \hline
                3 & $sp^{2}$ \\
                \hline
                4 & $sp^{3}$ \\
                \hline
                5 & $d^{1}sp^{3}$ \\
                \hline
                6 & $d^{2}sp^{3}$ \\
                \hline
            \end{tabular}
        \end{center}

\chapter{My Notes}

\hi{Periodic table}
    \hii{Terminology}
        \begin{itemize}
            \item The \impt{columns} are also called the \impt{groups}.
            \item The \impt{rows} are also called the \impt{periods}.
        \end{itemize}

    \hii{Determine the group}
        \par To determine whether an element belongs to group A or B, we depend on the subshell with
        highest energy level:
        \begin{itemize}
            \item s or p: group A
            \item d of f: group B
        \end{itemize}

\hi{Electrons}
    \hii{Core \& Valence electrons}
    \begin{itemize}
        \item \impt{Valence electrons}: the electrons that are located on the outershell of an atom
            and that can participate in the formation of a chemical bond.
        \item \impt{Core electrons}: the electrons that are not valence electrons and therefore
            do not participate in bonding.
    \end{itemize}

\chapter{Thermochemistry}

\hi{The nature of energy}
    \hiiBEGIN{System and Surroundings}
        \hiii{System}
            \par In the universe, the portion we single out to study is called the \impt{system}.
        \hiii{Surroundings}
            \par Everything other than the system in the universe is the \impt{surroundings}.
    \hiiEND
    
    \hiiBEGIN{Types of systems}
        \hiii{Open systems}
            \par An \impt{open system} is one in which matter and energy can be exchanged with the
            surroundings.
        \hiii{Closed systems}
            \par A \impt{closed system} can exchanged energy but not matter with it surroundings.
        \hiii{Isolated systems}
            \par An \impt{isolated system} is one in which neither energy nor matter can be
            exchanged with the surroundings.
    \hiiEND
        
\hi{The First Law of Thermodynamic}
    \hii{Internal Energy}
        \par The \impt{internal energy} of a system is the sum of all the kinetic and potential
        energies of all its components.
        \par The change in internal energy:
        \begin{eqbox}
            \Dt E = E_{final} - E_{initial}
        \end{eqbox}
        \begin{itemize}
            \item $\Dt E > 0$: internal energy increases
            \item $\Dt E < 0$: internal energy decreases
        \end{itemize}

    \hii{First Law of Thermodynamic}
        \par The change in the internal energy $\Dt U$ of a closed system is equal to the amount
        of heat $Q$ supplied to the system, plus the work $W$ done on the system.
        \begin{equation}
            \Dt U = P + Q
        \end{equation}

    \hii{Endothermic and Exothermic Processes}
        \par When the system \impt{absorbs} energy, we call the process \impt{endothermic}.
        \par When the system \impt{releases} energy, we call the process \impt{exothermic}.

    \hii{State Functions}
        \par The value of a \impt{state function} depends only on the present state of the system,
        not on the path the system took to reach that state.

    \hii{Enthalpy}
        \par A thermodynamic function called \impt{enthalpy} accounts for heat flow in processes
        occuring at constant pressure when no forms of work are performed other than P-V work.
        \begin{eqbox}
            H = E + PV
        \end{eqbox}
        where
        \begin{itemize}
            \item $H$: enthalpy
            \item $E$: internal energy
            \item $P$: pressure
            \item $V$: volume
        \end{itemize}
        \par Normally, the product $PV$ can be ignored, and the enthalpy can be considered to be
        approximately equals to the bond energy. (MIT OCW)
        \par Enthalpy is a state function.

\hi{Enthalpies of Reaction}
    \par \impt{Enthalpy} of reaction is also known as \impt{heat} of reaction.
    \begin{eqbox}
        \Dt H = H_{product} - H_{reactants}
    \end{eqbox}
    \begin{itemize}
        \item $\Dt H > 0$: \impt{endothermic} - reactions absorbing energy.
        \item $\Dt H < 0$: \impt{exothermic} - reaction releasing energy.
    \end{itemize}

\hi{Calorimetry}
    \hii{Calorimetry and calorimeter}
        \par The measurement of heat flow is \impt{calorimetry}, and the device used
        to measure heat flow is a \impt{calorimeter}.
    \hii{Heat Capacity and Specific Heat}
        \par Specific heat formula:
        \begin{equation}
            Q = mc \Dt t
        \end{equation}

\hi{Hess's Law}
    \par If a reaction is carried out in a series of steps, $\Dt H$ for the overall
    reaction will equal the sum of the enthalpy changes for the individual steps.
    \begin{eqbox}
        \Dt H = \SUM{\Dt H_{i}}
    \end{eqbox}

\hi{Enthalpy of Formation}

\hi{Spontaneous Processes}
    \par A \impt{spontaneous process} is one that proceeds on its own without any outside
    assistance.
    \par Processes that are spontaneous in one direction are nonspontaneous in the opposite
    direction.
    
\hi{Reversible and Irreversible process}
    \hii{Definition}
        \par In a \impt{reversible process}, a system is changed in such a way that the system and
            surroundings can be restored to their original state by exactly reversing the change.
        \par An \impt{irreversible process} is one that cannot simply be reversed to restore the
            system and its surroundings to their original states.
        \par A reversible change produces the maximum amount of work that can be achieved by the
            system on the surrounding.
        \begin{eqbox}
            w_{rev} = w_{max}
        \end{eqbox}
    \hii{Irreversibility of Spontaneous Processes}
        \par Any spontaneous process is irreversible.


\hi{Entropy and The Second Law of Thermodynamics}
    \hii{Entropy}
        \par \impt{Entropy} is a measure of the \impt{disorder} of a system.

    \hii{Change in Entropy}
        \par $\Dt S$ is a state function, which means it only depends on the initial and final
            state of the system.
        \begin{eqbox}
            \Dt S_{sys} = S_{final} - S_{initial}
        \end{eqbox}
        \par Special case: in an \impt{isothermal process}, $\Dt S$ is equal to the heat that would
            be tranferred if the process were \impt{reversible}, $q_{rev}$, divided by the
            temperature at which the process occurs.
        \begin{eqbox} \label{eq:EntropyIsothermalProcess}
            \Dt S_{sys} = \frac{q_{rev}}{T} \mbox{ (constant T)}
        \end{eqbox}
        \par Because $S$ is a state function, the formula \eqref{eq:EntropyIsothermalProcess} can
            be used for \impt{any isothermal process}, not just those that are reversible.
            In case the change is irreversible, we calculate $\Dt S$ by using a reversible path
            between the states.
        \begin{itemize}
            \item $\Dt S > 0$: increase in disorder
            \item $\Dt S < 0$: decrease in disorder
        \end{itemize}

    \hii{Comparing Entropy}
        \begin{itemize}
            \item \textbf{Based on state}: $S_{gas} > S_{liquid} > S_{solid}$
            \item \textbf{Based on molar mass of gas}: gas with higher molar mass has higher 
                entropy.
        \end{itemize}

    \hii{Second Law of Thermodynamics}
        \par The total entropy of the universe increases in any spontaneous process.
        \par In other words, any spontaneous change is accompanied by an overall increase in
            entropy.

    \hii{Entropy Changes in Chemical Reactions}
        \begin{eqbox}
            \Dt S \degree = \SUM{nS\degree \mbox{(products)}} - \SUM{mS\degree \mbox{(reactants)}}
        \end{eqbox}
        where $n$ and $m$ are coefficients of the chemical reaction.

    \hii{Entropy Changes in the Surroundings}
        \par The change in entropy of the surroundings will depend on how much heat is absorbed
            or given off by the system.
        \begin{eqbox}
            \Dt S_{surr} = \frac{-q_{sys}}{T}
        \end{eqbox}
        \par For a reaction occuring at constant pressure, $q_{sys}$ is simply the enthalpy change
        for the reaction.
        \begin{eqbox}
            \Dt S_{surr} = \frac{-\Dt H_{sys}}{T}
        \end{eqbox}

\hi{Gibbs Free Energy}
    \hii{Definition}
        \par \impt{Gibbs Free Energy} is a state function which is used to predict the spontaneity
            of a reaction.
        \begin{eqbox}
            G = H - TS
        \end{eqbox}
        \par For a process occurring at constant temperature (isothermal process), the change
            in free energy of the system is:
        \begin{eqbox}
            \Dt G = \Dt H - T \Dt S
        \end{eqbox}
        \par Under standard condition:
        \begin{eqbox}
            \Dt G \degree = \Dt H \degree - T \Dt S \degree
        \end{eqbox}
    \hii{The origin of $G$}
        \par For a reaction occurring at constant temperature and pressure:
        \begin{align*}
            \Dt S_{univ} = \Dt S_{sys} + \Dt S_{surr}
            = \Dt S_{sys} + \big(\frac{-\Dt H_{sys}}{T}\big)
        \end{align*}
        \par Multiplying both sides by $(-T)$ gives us:
        \begin{align*}
            -T \Dt S_{univ} = \Dt H_{sys} - T \Dt S_{sys}
        \end{align*}
        \par Substitute $G$ with $-T \Dt S_{univ}$ gives us the formula of Gibbs Free Energy.
        \par In a spontaneous process, $\Dt S_{univ} > 0$. Therfore,
            $-T \Dt S_{univ} = \Dt G < 0$.
    \hii{Sign of $G$}
        \begin{itemize}
            \item $\Dt G < 0$: the reaction is spontanous in the forward direction.
            \item $\Dt G = 0$: the reaction is at equilibrium.
            \item $\Dt G > 0$: the reaction is nonspontaneous; work must be supplied for the
                reaction to occur. Meanwhile, the reverse reaction is spontaneous.
        \end{itemize}
    \hii{Free Energy under Nonstandard Condition}
        \begin{eqbox}
            \Dt G = \Dt G \degree + RT \ln Q
        \end{eqbox}
        where
        \begin{itemize}
            \item $\Dt G$: change in free energy under nonstandard condition $[J]$
            \item $\Dt G \degree$: change in free energy under standard condition $[J]$
            \item $R$: ideal gas constant $[8.314 Jmol^{-1}K^{-1}]$
            \item $T$: temperature $[K]$
            \item $Q$: reaction quotient that corresponds to the particular reaction mixture of
                interest
        \end{itemize}

\chapter{Equilibrium}

\hi{Reaction Quotient}
    \begin{align*}
        aA + bB \Rightarrow cC + dD
    \end{align*}
    \begin{itemize}
        \item \textbf{In gaseous phase:}
            \begin{eqbox}
                Q_{P} = \frac{P_{C}^{c}P_{D}^{d}}{P_{A}^{a}P_{B}^{b}}
            \end{eqbox}
        \item \textbf{In solution:}
            \begin{eqbox}
                Q_{C} = \frac{[C]^{c}[D]^{d}}{[A]^{a}[B]^{b}}
            \end{eqbox}
        \item Common form:
            \begin{equation}
                Q = \frac{\mbox{product}}{\mbox{reactant}}
            \end{equation}
    \end{itemize}


\chapter{Solutions}

\hi{Terminology}
    \par \impt{Solvent}: the liquid in which a solute is dissolved to form a solution.
    \par \impt{Solute}: the minor component in a solution, dissolved in the solvent.

\hi{The Solution Process}

    \hii{The Effect of Intermocular Forces}
        \par The extent to which one substance is able to dissolve in another depends on the
            relative magnitudes of the solute-solvent, solute-solute, and solvent-solvent
            interactions involved in the solution process.

    \hii{Energy Changes and Solution Formation}
        \par There are three processes happen in the formation of a solution:
        \begin{itemize}
            \item breaking the solute - solute interaction
            \item breaking the solvent - solvent interaction
            \item forming the solute - solvent interaction
        \end{itemize}
        \par The overall enthalpy change in forming a solution, $\Dt H_{soln}$, is the sum of
            three terms associated with these three processes:
        \begin{eqbox}
            \Dt H_{soln} = \Dt H_{1} + \Dt H_{2} + \Dt H_{3}
        \end{eqbox}
        in which:
        \begin{itemize}
            \item $H_{1} > 0$
            \item $H_{2} > 0$
            \item $H_{3} < 0$
        \end{itemize}
        \par $\Dt H_{soln}$ can either be positive or negative depends on the intermolecular
            force.

\hi{Saturated Solutions and Solubility}
    \par As a solid solute begins to dissolve in a solvent, the concentration of solute
        particles in solution increases, thus increasing the chances of the solute particles
        colliding with the surface of the solid. Because of such a collision, the solute
        particle may become \textit{reattached} to the solid. This process is called
        \impt{crystallization}.
    \par There are two opposing processes occuring in a solution in contact with undissolve
        solute.
    \begin{center}
        \ce{Solute + Solvent <=>[dissolve][crystallize] Solution}
    \end{center}
    \par When the rates of these opposing processes become equal, there is no net change in
        the amount of solute in solution. The solution reaches \impt{equilibrium}.
    \par A solution that is in equilibrium with undissolved solute is \impt{saturated}.
        Additional solute will not dissolve if added to saturated solution.
    \par The \impt{solubility} is the maximum amount of solute that will dissolve in a given
        amount of solvent at a specified temperature, given that excess solute is present.
    \par A solution which contains less solute than that needed to form a saturated solution
        is \impt{unsaturated}.
    \par A solution which under suitable condition contains more solute than that needed to
        form a saturated solution is \impt{supersaturated}.

\hi{Factors affecting Solubility}
    
    \hii{Solute - Solvent Interactions}
        \par The stronger the attractions are between solute and solvent molecules, the greater
            the solubility.
        \par Substances with \impt{similar} intermolecular attractive forces tend to be soluble
            in one another.
        \par In other words, \textit{nonpolar} solutes tend to be insoluble in \textit{polar}
            solvents, and vice versa.

    \hii{Pressure Effects}
        \par The solubility of the gas increases in direct proportion to its partial pressure
            above the solution.
        \par \impt{Henry's law}:
        \begin{eqbox}
            S_{g} = kP_{g}
        \end{eqbox}
        where
        \begin{itemize}
            \item $S_{g}$: solubility of the gas in the solution phase (usually expressed as
                molarity).
            \item $P_{g}$: partial pressure of the gas over the solution
            \item $k$: Henry's law constant
        \end{itemize}
        \par The Henry's law constant is different for each solute-solvent pair and for
            different temperatures.

    \hii{Temperature Effects}
        \par The solubility of most \impt{solid solutes} in water \impt{increases} as the
            temperature of the solution increases.
        \par The solubility of \impt{gases} in water \impt{decreases} as the temperature of the
            solution increases.

\hi{Ways of expressing concentration}
    \hii{Mass Percentage, ppm and ppb}
        \begin{itemize}
            \item \impt{Mass percentage} ($C\%$):
                \begin{eqbox}
                    C\% = \frac{m_{\mbox{solute}}}{m_{\mbox{solute}} + m_{\mbox{solvent}}} \cdot 100
                    = \frac{m_{\mbox{solute}}}{m_{\mbox{solution}}} \cdot 100
                \end{eqbox}
            \item \impt{Parts per million} ($ppm$):
                \begin{eqbox}
                    ppm = \frac{m_{\mbox{solute}}}{m_{\mbox{solute}} + m_{\mbox{solvent}}} \cdot 10^6
                    = \frac{m_{\mbox{solute}}}{m_{\mbox{solution}}} \cdot 10^6
                \end{eqbox}
            \item \impt{Parts per billion} ($ppb$):
                \begin{eqbox}
                    ppm = \frac{m_{\mbox{solute}}}{m_{\mbox{solute}} + m_{\mbox{solvent}}} \cdot 10^9
                    = \frac{m_{\mbox{solute}}}{m_{\mbox{solution}}} \cdot 10^9
                \end{eqbox}
        \end{itemize}
    \hii{Mole Fraction, Molarity and Molality}
        \begin{itemize}
            \item \impt{Mole fraction} ($X$):
                \begin{eqbox}
                    X_{\mbox{solute}}
                    = \frac{n_{\mbox{solute}}}{n_{\mbox{solute}} + n_{\mbox{solvent}}}
                    = \frac{n_{\mbox{solute}}}{n_{\mbox{solution}}}
                \end{eqbox}
            \item \impt{Molarity} ($C_{M}$):
                \begin{eqbox}
                    C_{M} = \frac{n_{\mbox{solute}}}{V_{\mbox{solvent}}}
                \end{eqbox}
            \item \impt{Molality} ($C_{m}$):
                \begin{eqbox}
                    C_{m} = \frac{n_{\mbox{solute}}}{m_{\mbox{solvent}}}
                \end{eqbox}
                where:
                \begin{itemize}
                    \item $C_{m}$: molality $[mol/kg]$
                    \item $m_{\mbox{solvent}}$: mass of the solvent $[kg]$
                \end{itemize}
        \end{itemize}

\hi{Colligative Properties}
    \hii{Defition}
        \par Properties of solutions that depend on the \impt{quantity} (\impt{concentration}),
            not the kind or identity of the solute particles are called \impt{colligative
            properties}.

    \hiiBEGIN{Lowering the Vapor Pressure}
        \hiii{Vapor pressure}
            \par A liquid in a closed container will establish equilibrium with its vapor.
                When that equilibrium is reached, the pressure exerted by the vapor is
                called vapor pressure.
            \par A substance that has no measurable vapor pressure is \impt{nonvolatile}.
            \par A substance that exhibits a vapor pressure is \impt{volatile}.
        \hiii{Raoult's law}
            \begin{eqbox}
                P_{\mbox{solution}} = X_{\mbox{solute}} \cdot P_{\mbox{solvent}}
            \end{eqbox}
            where
            \begin{itemize}
                \item $P_{\mbox{solution}}$: vapor pressure of the solution
                \item $X_{\mbox{solute}}$: mole fraction of the solute
                \item $P_{\mbox{solvent}}$: vapor pressure of the solvent
            \end{itemize}
    \hiiEND

    \hii{Boiling-Point Elevation}
        \begin{eqbox}    
            \Dt T_{b} = K_{b} \cdot C_{m}
        \end{eqbox}    
        where
        \begin{itemize}
            \item $\Dt T_{b}$: change in boiling-point
            \item $K_{b}$: molal boiling-point-elevation constant, depending on the solvent
            \item $C_{m}$: molality
        \end{itemize}

    \hii{Freezing-Point Depression}
        \begin{eqbox}    
            \Dt T_{f} = K_{f} \cdot C_{m}
        \end{eqbox}    
        where
        \begin{itemize}
            \item $\Dt T_{f}$: change in freezing-point
            \item $K_{f}$: molal freezing-point-depression constant, depending on the solvent
            \item $C_{m}$: molality
        \end{itemize}

    \hii{Osmosis}
        \begin{eqbox}
            \pi = \frac{n}{V}RT = C_{M}RT
        \end{eqbox}
        where
        \begin{itemize}
            \item $R$: ideal-gas constant
        \end{itemize}
\end{document}
