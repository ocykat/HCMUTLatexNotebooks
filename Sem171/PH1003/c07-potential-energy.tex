\chapter{Potential Energy}
    \hi{Conservative and Nonconservative Forces}
        \par \impt{Conservative forces} have these two equivalent properties:
        \begin{itemize}
            \item The work done by a conservative force on a particle moving between any two
                points is independent of the path taken by the particle.
            \item The work done by a conservative force on a particle moving through any closed
                path is zero.
        \end{itemize}

   \hi{Gravitational Potential Energy}
        \begin{equation}
            U_{g} = mgy
        \end{equation}
        in which:
        \begin{itemize}
            \item $U_{g}$: gravitational potential energy $[J]$
            \item $m$: the mass of the object $[kg]$
            \item $g$: gravitational acceleration $[m/s^{2}]$
            \item $y$: the height of the object according to the reference configuration $[m]$
        \end{itemize}
        \par \impt{Note:} When dealing with problems related to gravitational energy, one
        must choose a reference configuration for which the gravitational potential energy
        is set equal to some value, which is normally zero. The choice of reference configuration
        is completely arbitrary because the important quantity is the \textit{difference} in
        potential energy and this difference is independent of the choice of reference configuration.

    \hi{Relationship between Work done and The Change in Gravitational Potential Energy}
        \begin{equation}
            W = \Delta U = U_{f} - U_{i}
        \end{equation}

    \hi{Isolated System}
        \par In an isolated system, the only type of forces allowed is \impt{conservative forces}.

    \hi{Mechanical Energy}
        \par \impt{Mechanical Energy} is the sum of $kinetic$ and $potential$ energies.
        \begin{equation}
            E_{mech} = K + U_{g}
        \end{equation}

    \hi{Conservation of Mechanical Energy}
        \par In an \impt{isolated system}:
        \begin{equation*}
            E_{i} = E_{f}
        \end{equation*}
        \begin{equation}
            \ra K_{i} + U_{i} = K_{f} + U_{f} 
        \end{equation}

    \hi{Elastic Potential Energy}
        \begin{equation}
            U_{s} = \dfrac{1}{2}kx^{2}
        \end{equation}