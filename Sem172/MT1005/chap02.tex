\chapter{Partial Derivatives}

\hi{Function of two variables}
  \hii{Definition}
    \par A function of two variables is a rule that assigns
    to each ordered pair of real numbers $(x, y)$ in a set a unique real number
    denoted by $f(x, y)$. The set $D$ is the \textbf{domain} of $f$ and its
    \textbf{range} is the set of values that $f$ takes on.
    \par We often write:
    \begin{eqbox}
      z = f(x, y)
    \end{eqbox}
    where:
    \begin{itemize}
      \item $x$, $y$: independent variables
      \item $z$: dependent variable
    \end{itemize}
  \hii{Domain}
    \par Given the function: $f(x, y)$. The domain of $f$ is the
    \textbf{Cartesian product} of the set $D_{x}$ and the set $D_{y}$, where
    $D_{x}$ and $D_{y}$ are the set of all possible values of $x$ and $y$,
    respectively. This is true for any multivariable function.
    \begin{eqbox}
      D = D_{x} \times D_{y} \subseteq \mathbb{R}^{2}
    \end{eqbox}
  \hii{Graph}
    \par If $f$ is a function of two variables with domain $D$, then the graph
    of $f$ is the set of all points $(x, y, z)$ in $\mathbb{R}^{3}$ such that
    $z = f(x, y)$ and $(x, y) \in D$.
  \hii{Level Curves}
    \par The \textbf{level curves} of a function $f$ of two variables are the curves
    with equations $f(x, y) = k$, where $k$ is a constant (in the range of $f$).
    \par The graph of level curves is called the \textbf{contour graph}.
    \par To construct a contour graph, the set of $k$ is required.

\newpage

\hi{Limits and Continuity}
  \hiiBEGIN{Limits}
    \hiii{Definition}
      \par Let $f$ be a function of two variables whose domain $D$ includes points
      arbitrarily close to $(a, b)$. Then we say that the \textbf{limit} of
      $f(x, y)$ as $(x, y)$ approaches $(a, b)$ is $L$ and we write
      \begin{eqbox}
        \lim_{(x, y) \to (a, b)} f(x, y) = L
      \end{eqbox}
      if for every number $\epsilon > 0$ there is a corresponding number
      $\delta > 0$ such that:
      \begin{center}
        if $(x, y) \in D$ and $0 < \sqrt{(x - a)^{2} - (y - b)^{2})} < \delta$, then
        $|f(x, y) - L| < \epsilon$
      \end{center}
    \hiii{Existence of limit}
      \par For a function of one variable
      \begin{center}
        If $\lim_{x \to a^{-}} f(x) \neq \lim_{x \to a^{+}}$, then
        $\lim_{x \to a} f(x)$ does not exists.
      \end{center}
      \par For a multivariable function, the limit at one point can be approaches
      from infinitely many directions. If there \textbf{exists two different paths}
      of approach along which the function $f(x, y)$ has different limits, then the
      limit at that point does not exists.
  \hiiEND
  \hii{Continuity}
    \par A function $f$ of two variables is called \textbf{continuous} at $(a, b)$ if:
    \begin{eqbox}
      \lim_{(x, y) \to (a, b)} f(x, y) = f(a, b)
    \end{eqbox}
    \par We say $f$ is continuous on $D$ if $f$ is continuous at every point $(a, b)$ in $D$.

\newpage

\hi{Partial Derivatives}
  \hii{Definition}
    \par Given the function $f(x, y)$.
    \par The \textbf{partial derivative} of $f$ with respect to $x$ at $(a, b)$, denoted by
    $f_{y}(a, b)$, is obtained by keeping $y$ fixed $(y = b)$ and finding the ordinary
    derivative at $a$ of the function $G(x) = f(x, b)$.
    \begin{eqbox}
      f_{x} (x, y) = \lim_{\Delta x \to 0} \frac{f(x + \Delta x, y) - f(x, y)}{\Delta x} \\
      f_{y} (x, y) = \lim_{\Delta y \to 0} \frac{f(x, y + \Delta y) - f(x, y)}{\Delta y}
    \end{eqbox}
  \hii{Notations}
    \begin{eqbox}
      f_{x} (x, y) = f_{x} = \pd{f}{x} = \pd{}{x} f(x, y) = \pd{z}{x} = D_{1}f = D_{x}f \\
      f_{y} (x, y) = f_{y} = \pd{f}{y} = \pd{}{y} f(x, y) = \pd{z}{y} = D_{2}f = D_{y}f
    \end{eqbox}
  \hii{Rule for Finding Partial Derivatives}
    \par Given the function $z = f(x, y)$.
    \begin{itemize}
      \item To find $f_{x}$, regard $y$ as a constant and differentiate $f(x, y)$
        with respect to $x$.
      \item To find $f_{y}$, regard $x$ as a constant and differentiate $f(x, y)$
        with respect to $y$.
    \end{itemize}
  \hii{Higher Derivatives}
    \begin{alignat*}{4}
      (f_{x})_{x} (x, y) = f_{xx} = f_{11}
        &= \pd{}{x}\bigg(\pd{f}{x}\bigg)
        &= \pd{^{2}f}{x^{2}}
        &= \pd{^{2}z}{x^{2}} \\
      (f_{x})_{y} (x, y) = f_{xy} = f_{12}
        &= \pd{}{y}\bigg(\pd{f}{x}\bigg)
        &= \pdd{^{2}f}{y}{x}
        &= \pdd{^{2}z}{y}{x} \\
      (f_{y})_{x} (x, y) = f_{yx} = f_{21}
        &= \pd{}{x}\bigg(\pd{f}{y}\bigg)
        &= \pdd{^{2}f}{x}{y}
        &= \pdd{^{2}z}{x}{y} \\
      (f_{y})_{y} (x, y) = f_{yy} = f_{22}
        &= \pd{}{y}\bigg(\pd{f}{y}\bigg)
        &= \pd{^{2}f}{y^{2}}
        &= \pd{^{2}z}{y^{2}}
    \end{alignat*}

\newpage

\hi{Tangent Plane and Linear Approximation}
  \hii{Differentials}
    \par For a differentiable function of two variables, $z = f(x, y)$, we define the
    differentials $dx$ and $dy$ to be independent variables. Then the differential $dz$,
    also called the total differntial, is defined by:
    \begin{eqbox}
      dz = f_{x}(x, y) dx + f_{y}(x, y) dy = \pd{f}{x} dx + \pd{f}{y} dy
    \end{eqbox}
    \par If we take $dx = \Delta x = x - a$ and $dy = \Delta y = y - b$,
    then the differential of $z$ is:
    \begin{alignat*}{2}
      dz &= f_{x} (x_{1}, y_{1}) \Delta x + f_{y} (x_{1}, y_{1}) \Delta y \\
      &= f_{x} (x_{1}, y_{1}) (x_{2} - x_{1}) + f_{y} (x_{1}, y_{1}) (y_{2} - y_{1})
    \end{alignat*}

\newpage

\hi{The Chain Rule}
  \hii{Case 1}
    \par Suppose that $z = f(x, y)$ is a differentiable function of $x$ and $y$, where
    $x = g(t)$ and $y = h(t)$ are both differentiable functions of $t$. Then $z$ is a
    differentiable function of $t$ and
    \begin{eqbox}
      \dif{z}{t} = \pd{z}{x} \dif{x}{t} + \pd{z}{y} \dif{y}{t}
    \end{eqbox}
  \hii{Case 2}
    \par Suppose that $z = f(x, y)$ is a differentiable function of $x$ and $y$, where
    $x = g(s, t)$ and $y = h(s, t)$ are both differentiable functions of $s$ and $t$.
    Then $z$ is a differentiable function of $s$ and $t$ and
    \begin{eqbox}
      \pd{z}{s} = \pd{z}{x} \dif{x}{s} + \pd{z}{y} \dif{y}{s} \\
      \pd{z}{t} = \pd{z}{x} \dif{x}{t} + \pd{z}{y} \dif{y}{t}
    \end{eqbox}

\newpage

\hi{Directional Derivatives and the Gradient Vector}
  \hiiBEGIN{Directional Derivatives}
    \hiii{Derivative in Unit Vector Direction}
      \par For a function $z = f(x, y)$, the partial derivatives:
      \begin{align*}
        f_{x} (x_{0}, y_{0})
          = \lim_{\Delta x \to 0}
              \frac{f({x_{0}} + \Delta x, y_{0}) - f(x_{0}, y_{0})}
                   {\Delta x} \\
        f_{y} (x_{0}, y_{0})
          = \lim_{\Delta y \to 0}
              \frac{f({x_{0}}, y_{0} + \Delta y) - f(x_{0}, y_{0})}
                   {\Delta y} \\
      \end{align*}
      represent the \tb{rates of change of $z$ in the $x$- and $y$- directions},
      or, the directions of the unit vectors $i$ and $j$.

    \hiii{Derivative in Arbitrary Direction}
      \par Let surface $S$ be the graph of the function $z = f(x, y)$.
      \par Let point $M(x_{0}, y_{0}, z_{0})$ (with $z_{0} = f(x_{0}, y_{0})$)
        be a point on the graph $S$.
      \par Suppose we want to obtain the rate of change of $z$ at $(x_{0}, y_{0})$ in
        the direction of an arbitrary unit vector $\vec{u} = \langle a, b \rangle$.
      \par The plane $P$ that passes through $M$ in the direction of $\vec{u}$
        intersects $S$ in a curve $C$.
      \par In the plane $P$, the slope of the tangent line $T$ to $C$ at the point $M$
        is the rate of change of $z$ in the direction of $\vec{u}$.
      \par Let $N(x, y, z)$ be another point on $C$. Let $M'$, $N'$ be the projections
        \footnotemark of $M$, $N$ onto the $xy$- plane. Then, $\vec{M'N'}$ is
        parallel to $\vec{u}$ and so:
          \footnotetext{The projection of (x, y, z) onto the $xy$-plane is the
          point $(x, y, 0)$}
        \[
          \vt{P'Q'} = h\vec{u} = \langle ha, hb \rangle
        \]
        for some arbitrary scalar $h$.
      \par Therefore:
        \[
          \begin{cases}
            x - x_{0} = ha \\
            y - y_{0} = hb
          \end{cases}
          \ra
          \begin{cases}
            x = x_{0} + ha \\
            y = y_{0} + hb \\
          \end{cases}
          \\
          \ra \frac{\dt z}{h} = \frac{z - z_{0}}{h}
             = \frac{f(x_{0} + ha, y_{0} + hb) - f(x_{0}, y_{0})}{h}
        \]
      \par If we take the limit as $h \to 0$, we obtain the \tb{rate of change}
        of $z$ in the direction of $\vec{u}$, which is called the \ti{directional
        derivative of $f$ in the direction of $\vec{u}$}.
      \par \tb{Definition}: The \tb{directional derivative} of $f$ at
        $(x_{0}, y_{0})$ in the direction of a unit vector $u = \langle a, b \rangle$
        is:
        \begin{eqbox}
          D_{u}f(x_{0}, y_{0})
            = \lim_{h \to 0}
              \frac{f(x_{0} + ha, y_{0} + hb) - f(x_{0}, y_{0})}{h}
        \end{eqbox}
        \ti{if this limit exists}.
  \hiiEND

\newpage

\hi{Maximum and Minimum Values}
  \hii{Definition}
    \par A function of two variables has a \textbf{local maximum} at $(a, b)$ if
    $f(x, y) \leq f(a, b)$ when $(x, y)$ is near $(a, b)$. The number $f(a, b)$ is
    called a \textbf{local maximum value}.
    \par A function of two variables has a \textbf{local minimum} at $(a, b)$ if
    $f(x, y) \geq f(a, b)$ when $(x, y)$ is near $(a, b)$. The number $f(a, b)$ is
    called a \textbf{local minimum value}.
    \par If the inequalities hold for all points $(x, y)$ in the domain of $f$, then $f$
    has an \textbf{absolute maximum} (or \textbf{absolute minimum}) at $(a, b)$.

  \hii{Theorem (Fermat theorem for multivariable function)}
    \par If $f$ has a local maximum or minimum at $(a, b)$ and the first-order partial
    derivatives of $f$ exist there, then $f_{x}(a, b) = 0$ and $f_{y} (a, b) = 0$.
    
  \hii{Second Derivatives Test}
    \par Suppose the second partial derivatives of $f$ are continuous on a disk with center
    $(a, b)$, and suppose that $f_{x}(a, b) = 0$ and $f_{y}(a, b) = 0$.
    \par Let:
    \begin{itemize}
      \item $A = f_{xx}(a, b)$
      \item $B = f_{xy}(a, b)$
      \item $C = f_{yy}(a, b)$
      \item $D = AC - B^{2}$
    \end{itemize}
    then
    \begin{itemize}
      \item $D > 0$ and $A > 0$, then $f(a, b)$ is a local minimum.
      \item $D > 0$ and $A < 0$, then $f(a, b)$ is a local maximum.
      \item $D < 0$, then $f(a, b)$ is not a local maximum or minimum. Instead, it
        is called a \textbf{saddle point}.
      \item $D = 0$, there is no conclusion can be made. $f(a, b)$ can either be a
        local maximum/minimum or a saddle point.
    \end{itemize}
    \par The formula of $D$ can be written in the form of the determinant:
    \begin{equation}
      D = 
      \begin{vmatrix}
        f_{xx} & f_{xy} \\
        f_{yx} & f_{yy}
      \end{vmatrix}
    \end{equation}

  \hii{Extreme Value Theorem}
    \par A \textbf{closed} set in $R^{2}$ is one that contains all its boundary
      points.
    \par \textit{Example:} $D = {(x, y) | x^{2} + y^{2} \leq 1}$
    \par The circle $x^{2} + y^{2} = 1$ is the set of all boundary points
    of $D$.
    \par A bounded set in $R^{2}$ is finite in extent and contained within
    some disk.
    \par \textbf{Extreme Value Theorem for Functions of Two Variables:}
    \par If $f$ is continuous on a closed, bounded set $D$ in $R^{2}$, then $f$
    attains an absolute maximum value $f(x_{1}, y_{1})$ and an absolute
    minimum value $f(x_{2}, y_{2})$ at some points $(x_{1}, y_{1})$ and
    $(x_{2}, y_{2})$ in $D$.
    \par To find the absolute maximum and minimum values of a continuous
    function $f$ on a closed, bounded set $D$:
    \begin{enumerate}
      \item Find the values of $f$ at the critical points of $f$ in $D$.
      \item Find the extreme values of $f$ on the boundary of $D$.
      \item The largest of the values from steps 1 and 2 is the absolute
      maximum value; the smallest of these values is the absolute minimum
      value.
    \end{enumerate}

\newpage

\hi{Lagrange Multipliers}
