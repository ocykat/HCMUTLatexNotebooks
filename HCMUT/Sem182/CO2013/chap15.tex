\chapter{Relational Database Design Algorithms and Further Dependencies}

\hi{Further Topics in Functional Dependencies: Inference Rules, Equivalence, and Minimal Cover}
  \hii{Inference Rules}
    \par Let $F$ be the set of functional dependencies that are specified on a relation schema $R$.

    \hiii{Inferred Functional Dependencies}
      \par An FD $X \to Y$ is \tb{inferred from} or \tb{implied by} a set of dependencies $F$ specified on $R$ if $X \to Y$ holds in \tb{every} legal relation state $r$ of $R$.

    \hiii{Closure of Functional Dependencies}
      \par The set of all dependencies that include $F$ as well as dependencies that can be inferred from $F$ is called the \tb{closure} of $F$ and is denoted by $F^{+}$.

    \hiii{Some Notations}
      \begin{itemize}
        \item $F |= X \to Y$: the FD $X \to Y$ is inferred from $F$
        \item $\{X, Y\} \to Z$ can be abbreviated to $XY \to Z$
        \item $\{X, Y, Z\} \to \{U, V\}$ can be abbreviated to $XYZ \to UV$
      \end{itemize}

    \hiii{Inference Rules}
      \par Aarmstrong's Axioms:
      \begin{itemize}
        \item \tb{IR1: Reflexive Rule}:
          $X \supseteq Y \Rightarrow X \to Y$
        \item \tb{IR2: Augmentation Rule}:
          $\{ X \to Y \} |= XZ \to YZ$
        \item \tb{IR3: Transitive Rule}:
          $\{ X \to Y, Y \to Z \} |= X \to Z$
      \end{itemize}
      \par Other rules:
      \begin{itemize}
        \item \tb{IR4: Decomposition Rule/Projective Rule}:
          $\{ X \to YZ \} |= X \to Y$
        \item \tb{IR5: Union Rule/Additive Rule}:
          $\{ X \to Y, X \to Z \} |= X \to YZ$
        \item \tb{IR6: Pseudotransitive Rule}:
          $\{ X \to Y, WY \to Z \} |= WX \to Z$
      \end{itemize}

    \hiii{Closure of Attributes under The FD Set}
      \par Let $X$ be a set of attributes of $R$. Then \tb{the closure of $X$ under $F$}, denoted by $X^{+}$, are all attributes that are functional determined by $X$ based on $F$.
      \par \tb{Algorithm}:
        \begin{algorithm}[H]
          \caption{Determine $X^{+}$ of $X \subseteq R$ under $F$}
          \begin{algorithmic}[1]
            \State $X^{+} := X$
            \While{\texttt{True}}
              \State $oldX^{+}$ := $X^{+}$
              \For{each $Y \to Z \in F$}
                \If{$X^{+} \supseteq Y$}
                  \State $X^{+}$ := $X^{+} \cup Z$
                \EndIf
              \EndFor
              \If{$oldX^{+} = X^{+}}$
                \State break
              \EndIf
            \EndWhile
          \end{algorithmic}
        \end{algorithm}

  \hii{Equivalence of Sets of Functional Dependencies}
      \hiii{Cover Relationship}
        \par A set of functional dependencies $F$ is said to cover another set of functional dependencies $E$ if every FD in $E$ is also in $F^+$ - meaning that every dependencies in $E$ can be \tb{inferred} from $F$.

      \hiii{Equivalent Set of FDs}
        \par Two sets of FDs $E$ and $F$ are \tb{equivalent} if $E^+ = F^+$.

  \hii{Minimal Sets of Functional Dependencies}
    \hiii{Extraneous Attribute}
      \par An attribute in a functional dependency is considered an extraneous attribute if we can remove it without changing the closure of the set of dependencies.

    \hiii{Canonical Form of Set of FDs}
      \par An FD is said to be in \tb{canonical form} if it only has a single attribute on the right-hand side.
      \par A set of FDs $F$ is said to be in \tb{canonical form} if for every FD in $F$, there is only a single attribute on the right-hand side.
      \par Example: If $X \to YZ \in F$, then $F$ is not in canonical form.

    \hiii{Minimal Set of FDs}
      \par A set of FDs $F$ is said to be \tb{minimal} if it satisfies the following conditions:
      \begin{itemize}
        \item $F$ is in canonical form
        \item Given that $X \supset Y$, we cannot replace $X \to A$ in $F$ with $Y \to A$ and still have a set of dependencies that is equivalent to $F$.
        \item We cannot remove any dependency from $F$ and still have a set of dependencies that is equivalent to $F$.
      \end{itemize}
      \par In short, a minimal set of dependencies must be in \tb{standard} or \tb{canonical form} (condition 1) and has \tb{no redundancies} (conditions 2 and 3).

    \hiii{Minimal Cover}
      \par A \tb{minimal cover} of a set of functional dependencies $E$ is a \tb{minimal set of dependencies} that is equivalent to $E$. We can always find \tb{at least one} minimal cover $F$ for any set of dependencies $E$ using the following algorithm:
        \begin{algorithm}[H]
          \caption{Determine the minimal cover $F$ for a set of FDs $E$}
          \Input{set of FDs $E$}
          \begin{algorithmic}[1]
            \State $F := E$
            \For{each FD $X \to \{ A_1, \ldots, A_n \} \in F$}
              \State Replace with $X \to A_1, \ldots, X \to A_n$
              \Comment{Place the FD in a canonical form}
            \EndFor
            \For{each FD $X \to A \in F$}
              \For{each attribute $B \in X$}
                \If{$ F - \{ X \to A \} + \{ (X - \{B\}) \to A \}$ is equivalent to $F$}
                  \State Replace $X \to A$ with $(X - \{B\}) \to A$ in $F$
                  \Comment{Remove all extraneous attributes}
                \EndIf
              \EndFor
            \EndFor
            \For{each remaining FD $X \to A \in F$}
              \If{$F - \{X \to A\}$ is equivalent to $F$}
                \State Remove $X \to A$ from $F$
                \Comment{Remove all remaining redundant FDs}
              \EndIf
            \EndFor
          \end{algorithmic}
        \end{algorithm}
      \par Note that on line 6 I used the plus ($+$) symbol instead of the union ($\cup$) symbol (like in the book) for easy-reading.

    \hiii{Determine the Key of a Relation}
      \begin{algorithm}[H]
        \caption{Determine the key $K$ of a relation $R$}
        \Input{$R$ and the set of FDs $F$ of $R$}
        \begin{algorithmic}[1]
          \State $K := R$
          \For{each attribute $A \in K$}
            \State Compute $(K - A)^+$ with respect to $F$
            \If{$(K - A)^+$ contains all attributes in $R$}
              \State $K := K - \{A\}$
            \EndIf
          \EndFor
        \end{algorithmic}
      \end{algorithm}

\hi{Properties of Relational Decompositions}
  \hii{Relation Decomposition and Insufficiency of Normal Forms}
    \par The relation schema $D = \{R_1, \ldots, R_m\}$ resulting from decomposing the relation schema $R$ is called a \tb{decomposition} of $R$.
    \par \tb{Attribute preservation condition of a decomposition}: each attribute in $R$ will appear in at least one relation schema $R_i$ in the decomposition so that no attributes are lost:
    \[
      \bigcup\limits_{i = 1}^{m} R_i = R
    \]
    \par Apart from the attribute preservation condition, another goal is to have each individual relation $R_i$ in $D$ be in BCNF or 3NF. However, this condition does not guarantee a good database design on its own.

  \hii{Dependency Preservation Property of a Decomposition}

