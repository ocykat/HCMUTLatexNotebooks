\chapter{Foundations}

\hi{Getting started}
  \hii{Insertion Sort}

    \begin{itemize}
      \item \textbf{Input:} A sequence of $n$ numbers $a_{1}, a_{2}, \ldots, a_{n}$
      \item \textbf{Output:} A permutation of the input sequence such that
        $a'_{1} \leq a'_{2} \leq \ldots \leq a'_{n}$
    \end{itemize}

    \begin{algorithm}
      \caption{Insertion-Sort}
      \begin{algorithmic}[1]
        \Statex
        \Function{Insertion-Sort}{$a$}
          \FOR{i \gets 2 \TO a.length}
            \LET{key}{a[i]}
            \LET{j}{i - 1}
            \WHILE{j > 0 \AND a[j] > key}
              \LET{a[j]}{a[j + 1]}
              \LET{j}{j - 1}
            \ENDWHILE
            \LET{a[j + 1]}{key}
          \ENDFOR
        \EndFunction
      \end{algorithmic}
    \end{algorithm}

  \hii{Loop Invariants}
    \par To prove that \textbf{Insertion Sort} is correct, we use \textbf{loop invariants}.
    \par \textbf{A loop invariant} consists of three things:
    \begin{itemize}
      \item \textbf{Initialization}: It is true prior to the first iteration of the loop.
      \item \textbf{Maintenance}: If it is true before an iteration of the loop, it remains
        true before the next iteration.
      \item \textbf{Termination}: When the loop terminates, the invariant gives us a useful
        property that helps show that the algorithm is correct.
    \end{itemize}

  \hii{Correctness of Insertion Sort}
    \begin{itemize}
      \item \textbf{Initialization}: When $i = 2$, the subarray $a[1..i - 1] = a[1..1]$ only
        contains 1 element. Therefore, it is already sorted.
      \item \textbf{Maintenance}: After each iteration, $a[i]$ is added to the sorted
        subarray in the following way: from the right to the left of the sorted subarray,
        find the suitable position for $a[i]$. While going right to left, all elements
        which are larger than $a[i]$ is shifted one position to the right to ensure that
        there is an empty position for $a[i]$ to occupy when the suitable position is
        found.
      \item \textbf{Termination}: After the last iteration $i = a.length$, the "subarray"
        $a[1..a.length]$ has been sorted. In order words, the whole array has been sorted.
    \end{itemize}

\hi{Analyzing algorithms}
  \hii{Analyzing Insertion Sort}
    \hiii{Evaluating Running Time}
      \par Let:
      \begin{itemize}
        \item $n = a.length$: the length of array $a$
        \item $t_{j}$: number of times the $j_{th}$ while-loop executes in one iteration
          of the outer for-loop
      \end{itemize}
      \begin{algorithm}
        \caption*{\textsc{Insertion-Sort}}
        \begin{algorithmic}[1]
          \Statex
          \Function{Insertion-Sort}{$a$}
            \FOR{i \gets 2 \TO a.length} \Comment{Times: $n$; Cost: $c_{1}$}
              \LET{key}{a[i]} \Comment{Times: $n - 1$; Cost: $c_{2}$}
              \LET{j}{i - 1} \Comment{Times: $n - 1$; Cost: $c_{3}$}
              \WHILE{j > 0 \AND a[j] > key}
              \Comment{Times: $\SUM{_{j = 2}^{n} t_{j}}$; Cost: $c_{4}$}
                \LET{a[j]}{a[j + 1]}
                \Comment{Times: $\SUM{_{j = 2}^{n} (t_{j} - 1)}$; Cost: $c_{5}$}
                \LET{j}{j - 1}
                \Comment{Times: $\SUM{_{j = 2}^{n} (t_{j} - 1)}$; Cost: $c_{6}$}
              \ENDWHILE
              \LET{a[j + 1]}{key}
              \Comment{Times: $n - 1$; Cost: $c_{7}$}
            \ENDFOR
          \EndFunction
        \end{algorithmic}
      \end{algorithm}
      \par The running time of the Insertion Sort algorithm can be calculated as follow:
      \begin{align*}
        T(n) = c_{1}n
          + c_{2}(n - 1)
          + c_{3}(n - 1)
          + c_{4} \SUM{_{j = 2}^{n} t_{j}}
          + c_{5} \SUM{_{j = 2}^{n} (t_{j} - 1)}
          + c_{6} \SUM{_{j = 2}^{n} (t_{j} - 1)}
          + c_{7}(n - 1)
      \end{align*}
      \par Running time may vary according to input.
    \hiii{Best-case Running Time}
      \par The \textbf{best case} occurs if the array is already sorted.
      In that case, $t_{j} = 1$ for $j = 2, 3, \ldots n$. Thus, the best-case running time
      is:
      \begin{align*}
        T(n) &= c_{1}n
          + c_{2}(n - 1)
          + c_{3}(n - 1)
          + c_{4}(n - 1)
          + c_{7}(n - 1) \\
          &= (c_{1} + c_{2} + c_{3} + c_{4} + c_{7})n
            - (c_{2} + c_{3} + c_{4} + c_{7}) \\
          &= an + b
      \end{align*}
      \par This is a \textbf{linear function} of $n$.
    \hiii{Worst-case Running Time}
      \par The \textbf{worst case} occurs if the array is reversely sorted.
      In that case, $t_{j} = j$ for $j = 2, 3, \ldots n$. Noting that:
      \begin{align*}
        \SUM{_{j = 2}{n} j} = \frac{n(n + 1)}{2} - 1
        \SUM{_{j = 2}{n} (j - 1)} = \frac{(n - 1)n}{2}
      \end{align*}
      \par Thus, the worst-case running time is:
      \begin{align*}
        T(n) &= c_{1}n
          + c_{2}(n - 1)
          + c_{3}(n - 1)
          + c_{4} \bigg[ \frac{n(n + 1)}{2} - 1 \bigg]
          + c_{5} \bigg[ \frac{(n - 1)n}{2} \bigg]
          + c_{6} \bigg[ \frac{(n - 1)n}{2} \bigg]
          + c_{7}(n - 1) \\
          &= c_{1}n
          + c_{2}n - c_{2}
          + c_{3}n - c_{3}
          + \frac{c_{4}}{2} [n(n + 1)] - c_{4}
          + \frac{c_{5}}{2} [(n - 1)n]
          + \frac{c_{6}}{2} [(n - 1)n]
          + c_{7}n - c_{7} \\
          &= c_{1}n
          + c_{2}n - c_{2}
          + c_{3}n - c_{3}
          + \frac{c_{4}}{2} [n^{2} + n] - c_{4}
          + \frac{c_{5}}{2} [n^{2} - n]
          + \frac{c_{6}}{2} [n^{2} - n]
          + c_{7}n - c_{7} \\
          &= \bigg(\frac{c_{4} + c_{5} + c_{6}}{2} \bigg) n^{2}
          + \bigg(c_{1} + c_{2} + c_{3}
            + \frac{c_{4}}{2} - \frac{c_{5}}{2} - \frac{c_{6}}{2} + c_{7} \bigg) n
          - (c_{2} + c_{3} + c_{4} + c_{7}) \\
          &= an^{2} + bn + c
      \end{align*}
      \par This is a \textbf{quadratic function} of $n$.
    \hiii{Average-case Running Time}
      \par On average, half the elements in $A[1..j - 1]$ are less than $A[j]$, and half the
      elements are greater. Therefore, on average, we have to check half of the subarray
      $A[1..j - 1]$, so $t_{j}$ is about $j/2$. Thus, the average-case running time is
      also a quadratic function.

  
  \hii{Worst-case and Average-case analysis}
    \par The \textbf{worst-case} running time should usually be concentrate on for three
    reasons:
    \begin{itemize}
      \item It gives us the upper bound on the running time of any input, which means the
        algorithm is guarantee to never take longer than the worst-case running time.
      \item Worst case occurs fairly often for some algorithms.
      \item The average case is often roughly as bad as the worse case. Insertion-Sort is
        one particular example.
    \end{itemize}

  \hii{Order of Growth}
    \par It is the \textbf{rate of growth}, or \textbf{order of growth}, of the running time
    that really interests us. We therefore consider only the \textbf{leading term} of the
    formula.
    We also \textbf{ignore the constant coefficient} of this term.
    \par For Insertion-Sort, we are left with the factor $n^{2}$. We write that:
    \begin{center}
      Insertion-Sort has the worst-case running time of $\Theta(n^{2})$.
    \end{center}

\hi{Designing algorithms}
  \hii{Common approaches}
    \par Some common approaches for algorithm design are:
    \begin{itemize}
      \item Incremental approach: insertion sort
      \item Divide-and-conquer: merge sort
    \end{itemize}
  \hii{Divide-and-conquer}
    \hiii{Introduction}
      \par Many algorithms are \textbf{recursive} in structure: to solve a given problem,
      they call themselves recursively one or more times to deal with closely related
      subproblems.
      \par These algorithms typically follow a \textbf{divide-and-conquer} approach:
      they break the problem into several subproblems that are similar to the original
      problem but smaller in size, solve the subproblems recursively, and then combine
      these solutions to create a solution to the original problem.
    \hiii{Steps}
      \par The divide-and-conquer paradigm involves three steps at each level of the
      recursion:
      \begin{itemize}
        \item \textbf{Divide} the problem into a number of subproblems that are smaller
          instances of the same problem.
        \item \textbf{Conquer} the subproblems by solving them recursively. If the
          subproblem sizes are small enough, however, just solve the subproblems in a
          straightforward manner.
        \item \textbf{Combine} the solutions to the subproblems into the solution for the
          original problem.
      \end{itemize}

  \hii{Merge Sort}

    \hiii{Description}
      \begin{itemize}
        \item \textbf{Divide}: Divide the $n$-element sequence to be sorted into two
          subsequences of $n / 2$ elements each.
        \item \textbf{Conquer}: Sort the two subsequences recursively using merge sort.
        \item \textbf{Combine}: Merge the two sorted subsequences to produce the sorted
          answer.
      \end{itemize}
      \par The recursion "bottoms out" when the sequence to be sorted has length 1, in which
      case there is no work to be done, since every sequence of length 1 is already sorted.

    \hiii{Pseudocode}
      \begin{algorithm}
        \caption{Merge-Sort}
        \begin{algorithmic}[1]
          \Statex
          \Function{Merge}{$a, start, mid, end$}
            \LET{n_{1}}{mid - start + 1} \Comment{$left$ includes $a[mid]$}
            \LET{n_{2}}{end - mid}
            \LET{left}{\mbox{new array} [1..n_{1} + 1]}
            \LET{right}{\mbox{new array} [1..n_{2} + 1]}
            \FOR{i \gets 1 \TO n_{1}}
              \LET{left[i]}{a[start + i - 1]}
            \ENDFOR
            \FOR{j \gets 1 \TO n_{2}}
              \LET{right[j]}{a[mid + j]}
            \ENDFOR
            \LET{left[n_{1} + 1]}{\infty}
            \LET{right[n_{2} + 1]}{\infty}
            \LET{i}{1}
            \LET{j}{1}
            \FOR{k \gets start \TO end}
              \IF{left[i] \leq right[j]}
                \LET{a[k]}{left[i]}
                \LET{i}{i + 1}
              \ELSE
                \LET{a[k]}{right[j]}
                \LET{i}{i + 1}
              \ENDIF
            \ENDFOR
          \EndFunction
          \State
          \Function{Merge-sort}{$a, start, end$}
            \IF{start < end}
              \LET{mid}{\floor{(start + end) / 2}}
              \CALLPROC{Merge-sort}{a, start, mid}
              \CALLPROC{Merge-sort}{a, mid + 1, end}
              \CALLPROC{Merge}{a, start, mid, end}
            \ENDIF
          \EndFunction
        \end{algorithmic}
      \end{algorithm}

    \hiii{The Merge Procedure}
      \par \tb{Correctness with Loop Invariant}
      \begin{itemize}
        \item \tb{Initialization}: Before the first iteration, $k = start$.
          The subarray $a[start..k - 1]$ is ``sorted", since it does not contain
          any element. $left[i]$ and $right[j]$ are each the smallest element in their
          arrays that has not been copied to $a$. Also, the largest element of $a$
          is smaller than both $left[i]$ and $right[j]$.
        \item \tb{Maintenance}: At the start of any iteration, the subarray
          $a[start..k - 1]$ is sorted. Then, the smaller of $left[i]$ and $right[j]$
          is copied to $a$. $k$ and $i$ or $j$ is increased by 1.
        \item \tb{Termination}: At termination, $k = end + 1$. By loop invariant, the
        subarray $a[start..k - 1] \equiv a[start..end]$ is sorted.
      \end{itemize}
      \par \tb{Running time}: $\Theta(n)$, where $n = end - start + 1$
      is the number of elements being merged.
    
  \hii{Analyzing divide-and-conquer algorithms}
    \par When an algorithm contains a recursive call to itself, the running
    time can be described by a \tb{recurrence equation} or \tb{recurrence}.
    \par For divide-and-conquer algorithms, the running time can be evaluated
    as follow:
    \begin{equation}
      T(n) = \begin{cases}
        \Theta(1) \qquad \mbox{ if } n \leq c \\
        a T(n / b) + D(n) + C(n) \quad \mbox {otherwise}
      \end{cases}
    \end{equation}
    where
    \begin{itemize}
      \item $n$: the size of the input
      \item $c$: the maximum size of the input where the solution can
      be evaluated in a straightforward way
      \item $a$: number of subproblems after dividing
      \item $n/b$: the size of the input of each subproblem (note that for
      some special cases, $a \neq b$)
      \item $D(n)$: dividing time
      \item $C(n)$: combining time
    \end{itemize}
  
  \hii{Analysis of Merge Sort}
    \par To simplify the analysis, we assume that the size of the array is
      a power of 2. Later, we will find out that this assumption does not
      affect the order of growth of the function.
    \par We reason as follow to set up the worst-case running time, $T(n)$,
      of merge sort:
    \begin{itemize}
      \item \tb{Divide}: The divide step just computes the middle of the
        subarray, which takes constant time. $D(n) = 1$.
      \item \tb{Conquer}: We recursively solve two subproblems, each of
        size $n/2$, which contributes $2T(n/2)$ to the running time.
      \item \tb{Combine}: The running time taken for merge is
        $C(n) = \Theta(n)$.
    \end{itemize}
    \begin{equation}
      T(n) = \begin{cases}
        \Theta(1) \qquad \mbox{ if } n = 1 \\
        2T(n/2) + \Theta(n) \quad \mbox{ if } n > 1 \\
      \end{cases}
    \end{equation}
    \par The ``master theorem" can be used to prove that
      $T(n) = \Theta(n\log(n))$.

\clearpage
  \begin{algorithm}
    \caption{Counting mismatches between two packed \DNA{} strings
      \label{alg:packed-dna-hamming}}
    \begin{algorithmic}[1]
      \Require{$x$ and $y$ are packed \DNA{} strings of equal length $n$}
      \Statex
      \Function{Distance}{$x, y$}
        \Let{$z$}{$x \oplus y$} \Comment{$\oplus$: bitwise exclusive-or}
        \Let{$\delta$}{$0$}
        \For{$i \gets 1 \textrm{ to } n$}
          \If{$z_i \neq 0$}
            \Let{$\delta$}{$\delta + 1$}
          \EndIf
        \EndFor
        \State \Return{$\delta$}
      \EndFunction
    \end{algorithmic}
  \end{algorithm}
