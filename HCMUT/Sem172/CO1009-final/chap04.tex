\begin{multicols}{2}
\hi{Combinational Logic Circuit}

  \hii{Terminology}
    \begin{itemize}
      \item \textbf{Sum-of-product (SOP) form}: consists of two or more AND terms
        that are ORed together.
      \item \textbf{Product-of-sum (POS) form}: consists of two or more OR terms
        that are ANDed together.
    \end{itemize}

  \hii{Designing Combinational Logic Circuits}
    \begin{itemize}
      \item Intepret the problem and set up a truth table to describe its
        operations.
      \item Write the AND (product) term for each case where the output is 1.
      \item Write the sum-of-product (SOP) expression for the output.
      \item \ti{Simplify the output expression if possible}.
      \item Implement the circuit for the final, simplify expression.
    \end{itemize}

  \hii{K-map}
    \hiii{Constructing}
      \par Gray-code order. MSB is placed vertically.
      \par For K-map with 5 or 6 variables, construct symmetrically.
    \hiii{Simplification Process}
      \begin{enumerate}
        \item Loop isolated 1s.
        \item Look for 1s that are adjacent to only one other 1 and loop
        those pairs.
        \item Loop any octet even if it contains some 1s that have been looped.
        \item Loop any quad that contains one or more 1s that have not already been looped,
          \textit{making sure to use the minimum number of loops}.
        \item Loop any pairs necessary to include any 1s that have not yet been looped,
          \textit{making sure to use the minimum number of loops}.
        \item Form the OR sum of all the terms generated by each loop.
      \end{enumerate}

\end{multicols}