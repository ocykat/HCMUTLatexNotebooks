\documentclass[12pt, a4paper]{report}

% MARGINS
\usepackage[margin=0.75in]{geometry}

% FONT
\renewcommand{\familydefault}{\sfdefault}

% LINE SPACING
\setlength{\parskip}{6pt}

% HEADINGS
\setcounter{secnumdepth}{4}
\newcommand{\hi}{\section}
\newcommand{\hii}{\subsection}
\newcommand{\hiii}{\subsubsection}
\newcommand{\hiiiBEGIN}[1]{\subsubsection \begin{enumerate}}
\newcommand{\hiiiEND}{\end{enumerate}}
\newcommand{\hiv}{\item\textbf}

% INDENTATION
\usepackage{indentfirst}

% TABLE OF CONTENTS
% links
\usepackage{hyperref}
\hypersetup{
  colorlinks,
  citecolor=black,
  filecolor=black,
  linkcolor=black,
  urlcolor=black
}

% APPENDICES
\usepackage[toc,page]{appendix}

% TEXT
% Bold, italic, underlined text
\newcommand{\tb}[1]{\textbf{#1}}
\newcommand{\ti}[1]{\textit{#1}}
\newcommand{\tbi}[1]{\textbf{\textit{#1}}}
\newcommand{\tu}[1]{\underline{#1}}
\newcommand{\tbu}[1]{\textbf{\underline{#1}}}
\newcommand{\smf}[1]{\small #1 \normalsize}
\newenvironment{smfont}{\small}{\normalsize}

% MATH
\usepackage{amsmath}
\usepackage{amssymb}
\usepackage{gensymb}

% Equation box
\usepackage{empheq}
\newenvironment{eqbox}
  {\setkeys{EmphEqEnv}{align}\setkeys{EmphEqOpt}{box=\fbox}\EmphEqMainEnv}
  {\endEmphEqMainEnv}

% Arrows
\newcommand{\ra}{\Rightarrow}
\newcommand{\lra}{\Leftrightarrow}

% Sum, product, limit
\newcommand{\SUM}[1]{\sum\limits #1}
\newcommand{\PROD}[1]{\prod\limits #1}

% Bold in math mode
\usepackage{bm}

% Absolute value
\DeclarePairedDelimiter\abs{\lvert}{\rvert}%
\DeclarePairedDelimiter\norm{\lVert}{\rVert}%
%   Swap the definition of \abs* and \norm*, so that \abs
%   and \norm resizes the size of the brackets, and the
%   starred version does not.
\makeatletter
\let\oldabs\abs
\def\abs{\@ifstar{\oldabs}{\oldabs*}}
\let\oldnorm\norm
\def\norm{\@ifstar{\oldnorm}{\oldnorm*}}
\makeatother

% Floor and ceiling
\usepackage{mathtools}
\DeclarePairedDelimiter\ceil{\lceil}{\rceil}
\DeclarePairedDelimiter\floor{\lfloor}{\rfloor}
\DeclarePairedDelimiter\set\{\}

% Inner Product
\newcommand{\iprod}[1]{\langle #1 \rangle}

% Derivatives
\newcommand{\dif}[2]{\dfrac{d #1}{d #2}}
\newcommand{\ddif}[2]{\dfrac{d^2 #1}{d #2^2}}
\newcommand{\difi}[2]{d #1/d #2}
\newcommand{\diff}[2]{\dfrac{d}{d #2} #1}
\newcommand{\pd}[2]{\dfrac{\partial #1}{\partial #2}}
\newcommand{\pdds}[2]{\dfrac{\partial^{2} #1}{\partial #2^{2}}}
\newcommand{\pdd}[3]{\dfrac{\partial^{2} #1}{\partial #2 \partial #3}}
\newcommand{\pddsi}[2]{{\partial^{2} #1} / {\partial #2^{2}}}
\newcommand{\pddi}[3]{{\partial^{2} #1} / {\partial #2 \partial #3}}

% Integrals
\usepackage{esint}
\newcommand{\INT}{\int \limits}
\newcommand{\OINT}{\oint \limits}
\newcommand{\IINT}{\iint \limits}
\newcommand{\IIINT}{\iiint \limits}

% Ratios
\newcommand{\ratio}[3]{\dfrac{#1_{#2}}{#1_{#3}}}

% Manual equation number
\newcommand{\eqnumber}[1]{\qquad\mbox{\tb{{(#1)}}}}

% FOOTNOTE
%   one foot note stays on one page
\interfootnotelinepenalty=10000

\newcommand{\fnmark}{\footnotemark}
\newcommand{\fnmarksame}{\footnotemark[\value{footnote}]}
\newcommand{\fntext}{\footnotetext}

% IMAGES
\usepackage{graphicx}
\usepackage{subcaption}
\usepackage{float}
\newcommand{\img}[2][]
  {
    \begin{figure}[H]
      \centering
      \includegraphics[#1]{#2}
    \end{figure}
  }

% PSEUDOCODE
\usepackage{algorithm}
\usepackage{algorithmicx}
\usepackage[noend]{algpseudocode}
\usepackage{caption}

\renewcommand{\thealgorithm}{\arabic{chapter}.\arabic{algorithm}}

\newcommand*\Let[2]{\State #1 $\gets$ #2}
\algrenewcommand\algorithmicrequire{\textbf{Input:}}
\algrenewcommand\algorithmicensure{\textbf{Output:}}
\newcommand{\INPUT}[1]{\Require{#1} \Statex}
\newcommand{\OUTPUT}[1]{\Ensure{#1} \Statex}
\newcommand{\INPUTOUTPUT}[2]{\Require{#1} \Ensure{#2} \Statex}
\newcommand{\LET}[2]{\Let{$#1$}{$#2$}}
\newcommand{\FOR}[2]{\For{$#1 \gets #2$}}
\newcommand{\ENDFOR}{\EndFor}
\newcommand{\TO}{\textrm{ \tb{to} }}
\newcommand{\DOWNTO}{\textrm{ \tb{downto} }}
\newcommand{\AND}{\textrm{ \tb{and} }}
\newcommand{\OR}{\textrm{ \tb{or} }}
\newcommand{\XOR}{\textrm{ \tb{xor} }}
\newcommand{\GETS}{\gets}
\newcommand{\IF}[1]{\If{$#1$}}
\newcommand{\ELSEIF}[1]{\ElsIf{$#1$}}
\newcommand{\ELSE}{\Else}
\newcommand{\ENDIF}{\EndIf}
\newcommand{\WHILE}[1]{\While{$#1$}}
\newcommand{\ENDWHILE}{\EndWhile}
\newcommand{\FUNCTION}[2]{\Function{#1}{$#2$}}
\newcommand{\ENDFUNCTION}{\EndFunction}
\newcommand{\PROCEDURE}[2]{\Procedure{#1}{$#2$}}
\newcommand{\ENDPROCEDURE}{\EndProcedure}
\newcommand{\CALLFUNC}[2]{\State \Call{#1}{$#2$}}
\newcommand{\CALLPROC}[2]{\State \Call{#1}{$#2$}}
\newcommand{\RETURN}[1]{\State \Return{#1}}
\newcommand{\SHIFTLEFT}{\ll}
\newcommand{\SHIFTRIGHT}{\gg}

% FRENCH CONVENTION FOR VERTICAL ARITHMETIC
\usepackage{xlop}

% CODE
\usepackage{listings}
%\makeatletter
%\lst@CCPutMacro\lst@ProcessOther {"2D}{\lst@ttfamily{-{}}{-{}}}
%\@empty\z@\@empty
%\makeatother
\usepackage{inconsolata}
\lstset{
  language=[x86masm]Assembler,
  aboveskip=1mm,
  belowskip=1mm,
  numbers=left,
  frame=tb,
  % language=[mips]Assembler,
  basicstyle=\ttfamily
  \footnotesize,
  breaklines=true,
  mathescape=false
}

% 
\usepackage{varwidth}
\newenvironment{fboxenv}
  {
    \begin{minipage}{\dimexpr\textwidth-2\fboxsep-2\fboxrule\relax}
  }
  {
    \end{minipage}
  }

% SOURCE
%\newcommand{\code}[1]{\texttt{#1}}
\newcommand{\refsource}[1]{Source: {#1}}

\begin{document}
\tableofcontents

\hi{Logic}

\hii{Natural Deduction Rules}

% AndIntroduction
\newsavebox\AndIntro
\sbox\AndIntro{
  \AxiomC{$\phi$}
  \AxiomC{$\psi$}
  \RightLabel{$\land i$}
  \BinaryInfC{$\phi \land \psi$}
  \DisplayProof
}

% AndElimination
\newsavebox\AndElim
\sbox\AndElim{
  \AxiomC{$\phi \land \psi$}
  \RightLabel{$\land e_1$}
  \UnaryInfC{$\phi$}
  \DisplayProof
  \hskip 1cm
  \AxiomC{$\phi \land \psi$}
  \RightLabel{$\land e_2$}
  \UnaryInfC{$\psi$}
  \DisplayProof
}

% Double-Negation Introduction
\newsavebox\DNegIntro
\sbox\DNegIntro{
  \AxiomC{$\phi$}
  \RightLabel{$\lnotnot i$}
  \UnaryInfC{$\lnotnot \phi$}
  \DisplayProof
}

% Double-Negation Elimination 
\newsavebox\DNegElim
\sbox\DNegElim{
  \AxiomC{$\lnotnot \phi$}
  \RightLabel{$\lnotnot e$}
  \UnaryInfC{$\phi$}
  \DisplayProof
}

% 

\begin{tabular}{|c|c|c|}
  \hline
  Rule              & Introduction      & Elimination      \\ \hline
  Conjunction (AND) & \usebox\AndIntro  & \usebox\AndElim  \\ \hline
  Double Negation   & \usebox\DNegIntro & \usebox\DNegElim \\ \hline
\end{tabular}

\setcounter{chapter}{2}
\begin{multicols*}{3}

\hi{Distributions}
  \hii{Discrete Distributions}

  \hiii{Binomial Distribution}
    \begin{itemize}
      \item \tb{Intepretation}:
        \begin{itemize}
          \item The number of winning game in $n$ games, given that the chance to win any one game is $p$.
          \item The number of red balls obtained after picking \tb{with replacement} from $n$ balls from a box of red and blue balls, given that the chance to pick a red ball at one time is $p$.
        \end{itemize}
      \item \tb{Notation}:
        \[
          X \sim B(n, p)
        \]
      \item \tb{P.M.F}:
        \[
          f(k) = P(X = k) = {n \choose k} p^k q^{n - k}
        \]
      \item \tb{Expectation}:
        \[
          \mu = E(X) = np
        \]
      \item \tb{Variance}:
        \[
          \sigma^2 = V(X) = npq = np(1 - p)
        \]
    \end{itemize}

  \hiii{Hypergeometric Distribution}
    \begin{itemize}
      \item \tb{Intepretation}:
        \begin{itemize}
          \item The number of red balls obtained after picking \tb{without replacement} $n$ balls from $N$ balls from a box of red and blue balls, given that the number of red balls in the box is $m$.
        \end{itemize}
      \item \tb{Notation}:
        \[
          X \sim H(N, m, n)
        \]
      \item \tb{P.M.F}:
        \[
          f(k) = P(X = k) = \dfrac{{m \choose k} {N - m \choose n - k}}{{N \choose n}}
        \]
      \item \tb{Expectation}:
        \[
          \mu = E(X) = np, \text{ where } p = \frac{m}{N}
        \]
      \item \tb{Variance}:
        \[
          \sigma^2 = V(X) = npq \frac{N - n}{N - 1}, \text{ where } p = \frac{m}{N}
        \]
    \end{itemize}

  \hiii{Poisson Distribution}
    \begin{itemize}
      \item \tb{Intepretation}:
        \begin{itemize}
          \item The number of events occuring in a fixed period of time.
        \end{itemize}
      \item \tb{Notation}:
        \[
          X \sim Po(\lambda)
        \]
      \item \tb{P.M.F}:
        \[
          f(k) = P(X = k) = \frac{\lambda^k}{k!} e^{-\lambda}
        \]
      \item \tb{Expectation}:
        \[
          \mu = E(X) = \lambda
        \]
      \item \tb{Variance}:
        \[
          \sigma^2 = V(X) = \lambda
        \]
    \end{itemize}

  \hii{Continuous Distributions}

  \hiii{Uniform Distribution}
    \begin{itemize}
      \item \tb{Notation}:
        \[
          X \sim U(a, b)
        \]
      \item \tb{P.D.F}:
        \[
          f(x) =
            \begin{cases}
              \dfrac{1}{b - a} & x \in [a, b]\\
              0 & \text{otherwise}
            \end{cases}
        \]
      \item \tb{C.D.F}:
        \[
          F(x) =
            \begin{cases}
              0 & x < a \\
              \dfrac{x - a}{b - a} & x \in [a, b)\\
              1 & x \geq b
            \end{cases}
        \]
      \item \tb{Expectation}:
        \[
          \mu = E(X) = \frac{a + b}{2}
        \]
      \item \tb{Variance}:
        \[
          \sigma^2 = V(X) = \frac{(b - a)^2}{12}
        \]
    \end{itemize}

  \hiii{Normal Distribution}
    \begin{itemize}
      \item \tb{Notation}:
        \[
          X \sim N(\mu, \sigma^2)
        \]
      \item \tb{Standardizing}:
        \[
          Y = \frac{X - \mu}{\sigma} \sim N(0, 1)
        \]
      \item \tb{C.D.F}:
        \par For $X \sim N(0, 1)$:
        \[
          \Phi(x) = \int\limits_{-\infty}^{x} f(u) du
        \]
      \item \tb{Expectation}:
        \[
          \mu = E(X) = \mu
        \]
      \item \tb{Variance}:
        \[
          \sigma^2 = V(X) = \sigma^2
        \]
      \item \tb{Properties}
        \begin{itemize}
          \item If $X \sim N(\mu, \sigma^2)$ and $Y = aX + b$:
            \[
              Y \sim N(a\mu + b, a^2 \sigma^2)
            \]
          \item If $X \sim N(\mu, \sigma^2)$:
            \[
              \sum\limits_{i = 1}^{n} X_i \sim N\bigg( \sum\limits_{i = 1}^{n} \mu_i, \sum\limits_{i = 1}^{n} \sigma_i^2 \bigg)
            \]
        \end{itemize}
    \end{itemize}


\end{multicols*}


\begin{appendices}

\end{appendices}

\end{document}
