\hi{Magnetic Fields and Forces}
    \hii{Magnetic Fields}
        \par A \impt{magnetic field} is a region around a magnetic material or a moving electric
        charge within which the force of magnetism acts.
        \par A magnetic field $B$ at some point in space can be defined in terms of the magnetic
        force $F_{B}$ that the field exerts on a charged particle moving with a velocity $v$.
        \begin{equation}
            \vt{F_{B}} = q \vt{v} \times \vt{B}
        \end{equation}
    \hii{Magnetic Force}
        \begin{itemize}
            \item Magnitude:
                \begin{equation}
                    F_{B} = |q|vB \sin(\vt{v}, \vt{B}) = |q|vB \sin(\theta)
                \end{equation}
            \item Unit: tesla (T)
                \begin{equation}
                    1 T = 1 \frac{N}{C \cdot m/s} = 1 \frac{N}{A \cdot m}
                \end{equation}
        \end{itemize}

\hi{Magnetic Force Acting on a Current-Carrying Conductor}
    \par Consider a \impt{straight} segment of wire of length $L$ and cross-sectional area $A$,
    carrying a current $I$ in a uniform magnetic field $B$.
    \par The magnetic force acting on a charge $q$:
    \begin{align*}
        \vt{F_{B/1q}} = q \vt{v} \cdot \vt{B}
    \end{align*}
    \par Define $n$ as the number of charges per unit volume. The magnetic forces acts on the
    wire:
    \begin{align*}
        \vt{F_{B}} = nALq \vt{v} \cdot \vt{B} 
    \end{align*}
    \par The current in the wire $I$ can be written as:
    \begin{align*}
        I = nqvA
    \end{align*}
    \par Therefore:
    \begin{align*}
        \vt{F_{B}} = I \vt{L} \times \vt{B}
    \end{align*}
    \par Now consider an arbitrarily shaped wire segment of uniform cross section in a magnetic
    field. The lenght of the wire is from $a$ to $b$.
    \begin{align*}
        F_{B} = I \INT{_{a}^{b}ds} \cdot B
    \end{align*}

