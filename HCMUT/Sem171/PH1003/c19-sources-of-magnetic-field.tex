\chapter{Sources of magnetic field}

\hi{The Biot-Savart Law}
    \par While performing an experiment on the force exerted by an electric current on a nearby
    magnet, Biot and Savart came up with the Biot-Savart Law:
    \par \textit{The magnetic field $dB$ at a point $P$ associated with a length $ds$ of a wire
    carrying a steady current $I$}:
    \begin{equation}
        dB = \frac{\mu_{0}}{4\pi} \frac{Id\textbf{s}\hat{r}}{r^{2}}
    \end{equation}
    where
    \begin{itemize}
        \item $B$: magnetic field \quad $[T]$
        \item $\mu_{0}$: permeability of free space \quad $(4 \pi \cdot 10^{-7} Tm/A)$
        \item $I$: the electric current \quad $[A]$
        \item $s$: the length of the conductor \quad $[m]$
        \item $r$: the distance from $ds$ to $P$ \quad $[m]$
    \end{itemize}
    \par By integration, we obtain:
    \begin{equation}
        B = \frac{\mu_{0}I}{4\pi} \INT{\frac{d\textbf{s}\hat{r}}{r^{2}}}
    \end{equation}
    \par \textbf{Example:} Consider a thin straight wire carrying a constant current $I$ and placed
    along the $x$ axis. Determine the magnitude and direction of the magnetic field at point P due to
    this current.
    \begin{flalign*}
        & \bullet d\textbf{s} \hat{r} = dx \sin(\vt{s}, \vt{r}) \hat{B} \mendl
        & \mbox{We also have:} \mendl
        & d\textbf{B} = \frac{\mu_{0}I}{4\pi} \frac{Id\textbf{s}\hat{r}}{r^{2}}\mendl
        & \ra d\textbf{B} = dB\hat{B}
            = \frac{\mu_{0}I}{4\pi} \frac{dx \sin(\vt{s}, \vt{r}) \hat{B}}{r^{2}} \mendl
        & \ra dB = \frac{\mu_{0}I}{4\pi} \frac{dx \sin(\vt{s}, \vt{r})}{r^{2}} \mendl
        &  = \frac{\mu_{0}I}{4\pi} \frac{dx \sin(\theta)}{r^{2}} \mendl
    \end{flalign*}
    \begin{flalign*}
        & \bullet \mbox{Let a = d(P, Ox)}\mendl
        & x = -a \cot(\theta) \mendl
        & \ra dx = \frac{a}{\sin^{2}(\theta)} d\theta \mendl
        & r = \frac{a}{\sin(\theta)}\mendl
        & \ra dB = \frac{\mu_{0} I}{4\pi}
            \frac{
                \frac{a}{sin^{2}(\theta)}
            }
            {
                \frac{a^{2}}{sin^{2}(\theta)}
            }
            \sin(\theta) d\theta \mendl
        & \ra dB = \frac{\mu_{0} I}{4\pi} \sin(\theta) d\theta \mendl
        & \ra B = \frac{\mu_{0} I}{4\pi a} \INT{_{\theta_{1}}^{\theta_{2}}\sin(\theta) d\theta} \mendl
        & \ra B = \frac{\mu_{0} I}{4\pi a} (\cos\theta_{1} - \cos\theta_{2}) \mendl
    \end{flalign*}
    \begin{flalign*}
        & \bullet \mbox{ If the wire is infinitely long }\mendl
        & \ra B = \frac{\mu_{0} I}{4\pi a} (\cos(0) - \cos(\pi)) \mendl
        & \ra B = \frac{\mu_{0} I}{2\pi a} \mendl
    \end{flalign*}

\hi{Ampere's Law}
    \begin{eqbox}
        B \Dt l = \mu_{0}I
    \end{eqbox}
