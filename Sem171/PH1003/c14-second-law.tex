\chapter{Heat Engines, Entropy, and the Second Law of Thermodynamics}

\hi{Heat Engines and the Second Law of Thermodynamics}
    \hii{Heat Engines}
        \par A \impt{heat engine} is a device that item takes in energy by heat and, operating in a cyclic process, expels a fraction of that energy by means of work.
        \par \impt{The cyclic process}:
        \begin{enumerate}
            \item The working substance absorbs energy by heat from a high-temperature
            energy reservoir
            \item Work is done by the engine
            \item Energy is expelled by heat to a lower-temperature reservoir.
        \end{enumerate}
    \hii{Work done by an engine}
        \begin{eqbox}
            W_{eng} = |Q_{h}| - Q_{c}
        \end{eqbox}
        \begin{itemize}
            \item $W_{eng}$: work done by the engine
            \item $Q_{h}$: heat absorbed from the hot reservoir
            \item $Q_{c}$: heat given up to the cold reservoir
        \end{itemize}
    \hii{Thermal Efficiency}
        \begin{eqbox}
            e = \frac{W_{eng}}{|Q_{h}|}
            = \frac{|Q_{h}| - |Q_{c}|}{|Q_{h}|}
            = 1 - \frac{|Q_{c}|}{|Q_{h}|}
        \end{eqbox}
    \hii{Kelvin-Planck form of the Second Law of Thermodynamics}
        \par It is impossible to construct a heat engine that, operating in a cycle,
        produces no effect other than the input of energy by heat from a reservoir and the
        performance of an equal amount of work.
        \par In order words:
        \begin{eqbox}
            W_{eng} < |Q_{h}| \mbox{ for all engines}
        \end{eqbox}

\hi{Heat Pumps and Refrigerators}
    \hii{Heat Pumps and Refrigerators Definition}
        \par The device that transfer energy from a cold reservoir to a hot reservoir is a
        \impt{heat pump} or \impt{refrigerator}.
        \par In a refrigerator, the engine takes in energy $|Q_{c}|$ from a cold reservoir
        and expels energy $|Q_{h}|$ to a hot reservoir. This can be accomplished only if
        \textit{work is done \textbf{on} the engine}.
    \hii{Clausius form of the Second Law of Thermodynamics}
        \par It is impossible to construct a cyclical machine whose sole effect is to
        transfer energy continuously by heat from one object to another object at a higher
        temperature without the input of energy by work.
        \par In simpler terms, \textit{energy does not transfer spontaneously by heat
        from a cold object to a hot object}.
    \hii{Coefficient of performance}
        \begin{eqbox}
            \mbox{COP (heating mode) } = \frac{|Q_{h}|}{W_{ref}}
        \end{eqbox}
        \begin{eqbox}
            \mbox{COP (cooling mode) } = \frac{|Q_{c}|}{W_{ref}}
        \end{eqbox}

\hi{Reversible and Irreversible Processes}
    \hii{Reversible and Irreversible Processes}
        \par In a \impt{reversible} process, the system undergoing the process can be
        returned to its initial conditions along the same path on a $PV$ diagram, and every
        point along this path is an equilibrium state.
        \par A process that does not sastisfy these requirements is \impt{irreversible}.
        \par All natural processes are known to be \textit{irreversible}.

\hi{The Carnot Engine}
    \hii{The Carnot Cycle}
        \par Assume that the working substance is an ideal gas contained in a cylinder
        fitted with a movable piston at one end. The cylinder's walls and the piston
        are thermally nonconducting.
        \par There are 4 stages in an Carnot cycle:
        \begin{enumerate}
            \item Process $A \to B$: \impt{isothermal expansion} at $T_{h}$.
                \par The gas is placed in thermal contact with an energy reservoir at
                temperature $T_{h}$.
                \par During the expansion, the gas:
                \begin{itemize}
                    \item absorbs energy $|Q_{h}|$ from the reservoir through the base
                    \item does work $W_{AB}$ in raising the piston.
                \end{itemize}
            \item Process $B \to C$: \impt{adiabatic expansion}
                \par The base of the cylinder is replaced by a thermally nonconducting wall,
                and the gas expands \impt{adiabatically} - no energy enters or leaves the
                system by heat. During the expansion, the temperature of the gas
                \impt{decreases} from $T_{h}$ to $T_{c}$ and the gas the work $W_{BC}$
                in raising the piston.
            \item Process $C \to D$: \impt{isothermal compression} at $T_{c}$
                \par The gas is placed in thermal contact with an energy reservoir at
                temperature $T_{c}$ and is compressed isothermally at temperature
                $T_{c}$.
                \par During the compression, the gas:
                \begin{itemize}
                    \item expels energy $|Q_{c}|$ to the reservoir
                    \item does work $W_{CD}$ on the piston
                \end{itemize}
            \item Process $D \to A$: \impt{adiabatic compression}
                \par The base of the cylinder is replaced by a nonconducting wall.
                \par During this process:
                \begin{itemize}
                    \item The temperature of the gas increases to $T_{h}$
                    \item The work done by the gas on the piston is $W_{CD}$.
                $W_{DA}$.
                \end{itemize}
        \end{enumerate}
        \begin{figure}[h!]
            \centering
            \includegraphics[width=5cm]{img/Carnot-cycle.jpg}    
            \caption{Carnot cycle}
        \end{figure}
    \hii{Carnot's theorem}
        \par No real heat engine operating between two energy reservoirs can be more
        efficient than a Carnot engine operating between the same two reservoirs.
    \hii{Efficiency of Carnot engine}
        \begin{eqbox}
            e_{C} = 1 - \frac{T_{c}}{T_{h}}
        \end{eqbox}
        \par \textbf{Proof}:
        \begin{flalign*}
            %
            &\bullet A \to B \mbox{ is an isothermal process} \mendl
            &\ra \Dt E_{int} = 0 \mendl
            &\ra |Q_{h}| = |-W_{AB}|
                = - \INT{_{V_{i}}^{V_{f}} PdV}
                = nRT_{h}\ln \Big(
                    \frac{V_{B}}{V_{A}}
                \Big) \quad (V_{B} > V_{A}) \quad (1) \mendl
            %
            &\bullet C \to D \mbox{ is an isothermal process} \mendl
            &\ra \Dt E_{int} = 0 \mendl
            &\ra |Q_{h}| = |-W_{CD}|
                = - \INT{_{V_{i}}^{V_{f}} PdV}
                = nRT_{c} \ln \Big(
                    \frac{V_{C}}{V_{D}}
                \Big) \quad (V_{C} > V_{D}) \quad (2) \mendl
            &\bullet \mbox{Divide (2) by (1)} \mendl
            &\ra \frac{|Q_{c}|}{|Q_{h}|}
                = \frac{T_{c}}{T_{h}} \cdot \frac{\ln(V_{C}/V_{D})}{\ln(V_{B}/V_{A})}
                \quad (3) \mendl
            %
        \end{flalign*}
        \begin{flalign*}
            &\bullet \mbox{ In an adiabatic process} \mendl
            &\Dt E_{int} = W \mendl
            &\ra \Dt E_{int} - W = 0 \mendl
            &\ra \frac{3}{2}nR \Dt T - P \Dt V = 0 \mendl
            &\ra \frac{3}{2}nRdT - nRTdV = 0 \mendl
            &\mbox{Divide both side for } nRT: \mendl
            &\ra \frac{3}{2}\frac{dT}{T} - \frac{dV}{V} = 0 \mendl
            &\ra \frac{3}{2}\INT{_{T_{i}}^{T_{f}} \frac{dT}{T}}
                - \INT{_{V_{i}}^{V_{f}} \frac{dV}{V}} = 0 \mendl
            &\ra \frac{3}{2} \ln\Big(\frac{T_{f}}{T_{i}}\Big)
                - \ln\Big(\frac{V_{f}}{V_{i}}\Big) = 0 \mendl
            &\ra \ln\Big(\frac{T_{f}^\frac{3}{2}}{T_{i}^\frac{3}{2}}\Big)
                - \ln\Big(\frac{V_{f}}{V_{i}}\Big) = 0 \mendl
            &\ra \ln\Big(
                    \frac{T_{f}^\frac{3}{2}}{T_{i}^\frac{3}{2}}
                    \cdot \frac{V_{i}}{V_{f}}
                    \Big) = 0 \mendl
            &\ra \frac{T_{f}^\frac{3}{2}}{T_{i}^\frac{3}{2}}
                \cdot \frac{V_{i}}{V_{f}} = 1 \quad \mendl
        \end{flalign*}
        \begin{flalign*}
            &\bullet B \to C \mbox{ is an adiabatic process } \mendl
            &\ra \frac{T_{C}^\frac{3}{2}}{T_{B}^\frac{3}{2}}
                \cdot \frac{V_{B}}{V_{C}} = 
                \frac{T_{c}^\frac{3}{2}}{T_{h}^\frac{3}{2}}
                \cdot \frac{V_{B}}{V_{C}} = 1 \quad (4) \mendl
            &\bullet D \to A \mbox{ is an adiabatic process } \mendl
            &\ra \frac{T_{A}^\frac{3}{2}}{T_{D}^\frac{3}{2}}
                \cdot \frac{V_{D}}{V_{A}} = 
                \frac{T_{h}^\frac{3}{2}}{T_{c}^\frac{3}{2}}
                \cdot \frac{V_{D}}{V_{A}} = 1 \quad (5) \mendl
            %
            &\bullet \mbox{ Multiply (4) and (5) } \mendl
            &\ra \frac{V_{B}}{V_{C}} \cdot \frac{V_{D}}{V_{A}} = 1 \mendl
            &\ra \frac{V_{B}}{V_{C}} = \frac{V_{A}}{V_{D}} \mendl
            &\ra \frac{V_{B}}{V_{A}} = \frac{V_{C}}{V_{D}} (6) \mendl
        \end{flalign*}

\hi{Entropy}
    \hii{Statement on Entropy of the Second Law of Thermodynamics}
        \par The entropy of the Universe increases in all real processes.
    \hii{Formula}
        \par Consider any infinitesimal process in which a system changes from
        one equilibrium state to another. If $dQ_{r}$ is the amount of energy
        transferred by heat when the system follows a reversible path between
        the states, then the change in entropy $dS$ is equal to this amount of
        energy for the reversible process divided by the absolute temperature
        of the system.
        \begin{eqbox}
            dS = \frac{dQ_{r}}{T}
        \end{eqbox}
    \hii{Entropy as a State Variable}
        \par The change in entropy during a process depends only on the end
        points and therefore is independent of the actual path followed.
        Consequently, the entropy change for an irreversible process can be
        determined by calculating the entropy change for a reversible process
        that connects the same initial and final states.
    \hii{Entropy in the Carnot cycle}
        \par In a Carnot cycle, there are two processes where the engine exchanges
            energy with the reservoirs:
        \begin{itemize}
            \item Process $A \to B$ where the engine absorbs heat from the
                hot reservoir
            \item Process $C \to D$ where the engine expels heat to the
                cold reservoir
        \end{itemize}
        \begin{flalign*}
            & \Dt S = \Dt S_{AB} + \Dt S_{CD} \mendl
            & \ra \Dt S = \INT{\frac{dQ_{h}}{T_{h}}} + \INT{\frac{dQ_{c}}{T_{c}}} \mendl
        \end{flalign*}
        \par As we have mentioned earlier about the efficiency of a Carnot engine:
        \begin{flalign*}
            & \frac{Q_{h}}{Q_{c}} = \frac{T_{h}}{T_{c}} \mendl
        \end{flalign*}
        \par Therefore:
        \begin{flalign*}
            & \Dt S = 0 \mendl
        \end{flalign*}
        \par In general:
        \begin{flalign*}
            & \OINT{\frac{dQ_{r}}{T}} = 0 \mendl
        \end{flalign*}
