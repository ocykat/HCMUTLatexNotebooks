\chapter{Hardware Description Language}

\hi{ASIC and FPGA}
    \hii{ASIC}
        \par ASIC stands for \textbf{application-specific integrated circuit}.
        It is a IC customized for a particular use, rather than intended for
        general-purpose use.
    \hii{FPGA}
        \par A field-programmable gate array (FPGA) is an integrated circuit
        designed to be configured by a customer or a designer after manufacturing
        – hence "field-programmable". The FPGA configuration is generally
        specified using a hardware description language (HDL), similar to that
        used for an application-specific integrated circuit (ASIC).
    \hii{Differences between ASIC and FPGA}
    \begin{center}
        \begin{tabular}{|c|c|}
            \hline 
            ASIC & FPGA \\ 
            \hline 
            specific-purpose IC & reprogrammable IC \\ 
            \hline 
            cannot be altered after production & can be corrected/updated \\ 
            \hline 
            & commonly used to design and test \\ 
            \hline 
            waste very little material &  \\ 
            \hline 
            & \specialcell{more cost-effective for smaller designs \\ or lower production volumes} \\ 
            \hline 
        \end{tabular} 
    \end{center}

\hi{FPGA design flow}
    \hii{Write a specification}
        \par Specification allows engineers to understand the design.
    \hii{Design - top-down method}
        \par High level functions are defined first, then the lower level implementation
        details are filled in later.
    \hii{Simulate}
        \par This step is for debugging.
    \hii{Synthesize}
        \par Turn the abstract form of desired circuit behavior into the design
        implementation in terms of logic gates.
    \hii{Place and Route}
    \hii{Resimulate}

\hi{Using Verilog in Quartus}
    \hii{Create a Verilog file}
        \begin{itemize}
            \item File $\to$ New $\to$ Verilog HDL File
            \item File $\to$ Save As...
            \item Start coding. Note that the module name must be similar to the
            file name.
        \end{itemize}
    \hii{Using block diagram/schemantic file}
        \begin{itemize}
            \item Change "Hierarchy" to "File" in the dropdown menu on the file pane.
            \item Right click the file $\to$ Set as top-level entity
            \item Right click the file $\to$ Create Symbol Files for Current File
            \item File $\to$ New $\to$ Verification/Debugging Files/University
                Program VWF
            \item Click on the Symbol Tool, choose the verilog file to load the IC.
            \item To create new input/output, use the Pin Tool.
        \end{itemize}
    \hii{Simulate the design}
        \begin{itemize}
            \item File $\to$ New $\to$ Block diagram/Schemantic File
            \item Edit $\to$ Insert $\to$ Insert Node or Bus...
            \item Click on Node Finder
            \item Click on List to list all available nodes
            \item Add nodes to the right pane and click OK twice.
        \end{itemize}