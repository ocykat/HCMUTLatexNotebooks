\chapter{Data Structures}

\hi{Heap}
  \hii{Definition}
    \par A \tb{max heap} is a nearly completed binary tree in which the
      \tb{max heap property} holds:
      \begin{align*}
        heap[parent(i)] \geq heap[i]
      \end{align*}
    \par Usually, a heap can be implemented using an array.
    \par If the array is 1-based, then for a node $i$:
    \begin{itemize}
      \item Parent: $\floor{i / 2}$
      \item Left child: $2i$
      \item Right child: $2i + 1$
    \end{itemize}
  \hii{Maintaining heap property}
    \par The function \texttt{MAX-HEAPIFY(i)} assumes that: \tb{the binary
      trees rooted at \texttt{LEFT(i)} and \texttt{RIGHT(i)} are heaps
      already.}
    \par Complexity of \texttt{MAX-HEAPIFY(i)}: $O(\log(n))$
    \begin{algorithm}
      %\caption{}
      \begin{algorithmic}[1]
        \FUNCTION{MAX-HEAPIFY}{i}
        \ENDFUNCTION
      \end{algorithmic}
    \end{algorithm}
  \hii{Building Heap}
    \begin{algorithm}
      %\caption{}
      \begin{algorithmic}[1]
        \PROCEDURE{BUILD-MAX-HEAP}{a}
          \FOR{i}{\floor{a.length/2} \DOWNTO 1}
            \CALLFUNC{MAX-HEAPIFY}{i}
          \ENDFOR
        \ENDPROCEDURE
      \end{algorithmic}
    \end{algorithm}
    \par Loop invariant of \texttt{BUILD-MAX-HEAP}:
    \begin{itemize}
      \item \tb{Initialization}: Prior to the first iteration,
        $i = \floor{n/2}$. $\forall j > \floor{n/2}$, $j$ is a leaf and thus
        the root of a trivial max heap.
      \item \tb{Maintenance}: $left(i) > i$ and $right(i) > i$ are both root
        of max heaps. This is the required condition for \texttt{MAX-HEAPIFY}.
        \texttt{MAX-HEAPIFY} ensures that all nodes from $i$ to $n$ are root
        of max heap. Decrementing $i$ reestablishes the loop invariant for the
        next iteration.
      \item \tb{Termination}: At termination, $i = 0$. By the loop invariant,
        all nodes from 1 to $n$ are roots of max heaps.
    \end{itemize}

