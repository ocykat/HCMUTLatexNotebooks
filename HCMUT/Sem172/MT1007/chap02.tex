\chapter{Vector Space}


\hi{Vector Space}
  \hii{Definition}
    \par Let $V$ be a non-empty set. Elements in $V$ are vectors. There exists
    2 operations in $V$:
    \begin{itemize}
      \item Adding: Sum of two vectors
      \item Scaling: Multiplication of a vector by a number
    \end{itemize}

    \par The set $V$ is called a vector space over $K$, if the 10 following
    axioms are held:
    \begin{enumerate}
      \item $\forall x, y \in V: x + y \in V$.
      \item $\forall x \in V, \forall \alpha \in K: \alpha x \in V$.
      \item $\forall x, y \in V: x + y = y + x$.
      \item $\forall x, y \in V: (x + y) + z = x + (y + z)$.
      \item $\forall x \in V: \exists v = 0: x + v = x + 0 = x$.
      \item $\forall x \in V: \exists v = (-x): x + v = x + -x = 0$
      \item $\forall x, y \in V: \forall \alpha \in K:
        \alpha (x + y) = \alpha x + \alpha y$.
      \item $\forall x \in V: \forall \alpha, \beta \in K:
        (\alpha + \beta) x = \alpha x + \beta x$.
      \item $\forall x \in V: \forall \alpha, \beta \in K:
        (\alpha\beta) x = \alpha(\beta x)$.
      \item $\forall x \in V: x \cdot 1 = x$.
    \end{enumerate}

  \hii{Notations}
    \hiii{Real Vector Space}
      \par Let $V$ be a vector space over $\mathbb{R}$ that contains all
        tuples of $n$ numbers. $(n \in \mathbb{N*})$
      \par $V$ is denoted by: $R_{n}$.
      \par Note that $R_{1}$ is also a vector space.

    \hiii{Matrix Vector Space}
      \par Let $V$ be a vector space that contains all real matrices of size
      $m \times n$.
      \par $V$ is denoted by: $M_{m \times n}[R]$.
      \par If $V$ is a square matrix of order $n$, then $V$ can also be denoted
      by $M_{n}[R]$.

    \hiii{Polynomial Vector Space}
      \par Let $V$ be a vector space that contains all polynomials with real
      coefficients and degree of $n$ or less.
      \par $V$ is denoted by: $P_{n}[x]$.


\hi{Linear Independence}
  \par Let $M = \{v_{1}, v_{2}, \ldots, v_{m}\} \subseteq V$.

  \hii{Linear Dependent Set \& Linear Independent Set}

    \hiii{Linear Dependent Set}
      \par $M$ is called a \tb{linear dependent} set if:
      \[
        \exists (\alpha_{1}, \alpha_{2}, \ldots, \alpha_{m}):
        \begin{cases}
          \exists i: i \in [1, m], \alpha_{i} \neq 0 \\
          \alpha_{1} v_{1} + \alpha_{2} v_{2}
            + \ldots + \alpha_{m} v_{m} = 0
        \end{cases}
      \]

    \hiii{Linear Dependent Set}
      \par $M$ is called a \tb{linear independent} set if it is
      not a linear dependent set.
      \par \ti{Equivalent definition}:
      \par If
      \begin{align*}
          \alpha_{1} v_{1} + \alpha_{2} v_{2}
            + \ldots + \alpha_{m} v_{m} = 0 \\
          \rightarrow \alpha_{1} = \alpha_{2}
            = \ldots = \alpha_{m} = 0
      \end{align*}
      then $M$ is independent.

  \hii{Linear Combination}
       \par $v$ is called a \tb{linear combination} of $M$ if:
      \[
        \exists (\alpha_{1}, \alpha_{2}, \ldots, \alpha_{m}):
          \alpha_{1} v_{1} + \alpha_{2} v_{2}
            + \ldots + \alpha_{m} v_{m} = v
      \]

  \hii{Spanning Set}
    \par $M$ is called a \tb{spanning set} of $V$ if:
    \[
      \forall v \in V: v \mbox{ is a linear combination of } M
    \]

  \hii{Basis}
    \par $M$ is called a \tb{basis} of $V$ if:
    \begin{itemize}
      \item $M$ is a \tb{spanning set} of $V$.
      \item $M$ is \tb{independent}.
    \end{itemize}

  \hii{Dimension}
    \par If $V$ has a basis that contains a finite number of elements, then
      $V$ is a \tb{finite dimensional vector space}.
    \par The number of elements in the basis is called the \tb{dimension} of
      $V$ and is denoted by $dim(V)$.
    \par Note: $dim(R_{n}) = n$.


\hi{Rank of a Set of Vectors}
  \hii{Definition}
    \par Let $M = \{v_{1}, v_{2}, \ldots, v_{m}\} \subseteq V$.
    \par The \tb{rank} of the set $M$ is the \ti{maximum number of linear
    independent} vectors from $M$.
    \par $rank(M) = x$ means that:
    \begin{itemize}
      \item There exists $x$ linear independent vectors from $M$.
      \item Any subset of $M$ that contains more than $x$ vectors
      is \ti{dependent}.
    \end{itemize}

  \hii{Relationship between Rank of a Set of Vectors and Rank of a Matrix}
    \par Given a matrix $A$.
    \par If $M$ is the \tb{set of row vectors} of $A$, and
            $N$ is the \tb{set of column vectors} of $A$, then:
    \[
      rank(A) = rank(M) = rank(N)
    \]

  \hii{Properties}
    \par Let $M = \{v_{1}, v_{2}, \ldots, v_{m}\} \subseteq V$.
    \begin{enumerate}
      \item If $rank(M) = m$ then $M$ is \tb{independent}.
      \item If $rank(M) < m$ then $M$ is dependent.
      \item If $rank(M) = dim(V)$ then $M$ is a \tb{spanning set} of $V$.
      \item If $rank(M) < dim(V)$ then $M$ is NOT a \tb{spanning set} of $V$.
      \item If $rank(M) = dim(V) = m$ then $M$ is a \tb{basis} of $V$.
    \end{enumerate}


\hi{Standard basis}
  \hii{Standard basis of a Real Vector}
    \begin{align*}
      dim(R_{n}) = n \\
      %E = \{1, x, \ldots, x^{n}\}
    \end{align*}

  \hii{Standard basis of a Matrix Vector}
    \begin{align*}
      dim(M_{m \times n}[R]) = m \times n \\
      %E = \{1, x, \ldots, x^{n}\}
    \end{align*}

  \hii{Standard basis of a Polynomial Vector}
    \begin{align*}
      dim(P_{n}[x]) = n + 1 \\
      E = \{1, x, \ldots, x^{n}\}
    \end{align*}


\hi{Coordinates}
  \hii{Coordinates}
  \begin{itemize}
    \item \tb{Definition}
    \par Let $E = \{e_{1}, e_{2}, \ldots, e_{n}\}$ be an \tb{ordered basis} for
    $V$ and $x \in V$.
    \[
      x \mbox{ is a linear combination of } E
        \Leftrightarrow x = x_{1} e_{1} + x_{2} e_{2} + \ldots + x_{n} e_{n}
    \]
    \par $n$ numbers $(x_{1}, x_{2}, \ldots, x_{n})$ are called \tb{coordinates}
    of the vector $x$ with respect to $E$ and are denoted by:
    \[
      [x]_{E} =
      \begin{pmatrix}
        x_{1}  \\
        x_{2}  \\
        \ldots \\
        x_{n}  \\
      \end{pmatrix}
    \]
    \par \tb{Equation}
    \begin{eqbox}
      [x]_{E} = E^{-1} \cdot x
    \end{eqbox}

    \par \tb{Explanantion}
      \begin{smfont}
        \par This form is a bit easier to interpret:
          \begin{eqbox}
            x = E \cdot [x]_{E}
          \end{eqbox}
        \par Interpretation: $[x]_{E}$ is the original vector.
        $x$ is the result after applying the linear transformation $E$
        on $[x]_{E}$.
      \end{smfont}
  \end{itemize}

  \hii{Change-of-basis Matrix}
    \begin{itemize}
      \item \tb{Equations}
        \begin{eqbox}
          [x]_{E} = A [x]_{E'} \\
          A = E^{-1} \cdot E'
        \end{eqbox}
      \item \tb{Proof 1}
        \begin{align*}
          E  = \{e_{1},  e_{2},  \ldots, e_{n}\}  \\
          E' = \{e_{1}', e_{2}', \ldots, e_{n}'\} \\
        \end{align*}
        be 2 bases of $V$.

        \par Let $x$ be an element of $V$.
        \begin{align*}
          x = x_{1} e_{1}   + x_{2} e_{2}   + \ldots + x_{n} e_{n}
            \Rightarrow [x]_{E} =
              \begin{pmatrix}
                x_{1}  \\
                x_{2}  \\
                \ldots \\
                x_{n}  \\
              \end{pmatrix} \qquad (1) \\
          x = x_{1}' e_{1}' + x_{2}' e_{2}' + \ldots + x_{n}' e_{n}'
            \Rightarrow [x]_{E'} =
              \begin{pmatrix}
                x_{1}'  \\
                x_{2}'  \\
                \ldots \\
                x_{n}'  \\
              \end{pmatrix} \qquad (2) \\
        \end{align*}

        \begin{align*}
          e_{1}' = a_{11} e_{1} + a_{21} e_{2} + \ldots + a_{n1} e_{n} \\
          e_{2}' = a_{12} e_{1} + a_{22} e_{2} + \ldots + a_{n1} e_{n} \\
          \ldots \\
          e_{n}' = a_{1n} e_{1} + a_{2n} e_{2} + \ldots + a_{nn} e_{n} \\
        \end{align*}

        \par Substitute into $(2)$:
        \begin{align*}
          x &= x_{1}'(a_{11} e_{1} + a_{21} e_{2} + \ldots + a_{n1} e_{n}) & \\
            &+ x_{2}'(a_{12} e_{1} + a_{22} e_{2} + \ldots + a_{n2} e_{n}) & \\
            &+ \ldots &\\
            &+ x_{n}'(a_{1n} e_{1} + a_{2n} e_{2} + \ldots + a_{nn} e_{n}) &\\
        \end{align*}

        \begin{align*}
          \Rightarrow \left\{
          \begin{aligned}
            x_{1} &= a_{11}x_{1}' + a_{12}x_{2}' + \ldots + a_{1n}x_{n}' \\
            x_{2} &= a_{21}x_{1}' + a_{22}x_{2}' + \ldots + a_{2n}x_{n}' \\
            \ldots & \ldots \\
            x_{n} &= a_{n1}x_{1}' + a_{n2}x_{2}' + \ldots + a_{nn}x_{n}' \\
          \end{aligned}
          \right.
        \end{align*}

        \begin{align*}
          \Leftrightarrow
          \begin{pmatrix}
            x_{1}  \\
            x_{2}  \\
            \ldots \\
            x_{n}  \\
          \end{pmatrix}
          =
          \begin{pmatrix}
            a_{11} & a_{12} & \ldots & a_{1n} \\
            a_{21} & a_{22} & \ldots & a_{2n} \\
            \vdots & \vdots & \vdots & \vdots \\
            a_{n1} & a_{n2} & \ldots & a_{nn} \\
          \end{pmatrix}
          \begin{pmatrix}
            x_{1}' \\
            x_{2}' \\
            \ldots \\
            x_{n}' \\
          \end{pmatrix}
        \end{align*}

        \begin{align*}
          [x]_{E} = A[x]_{E'}
        \end{align*}

        \par The matrix $A$ is called a \tb{change-of-basis matrix} from $E$ to $E'$.

        \par Also:
        \begin{align*}
          A &=
          \begin{pmatrix}
            [e_{1}']_{E} & [e_{2}']_{E} & \ldots & [e_{n}']_{E}
          \end{pmatrix} \\
          &=
          \begin{pmatrix}
            E^{-1}e_{1}' & E^{-1}e_{2}' & \ldots & E^{-1}e_{n}'
          \end{pmatrix} \\
          &= E^{-1}
          \begin{pmatrix}
            e_{1}' & e_{2}' & \ldots & e_{n}'
          \end{pmatrix} \\
          &= E^{-1} \times E'
        \end{align*}
      \item \tb{Proof 2}
        \begin{smfont}
          \par Let $[x]_{E}$ and $[x]_{E'}$ be two vectors.
          \par After applying the linear transformations $E$ onto $[x]_{E}$ and
            $E'$ onto $[x]_{E'}$, we obtain the same result x for both transformations.
          \begin{flalign*}
            & x = E \cdot [x]_{E} = E' \cdot [x]_{E'} && \\
            & \ra [x]_{E} = E^{-1} \cdot E' \cdot [x]_{E} && \\
            & \ra A = E^{-1} \cdot E' && \\
          \end{flalign*}
        \end{smfont}
    \end{itemize}


\hi{Subspace}
  \hii{Definition}
    \par A subset $F$ of the vector space $V$ is called a \tb{subspace}
      of $V$, if the subset $F$ is a vector space.

  \hii{Theorem}
    \par A nonempty subset $F$ of the vector space $V$ over $K$ is a
      subspace of $V$, if the following conditions hold:
    \begin{itemize}
      \item $\forall f_{1}, f_{2} \in F: (f_{1} + f_{2}) \in F$.
      \item $\forall f \in F, \forall \alpha \in K: \alpha f \in F$.
    \end{itemize}

  \hii{Examples}
    \begin{itemize}
      \item $F$ is a straight line that passes through the origin.
        \begin{itemize}
          \item Basis: $\{\vec{a}\}$ where $\vec{a}$ is the
            directional vector of the line.
          \item $dim(F) = 1$ since there is $1$ element in the basis.
        \end{itemize}
      \item $F$ is a plane that contains the origin.
        \begin{itemize}
          \item Basis: $\{\vec{a_{1}}, \vec{a_{2}}\}$, where
            $\vec{a_{1}}, \vec{a_{2}}$ are two vectors lying in
            the plane.
          \item $dim(F) = 2$
        \end{itemize}
      \item $F = \{(0, 0, 0)\}$
        \begin{itemize}
          \item Basis: No basis, because the zero vector cannot exists
            in a basis.
          \item $dim(F) = 0$
        \end{itemize}
      \item $F = R^{3}$
        \begin{itemize}
          \item Basis: $\{\vec{OA}, \vec{OB}, \vec{OC}\}$ where
            $OABC$ is a tetrahedron.
          \item $dim(F) = 3$
        \end{itemize}
    \end{itemize}

  \hii{Subspace and the Origin}
    \par \tb{Problem}: Prove that any subspace $F$ in $V = R^{3}$ must contain
      the origin $(0, 0, 0)$.
    \par \tb{Proof}: \\
    \ti{
      Suppose that $(0, 0, 0) \notin F$. \\
      According to the theorem: $\forall f \in F, \forall \alpha \in K:
        \alpha f \in F$.
      \begin{flalign*}
        &\ra 0 \cdot f = (0, 0, 0) \in F && \\
      \end{flalign*}
      This contradicts the assumption. Therefore, the subspace $F$ must contain
      the origin.
    }

  \hii{Relationship of two subspaces}
    \hiii{Intersection and Sum}
      \par Let $F$ and $G$ be two subspaces of V.
      \begin{itemize}
        \item The subset $F \cap G = \{\forall x \in V | x \in F, x \in G\}$ is
          called the \tb{intersection} of $F$ and $G$.
        \item The subset $F + G = \{f + g | f \in F, g \in G\}$ is
          called the \tb{sum} of $F$ and $G$.
      \end{itemize}
    \hiii{Theorem}
      \par Let $F$ and $G$ be two subspaces of $V$.
      \begin{itemize}
        \item Both the intersection and the union of $F$ and $G$ are also
          subspaces of $V$.
        \item $dim(F + G) = dim(F) + dim(G) - dim(F \cap G)$
      \end{itemize}
