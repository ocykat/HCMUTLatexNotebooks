\chapter{Introductory Concepts}

\hi{Numerical Representation}
    \hii{Analog Representations}
        \par In \textbf{analog representation}, a quantity is represented by a continuously
        variable, proportional indicator.
        \par Analog quantities can vary over a continuous range of values.
    \hii{Digital Representations}
        \par In \textbf{digital representation} the quantities are represented not by
        continuously variable indicators but by symbols called digits.
    \hii{Difference between two types of representation}
        \begin{center}
            analog $\equiv$ continuous \\
            digital $\equiv$ discrete (step by step)
        \end{center}

\hi{Digital and Analog Systems}
    \hii{Digital and Analog Systems}
        \par A \textbf{digital system} is a combination of devices designed to manipulate logical
        information or physical quantities that represented in digital form.
        \par A \textbf{analog system} contains devices that manipulate physical quantities that
        are represented in analog form.
    \hii{Advantages of Digital Techniques}
        \begin{enumerate}
            \item Ease of design
            \item Ease of information storage
            \item Accuracy/Precision
            \item Programmable operation
            \item More digital circuitry can be fabricated \footnotemark on IC chips.
                \footnotetext{fabricate: construct or manufacture (something, especially an
                industrial product), especially from prepared components}
        \end{enumerate}
    \hii{Limitations of Digital Techniques}
        \begin{enumerate}
            \item The real world is analog
            \item Processing digitized signals takes time.
        \end{enumerate}
    \hii{Implement Digital Techniques}
        \par To take advantage of digital techniques when dealing with analog inputs and outputs,
        four steps must be followed:
        \begin{enumerate}
            \item Convert the physical variable to an electrical (analog) signal.
            \item Convert the electrical (analog) signal into digital form.
            \item Process (operate on) the digital information.
            \item Convert the digital outputs back to real-world analog form.
        \end{enumerate}
    
\hi{Digital Number Systems}
    \par \textbf{Terminology}
    \begin{itemize}
        \item bit: binary digit
        \item most significant bit (MSB): leftmost bit (largest bit)
        \item least significant bit (LSB): rightmost bit (smallest bit)
    \end{itemize}

\hi{Representing Binary Quantities}
    \par In electronic digital systems, binary information is represented by voltages (or
    currents).
    \par Because of circuit variations, the 0 and 1 would be represented by voltage ranges.
    For example:
    \begin{itemize}
        \item 0: 0 - 0.8V
        \item 1: 2 - 5V
    \end{itemize}
    \par Another difference between digital and analog systems: in digital systems, the exact
    value of a voltage is not important. Therefore, the design of accurate digital circuitry is
    simpler.

\hi{Digital Circuits/Logic Circuits}
    \hii{Digital circuits}
        \par Digital circuits are designed to produce output voltage that fall within the
        prescribed 0 and 1 voltage ranges.
    \hii{Logic Circuits}
        \par Each type of digital circuit obeys a certain set of logic rules. For this reason, 
        digital circuits are also called logic circuit.
    \hii{Digital Integrated Circuits}
        \par An integrated circuit (or IC, a chip, or a microchip) is a set of electronic
        circuits on one small flat piece (or "chip") of semiconductor material, normally silicon.
        \par Almost all of the digital circuits used in modern digital systems are integrated
        circuits (ICs). The wide variety of available logic ICs has made it possible to construct
        complex digital systems that are smaller and more reliable than their discrete-component
        counterparts.

\hi{Parallel and Serial Transmission}
    \par There are two basic methods for digital information transmission: parallel and serial.
    \begin{itemize}
        \item Parallel transmission: information is sent in $n$ bit at a time through $n$ wires.
        \item Serial transmission: information is sent in 1 bit at a time through a single wire.
    \end{itemize}
    \par The principle trade-off between parallel and serial representations is one of speed
    versus circuit simplicity: parallel transmission is faster but it requires more wires to
    operate.

\hi{Memory}
    \par In a nonmemory circuit, the output changes when the input signal is applied and returns
    to its original state when the input signal is removed.
    \par In a memory circuit, the output also changes when the input signal is applied. However,
    it will remain in the new state even after the input is removed.

\hi{Digital Computers}
    \hii{Digital Computers}
        \par In simplest terms, a computer is a system of hardware that performs arithmetic
        operations, manipulates data (usually in binary form), and make decisions.
    \hii{Programs}
        \par A program is a set of instructions for a computer to carry out a particular job.
    \hii{Major Parts of a Computer}
        \begin{tikzpicture}[node distance=5cm]
            \node (input)   [rect] {Input};
            \node (control) [rect, right of=input] {Control};
            \node (arith)   [rect, above of=control, yshift=-2cm] {Arithmetic/Logic};
            \node (memory)  [rect, below of=control, yshift=2cm] {Memory};
            \node (output)  [rect, right of=control] {Output};
            \draw [twarrow]
                (input)   |- (arith);
            \draw [arrow]
                (input)   |- (memory);
            \draw [twarrow]
                (arith)   .. controls (2, 1) and (2, -1) .. (memory);
            \draw [darrow]
                (control) -- (input);
            \draw [darrow]
                (control) -- (arith);
            \draw [darrow]
                (control) -- (memory);
            \draw [twarrow, transform canvas={xshift=0.5cm}]
                (control) -- (memory);
            \draw [darrow]
                (control) -- (output);
            \draw [arrow]
                (arith) -| (output);
            \draw [arrow]
                (memory) -| (output);
            \draw [darrow]
                (control) -- (output);
        \end{tikzpicture}
    
        \begin{itemize}
            \item \textbf{Input unit}: Through this unit, a complete set of instructions and data
                is fed into the computer system and into the memory unit, to be stored until
                needed.
            \item \textbf{Memory unit}: The memory stores the instructions and data received from
                the input unit. It stores the results of arithmetic operations received from
                the arithmetic unit. It also supplies information to the output unit.
            \item \textbf{Control unit}: This unit takes instructions from the memory unit one
                at a time and interprets them. It then sends appropriate signals to all the
                other units to cause the specific instruction to be executed.
            \item \textbf{Arithmetic/logic unit}: All arithmetic calculations and logical
                decisions are performed in this unit, which can then send results to the memory
                unit to be stored.
            \item \textbf{Output unit}: This unit takes data from the memory unit and prints out,
                displays, or otherwise presents the information to the operator (or process,
                in the case of a process control computer).
        \end{itemize}
    \hii{Central Processing Unit (CPU)}
        \par The control and arithmetic/logic units are often considered as one unit, called the
        \textbf{central processing unit (CPU)}.
