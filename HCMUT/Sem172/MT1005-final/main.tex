% \documentclass[9pt, landscape, a4paper]{article}
\documentclass[9pt, landscape, a4paper]{scrartcl}
\addtokomafont{disposition}{\rmfamily}
% \KOMAoptions{
%     % DIV=calc,
%     fontsize=10pt,
% }


% MARGINS
\usepackage{geometry}
\geometry{top=0.75cm,bottom=0.75cm,left=0.75cm,right=0.75cm,includehead,includefoot}

% LANDSCAPE
% \usepackage{pdflscape}

% MULTI-COLUMN
\usepackage{multicol}
\setlength{\columnsep}{0.5cm}
\setlength{\columnseprule}{0.2pt}
\newcommand{\colbreak}{\vfill\null\columnbreak}

% INDENTATION
\usepackage{indentfirst}

% TABLE OF CONTENT
% \usepackage[tocflat]{tocstyle}
% \usetocstyle{standard}
% \usepackage[hidelinks]{hyperref}
% \hypersetup{
%   colorlinks,
%   citecolor=black,
%   filecolor=black,
%   linkcolor=black,
%   urlcolor=black
% }


% HEADINGS
% counter depth
\setcounter{secnumdepth}{4}
% macros
\newcommand{\hi}{\section}
\newcommand{\hii}{\subsection}
\newcommand{\hiii}{\subsubsection}
\newcommand{\hiiiBEGIN}[1]{\subsubsection \begin{enumerate}}
\newcommand{\hiiiEND}{\end{enumerate}}
\newcommand{\hiv}{\item\textbf}


% IMAGES
\usepackage{graphicx}
\usepackage{subcaption}
\usepackage{float}
\newcommand{\img}[2][]
  {
    \begin{figure}[H]
      \centering
      \includegraphics[#1]{#2}
    \end{figure}
  }

% MATH
\usepackage{amsmath}
\usepackage{amssymb}
\usepackage{gensymb}
\newcommand{\sqbr}[1]{[#1]}
\newcommand{\ls}{<}
\newcommand{\gr}{>}
% equation boxes
\usepackage{empheq}
\newenvironment{eqbox}
  {\setkeys{EmphEqEnv}{align}\setkeys{EmphEqOpt}{box=\fbox}\EmphEqMainEnv}
  {\endEmphEqMainEnv}
% bold in math env
\usepackage{amsbsy}
% word above equal sign
\usepackage{mathtools}
% \newcommand\eqsign[1]{\stackrel{\mathclap{\normalfont\mbox{#1}}}{=}}
\newcommand\eqsign[1]{\mathrel{\overset{\makebox[0pt]{\mbox{\normalfont\tiny\sffamily #1}}}{=}}}
% macros
\newcommand{\vt}{\overrightarrow}
\newcommand{\avg}{\overline}
\newcommand{\ra}{\Rightarrow}
\newcommand{\lra}{\Leftrightarrow}
\newcommand{\dt}{\Delta}
\newcommand{\dif}[2]{\frac{d #1}{d #2}}
\newcommand{\pd}[2]{\frac{\partial #1}{\partial #2}}
\newcommand{\pdd}[3]{\frac{\partial #1}{\partial #2 \partial #3}}
\newcommand{\INT}{\int \limits}
\newcommand{\IINT}{\iint \limits}
\newcommand{\IIINT}{\iiint \limits}
\newcommand{\OINT}{\oint \limits}
\newcommand{\SUM}{\sum \limits}

% TEXT
% Bold, italic, underlined text
\newcommand{\tb}[1]{\textbf{#1}}
\newcommand{\ti}[1]{\textit{#1}}
\newcommand{\tbi}[1]{\textbf{\textit{#1}}}
\newcommand{\tu}[1]{\underline{#1}}
\newcommand{\tbu}[1]{\textbf{\underline{#1}}}


% FOOTNOTE
% + one footnote stays on one page
\interfootnotelinepenalty=10000

% INFO
\title{\vspace{-4ex}\Large{Calculus 2 - Final Exam Notes}}
\author{Nhat M. Nguyen}
\date{June 2018}

\begin{document}
  % \begin{landscape}
    \maketitle
    \img[width=3cm]{logo.jpeg}
      % \setcounter{secnumdepth}{2}
      \setcounter{tocdepth}{2}
      % \begin{multicols}{2}
      \tableofcontents
      % \end{multicols}
      % \begin{minipage}{\columnwidth}
  \hi*{Test Contents}
   \begin{itemize}
      \item Triple Integral
        \begin{itemize}
          \item Separable form
          \item Cylindrical form
          \item Spherical form
        \end{itemize}
      \item Line Integral
        \begin{itemize}
          \item $1^{st}$ type
          \item $2^{st}$ type
          \item Independence of Path
          \item Green's Theorem
        \end{itemize}
      \item Surface Integral
        \begin{itemize}
          \item $1^{st}$ type
          \item $2^{st}$ type
        \end{itemize}
      \item Series
        \begin{itemize}
          \item Numeric Series
          \item Power Series
        \end{itemize}
    \end{itemize}
\end{minipage}
      \clearpage

\begin{multicols}{3}

\hi{Triple Integrals}

  \par \tb{General form}
    \begin{eqbox}
      I = \IIINT_{E} f(x, y, z) dV
    \end{eqbox}

  \hii{Separable Form}
    \hiii{Domain}
      \begin{equation*}
        E = [a, b] \times [c, d] \times [r, s]
      \end{equation*}

    \hiii{Method}
      \par If $f(x, y, z) = f_{1}(x) \times f_{2}(y) \times f_{3}(z)$
        then
        \begin{equation}
          I = \INT_{a}^{b} f_{1}(x)dx \times
              \INT_{c}^{d} f_{2}(y)dy \times
              \INT_{r}^{s} f_{3}(z)dz
        \end{equation}

    \hiii{Applications}
      \begin{itemize}
        \item \tb{Volume of solid}
          \begin{equation*}
            V = \IIINT_{E} dV
          \end{equation*}
        \item \tb{Mass}
          \par Given volume mass density $\rho(x, y, z)$:
          \begin{equation*}
            m = \IIINT_{E} \rho dV
          \end{equation*}
        \item \tb{Charge}
          \par Given charge density $\sigma(x, y, z)$:
          \begin{equation*}
            q = \IIINT_{E} \sigma dV
          \end{equation*}
      \end{itemize}

\colbreak

  \hii{Cylindrical Form}
    \hiii{Domain}
      \par One of the three variables appears \tb{twice} $\to$ height.
      \par The other two variables $\to$ D. If $D$ is \tb{infinite}
        $\to$ \tb{intercept equation}.
      \par W.L.O.G:
        \footnotemark
        \footnotetext{any of the three variables can be the height}
        \begin{equation*}
          E =
          \begin{cases}
            h: z_{2} - z_{1} \\
            D: (x, y)
          \end{cases}
        \end{equation*}

    \hiii{Method}
      \begin{equation}
        I = \IINT_{D} \INT_{z_{1}}^{z_{2}} f(x, y, z) dV
      \end{equation}

    \hiii{Polar Coordinate}
      \par If $D$ is related to circle/disk:
        \begin{equation*}
          x^{2} + y^{2} = r^{2}
        \end{equation*}
        then
        \begin{equation*}
          E =
          \begin{cases}
            D_{r \theta} \mbox{ where } 
            \begin{cases}
              x = r\cos\theta \\
              y = r\sin\theta \\
            \end{cases} \\
            h: z_{2}(r\cos\theta, r\sin\theta) - z_{1}(r\cos\theta, r\sin\theta)
          \end{cases}
        \end{equation*}
      \begin{equation}
        I = \IINT_{D_{r \theta}}
          \INT_{z_{1}}^{z_{2}} f(r\cos\theta, r\sin\theta, z) \pmb{r} dz d\theta dr
      \end{equation}

\colbreak

  \hii{Spherical Form}
    \hiii{Domain}
      \par $E$ is related to
        \begin{itemize}
          \setlength\itemsep{0em}
          \item \tb{Sphere}: $x^{2} + y^{2} + z^{2} = R^{2}$
          \item \tb{Ball}: $x^{2} + y^{2} + z^{2} \leq R^{2}$
        \end{itemize}
    \hiii{Method}
        \begin{equation*}
          E =
            \begin{cases}
              0 \leq \rho \leq R \\
              0 \leq \theta \leq 2 \pi \\
              0 \leq \varphi \leq \pmb{\pi} \mbox{ (!) }
            \end{cases}
        \end{equation*}

        where
        \begin{equation*}
          \begin{cases}
            x = \rho \sin \varphi \cos \theta \\
            y = \rho \sin \varphi \sin \theta \\
            z = \rho \cos \varphi
          \end{cases}
        \end{equation*}

        \begin{equation}
          I = \IIINT_{E_{\rho \theta \varphi}}
            f(\rho \sin \varphi \cos \theta,
              \rho \sin \varphi \sin \theta,
              \rho \cos \varphi
            ) \pmb{\rho^{2} \sin \varphi}
            d \varphi d \theta d \rho
        \end{equation}

\end{multicols}
      \clearpage

\begin{multicols}{4}

\hi{Line Integrals}

  \hii{Type 1}
    \hiii{Form}
      \begin{itemize}
        \item \tb{Domain}: Curve $(C)$ between two points.
        \item \tb{Integral}:
          \begin{equation*}
            I = \INT_{(C)} f(x, y)\pmb{ds}
          \end{equation*}
      \end{itemize}

    \hiii{Method}
      \par General idea: convert to single integral of one variable.
        \begin{itemize}
          \item \tb{Case 1}:
            \begin{equation*}
              (C):
                \begin{cases}
                  y = g(x) \\
                  x_{1} \leq x \leq x_{2}
                \end{cases}
            \end{equation*}          

            \begin{equation}
              I = \INT_{x_{1}}^{x_{2}} f(x, g(x)) \sqrt{1 + [g'(x)]^{2}} dx
            \end{equation}

          \item \tb{Case 2}:
            \begin{equation*}
              (C):
                \begin{cases}
                  x = g(y) \\
                  y_{1} \leq y \leq y_{2}
                \end{cases}
            \end{equation*}

            \begin{equation}
              I = \INT_{y_{1}}^{y_{2}} f(g(y), y) \sqrt{1 + [g'(y)]^{2}} dy
            \end{equation}

        \end{itemize}

      \hiii{Application}
        \par Mass of string $(C)$ given length density $\rho(x, y)$.

\colbreak


  \hii{Type 2}
    \hiii{Form}
      \begin{itemize}
        \item \tb{Domain}: Curve $(C)$ between two points.
        \item \tb{Integral}:
          \begin{equation*}
            I = \INT_{(C)} \big[P(x, y)\pmb{dx} + Q(x, y)\pmb{dy}\big]
          \end{equation*}
      \end{itemize}

    \hiii{Method}
      \par General idea: convert to single integral of one variable.
        \begin{itemize}
          \item \tb{Case 1}:
            \begin{equation*}
              (C):
                \begin{cases}
                  y = g(x) \ra dy = g'(x)dx \\
                  x_{1} \to x_{2} \mbox{ (!) }
                \end{cases}
            \end{equation*}          

            \begin{equation}
              I = \INT_{(C)} \big[P(x, g(x)) + Q(x, g(x)) \cdot g'(x)\big] \pmb{dx}
            \end{equation}

          \item \tb{Case 2}:
            \begin{equation*}
              (C):
                \begin{cases}
                  x = g(y) \ra dx = g'(y)dy \\
                  y_{1} \to y_{2} \mbox{ (!) }
                \end{cases}
            \end{equation*}

            \begin{equation}
              I = \INT_{(C)} \big[P(g(y), y) \cdot g'(y) + Q(g(y), y)\big] \pmb{dy}
            \end{equation}
        \end{itemize}

    \hiii{Application}
      \par Work done by force field along the path $(C)$.
        \begin{itemize}    
          \item \tb{Force field}
            \begin{equation*}
              \vec{F} = [P(x, y), Q(x, y)] = P\vec{i} + Q\vec{j}
            \end{equation*}
          \item \tb{R.O.C of Displacement}
            \begin{equation*}
              d\vec{r} = (dx, dy)
            \end{equation*}
          \item \tb{Work}
            \begin{flalign*}
              & W = \INT_{(C)} \vec{F} d\vec{r} && \\
              &   \stackrel{\text{dot product}}{=}
              &      \INT_{(C)} \big[P(x, y)dx + Q(x, y)dy\big] &&
            \end{flalign*}
        \end{itemize}

% \colbreak

  \hii{Independent of Path}
    \hiii{Condition}
      \par The line integral \tb{type 2} is \ti{independent of path}
        if:
        \begin{equation}
          \pd{P}{y} = \pd{Q}{x}
        \end{equation}

    \hiii{Method}
      \par Simplest path:
        \begin{itemize}
          \item Going horizontally $x_{A} \to x_{B}$ with $y = y_{A}$.
          \item Going vertically $y_{A} \to y_{B}$ with $x = x_{B}$.
        \end{itemize}

        \begin{equation}
          I = \INT_{x_{A}}^{x_{B}} P(x, y_{A})
            + \INT_{y_{A}}^{y_{B}} Q(x_{B}, y)
        \end{equation}

    \hiii{Application}
      \par If force field $\vec{F}$ is conservative:
        \begin{align*}
          \pd{P}{y} = \pd{Q}{x}
        \end{align*}

\colbreak

  \hii{Green's theorem}
    \hiii{Form}
      \begin{itemize}
        \item $(C)$ is a \tb{closed curve}.
        \item \tb{Integral}
          \begin{equation*}
            I = \INT_{(C)} \big[P(x, y)\pmb{dx} + Q(x, y)\pmb{dy}\big]
          \end{equation*}
        \item \tb{Orientation}: walking on the boundary $(C)$ of the domain
          $\to$ the domain is on the \tb{left}.
      \end{itemize}
    \hiii{Method}
      \par Green's theorem
        \begin{equation}
          \OINT_{(C)} \big[P(x, y)dx + Q(x, y)dy\big]
          = \INT_{D} \bigg( \pd{Q}{x} - \pd{P}{y} \bigg)dA
        \end{equation}


\end{multicols}
      \clearpage

\begin{multicols}{2}

\hi{Surface Integrals}

  \hii{Type 1}
    \hiii{Form}
      \begin{itemize}
        \item \tb{Domain}: Surface $S$:
          \begin{equation*}
            S:
              \begin{cases}
                z = z(x, y) \\
                (x, y) \in D \subset (Oxy)
              \end{cases}
          \end{equation*}
        \item \tb{Integral}:
          \begin{equation*}
            I = \IINT_{S} f(x, y, z)\pmb{dS}
          \end{equation*}
      \end{itemize}

    \hiii{Method}
      \par General idea: convert to double integral of two variable.
        \begin{equation}
          I = \IINT_{(D)} f(x, y, z(x, y)) \sqrt{1 + z_{x}^{2} + z_{y}^{2}} dA
        \end{equation}

      \hiii{Application}
        \par Mass of lamina $S$ given surface density $\rho(x, y, z)$.

  \hii{Type 2}
    \hiii{Form}
      \begin{itemize}
        \item \tb{Domain}: oriented surface $\sigma$:
          \begin{equation*}
            \sigma:
              \begin{cases}
                z = z(x, y) \\
                (x, y) \in D \subset (Oxy) \\
                \mbox{ orientation: } +/-
              \end{cases}
          \end{equation*}
        \item \tb{Integral}:
          \begin{equation*}
            I = \IINT_{\sigma} \big[P(x, y, z)\pmb{dydz} + Q(x, y, z)\pmb{dzdx} + R(x, y, z)\pmb{dxdy}\big]
          \end{equation*}
      \end{itemize}

    \hiii{Method}
      \begin{itemize}
        \item $(+)$ orientation: $\vec{n} = (-z_{x}, -z_{y}, 1)$
          \begin{equation}
            I = \IINT_{D} \big[-Pz_{x} - Qz_{y} + R\big] dA
          \end{equation}
        \item $(-)$ orientation: $\vec{n} = (z_{x}, z_{y}, -1)$
          \begin{equation}
            I = \IINT_{D} \big[+Pz_{x} + Qz_{y} - R\big] dA
          \end{equation}
      \end{itemize}

    \hiii{Application}
      \par Flux through a surface $\sigma$.
      \begin{itemize}
        \item \tb{Vector field}
          \begin{equation*}
            \vec{F} = \big[P(x, y, z), Q(x, y, z), R(x, y, z)\big]
                    = P\vec{i} + Q\vec{j} + R\vec{k}
          \end{equation*}
        \item \tb{Normal vector} $\vec{n}$
        \item \tb{Flux}
          \begin{equation}
            I = \IINT_{\sigma} \vec{F}\vec{n}dS
              = \IINT_{\sigma} \big[Pdydz + Qdzdx + Rdxdy\big]
          \end{equation}
      \end{itemize}


\end{multicols}
      \clearpage

\begin{multicols}{2}

\hi{Infinite Series}
  \hii{Numerical Series}
    \hiii{Definition}
      \begin{itemize}
        \item \tb{Series}
          \begin{equation*}
            \sum a_{n} = a_{1} + a_{2} + \ldots + a_{n}
          \end{equation*}
        \item \tb{Partial Sum}
          \begin{equation*}
            \sum S_{n} = a_{1} + a_{2} + \ldots + a_{n} = \SUM_{i = 1}^{n} a_{i}
          \end{equation*}
      \end{itemize}

    \hiii{Convergence - Divergence}
      \begin{itemize}
        \item Case 1: $lim_{n \to \infty} S_{n}$ exists.
          \par $\ra S_{n}$ converges, $\sum a_{n}$ is \tb{convergent}.
        \item Case 2: $lim_{n \to \infty} S_{n}$ does not exists.
          \par $\ra S_{n}$ diverges, $\sum a_{n}$ is \tb{divergent}.
      \end{itemize}

    \hiii{Geometric series}
      \begin{itemize}
        \item \tb{Series}
          \begin{equation*}
            a, ar^{1}, ar^{2}, \ldots, ar^{n}, \ldots
          \end{equation*}
        \item \tb{Simplified Series (SS)}
          \begin{equation*}
            \pmb{1}, r^{1}, r^{2}, \ldots, r^{n}, \ldots
          \end{equation*}        
        \item \tb{Sum of SS}
          \begin{equation}
            S = \pmb{1} + r^{1} + r^{2} + \ldots + r^{n} + \ldots
              = \SUM_{\pmb{n = 0}}^{\infty} r^{n}
          \end{equation}
          \par \ti{Note: $n$ starts at $0$, first term is $1$}.
          \begin{itemize}          
            \item $|r| < 1$: converges
              \begin{equation}
                S = \frac{1}{1 - r}
              \end{equation}
            \item $|r| > 1$: diverges
          \end{itemize}
      \end{itemize}

  \hii{Power series}
    \hiii{Form}
      \begin{equation*}
        S = c_{0} + c_{1}x + c_{2}x^{2} + \ldots + c_{n}x^{n} \ldots = \SUM_{n = 0}^{n} c_{n}x^{n}
      \end{equation*}
    \hiii{Interval of Convergence (IoC)}
      \begin{itemize}
        \item \tb{Special case}: $\forall n: c_{n} = 1$ (geometric series)
          \par $S$ converges $\lra |x| < 1$
          $\ra$ \tb{IoC}: $x \in (-1, 1)$
        \item \tb{General case}:
          \par \tb{IoC}: $x \in (-R, R)$
      \end{itemize}
      \par \tb{Finding IoC}:
        \begin{align*}
          \lim_{n \to \infty} \sqrt[n]{|c_{n}x^{n}|} = |x| \cdot R^{-1} < 1 \ra x \in (-R, R)
        \end{align*}
    \hiii{Some useful limits}
      \begin{flalign*}
        &\lim_{n \to \infty} \sqrt[n]{n} = 1 && \\
        &\lim_{n \to \infty} \bigg( 1 + \frac{1}{n} \bigg)^{n} = e && \\
        &\lim_{n \to \infty} \frac
                            {a_{\alpha}n^{\alpha} + a_{\alpha - 1}n^{\alpha - 1} + \ldots}
                            {b_{\alpha}n^{\alpha} + b_{\alpha - 1}n^{\alpha - 1} + \ldots}
                            = \frac{a_{\alpha}}{b_{\alpha}} &&
      \end{flalign*}

\end{multicols}
  % \end{landscape}
\end{document}
