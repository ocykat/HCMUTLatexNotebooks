\documentclass[12pt, a4paper]{report}

% === MARGINS ===
\usepackage[margin=0.75in]{geometry}

% === HEADINGS ===
\setcounter{secnumdepth}{4}
\newcommand{\hi}{\section}
\newcommand{\hii}{\subsection}
\newcommand{\hiii}{\subsubsection}
\newcommand{\hiiiBEGIN}[1]{\subsubsection \begin{enumerate}}
\newcommand{\hiiiEND}{\end{enumerate}}
\newcommand{\hiv}{\item\textbf}

% === INDENTATION ===
\usepackage{indentfirst}

% === LINE SPACING ===
\setlength{\parskip}{6pt}

% === TABLE OF CONTENTS ===
% links
\usepackage{hyperref}
\hypersetup{
  colorlinks,
  citecolor=black,
  filecolor=black,
  linkcolor=black,
  urlcolor=black
}

% === APPENDICES ===
\usepackage[toc, page]{appendix}

% Bold, italic, underlined text
\newcommand{\tb}[1]{\textbf{#1}}
\newcommand{\ti}[1]{\textit{#1}}
\newcommand{\tbi}[1]{\textbf{\textit{#1}}}
\newcommand{\tu}[1]{\underline{#1}}
\newcommand{\tbu}[1]{\textbf{\underline{#1}}}
\newcommand{\smf}[1]{\small #1 \normalsize}
\newenvironment{smfont}{\small}{\normalsize}

% === ENUMERATE ===
\usepackage[shortlabels]{enumitem}
% Usage: \begin{enumerate}[a.] or [a)] or [(A)] etc.

% === MATH ===
% Basic packages
\usepackage{amsmath}
\usepackage{amssymb}
\usepackage{gensymb}

% Equation box
\usepackage{empheq}
\newenvironment{eqbox}
  {\setkeys{EmphEqEnv}{align}\setkeys{EmphEqOpt}{box=\fbox}\EmphEqMainEnv}
  {\endEmphEqMainEnv}

% Arrows
\newcommand{\ra}{\Rightarrow}
\newcommand{\lra}{\Leftrightarrow}

% Sum, product, limit
\newcommand{\SUM}[1]{\sum\limits #1}
\newcommand{\PROD}[1]{\prod\limits #1}

% Bold in math mode
\usepackage{bm}

% Absolute value
\DeclarePairedDelimiter\abs{\lvert}{\rvert}%
\DeclarePairedDelimiter\norm{\lVert}{\rVert}%
%   Swap the definition of \abs* and \norm*, so that \abs
%   and \norm resizes the size of the brackets, and the
%   starred version does not.
\makeatletter
\let\oldabs\abs
\def\abs{\@ifstar{\oldabs}{\oldabs*}}
\let\oldnorm\norm
\def\norm{\@ifstar{\oldnorm}{\oldnorm*}}
\makeatother

% Floor and ceiling
\usepackage{mathtools}
\DeclarePairedDelimiter\ceil{\lceil}{\rceil}
\DeclarePairedDelimiter\floor{\lfloor}{\rfloor}

% Inner Product
\newcommand{\iprod}[1]{\langle #1 \rangle}

% Derivatives
\newcommand{\dif}[2]{\dfrac{d #1}{d #2}}
\newcommand{\ddif}[2]{\dfrac{d^2 #1}{d #2^2}}
\newcommand{\difi}[2]{d #1/d #2}
\newcommand{\diff}[2]{\dfrac{d}{d #2} #1}
\newcommand{\pd}[2]{\dfrac{\partial #1}{\partial #2}}
\newcommand{\pdds}[2]{\dfrac{\partial^{2} #1}{\partial #2^{2}}}
\newcommand{\pddsi}[2]{{\partial^{2} #1} / {\partial #2^{2}}}
\newcommand{\pdd}[3]{\dfrac{\partial^{2} #1}{\partial #2 \partial #3}}
\newcommand{\pddi}[3]{{\partial^{2} #1} / {\partial #2 \partial #3}}

% Integrals
\usepackage{esint}
\newcommand{\INT}{\int \limits}
\newcommand{\OINT}{\oint \limits}
\newcommand{\IINT}{\iint \limits}
\newcommand{\IIINT}{\iiint \limits}

% Ratios
\newcommand{\ratio}[3]{\dfrac{#1_{#2}}{#1_{#3}}}

% === FOOTNOTE ===
%   one foot note stays on one page
\interfootnotelinepenalty=10000

\newcommand{\fnmark}{\footnotemark}
\newcommand{\fnmarksame}{\footnotemark[\value{footnote}]}
\newcommand{\fntext}{\footnotetext}

% === IMAGES ===
\usepackage{graphicx}
\usepackage{caption}
\usepackage{subcaption}
\usepackage{float}
\newcommand{\img}[3][]
{
  \begin{figure}[H]
    \centering
    \includegraphics[#1]{#2}
    \caption*{#3}
  \end{figure}
}
% Usage: \img[width=...]{<path_to_img>}{<caption>}

% === PSEUDOCODE ===
\usepackage{algorithm}
\usepackage{algorithmicx}
\usepackage[noend]{algpseudocode}
\usepackage{caption}

\renewcommand{\thealgorithm}{\arabic{chapter}.\arabic{algorithm}}

\newcommand*\Let[2]{\State #1 $\gets$ #2}
\algrenewcommand\algorithmicrequire{\textbf{Input:}}
\algrenewcommand\algorithmicensure{\textbf{Output:}}
\newcommand{\INPUT}[1]{\Require{#1} \Statex}
\newcommand{\OUTPUT}[1]{\Ensure{#1} \Statex}
\newcommand{\INPUTOUTPUT}[2]{\Require{#1} \Ensure{#2} \Statex}
\newcommand{\LET}[2]{\Let{$#1$}{$#2$}}
\newcommand{\FOR}[2]{\For{$#1 \gets #2$}}
\newcommand{\ENDFOR}{\EndFor}
\newcommand{\TO}{\textrm{ \tb{to} }}
\newcommand{\DOWNTO}{\textrm{ \tb{downto} }}
\newcommand{\AND}{\textrm{ \tb{and} }}
\newcommand{\OR}{\textrm{ \tb{or} }}
\newcommand{\XOR}{\textrm{ \tb{xor} }}
\newcommand{\GETS}{\gets}
\newcommand{\IF}[1]{\If{$#1$}}
\newcommand{\ELSEIF}[1]{\ElsIf{$#1$}}
\newcommand{\ELSE}{\Else}
\newcommand{\ENDIF}{\EndIf}
\newcommand{\WHILE}[1]{\While{$#1$}}
\newcommand{\ENDWHILE}{\EndWhile}
\newcommand{\FUNCTION}[2]{\Function{#1}{$#2$}}
\newcommand{\ENDFUNCTION}{\EndFunction}
\newcommand{\PROCEDURE}[2]{\Procedure{#1}{$#2$}}
\newcommand{\ENDPROCEDURE}{\EndProcedure}
\newcommand{\CALLFUNC}[2]{\State \Call{#1}{$#2$}}
\newcommand{\CALLPROC}[2]{\State \Call{#1}{$#2$}}
\newcommand{\RETURN}[1]{\State \Return{#1}}

% === CODE ===
\usepackage{listings}
\usepackage{inconsolata}
\usepackage{color}

\definecolor{dkgreen}{rgb}{0,0.6,0}
\definecolor{gray}{rgb}{0.5,0.5,0.5}
\definecolor{mauve}{rgb}{0.58,0,0.82}

\lstset {
  language=C++,
  aboveskip=1mm,
  belowskip=1mm,
  numbers=left,
  frame=tb,
  basicstyle=\ttfamily,
  numberstyle=\tiny\color{gray},
  commentstyle = \color{dkgreen},
  keywordstyle = \color{blue},
  stringstyle = \color{mauve}
}

% === TEXTBOX ===
\usepackage{varwidth}
\newenvironment{fboxenv}
  {
    \begin{minipage}{\dimexpr\textwidth-2\fboxsep-2\fboxrule\relax}
  }
  {
    \end{minipage}
  }

\newcommand{\textbox}[1]{\framebox{\parbox{\dimexpr\linewidth-2\fboxsep-2\fboxrule}{\itshape #1 }}}


% === BOOK SECTION MARKER ===
\newcommand{\booktitle}{}
\newcommand{\booksection}[1]{\smf{\ti{Section #1 - \booktitle}}}

\begin{document}

\setcounter{chapter}{1}
\chapter{Partial Derivatives}

\hi{Function of two variables}
    \hii{Definition}
        \par A function of two variables is a rule that assigns
        to each ordered pair of real numbers $(x, y)$ in a set a unique real number
        denoted by $f(x, y)$. The set $D$ is the \textbf{domain} of $f$ and its
        \textbf{range} is the set of values that $f$ takes on.
        \par We often write:
        \begin{eqbox}
            z = f(x, y)
        \end{eqbox}
        where:
        \begin{itemize}
            \item $x$, $y$: independent variables
            \item $z$: dependent variable
        \end{itemize}
    \hii{Domain}
        \par Given the function: $f(x, y)$. The domain of $f$ is the
        \textbf{Cartesian product} of the set $D_{x}$ and the set $D_{y}$, where
        $D_{x}$ and $D_{y}$ are the set of all possible values of $x$ and $y$,
        respectively. This is true for any multivariable function.
        \begin{eqbox}
            D = D_{x} \times D_{y} \subseteq \mathbb{R}^{2}
        \end{eqbox}
    \hii{Graph}
        \par If $f$ is a function of two variables with domain $D$, then the graph
        of $f$ is the set of all points $(x, y, z)$ in $\mathbb{R}^{3}$ such that
        $z = f(x, y)$ and $(x, y) \in D$.
    \hii{Level Curves}
        \par The \textbf{level curves} of a function $f$ of two variables are the curves
        with equations $f(x, y) = k$, where $k$ is a constant (in the range of $f$).
        \par The graph of level curves is called the \textbf{contour graph}.
        \par To construct a contour graph, the set of $k$ is required.

\hi{Limits and Continuity}
    \hiiBEGIN{Limits}
        \hiii{Definition}
            \par Let $f$ be a function of two variables whose domain $D$ includes points arbitrarily
            close to $(a, b)$. Then we say that the \textbf{limit} of $f(x, y)$ as $(x, y)$ approaches
            $(a, b)$ is $L$ and we write
            \begin{eqbox}
                \lim_{(x, y) \to (a, b)} f(x, y) = L
            \end{eqbox}
            if for every number $\epsilon > 0$ there is a corresponding number $\delta > 0$ such that:
            \begin{center}
                if $(x, y) \in D$ and $0 < \sqrt{(x - a)^{2} - (y - b)^{2})} < \delta$, then
                $|f(x, y) - L| < \epsilon$
            \end{center}
        \hiii{Existence of limit}
            \par For a function of one variable
            \begin{center}
                If $\lim_{x \to a^{-}} f(x) \neq \lim_{x \to a^{+}}$, then $\lim_{x \to a} f(x)$ does
                not exists.
            \end{center}
            \par For a multivariable function, the limit at one point can be approaches from
            infinitely many directions. If there \textbf{exists two different paths} of approach
            along which the function $f(x, y)$ has different limits, then the limit at that point
            does not exists.
    \hiiEND
    \hii{Continuity}
        \par A function $f$ of two variables is called \textbf{continuous} at $(a, b)$ if:
        \begin{eqbox}
            \lim_{(x, y) \to (a, b)} f(x, y) = f(a, b)
        \end{eqbox}
        \par We say $f$ is continuous on $D$ if $f$ is continuous at every point $(a, b)$ in $D$.

\hi{Partial Derivatives}
    \hii{Definition}
        \par Given the function $f(x, y)$.
        \par The \textbf{partial derivative} of $f$ with respect to $x$ at $(a, b)$, denoted by
        $f_{y}(a, b)$, is obtained by keeping $y$ fixed $(y = b)$ and finding the ordinary
        derivative at $a$ of the function $G(x) = f(x, b)$.
        \begin{eqbox}
            f_{x} (x, y) = \lim_{\Delta x \to 0} \frac{f(x + \Delta x, y) - f(x, y)}{\Delta x} \\
            f_{y} (x, y) = \lim_{\Delta y \to 0} \frac{f(x, y + \Delta y) - f(x, y)}{\Delta y}
        \end{eqbox}
    \hii{Notations}
        \begin{eqbox}
            f_{x} (x, y) = f_{x} = \pd{f}{x} = \pd{}{x} f(x, y) = \pd{z}{x} = D_{1}f = D_{x}f \\
            f_{y} (x, y) = f_{y} = \pd{f}{y} = \pd{}{y} f(x, y) = \pd{z}{y} = D_{2}f = D_{y}f
        \end{eqbox}
    \hii{Rule for Finding Partial Derivatives}
        \par Given the function $z = f(x, y)$.
        \begin{itemize}
            \item To find $f_{x}$, regard $y$ as a constant and differentiate $f(x, y)$ with respect
                to $x$.
            \item To find $f_{y}$, regard $x$ as a constant and differentiate $f(x, y)$ with respect
                to $y$.
        \end{itemize}
    \hii{Higher Derivatives}
        \begin{alignat*}{4}
            (f_{x})_{x} (x, y) = f_{xx} = f_{11}
                &= \pd{}{x}\bigg(\pd{f}{x}\bigg)
                &= \pd{^{2}f}{x^{2}}
                &= \pd{^{2}z}{x^{2}} \\
            (f_{x})_{y} (x, y) = f_{xy} = f_{12}
                &= \pd{}{y}\bigg(\pd{f}{x}\bigg)
                &= \pdd{^{2}f}{y}{x}
                &= \pdd{^{2}z}{y}{x} \\
            (f_{y})_{x} (x, y) = f_{yx} = f_{21}
                &= \pd{}{x}\bigg(\pd{f}{y}\bigg)
                &= \pdd{^{2}f}{x}{y}
                &= \pdd{^{2}z}{x}{y} \\
            (f_{y})_{y} (x, y) = f_{yy} = f_{22}
                &= \pd{}{y}\bigg(\pd{f}{y}\bigg)
                &= \pd{^{2}f}{y^{2}}
                &= \pd{^{2}z}{y^{2}}
        \end{alignat*}


\hi{Tangent Plane and Linear Approximation}
    \hii{Differentials}
        \par For a differentiable function of two variables, $z = f(x, y)$, we define the
        differentials $dx$ and $dy$ to be independent variables. Then the differential $dz$,
        also called the total differntial, is defined by:
        \begin{eqbox}
            dz = f_{x}(x, y) dx + f_{y}(x, y) dy = \pd{f}{x} dx + \pd{f}{y} dy
        \end{eqbox}
        \par If we take $dx = \Delta x = x - a$ and $dy = \Delta y = y - b$,
        then the differential of $z$ is:
        \begin{alignat*}{2}
            dz &= f_{x} (x_{1}, y_{1}) \Delta x + f_{y} (x_{1}, y_{1}) \Delta y \\
            &= f_{x} (x_{1}, y_{1}) (x_{2} - x_{1}) + f_{y} (x_{1}, y_{1}) (y_{2} - y_{1})
        \end{alignat*}

\begin{multicols*}{3}

\hi{Distributions}
  \hii{Discrete Distributions}

  \hiii{Binomial Distribution}
    \begin{itemize}
      \item \tb{Intepretation}:
        \begin{itemize}
          \item The number of winning game in $n$ games, given that the chance to win any one game is $p$.
          \item The number of red balls obtained after picking \tb{with replacement} from $n$ balls from a box of red and blue balls, given that the chance to pick a red ball at one time is $p$.
        \end{itemize}
      \item \tb{Notation}:
        \[
          X \sim B(n, p)
        \]
      \item \tb{P.M.F}:
        \[
          f(k) = P(X = k) = {n \choose k} p^k q^{n - k}
        \]
      \item \tb{Expectation}:
        \[
          \mu = E(X) = np
        \]
      \item \tb{Variance}:
        \[
          \sigma^2 = V(X) = npq = np(1 - p)
        \]
    \end{itemize}

  \hiii{Hypergeometric Distribution}
    \begin{itemize}
      \item \tb{Intepretation}:
        \begin{itemize}
          \item The number of red balls obtained after picking \tb{without replacement} $n$ balls from $N$ balls from a box of red and blue balls, given that the number of red balls in the box is $m$.
        \end{itemize}
      \item \tb{Notation}:
        \[
          X \sim H(N, m, n)
        \]
      \item \tb{P.M.F}:
        \[
          f(k) = P(X = k) = \dfrac{{m \choose k} {N - m \choose n - k}}{{N \choose n}}
        \]
      \item \tb{Expectation}:
        \[
          \mu = E(X) = np, \text{ where } p = \frac{m}{N}
        \]
      \item \tb{Variance}:
        \[
          \sigma^2 = V(X) = npq \frac{N - n}{N - 1}, \text{ where } p = \frac{m}{N}
        \]
    \end{itemize}

  \hiii{Poisson Distribution}
    \begin{itemize}
      \item \tb{Intepretation}:
        \begin{itemize}
          \item The number of events occuring in a fixed period of time.
        \end{itemize}
      \item \tb{Notation}:
        \[
          X \sim Po(\lambda)
        \]
      \item \tb{P.M.F}:
        \[
          f(k) = P(X = k) = \frac{\lambda^k}{k!} e^{-\lambda}
        \]
      \item \tb{Expectation}:
        \[
          \mu = E(X) = \lambda
        \]
      \item \tb{Variance}:
        \[
          \sigma^2 = V(X) = \lambda
        \]
    \end{itemize}

  \hii{Continuous Distributions}

  \hiii{Uniform Distribution}
    \begin{itemize}
      \item \tb{Notation}:
        \[
          X \sim U(a, b)
        \]
      \item \tb{P.D.F}:
        \[
          f(x) =
            \begin{cases}
              \dfrac{1}{b - a} & x \in [a, b]\\
              0 & \text{otherwise}
            \end{cases}
        \]
      \item \tb{C.D.F}:
        \[
          F(x) =
            \begin{cases}
              0 & x < a \\
              \dfrac{x - a}{b - a} & x \in [a, b)\\
              1 & x \geq b
            \end{cases}
        \]
      \item \tb{Expectation}:
        \[
          \mu = E(X) = \frac{a + b}{2}
        \]
      \item \tb{Variance}:
        \[
          \sigma^2 = V(X) = \frac{(b - a)^2}{12}
        \]
    \end{itemize}

  \hiii{Normal Distribution}
    \begin{itemize}
      \item \tb{Notation}:
        \[
          X \sim N(\mu, \sigma^2)
        \]
      \item \tb{Standardizing}:
        \[
          Y = \frac{X - \mu}{\sigma} \sim N(0, 1)
        \]
      \item \tb{C.D.F}:
        \par For $X \sim N(0, 1)$:
        \[
          \Phi(x) = \int\limits_{-\infty}^{x} f(u) du
        \]
      \item \tb{Expectation}:
        \[
          \mu = E(X) = \mu
        \]
      \item \tb{Variance}:
        \[
          \sigma^2 = V(X) = \sigma^2
        \]
      \item \tb{Properties}
        \begin{itemize}
          \item If $X \sim N(\mu, \sigma^2)$ and $Y = aX + b$:
            \[
              Y \sim N(a\mu + b, a^2 \sigma^2)
            \]
          \item If $X \sim N(\mu, \sigma^2)$:
            \[
              \sum\limits_{i = 1}^{n} X_i \sim N\bigg( \sum\limits_{i = 1}^{n} \mu_i, \sum\limits_{i = 1}^{n} \sigma_i^2 \bigg)
            \]
        \end{itemize}
    \end{itemize}


\end{multicols*}

\begin{multicols*}{3}

\hi{Appendix}

  \hii{Integrals}
    \begin{itemize}
      \item \[\int x^n dx = \frac{1}{n + 1} x^{n + 1} + C \]
      \item \[\int \frac{1}{x} dx = \ln|x| + C \]
      \item \[\int \frac{1}{t^2} dx = - \frac{1}{t} + C\]
      \item \[\int e^x dx = e^x + C\]
      \item \[\int a^x dx = \frac{a^x}{\ln(a)} + C\]
      \item \[\int e^{ax} dx = \frac{e^{ax}}{a} + C\]
      \item \[\int x e^x dx = (x - 1)e^x + C\]
      \item \tb{Integration by parts}:
        \[
          \int udv = uv - v \int du
        \]
        \par Choose $u$ in this order: log - algebraic - trig - exp
    \end{itemize}

\end{multicols*}

\begin{multicols}{3}
\hi{Confidence Intervals}

  \hii{Definitions and Terminology}
    \hiii{Terminology}
      \begin{itemize}
        \item A \tb{population} is a collection of objects about which information is sought.
        \item A \tb{sample} is a part of the population that is observed.
        \item A \tb{parameter} is a numerical characteristic of a population.
        \item A \tb{statistics} is a numerical function of the sampled data, used to estimate an unknown parameter
      \end{itemize}

    \hiii{Mean}
      \begin{itemize}
      \item \tb{Population Mean}: denoted $\mu$, is the average of all values in the \tb{entire population}.
      \item \tb{Sample Mean}: denoted $\hat{\mu} = \bar{X}$, is the average value of a \tb{sample}. For a sample with $n$ values:
        \[
          \hat{\mu} = \bar{X} = \frac{x_1 + \ldots + x_n}{n}
        \]
      \end{itemize}

    \hiii{Median}
      \par List all data values in sorted order from smallest to largest. For a sample with $n$ values:
      \begin{itemize}
        \item If $n \not \mid 2$: the mean is one single value at the middle position.
        \item If $n \mid 2$: the mean is the average of two values at the middle positions.
      \end{itemize}

    \hiii{Variance}
      \par The \tb{sample variance}, denoted $s^2$, is the approximation of the \tb{population variance} $\sigma^2$.
      \[
        s^2 = \frac{\sum (x_i - \bar{X})^2}{n - 1} = \frac{\sum (x_i - \hat{\mu})^2}{n - 1}
      \]
      where $n - 1$ is called the \tb{degree of freedom}.

    \hiii{Confidence Level and Significance Level}
      \begin{itemize}
        \item \tb{Confidence level} $\gamma$ is the measure of the degree of reliability of the interval.
        \item \tb{Significance level} $\alpha$ is the probability we allow ourselves to be wrong when we are estimating a parameter with a confidence interval.
      \end{itemize}
      \[
        \gamma + \alpha = 1
      \]

  \hii{Case 1: Normal Population with Known Standard Deviation}
    \hiii{Confidence Interval of the Population Mean}
    \par If $x_1, \ldots, x_n$ are independent and identically distributed (i.i.d.), following a normal distribution $N(\mu, \sigma^2)$ and $\alpha = 1 - \gamma$, then the \tb{confidence interval of the population mean} is:
    \[
      \mu = \hat{\mu} \pm z_{\alpha / 2} \times \frac{\sigma}{\sqrt{n}}
    \]

    \hii{Sample Size According to Maximum Error}
      \par If $\bar{X}$ is used to estimate $\mu$, then we can be $100(1 - a)\%$ confident that the error will not exceed a specified amount $\epsilon$ when the sample size is:
      \[
        n = \bigg\lceil \frac{\sigma z_{\alpha/2}}{\epsilon} \bigg\rceil^2
      \]

    \hii{One-Sided Convidence Interval of the Population Mean}
      \begin{itemize}
        \item A $100(1 - \alpha)\%$ upper-confidence bound for $\mu$ is
        \[
          \mu \leq \hat{\mu} + z_{\alpha} \times \frac{\sigma}{\sqrt(n)}
        \]
        \item A $100(1 - \alpha)\%$ lower-confidence bound for $\mu$ is
        \[
          \mu \geq \hat{\mu} - z_{\alpha} \times \frac{\sigma}{\sqrt(n)}
        \]
      \end{itemize}

  \hii{Case 2: Normal Population with Unknown Standard Deviation}
    \hiii{Confidence Interval of the Population Mean}
     \par If $x_1, \ldots, x_n$ are independent and identically distributed (i.i.d.) then the \tb{confidence interval of the population mean} is:
     \[
      \mu = \hat{\mu} \pm t_{n - 1, \alpha / 2} \times \frac{s}{\sqrt(n)}
     \]
    \hiii{Confidence Interval of the Population Variance}
      \begin{itemize}
        \item \tb{Two-sided Confidence Interval}
        \par Choose $c_1$ and $c_2$ so that the area in each tail of $\chi_{n - 1}^2$ distribution is $\alpha / 2$. Then the $\gamma$-confidence interval
        for the unknown variance $\sigma^2$ is:
        \[
          \frac{(n - 1)s^2}{c_2} \leq \sigma^2 \leq \frac{(n - 1)s^2}{c_1}
        \]
        \item \tb{One-sided Confidence Interval}
        \par Choose $c_1$ and $c_2$ so that the area in each tail of $\chi_{n - 1}^2$ distribution is $\alpha$. Then the $\gamma$-confidence interval
        for the unknown variance $\sigma^2$ are:
        \[
          \sigma^2 \geq \frac{(n - 1)s^2}{c_2}
        \]
        and
        \[
          \sigma^2 \leq \frac{(n - 1)s^2}{c_1}
        \]
      \end{itemize}

    \hii{Case 3: Large Sample Size}
      \hiii{Confidence Interval of the Population Mean}
      \par If $x_1, \ldots, x_n$ are independent and identically distributed (i.i.d.) and $n$ is large, then:
        \[
          \mu = \hat{\mu} \pm z_{\alpha / 2} \times \frac{s}{\sqrt(n)}
        \]

\end{multicols}

\clearpage


\tableofcontents

\begin{appendices}
\end{appendices}

\end{document}
